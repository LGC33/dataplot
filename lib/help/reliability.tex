This is file RELDEF.TEX     6/1/94
Reliability Analysis Definitions
 
 
Acceleration Factors
   The physical factors which are used in accelerated
   testing to hasten the occurance of failures.  Common
   stress variables include: temperature, load, load
   cycling, chemical contaminants, chemical concentrations,
   speed, solar radiation, etc.
   Synonym: Stress variables
   Reference: Nelson, Accelerated Testing, p. 11-12.
 
Accelerated Testing
   Tests which are carried out at high stress (e.g., high
   temperature) conditions so as to yield failures quickly.
   The underlying assumption is that such high-stress
   testing yields useful information about product life at
   (normal-stress) design conditions.  The fundamental
   problem in analyzing accelerated testing data is how to
   validly map such high-stress lifetimes into normal-stress
   lifetimes.
   Reference: Nelson, Accelerated Testing, p. xi.
 
Censored Item
   An item whose true failure time is unknown, but is known
   to have failed before / after a specified cutoff value.
 
Censored Data Point
   The failure time "assigned" to an item which failed before / after
   a cutoff time (and thus the true failure time is unknown).
   The assigned value is not the true (unknown) failure
   time t, but rather is simply designated as "<c" / ">c".
 
Censored Sample
   The set of data that results when
      1) the entire target population is available
         to be sampled from;
      2) the experiment design has a cutoff time before / after
         which the failure time was unknown or not recorded;
      3) one or more items failed before / after the cutoff time.
 
Censor Time for a Test
   The test cutoff time at which
      1) failure times start to be recorded (for left-censoring)
      2) failure times stop  being recorded (for right-censoring)
 
CME Function
   Synonym: Mean residual life function (see write-up)
   Synonym: Life expectance function
   Synonym: Conditional mean exceedance function
   Synonym: CME function
 
Complete Sample
   A sample with no censored items; that is, all of the items
   failed within the time span of the test.
 
Competing Risks
   Synonym: Failure modes (see write-up)
 
Conditional Mean Exceedance Function
   Synonym: Mean residual life function (see write-up)
   Synonym: Life expectancy function
   Synonym: CME function
 
Confidence Interval
   [a,b] is a 95% confidence interval for a parameter
   if the data analysis leads the analyst to be 95%
   confident that the true value of the parameter lies
   in the interval [a,b]; that is
 
      Prob[ a < parameter < b ] = .95
 
   a and b are said to be the "confidence limits".
   Strictly speaking, a confidence interval is a random
   interval--the interval itself will change randomly
   from one data sample to the next.  The 95%
   confidence interval from a procedure (for example:
   t, chi-squared, F, etc.) has the property that 95%
   of these random intervals obtained in this
   systematic way from (conceptually) repeated samples
   will have the property that they cover the true,
   unknown (and will always be unknown), fixed value of
   the parameter.
 
Cumulative Distribution Function F(y)
   F(y) = The probability of an item failing before age y
          (a monotonically-increasing function).
        = Prob(Y <= y)
   Reference: Nelson, Applied Life Data Analysis, p. 19.
 
Cumulative Hazard Function H(y)
   H(y) = The cumulative integral of the hazard function h(y).
        = a monotonically-increasing function.
        = integral h(y)
        = -log(R(y))
        = -log(1 - F(y))
   Reference: Nelson, Applied Life Data Analysis, p. 27.
 
Cumulative Hazard Rate Function
   Synonym: cumulative hazard function
 
Degradation Mechanisms
   The physical mechanisms which cause an item to fail.
   They include: fatigue, creep, cracking, wear, corrosion,
   oxidation, weathering, etc.
 
Distribution (of a Random Variable)
   A probability model which describes the relative
   frequency (likelihood of occurance) of failure times
   of (target) population items.
   Common reliability analysis distributions include
   exponential, lognormal, logistic, normal, Weibull, etc.
 
Distribution Parameters
   Numbers which describe the location,
   the spread, and the shape of a distribution.
 
Durability
   to be done
 
Exponential Distribution
   A distribution with z = (x-mu)/sigma   and   y>= mu
      f(y)   = (1/sigma) * exp(-z)
      F(y)   = 1 - exp(-z)
      ppf(p) = mu + sigma * log(1/(1-p))
      R(y)   = exp(-z)
      h(y)   = 1/sigma
      H(y)   = y/sigma
      CME(y) = ???
 
Experiment Design
   The organization, planning, and execution of the testing
   so as to carry out the engineering objectives of the project.
 
Failure Modes
   This term is applicable only for items or systems which
   can fail due to multiple physical failure mechanisms.
   Each different potential failure mechanism is referred to
   as a different "failure mode" or (equivalently) a
   different "competing risk".
   Synonym: Competing risks
 
Failure Rate Function
   Synonym: Hazard function h(y)  (See write-up)
   Synonym: Hazard rate
   Synonym: Mortality rate
   Synonym: Force of mortality
 
Failure Time (of an Item) Y
   The time at which an item under test did in fact fail/die.
   Synonym: Lifetime
 
Force of Mortality (Function)
   Synonym: Hazard function h(y)  (See write-up)
   Synonym: Hazard rate (function)
   Synonym: Failure rate function
   Synonym: Mortality rate (function)
   Synonym: Intensity function
 
Hazard Function h(y)
   h(y) = The instantaneous failure rate of an item at age y.
        = The propensity to failure as a function of age.
        = The prob. of failure at age y given survival to age y.
        = The (conditional) probability of failure at age y.
        = Prob(Y = y) / Prob(Y > y)
        = f(y) / (1 - F(y))
        = f(y) / R(y)
   Interpretation: h(y) decreasing? increasing?
   Synonym: Hazard rate (function)
   Synonym: Failure rate function
   Synonym: Mortality rate (function)
   Synonym: Force of mortality (function)
   Synonym: Intensity function
   Reference: Nelson, Applied Life Data Analysis, p. 25.
 
Hazard Plot
   A methodology for determining the best-fit distribution
   for lifetime data.  Each individual distribution has its own
   hazard plot.  For a given data set, an exponential
   hazard plot will appear different than a normal hazard
   plot, a Weibull hazard plot, etc.  Hazard plots are
      1) graphical;
      2) parametric (i.e., output = a distribution);
      3) similar to probability plotting;
      4) usable for both complete and incomplete samples;
      5) easy to interpret--a linear hazard plot
         implies a good distributional fit.
 
Hazard Rate (Function)
   Synonym: Hazard function h(y)  (See write-up)
   Synonym: Failure rate function
   Synonym: Mortality rate (function)
   Synonym: Force of mortality (function)
   Synonym: Intensity function
 
Incomplete Sample
   A sample with one or more censored items; that is, some item
   failed outside of the time span of the test.
 
Interval Censored Item
   An item whose exact failure time is unknown, but is
   known to have failed in the time interval spanned
   by 2 known cutoff values.
 
Intensity Function
   Synonym: Hazard function h(y)  (See write-up)
   Synonym: Hazard rate (function)
   Synonym: Failure rate function
   Synonym: Mortality rate (function)
   Synonym: Force of mortality (function)
   Synonym: Intensity function
 
Item
   The physical object whose time to failure is the focus of study.
   Synonym: Test Item
   Synonym: Unit
 
Kaplan-Meier Estimation
   A non-parametric method for estimating the values of the
   cumulative distribution function at the observed sample
   values.  It may be used for both complete and for incomplete
   samples.
   Reference: Nelson, Applied Life Data Analysis, p. 149.
 
Left-Censored Item
   An item whose true failure time is unknown, but is
   known to have failed before a specified cutoff value.
 
Left-Censored Data Point
   The failure time "assigned" to an item which failed before
   a cutoff time (and thus the true failure time is unknown).
   The assigned value is not the true (unknown) failure
   time t, but rather is simply designated as "<c".
 
Left-Censored Sample
   The set of data that results when
      1) the entire target population is available
         to be sampled from;
      2) the experiment design has a cutoff time before
         which the failure time was unknown or not recorded;
      3) one or more items failed before the cutoff time.
 
Left-Truncated item
 
Left-Truncated data point
 
Left-Truncated sample
 
Life Expectancy Function
   Synonym: Mean residual life function (see write-up)
   Synonym: Life expectance function
   Synonym: Conditional mean exceedance function
   Synonym: CME function
 
Lifetime (of an Item) Y
   Synonym: Failure Time (see write-up)
 
Lifetime Analysis
   Synonym: Reliability analysis (see write-up)
   Synonym: Survival analysis
 
Location of a Distribution
   A number (= a parameter) which describes "where" a
   distribution is positioned on the response variable axis.
   For some distributions (e.g., normal), this number
   describes the "middle" of the distribution.  For other
   distribution (e.g., exponential and Weibull), this number
   may describe where the distribution "starts".
 
Lognormal Distribution
   A distribution with z = (x-mu)/sigma   and   y>= mu
      f(y)   =
      F(y)   =
      ppf(p) =
      R(y)   =
      h(y)   =
      H(y)   =
      CME(y) = ???
 
Martingale
 
Maximum Likelihood Estimation
   A general parametric method for estimating the parameters
   of a distribution.  The maximum likelihood estimators of
   the distributional parameters are those values which
   maximize the likelihood function (= product of density
   functions).  A vast body of literature exists for maximum
   likelihood estimation.  It may be used for both complete
   and incomplete samples.
   Reference: Nelson, Applied Life Data Analysis, p. 313ff.
 
Maximum PPCC Estimation
   A parametric method for estimating the shape and/or tail
   length parameter of a distributional family (e.g.,
   Weibull).  The maximum PPCC (PPCC = probability plot
   correlation coefficient) estimator of the distributional
   family shape and/or tail length parameter is the value
   which maximizes the linearity of the probability plot of
   the distribution.  It may be used for both complete and
   incomplete samples.
   Reference: Filliben...
 
Mean Life M
   That time value M which serves as the centroid balance point
   of the distribution of population failure times.
   The mean life M is defined as
      M = integral y*f(y) dy
   Synonym: Mean Life (see write-up)
   Synonym: Mean Time to Failure MTTF
   Synonym: Mean Time between Failures MTBF
 
Mean Residual Life Function
   MRL(y) = The typical remaining time to failure as a function of age.
          = The average remaining life at age y given survival to age y.
          = The (conditional) mean time to failure beyond age y.
          = integral y*h(y) dy ???
          = (integral (y,infinity) y*f(y) dy) / (1 - F(y))  ???
   Synonym: Life expectancy function
   Synonym: Conditional mean exceedance function
   Synonym: CME function
 
Mean Time to Failure MTTF
   Synonym: Mean Life (see write-up)
   Synonym: Mean Time between Failures MTBF
 
Mean Time Between Failures MTBF
   Synonym: Mean Life (see write-up)
   Synonym: Mean Time to Failure MTTF
 
Median Life
   That time value for which exactly 50% of the population
   failure times fall below (and necessarily 50% fall above).
   Reference: Nelson, Applied Life Data Analysis, p. 22.
 
Mortality Rate (Function)
   Synonym: Hazard function h(y)  (See write-up)
   Synonym: Hazard rate (function)
   Synonym: Failure rate function
   Synonym: Force of mortality (function)
   Synonym: Intensity function
 
Multiply-Censored Sample
   A (usually) right-censored sample in which
      1) different items go "on test" at different times;
      2) there was only a single stop (censor time);
      3) at least one item was censored.
   The net effect is that the resulting sample is intermixed
   with known life times for some items which exceed bounded
   life times for other items.  Distributional estimation
   of multiply-censored samples may be done via hazard plotting.
 
Nelson-Aalen Estimator (of chf)
 
Non-parametric Methods
   In a reliability analysis context, any statistical
   estimation procedure which yields estimated
      1) mean lifes
      2) quantiles
      3) predicted distribution/reliability/hazard/etc. values
   without having to resort to intermediate specific
   best-fit distribution functions (such as normal,
   exponential, Weibull, etc.) An example of a
   non-parametric methods is the Kaplan-Meier estimate of
   cumulative distribution function values.
 
Normal Distribution
   A distribution with z = (x-mu)/sigma
      f(y)   = (1/sqrt(2*pi)*sigma) * exp(-0.5*z**2)
      F(y)   =
      ppf(p) =
      R(y)   =
      h(y)   =
      H(y)   =
      CME(y) =
 
On Test
   To initiate the life testing of the item under study.
 
Parameter of a Distribution
   Numbers which characterize the
      1) location,
      2) variation,
      3) shape, or
      4) tail-length
   of the distribution.
 
Parametric Methods
   In a reliability analysis context, any statistical
   estimation procedure which yields estimated
      1) mean lifes
      2) quantiles
      3) predicted distribution/reliability/hazard/etc. values
      4) distribution/reliability/hazard/etc. functions
      5) distribution/reliabilit/hazard/etc. function parameters
   by means of first estimating the best-fit distribution
   function (e.g., normal, exponential, Weibull, etc.)
   and the parameters of that best-fit distribution.
   Examples of parametric methods is normal hazard plotting,
   exponential hazard plotting, and Weibull hazard plotting.
 
Percent Point Function
 
Percentile
   Synonym: Quantile
 
Population (= target population)
   1. The set of items for which the analyst wishes
      to make conclusions about; or
   2. The failure times of those items for which the
      analyst wishes to make conclusions about.
 
PPCC Plotting
 
Probability Density Function f(y)
   f(y) = The instantaneous probability of failure at age y.
        = The (unconditional) probability of failure at age y.
        = Prob(Y = y)
   Interpretation: f(y) decreasing? increasing?
   Reference: Nelson, Applied Life Data Analysis, p. 16.
 
Probability Plot
   A methodology for determining the best-fit distribution
   for lifetime data.  Each individual distribution has its own
   probability plot.  For a given data set, an exponential
   probability plot will appear different than a normal probability
   plot, a Weibull probability plot, etc.  Probability plots are
      1) graphical;
      2) parametric (i.e., output = a distribution);
      3) similar to hazard plotting;
      4) usable for both complete and incomplete samples;
      5) easy to interpret--a linear probability plot
         implies a good distributional fit.
 
Probability Plot Estimation
   A parametric method for estimating the location and scale
   parameters of a distribution (e.g., normal or
   exponential), or for a member of a distributional family
   (e.g., Weibull with shape parameter 2.5, or gamma with
   shape parameter 6).  The probability plot estimates of
   the location and scale parameters are the estimated
   intercept and slope from the least squares linear fit of
   the probability plot.  Probability plot estimation may be
   used for both complete and incomplete samples.
   Reference: Filliben...
 
 
Product Life Estimation
   A general non-parametric method for estimating the values
   of the cumulative distribution function (and other
   distribution functions such as the reliability function)
   at the observed sample values.  The Kaplan-Meier
   estimation method is a special case of the product-limit
   estimation method.  The product-limit method may be used
   for both complete and for incomplete samples.
   Reference: Nelson, Applied Life Data Analysis, p. 149.
 
Quantile
   Synonym: Percentile
 
Random Variable Y
 
Reliability Analysis
   The data analysis of age data whose purpose is to make
   inferences about a target population of items; such
   inferences involve the estimation of
      1) mean life
      2) quantiles
      3) predicted distribution values
      4) best-fit distribution identification
      5) best-fit distributional parameters
      6) best-fit distributional functions
   and the attachment of uncertainty values and/or
   uncertainty limits for the above estimates.
   Synonym: Survival analysis
   Synonym: Lifetime analysis
 
Reliability Function R(y)
   R(y) = the probability of an item surviving beyond age y
          (a monotonically-decreasing function).
        = Prob(Y > y)
        = 1 - Prob(Y <= y)
        = 1 - cdf(y)
        = 1 - F(y)
   Synonym: Survival function
   Synonym: Survivorship function
   Reference: Nelson, Applied Life Data Analysis, p. 21.
 
Right-Censored Item
   An item whose true failure time is unknown, but is
   known to have failed only after a specified cutoff value.
 
Right-Censored Data Point
   The failure time "assigned" to an item which failed after
   a cutoff time (and thus the true failure time is unknown).
   The assigned value is not the true (unknown) failure
   time t, but rather is simply designated as ">c".
 
Right-Censored Sample
   The set of data that results when
      1) the entire target population is available
         to be sampled from;
      2) the experiment design has a cutoff time after
         which the failure time was unknown or not recorded;
      3) one or more items failed after the cutoff time.
 
Right-Truncated sample
 
Right-Truncated Data Point
 
Right-Truncated Sample
 
Sample
   The set of lifetime data values collected during a test.
 
Shape of a Distribution
   A number (= a parameter) which describes whether a
   distribution is symmetric (with probability density
   function being a mirror-image about a center point, such
   as the normal distribution), or skewed (non-symmetric,
   such as the exponential, lognormal, and Weibull
   distributions).
 
Singly-Censored Sample
   A (usually) right-censored sample in which
      1) all items were started on test at the same time;
      2) there was only a single stop (censor time);
      3) at least one item was censored.
   Distributional estimation of singly-censored samples may
   be done via probability plotting.
 
Spread of a Distribution
   Synonym: Variation of a distribution (see write-up)
 
Standard Deviation
   The standard deviation of a random variable or a statistic is a
   number which describes how "spread out" (how "variable")
   the distribution of the random variable or statistic is.
   The standard deviation is defined as the
      square root of the integral of (x-mu)**2 * f(x) dx
   where mu is the mean of the random variable and
   f(x) is the probability density function of the random
   variable.  The standard deviation of a random variable or a
   statistic is identically the square root of the variance
   of the random variable or the statistic.  In choosing
   between the standard deviation or the variance to
   describe the spread of a distribution, we recommend
   the use of the standard deviation (which is in units of
   the data) as opposed to the variance (which is
   artifically in squared units of the data).
 
Stress Variables
   Synonym: Acceleration factors   (see write-up)
 
Survival Analysis
   Synonym: Reliability analysis (see write-up)
   Synonym: Lifetime analysis
 
Survival Function
   Synonym: Reliability function  (see write-up)
   Synonym: Survivorship function
 
Survivorship Function
   Synonym: Reliability function  (see write-up)
   Synonym: Survival function
 
Survival Distribution Function
 
Tail-Length of a Distribution
   A number (= a parameter) which describes the propensity
   of a spread-standardized distribution to yield
   "outlying-appearing" values.  Short-tailed distributions
   (such as the uniform distribution) have finite-domain and
   so no "tails" and thus yield no "outliers".
   Moderate-tailed distributions (such as the normal
   distribution) have infinite-domain probability density
   functions but which rapidly approach zero and so have
   small probability of "outliers".  Long-tailed
   distributions (such as the Cauchy distribution) have
   infinite-domain probability density functions but which
   approach zero only slowly and so have high probability of
   "outliers".
 
Test
   The process of executing a lifetime experiment and
   recording failure times from such an experiment.  If the
   experiment is carried out under higher-than-normal stress
   levels so as to hasten failures and thus shorten the
   total elapsed time of the test, then such a test is
   referred to as an "accelerated test".
 
Test Item
   Synonym: Item (see write-up)
   Synonym: Unit
 
Tolerance Limits
 
Truncated Data Point
 
Truncated Sample
   The set of data that results when a certain subset
   of the target population is unavailable to be put on test.
 
Type I Censored Sample
   A censored sample in which the cutoff time is pre-fixed,
   and thus the actual number of censored items will be random.
   Example--stop the test at 100 days.
   Type I censoring is quite common in practice.
 
Type II Censored Sample
   A censored sample in which the cutoff time is not fixed
   but rather is defined via the actual number of failures
   or the proportion of failures.  Thus the cutoff time is
   random and the number (or proprotion) of failed is fixed.
   Example--stop the test when 90% of the items have failed.
   Type II censoring is less common in practice.
 
Unit
   Synonym: Item (see write-up)
   Synonym: Test item
 
Variance
   The variance of a random variable or a statistic is a
   number which describes how "spread out" (how "variable")
   the distribution of the random variable or statistic is.
   The variance is defined as the
      integral of (x-mu)**2 * f(x) dx
   where mu is the mean of the random variable and
   f(x) is the probability density function of the random
   variable.  The variance of a random variable or a
   statistic is identically the squared standard deviation
   of the random variable or the statistic.  In choosing
   between the standard deviation or the variance to
   describe the spread of a distribution, we recommend
   the use of the standard deviation (which is in units of
   the data) as opposed to the variance (which is
   artifically in squared units of the data).
 
Variation of a Distribution
   A number (= a parameter) which describes "how spread out" a
   distribution is along the response variable axis.
   Synonym: Spread of a distribution
 
Warranty
   to be done
 
Weibull Distribution
   This is a commonly-employed probability model for failure
   times.  "The" Weibull distribution is a misnomer-- it is
   really a family of distributions--ranging over a wide
   variety of shapes.  The most commonly-employed form of
   the Weibull distribution has the following distributional
   function forms (with z = (x-mu)/sigma and y>= mu):
 
      f(y)   = gamma * (x**(gamma-1)) * exp(-(x**gamma))
      F(y)   = 1 - exp(-(x**gamma))
      ppf(p) = mu + sigma * (log(1/(1-p)))**(1/gamma)
      R(y)   =
      h(y)   =
      H(y)   =
      CME(y) = ???
   The above general representation of the Weibull family is
   referred to as the "3-parameter" Weibull distribution.
   It is extremely common (but yields less-flexible
   probability models) in the lifetime literature to set mu
   = 0--this is referred to as the "2-parameter" Weibull
   distribution.
