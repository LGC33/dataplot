7000.       (VERSION 2019.04)  total number of lines in file (including this line)
29.                          number of     sections below
   50.    OVERVIEW           first line number of OVERVIEW          section
  150.    GRAPHICS           first line number of GRAPHICS          section
  300.    DIAGRAMS           first line number of DIAGRAMS          section
  450.    ANALYSIS           first line number of ANALYSIS          section
  550.    PLOT CONTROL       first line number of PLOT CONTROL      section
  850.    SUPPORT            first line number of SUPPORT           section
 1100.    OUTPUT DEVICE      first line number of OUTPUT DEVICE     section
 1300.    KEYWORDS           first line number of KEYWORDS          section
 1450.    FUNCTIONS          first line number of FUNCTIONS         section
 1500.    MATH FUNCTIONS     first line number of MATH FUNCTIONS    section
 1800.    TRIG FUNCTIONS     first line number of TRIG FUNCTIONS    section
 1900.    PROBABIL FUNCTIONS first line number of PROB FUNCTIONS    section
 2700.    LET SUBCOMMANDS    first line number of LET SUBCOMMANDS   section
 2800.    STATISTICS         first line number of STATISTICS        section
 3700.    MATH OPERATIONS    first line number of MATH OPERATIONS   section
 3900.    MATRIX OPERATIONS  first line number of MATH OPERATIONS   section
 4000.    RANDOM NUMBERS     first line number of RANDOM NUMBERS    section
 4400.    TEXT SUBCOMMANDS   first line number of TEXT SUBCOMMANDS  section
 4500.    CAPITALIZATION     first line number of CAPITALIZATION    section
 4600.    SUBSCRIPTS         first line number of SUBSCRIPTS        section
 4700.    GREEK SYMBOLS      first line number of GREEK SYMBOLS     section
 4800.    MATH SYMBOLS       first line number of MATH SYMBOLS      section
 4900.    MISC SYMBOLS       first line number of MISC SYMBOLS      section
 5000.    CHARACTER TYPES    first line number of CHARACTER TYPES   section
 5100.    LINE TYPES         first line number of LINE TYPES        section
 5200.    COLOR TYPES        first line number of COLOR TYPES       section
 5400.    ASCII FILES        first line number of ASCII FILES       section
 6400.    SYSTEM LIMITS      first line number of SYSTEM LIMITS     section
 6600.    PROBABILITY DIST   first line number of DISTRIBUTIONS     section
 
----------------------------------------------------------
















-------------------------  *OVERVIEW*  -------------------

OVERVIEW
DATAPLOT is a language for
 
   1) graphics (continuous or discrete);
   2) fitting (non-linear or linear);
   3) general data analysis;
   4) mathematics.
 
DATAPLOT commands are high-level, English-syntax, and
self-descriptive, such as
 
   PLOT Y X
   PLOT EXP(-X**2) FOR X = -3 .1 3
   FIT Y = A+B*EXP(-ALPHA*X)
   BOX PLOT Y X
   ANOVA Y X1 X2 X3
   LET A = ROOTS SIN(X**2)+EXP(-X) FOR X = 0 TO 5
 
The 3 most important commands are PLOT, FIT, and LET.
The "average" analyst commonly uses about 20 commands.
The language as a whole consists of over 200 commands.
These 200+ commands are in 7 command categories.  For a
list of commands within each command category, enter
HELP followed by the category name--
 
   1) Graphics               HELP GRAPHICS
   2) Diagrammatic Graphics  HELP DIAGRAMMATIC GRAPHICS
   3) Analysis               HELP ANALYSIS
   4) Plot Control           HELP PLOT CONTROL
   5) Support                HELP SUPPORT
   6) Output Devices         HELP OUTPUT DEVICES
   7) Keywords               HELP KEYWORDS
 
For syntax, default, etc.  information about an
individual command, enter HELP followed by the command
name, as in
 
   HELP PLOT
   HELP FIT
   HELP LET
 
For a listing of built-in library functions (which can
be employed in any PLOT, FIT, LET, etc.  command),
enter HELP FUNCTIONS.
 
For a listing of subcommands under the LET command (for
computing statistics, carrying out math operations, and
generating random numbers), enter HELP LET SUBCOMMANDS.
 
For information on capitalization and subscripting, and
a listing of Greek, math, and other special symbols in
the TEXT, TITLE, LABEL, and LEGEND commands, enter HELP
TEXT SUBCOMMANDS.
 
For a listing of available character types, line types,
and color types, respectively, enter
 
   HELP CHARACTER TYPES     or      LIST CHARACTERS.
   HELP LINE TYPES          or      LIST LINES.
   HELP COLOR TYPES         or      LIST COLORS.
 
Dataplot has a number of useful reference files.  The
following reference ASCII files may be scanned via the
LIST and SEARCH commands (do not forget the . at the
end of the file name):
 
     File         LIST Example        SEARCH Example
  ...................................................................
   DATASETS.      LIST DATASETS.      SEARCH DATASETS. REGRESSION
   DISTRIBUTIONS. LIST DISTRIBUTIONS. SEARCH DISTRIBUTIONS. SYMMETRIC
   DESIGNS.       LIST DESIGNS.       SEARCH DESIGNS. L18
 
   COMMANDS.      LIST COMMANDS.      SEARCH COMMANDS. LABEL
   SYNTAX.        LIST SYNTAX.        SEARCH SYNTAX FIT
   FUNCTIONS.     LIST FUNCTIONS.     SEARCH FUNCTIONS. NORMAL
   SUBCOMMANDS.   LIST SUBCOMMANDS.   SEARCH SUBCOMMANDS. MEAN
   PROGRAMS.      LIST PROGRAMS.      SEARCH PROGRAMS. DEX
   MACROS.        LIST MACROS.        SEARCH MACROS. DEX
 
   CHARACTERS.    LIST CHARACTERS.    SEARCH CHARACTERS. SQUARE
   LINES.         LIST LINES.         SEARCH LINES. DOTTED
   COLORS.        LIST COLORS.        SEARCH COLORS. YELLOW
 
   GREEKSYMB.     LIST GREEKSYMB.     SEARCH GREEKSYMB. ALPHA
   MATHSYMB.      LIST MATHSYMB.      SEARCH MATHSYMB. INTEGRATION
   MISCSYMB.      LIST MISCSYMB.      SEARCH MISCSYMB. BAR
   INLINE.        LIST INLINE.        SEARCH INLINE. CIRCLE
 
   DEFAULTS.      LIST DEFAULTS.      SEARCH DEFAULTS. CHARACTERS
   FAQS.          LIST FAQS.          SEARCH FAQS. POSTSCRIPT
 
----------------------------------------------------------
 
 
 
 
 
 
-------------------------  *GRAPHICS*  -------------------
 
GRAPHICS
Graphics Commands
 
Commands in this category generate various kinds of plots, such as
y-x scatter plots, histograms, and spectral plots.  Examples include
PLOT, HISTOGRAM, and SPECTRUM.  The commands in this category are--
 
X-Y PLOTS:
   PLOT                  Generate a plot of a variable and/or function
   ERROR BAR PLOT        Generate an error bar plot
   VECTOR PLOT           Generate a vector plot (pairs of points
                         connected with arrows)
 
3-D PLOTS:
   3-D PLOT              Generate a 3-dimensional plot of a variable
                         and/or function
   CONTOUR PLOT          Generate a contour plot
 
DISTRIBUTIONAL PLOTS:
   ... HISTOGRAM         Generate a histogram--count, cumulative count,
                         relative, or cumulative relative
   ... BIHISTOGRAM       Generate a bi-histogram--count, cumulative,
                         relative, or cumulative relative
   ... FREQUENCY PLOT    Generate a frequency plot--count, cumulative,
                         relative, or cumulative relative
   ... ROOTOGRAM         Generate a rootogram plot
   STEM AND LEAF PLOT    Generate a stem and leaf diagram
   PIE CHART             Generate a pie chart
   ... PROBABILITY PLOT  Generate a probability plot (24 distributions)
   ... PPCC PLOT         Generate a probability plot correlation
                         coefficient plot (9 families)
   NORMAL PLOT           Generate a normal plot
   BOX-COX NORM PLOT     Generate a Box-Cox normality plot
   BOX-COX LINE PLOT     Generate a Box-Cox linearity plot
   BOX-COX HOMO PLOT     Generate a Box-Cox homoscedasticity plot
   PERCENT POINT PLOT    Generate a percent point plot
   QUANT-QUANT PLOT      Generate a quantile-quantile plot
   SYMMETRY PLOT         Generate a symmetry plot
   4 PLOT                Generate 4-plot for univariate analysis
 
TIME SERIES PLOTS:
   RUN SEQUENCE PLOT     Generate a run sequence plot
   LAG ... PLOT          Generate a lag plot for a given lag number
   ... CORRELATION PLOT  Generate an auto- or cross-correlation plot
   ... SPECTRAL PLOT     Generate auto-, cross-, etc spectral plot
   ... PERIODOGRAM       Generate auto- or cross-periodogram
   COMP DEMOD PLOT       Generate complex demodulation amp/phase plot
   AV PLOT               Generate an Allan variance plot
   ASD PLOT              Generate an Allan standard deviation plot
 
QUALITY CONTROL PLOTS:
   ... CONTROL CHART     Generate a C, U, N, NP, mean, sd, or range
                         control chart
   Q ... CONTROL CHART   Generate Quesenberry style control charts
   TAGUCHI ... PLOT      Generate a Taguchi signal-to-noise ratio chart
 
MULTUVARIATE PLOTS:
   ANDREWS PLOT          Generate Andrews curves
   PROFILE PLOT          Generate a profile plot
   STAR PLOT             Generate a star plot
   SYMBOL PLOT           Generate a plot of variable with character
                         attributes controlled by other variables
 
ANALYSIS OF VARIANCE, DESIGN OF EXPERIMENTS PLOTS:
   ... BLOCK PLOT        Generate a block plot
   BOX PLOT              Generate box plot
   DEX ... PLOT          Generate a wide variety of design of
                         experiment plots
 
STATISTICAL PLOTS:
   ANOP PLOT             Generate a analysis of proportions plot
   BOOT ... STAT PLOT    Generate a bootstrap plot for a statistic
   JACK ... STAT PLOT    Generate a jackknife plot for a statistic
   FRACTAL PLOT          Generate a fractal plot
   I PLOT                Generate an I plot
   PARETO PLOT           Generate a Pareto plot
   PHASE PLANE DIAGRAM   Generate a phase plane diagram plot
   ... STATISTIC PLOT    Generate a plot of a given statistic against
                         subsets of the data
 
RELIABILITY, EXTREME VALUE ANALYSIS
   TAIL AREA PLOT        Generate a tail area plot
   CME PLOT              Generate a conditional mean exceedance plot
   WEIBULL PLOT          Generate a Weibull plot
 
The ... in some of the commands indicates user-defined options for the
command, as in
   NORMAL PROBABILITY PLOT, UNIFORM PROBABILITY PLOT, etc.
   AUTOCORRELATION PLOT, CROSS-CORRELATION PLOT
   MEAN CONTROL CHART, RANGE CONTROL CHART, etc.
 
The first 4 letters of most commands will usually suffice.  Use spaces
(not commas) to separate arguments in a command.
 
For further information on a given command, enter HELP   followed by
the command name, as in
   HELP PLOT
   HELP PROBABILITY PLOT
   HELP CONTROL CHART
 
----------------------------------------------------------
 
 
 
 
 
 
 
 
 
 
 
 
 
 
 
 
 
 
 
 
 
 
 
 
 
 
 
 
 
 
 
 
 
 
 
 
 
 
 
 
 
 
 
 
 
 
 
-------------------------  *DIAGRAMMATIC GRAPHICS*  ------
 
DIAGRAMMATIC GRAPHICS
Diagrammatic Graphics Commands
 
Commands in this category generate text strings, diagrams, schematics,
word charts, etc., and specify details (e.g., character font and
character height) of elements on such diagrams.  Examples include
MOVE, DRAW, CIRCLE, BOX, TEXT, FONT, and HEIGHT.  The commands in this
category are--
 
DEVICE ATTRIBUTES
   WINDOW                Specify the graphics region in 0 to 100 units
   BACKGROUND COLOR      Specify the color of the background after the
                         next screen erase
   ERASE                 Erase the current screen
   COPY                  Copy the current screen onto local hardcopy
   RING BELL             Ring the bell
 
TEXT ATTRIBUTES
   FONT                  Specify the text font (TRIPLEX, COMPLEX, etc.)
   CASE                  Specify the text case (UPPER, LOWER)
   HEIGHT                Specify the text height in 0 to 100 units
   WIDTH                 Specify the text width in 0 to 100 units
   HW                    Specify the text height and width
   VERTICAL SPACING      Specify the vertical spacing between lines in
                         0 to 100 units
   HORIZONTAL SPACING    Specify the horizontal spacing between
                         characters in 0 to 100 units
   THICKNESS             Specify the text line width in 0 to 100 units
   COLOR                 Specify the text color (RED, BLUE, etc.)
   JUSTIFICATION         Specify the text justification (LEFT, CENTER,
                         RIGHT)
   CR                    Specify automatic carriage return after TEXT
   LF                    Specify an automatic line feed after TEXT
   CRLF                  Specify an automatic carriage return/line feed
                         after TEXT
   MARGIN                Specify the position for carriage return in 0
                         to 100 units
   ()                    Specify a math or Greek character in TEXT
 
LINE ATTRIBUTES
   LINES                 Specify the line type for figures (SOLID, DOT,
                         etc.)
   LINE THICKNESS        Specify the line thicknesses in 0 to 100 units
   LINE COLORS           Specify the line colors (RED, BLUE, etc.)
 
GRAPHICS INPUT
   CROSS-HAIR (or CH)    Activate and read the cross-hair (returned
                         value in 0 to 100 units)
 
TEXT
   TEXT                  Write out text
 
GRAPHICAL FIGURES (coordinates specified in 0 to 100 units)
   MOVE                  Move to a point
   MOVEDATA              Move to a point (in units of the most recent
                         plot)
   DRAW                  Draw a line
 
   POINT                 Draw a point
   ARROW                 Draw an arrow
   TRIANGLE              Draw a triangle
   BOX                   Draw a box
   HEXAGON               Draw a hexagon
   CIRCLE                Draw a circle
   SEMI-CIRCLE           Draw a semi-circle
   ARC                   Draw an arc
   ELLIPSE               Draw an ellipse
   OVAL                  Draw an oval
   DIAMOND               Draw a diamond
   CUBE                  Draw a cube
   PYRAMID               Draw a pyramid
   LATTICE               Draw a lattice
 
   AMPLIFIER             Draw an amplifier
   CAPACITOR             Draw a capacitor
   GROUND                Draw a ground
   INDUCTOR              Draw an inductor
   RESISTOR              Draw a resistor
 
   AND                   Draw an and box
   OR                    Draw an or box
   NAND       [not work] Draw a nand box
   NOR                   Draw a nor box
 
The first 4 letters of most commands will usually suffice.  Use spaces
(not commas) to separate arguments in a command.
 
For further information on a given command, enter HELP   followed by
the command name, as in
   HELP FONT
   HELP TEXT
   HELP AMPLIFIER
 
----------------------------------------------------------
 
 
 
 
 
 
 
 
 
 
 
 
 
 
 
 
 
 
 
 
 
 
 
 
 
 
 
 
 
 
 
 
 
 
 
 
 
 
 
 
 
 
 
 
 
 
 
 
 
 
 
 
 
 
-------------------------  *ANALYSIS*  -------------------
 
ANALYSIS
Analysis Commands
 
Commands in this category carry out mathematical/statistical analyses
such as fitting, transformations, and smoothing.  Examples of such
commands include FIT, LET, and SMOOTH.  The commands in this category
are--
 
DATA AND FUNCTION TRANSFORMATIONS
   LET                     Define variables and parameters, calculate
                           statistics, find roots, derivatives, and
                           integrals
   LET FUNCTION            Define and operate on functions and
                           differentiate functions
 
STATISTICAL SUMMARIES AND TESTS
   SUMMARY                 Compute summary statistics
   T TEST                  Carry out a 1- or 2-sample t test
   F TEST                  Carry out a 2-sample F test
   CHI-SQUARE TEST         Carry out a 1-sample chi-square test
   CONFIDENCE LIMITS MEAN  Compute the confidence limits for the mean
   RUNS                    Carry out a runs analysis
   TABULATE                Tabulate counts, means, standard deviations,
                           and ranges of data
   CROSS TABULATE          Tabulate counts, means, standard deviations,
                           and ranges of data from a pair of variables
 
FITTING AND SMOOTHING
   ... FIT                 Perform least squares linear or non-linear
                           fit
   ... PRE-FIT             Perform pre-fit analysis for starting values
   EXACT ... RATIONAL FIT  Perform an exact rational function fit
   ... SPLINE FIT          Perform a spline fit
   ... SMOOTH              Perform various types of smoothing for
                           equi-spaced data
   LOWESS SMOOTH           Perform locally weighted least squares
                           smoothing
 
EXPERIMENT DESIGN AND ANALYSIS OF VARIANCE
   ANOVA                   Perform an analysis of variance
   MEDIAN POLISH           Perform a robust analysis of variance
   YATES ANALYSIS          Perform a Yates analysis
   DEX PHD                 Perform a pHd (principal Hessian directions)
                           analysis of a Yates design
 
QUALITY CONTROL
   CAPABILITY ANALYSIS     Generate a table of capability analysis
                           statistics
 
The ... in some of the commands indicates user-defined options for the
command, as in
   LINEAR SPLINE FIT, CUBIC SPLINE FIT, etc.
   LINEAR SMOOTH, CUBIC SMOOTH, ROBUST SMOOTH, etc.
   EXACT 1/1 RATIONAL FIT, EXACT 2/3 RATIONAL FIT, etc.
 
The first 4 letters of most commands will usually suffice.  Use spaces
(not commas) to separate arguments in a command.
 
For further information on a given command, enter HELP   followed by
the command name, as in
   HELP ANOVA
   HELP LET
   HELP MEDIAN POLISH
 
----------------------------------------------------------
 
 
 
 
 
 
 
 
 
 
 
 
 
 
 
 
 
 
 
 
 
 
 
 
 
 
 
 
 
 
 
 
 
-------------------------  *PLOT CONTROL*  ---------------
 
PLOT CONTROL
Plot Control Commands
 
Commands in this category specify details of subsequent plots, such as
line type, labels, and log scale.  Examples include LINES, LABEL, and
LOG.  The commands in this category are--
 
PAGE CONTROL
   MULTIPLOT CORNER COORD     Specify the location of the multi-plot
                              region
   MULTIPLOT                  Specify the number of plot regions on a
                              page
   WINDOW CORNER COORD        Specify the portion of the device area
                              to use
   ORIENTATION                Specify whether plots are generated in
                              landscape, portrait, or poster mode
 
TITLE ATTRIBUTES
   TITLE                      Specify the title at the top of the plot
   TITLE AUTOMATIC            Specify an automatically generated title
   TITLE CASE                 Specify the case for the plot title
   TITLE COLOR                Specify the color for the plot title
   TITLE DISPLACEMEMNT        Specify the distance from frame to title
   TITLE FONT                 Specify the font for the plot title
   TITLE SIZE                 Specify the size (height) for plot title
   TITLE THICKNESS            Specify the thickness for the plot title
 
AXIS LABEL ATTRIBUTES
   ...LABEL                   Specify axis labels to appear at the
                              sides and bottom of the plot
   ...LABEL AUTOMATIC         Specify automatically generated labels
   ...LABEL CASE              Specify the case for plot labels
   ...LABEL COLOR             Specify the colors for plot labels
   ...LABEL DISPLACEMENT      Specify the distance from frame to labels
   ...LABEL FILL              Specify the fill switch for axis labels
   ...LABEL FONT              Specify the font for plot labels
   ...LABEL SIZE              Specify the size (height) for plot labels
   ...LABEL THICKNESS         Specify the thickness for plot labels
 
LEGEND ATTRIBUTES
   LEGEND ...                 Specify the text for plot legends
   LEGEND ... ANGLE           Specify the angle for plot legends
   LEGEND ... CASE            Specify the case for plot legends
   LEGEND ... COLOR           Specify the color for plot legends
   LEGEND ... COORDINATES     Specify the positions for plot legends
   LEGEND ... DIRECTION       Specify the direction for plot legends
   LEGEND ..  FILL            Specify the fill switch for plot legends
   LEGEND ... FONT            Specify the font for plot legends
   LEGEND ... HW              Specify height and width for plot legends
   LEGEND ... JUSTIFICATION   Specify justification for plot legends
   LEGEND ... SIZE            Specify size (height) for plot legends
   LEGEND ... THICKNESS       Specify the thickness for plot legends
 
CHARACTER ATTRIBUTES
   CHARACTERS                 Specify the plot character types (X,
                              SQUARE, etc.)
   CHARACTER ANGLE            Specify the angle for plot characters
   CHARACTER AUTOMATIC        Specify a variable to use as arguments
                              for the CHARACTERS command
   CHARACTER CASE             Specify the case for plot characters
   CHARACTER COLORS           Specify the colors for plot characters
   CHARACTER FILL             Specify fill switch for plot characters
   CHARACTER FONT             Specify the font for plot characters
   CHARACTER HW               Specify the character height and width
   CHARACTER JUSTIFICATION    Specify justification for plot characters
   CHARACTER OFFSET           Specify the offset (i.e., displacement)
                              for plot characters
   CHARACTER SIZES            Specify the height for plot characters
   CHARACTER THICKNESS        Specify the thickness for plot characters
   CHARACTER WIDTH            Specify the width for plot characters
 
LINE ATTRIBUTES
   LINES                      Specify the line types (SOLID, DOT, DASH,
                              etc.) for plot lines
   LINE THICKNESS             Specify the thicknesses for plot lines
   LINE COLORS                Specify the colors for plot lines
 
SPIKE ATTRIBUTES
   SPIKE                      Specify the existence (ON/OFF) of plot
                              spikes
   SPIKE BASE                 Specify base locations for plot spikes
   SPIKE COLOR                Specify the colors for plot spikes
   SPIKE DIRECTION            Specify the directions (H or V) for plot
                              spikes
   SPIKE LINE                 Specify the line types for plot spikes
   SPIKE THICKNESS            Specify the thicknesses for plot spikes
 
BAR ATTRIBUTES
   BAR                        Specify existence (ON/OFF) of bars on plots
   BAR BASE                   Specify the base locations for plot bars
   BAR BORDER COLOR           Specify the plot bar border colors
   BAR BORDER LINE            Specify the plot bar border line types
   BAR BORDER THICKNESS       Specify the plot bar border thicknesses
   BAR DIMENSION              Specify the bar dimensions to be 2d or 3d
   BAR DIRECTION              Specify the bar directions to be
                              horizontal or vertical
   BAR FILL                   Specify the existence (ON/OFF) of bar
                              fills
   BAR FILL COLOR             Specify the bar fill (background) colors
   BAR PATTERN                Specify the bar fill pattern types
   BAR PATTERN COLOR          Specify the bar fill pattern colors
   BAR PATTERN LINE TYPE      Specify the bar fill pattern line types
   BAR PATTERN SPACING        Specify bar fill pattern line spacings
   BAR PATTERN THICKNESS      Specify bar fill pattern line thicknesses
   BAR WIDTH                  Specify the widths for plot bars
 
REGION ATTRIBUTES
   REGION BASE                Specify base locations for plot regions
   REGION FILL                Specify the existence (ON/OFF) of
                              regions on plots
   REGION FILL COLOR          Specify the region solid fill colors
   REGION PATTERN             Specify the region fill pattern types
   REGION PATTERN COLOR       Specify the region hatch pattern colors
   REGION PATTERN LINE TYPE   Specify region fill pattern line types
   REGION PATTERN SPACING     Specify region fill pattern line spacings
   REGION PATTERN THICKNESS   Specify the region fill pattern line
                              thicknesses
 
BACKGROUND ATTRIBUTES
   BACKGROUND COLOR           Specify background color inside the frame
   MARGIN COLOR               Specify background color outside the
                              frame
 
FRAME ATTRIBUTES
   ...FRAME                   Specify existence (ON/OFF) of plot frame
   FRAME CORNER COORDINATES   Specify the plot frame location and shape
   ...FRAME COLOR             Specify the plot frame colors
   ...FRAME THICKNESS         Specify the plot frame line thicknesses
   ...FRAME PATTERN           Specify the plot frame line types
 
SCALE ATTRIBUTES
   ...MINIMUM                 Specify minima to appear on plot frame
   ...MAXIMUM                 Specify maxima to appear on plot frame
   ...LIMITS                  Specify the limits (minimum and maximum)
                              for the plot frame
   ...LOG                     Specify the existence (ON/OFF) of
                              a logarithmic scale
 
GRID ATTRIBUTES
   ...GRID                    Specify existence (ON/OFF) of grid lines
   ...GRID LINE               Specify the line types of the plot grid
   ...GRID COLOR              Specify the line colors of the plot grid
   ...GRID THICKNESS          Specify line thicknesses of the plot grid
   GMINOR                     Specify the existence of minor grid lines
 
TIC MARK ATTRIBUTES
   ...TIC MARK                Specify existence (ON/OFF) of tic marks
   ...TIC MARK COLOR          Specify the plot tic mark colors
   ...TIC MARK OFFSET         Specify the distance from the frame
                              corner to the first and last tic marks
   TIC OFFSET UNITS           Specify the tic offset units (data units
                              or DATAPLOT 0 to 100 units)
   ...TIC MARK POSITION       Specify the plot tic mark positions
                              (in/out/thru)
   ...TIC MARK SIZE           Specify the plot tic mark sizes
   ...TIC MARK THICKNESS      Specify the plot tic mark thicknesses
   ...MAJOR TIC MARK NUMBER   Specify the number of major tic marks
   ...MINOR TIC MARK NUMBER   Specify the number of minor tic marks
 
TIC MARK LABEL ATTRIBUTES
   ...TIC MARK LABEL          Specify existence (ON/OFF) of tic mark labels
   ...TIC MARK LABEL ANGLE    Specify the plot tic mark label angles
   ...TIC MARK LABEL CASE     Specify the plot tic mark label cases
   ...TIC MARK LABEL COLOR    Specify the plot tic mark label colors
   ...TIC MARK LABEL CONTENT  Specify alphanumeric tic mark labels
   ...TIC MARK LABEL DECIMAL  Specify the number of digits to the right
                              of the decimal point
   ...TIC MARK LABEL DIRECT   Specify the tic mark label directions
   ...TIC MARK LABEL DISPLAC  Specify tic mark label to frame distances
   ...TIC MARK LABEL FONT     Specify the plot tic mark label fonts
   ...TIC MARK LABEL FORMAT   Specify the plot tic mark label formats
                              (real/exponential/power/alpha)
   ...TIC MARK LABEL HW       Specify tic mark label heights and widths
   ...TIC MARK LABEL JUST     Specify the tic mark label justifications
   ...TIC MARK LABEL SIZE     Specify the plot tic mark label heights
   ...TIC MARK LABEL THICK    Specify the tic mark label thicknesses
 
ARROW ATTRIBUTES
   ARROW ... COORDINATES      Specify the location of arrows
   ARROW ... COLOR            Specify the colors for arrows
   ARROW ... PATTERN          Specify the line types for arrows
   ARROW ... THICKNESS        Specify line thicknesses for arrows
 
BOX ATTRIBUTES
   BOX ... CORNER COORDINATES Specify location of plot boxes
   BOX ... COLOR              Specify the frame colors for boxes
   BOX ... PATTERN            Specify the frame line types for boxes
   BOX ... THICKNESS          Specify the frame thicknesses for boxes
   BOX ... FILL COLOR         Specify the pattern fill colors for boxes
   BOX ... FILL GAP           Specify pattern fill line spacings for
                              boxes
   BOX ... FILL LINE          Specify pattern fill line types for boxes
   BOX ... FILL PATTERN       Specify the pattern fill types for boxes
   BOX ... FILL THICKNESS     Specify the pattern fill line thickness
                              for boxes
   BOX ... SHADOW HW          Specify the shadow sizes for boxes
 
SEGMENT ATTRIBUTES
   SEGMENT ... COORDINATES    Specify location of plot line segments
   SEGMENT ... COLOR          Specify the colors for plot line segments
   SEGMENT ... PATTERN        Specify line types for plot line segments
   SEGMENT ... THICKNEESS     Specify line thicknesses for plot line
                              segments
 
3D ATTRIBUTES
   EYE COORDINATES            Specify the eye location for a 3d plot
   ROTATE EYE                 Rotate the eye coordinates
   3DFRAME                    Specify the type of frame to draw on a
                              3D plot
   ORIGIN COORDINATES         Specify the reference origin for 3d plot
   PEDESTAL        [not work] Specify the existence (ON/OFF) of a
                              pedestal on 3d plots
   PEDESTAL SIZE   [not work] Specify the pedestal size on 3d plots
   PEDESTAL COLOR  [not work] Specify the pedestal color on 3d plots
   VISIBLE         [not work] Specify whether background lines are
                              visible on 3d plots
 
DESIGN OF EXPERIMENT PLOT ATTRIBUTES
   DEX DEPTH                  Specify depth of DEX interaction terms
   DEX HORIZONTAL AXIS        Specify horizontal axis for DEX plots
   DEX WIDTH                  Specify the width of levels for DEX plots
 
MISCELLANEOUS ATTRIBUTES
   PRE-ERASE                  Specify whether subsequent plots perform
                              an initial screen erase (ON/OFF)
   BELL                       Specify whether subsequent plots ring the
                              bell before plotting (ON/OFF)
   SEQUENCE                   Specify whether subsequent plots contain
                              an automatic sequence number (ON/OFF)
   HARDCOPY                   Specify whether subsequent plots have
                              automatic hardcopy generated (ON/OFF)
   PRE-SORT                   Specify whether subsequent plots pre-sort
                              the data before plotting (ON/OFF)
 
   HORIZONTAL SWITCH          Specify whether plots are generated
                              horizontally or vertically
 
 
The ... in some of the commands indicates user-defined options for
the command, as in
   X1LABEL, X2LABEL, X3LABEL, Y1LABEL, Y2LABEL
   LEGEND 1 COORDINATES, LEGEND 2 COORDINATES, etc.
   XLOG, YLOG, X1LOG, X2LOG, Y1LOG, Y2LOG
 
The first 4 letters of most commands will usually suffice.  Use spaces
(not commas) to separate arguments in a command.
 
For further information on a given command, enter HELP   followed by
the command name, as in
   HELP TITLE
   HELP LOG
   HELP ARROW COLOR
 
----------------------------------------------------------
 
 
 
 
 
 
 
 
 
 
 
 
 
 
 
 
 
 
 
 
 
 
 
 
 
 
 
 
 
 
 
 
 
 
 
 
 
 
 
 
 
 
 
 
-------------------------  *SUPPORT*  --------------------
 
SUPPORT
Support Commands
 
Commands in this category carry out secondary operations, such as input
and output and defining trigonometric units.  Examples of commands are
READ, WRITE, and DEGREES.  The commands in this category are--
 
ONLINE HELP
   HELP                    Print short documentation for a command
   STATUS                  Print the status of all lines, characters,
                           variables, and parameters
   NEWS                    Print general news from the DATAPLOT service
                           organization (documents new commands)
   MAIL                    Print a message from the DATAPLOT service
                           organization to a user
   MESSAGE   [not working] Send a message to the DATAPLOT service
                           organization
   BUGS                    List known bugs
   EXPERT [not working]    Invoke the expert subsystem
 
INPUT AND OUTPUT
   READ                    Read variables
   SERIAL READ             Read variables serially
   READ PARAMETER          Read parameters
   READ FUNCTION           Read 1 line of functions (= READ STRING)
   READ STRING             Read 1 line of strings (= READ FUNCTION)
   READ MATRIX             Read a matrix
   WRITE (or PRINT)        Write variables, parameters, functions, or
                           matrices to either the terminal or a file
   SKIP                    Specify the number of header lines to skip
                           for subsequent READ and SERIAL READ commands
   ROW LIMITS              Specify READ and SERIAL READ row limits
   COLUMN LIMITS           Specify READ and SERIAL READ column limits
   END OF DATA             Define end of data for READ and SERIAL READ
 
RE-EXECUTE PREVIOUS COMMANDS AND TERMINAL CONTROL
   REPEAT                  Re-execute one or more of last 20 commands
   SAVE                    Save one or more of the last 20 commands
   /                       Re-execute saved commands
   PAUSE                   Wait for a carriage return before
                           continuing execution
   PROMPT                  Specify whether a DATAPLOT prompt is
                           printed after a command completes
 
SAVING, RE-DIRECTING, AND PRINTING OUTPUT
   CAPTURE                 Re-direct alphanumeric output to a file
   END OF CAPTURE          Re-direct alphanumeric output to the screen
   / LP  [host dependent]  Re-execute the saved commands and send the
                           alphanumeric output to a printer
   / LPT1                  Synonym for "/ LP"
   / PRINTER               Synonym for "/ LP"
   / <file> [host depend]  Re-execute the saved commands and send the
                           alphanumeric output to the named file
   PP  [host dependent]    Send a copy of most recent plot to printer
 
LISTING
   LIST                    List the last 20 commands or print the
                           contents of a file
   NLIST                   Print the contents of a file with line
                           numbers
   COLUMN RULER            Prints out a column header denoting columns
                           1 through 80
 
DATAPLOT FEEDBACK
   ECHO                    Specify automatic echo of command lines
                           (ON/OFF)
   FEEDBACK                Allow/suppress feedback printing (ON/OFF)
   PRINTING                Allow/suppress analysis printing (ON/OFF)
 
REINITIALIZING/EXITING
   RESET                   "Zero-out" all variables, parameters,
                           functions, etc
   SAVE MEMORY             Dump all variables, parameters, and
                           functions to a mass storage file
   RESTORE MEMORY          Restore all saved variables, parameters, and
                           functions from mass storage
   EXIT                    Exit from DATAPLOT (Synonyms are STOP, END,
                           HALT, QUIT)
 
WORKSPACE
   DIMENSION               Specify dimensions of internal data storage
 
MODIFYING VARIABLES
   DELETE                  Delete variables or elements of a variable
   RETAIN                  Retain variables or elements of a variable
   NAME                    Assign additional names to a variable
   APPEND                  Append one variable to the end of another
                           variable
   EXTEND                  Extend one variable by attaching another
                           variable to the end of it
 
COMMENTS
   COMMENT                 Insert a comment line in code
   .                       Insert a comment line in code
   COMMENT CHECK           Check data files for comment lines (ON/OFF)
   COMMENT CHARACTER       Define the comment character for data files
 
DATAPLOT MACROS AND PROGRAMMING STRUCTURES
   CREATE                  Create a subprogram
   END OF CREATE           End creation of a subprogram
   CALL                    Execute a DATAPLOT subprogram stored on a
                           mass storage file
   LOOP                    Initiate a loop
   END OF LOOP             Terminate a loop
   BREAK LOOP              Terminate a loop before the last iteration
   IF                      Define start of conditionally-executed code
   END OF IF               Define end of conditionally-executed code
 
TRIGONOMETRIC UNITS
   ANGLE UNITS             Specify type of trigonometric units to use
   RADIANS                 Specify the use of radians for
                           trigonometric calculations (ON/OFF)
   DEGREES                 Specify the use of degrees for
                           trigonometric calculations (ON/OFF)
   GRADS                   Specify the use of grads for
                           trigonometric calculations (ON/OFF)
 
ACCESSING INTERNAL DATAPLOT VARIABLES
   PROBE                   Print value of underlying FORTRAN parameter
   SET                     Set value of an underlying FORTRAN parameter
 
SEARCHING AND EDITING FILES
   SEARCH                  Search file for the first occurrence (or all
                           occurrences) of a string
   EDIT (or FED)           Edit a file with a line mode editor
 
DEFINE SPECIAL CHARACTERS AND STRINGS
   TERMINATOR  CHARACTER   Specify the character to terminate commands
   CONTINUE CHARACTER      Specify the character to continue commands
   SUBSTITUTE CHARACTER    Specify the substitution character
   DEFINE                  Define general ASCII string commands
   DEFINE POSTHELP         Define ASCII string to succeed HELP
   DEFINE PREHELP          Define ASCII string to precede HELP
   DEFINE POSTPLOT         Define ASCII string to succeed HELP
   DEFINE PREPLOT          Define ASCII string to precede HELP
   PREPLOT                 Specify the preplot and postplot device
 
PLOT SUPPORT
   ANDREWS INCREMENT       Specify the horizontal axis increment for
                           the ANDREWS PLOT command
   ANOP LIMITS             Specify limits for regions in an ANOP plot
   CLASS ...LOWER          Specify the  first class lower limit for
                           the HISTOGRAM and related commands
   CLASS ...UPPER          Specify the last class upper limit for
                           the HISTOGRAM and related commands
   CLASS ...WIDTH          Specify the class width for the HISTOGRAM
                           and related commands
   CURSOR COORDINATES      Specify the cursor coordinates after a plot
   CURSOR SIZE             Specify the cursor size after a plot
   ERASE DELAY             Specify the delay factor for an erase
   FENCE                   Specify whether or not fences are drawn on
                           box plots
   FRACTAL TYPE            Specify the type of input for fractal plots
   FRACTAL ITERATIONS      Specify the number of points to generate for
                           a fractal plot
   HARDCOPY DELAY          Specify the delay factor for a hardcopy
   NEGATE                  Specify whether or not the vertical axis is
                           negated
   VECTOR ARROW            Specify the attributes for the arrow on a
                           vector plot
   VECTOR FORMAT           Specify the data format for vector plots
 
SWITCHES FOR ANALYSIS COMMANDS
   BOOTSTRAP SAMPLE SIZE   Set the sample size for bootstrap plots
   DEMODULATION FREQUENCY  Specify frequency for complex demodulation
   FILTER WIDTH            Specify filter width for SMOOTH
   FIT CONSTRAINTS[nt work]Specify FIT and PRE-FIT constraints
   FIT ITERATIONS          Specify an upper bound on iterations for FIT
   FIT POWER               Specify the fit criterion power for PRE-FIT
                           and FIT
   FIT STANDARD DEVIATION  Specify the lower bound on the residual
                           standard deviation for the FIT
   LOWESS FRACTION         Set interval for LOWESS SMOOTH (as fraction)
   LOWESS PERCENT          Set interval for LOWESS SMOOTH (as percent)
   KNOTS                   Specify the knots variable for SPLINE FIT
   PRINCIPAL COMP TYPE     Specify the type of input data for the
                           PRINCIPAL COMPONENTS command
   POLYNOMIAL DEGREE       Specify the polynomial degree for certain
                           variations of the FIT, SMOOTH, and SPLINE
                           FIT commands
   ROOT ACCURACY           Specify the accuracy tolerance for the
                           ROOTS command
   SEED                    Specify the seed for random number
                           generation
   WEIGHTS                 Specify the weights variable for the FIT,
                           PRE-FIT, and related commands
   YATES PRINT             Specify what sections of the YATES ANALYSIS
                           output to print
   YATES CUTOFF            Specify which factor effects from the YATES
                           ANALYSIS command to print based on various
                           cutoff criterion
 
HOST COMMUNICATIONS
   COMMUNICATIONS LINK     Specify link (phone, network, etc.) to host
   BAUD RATE               Specify the baud rate
   HOST                    Specify the host computer
   SYSTEM [host dependent] Send a command to the host operating system
   OPERATOR     [not work] Send a message to the host console operator
 
MISCELLANEOUS
   EXECUTE STRING          Execute a command line with string
                           substitutions
   TIME   [host dependent] Display the time and date
   IMPLEMENT [obsolete]    Activate local change to the DATAPLOT
                           implementation
   TRANSLATE               Define an automatic translation of graphic
                           strings
 
The ... in some of the commands indicates user-defined options for
the command, as in
   CLASS XLOWER, CLASS YLOWER
   CLASS XWIDTH, CLASS YWIDTH
   CLASS XUPPER, CLASS YUPPER
 
The first 4 letters of most commands will usually suffice.  Use spaces
(not commas) to separate arguments in a command.
 
For further information on a given command, enter HELP   followed by
the command name, as in
   HELP ECHO
   HELP READ
   HELP FEEDBACK
 
----------------------------------------------------------
 
 
 
 
 
 
 
 
 
 
 
 
 
 
 
 
 
 
 
 
 
 
 
 
-------------------------  *OUTPUT DEVICE*  --------------
 
OUTPUT DEVICE
Output Device Commands
 
DATAPLOT supports the following built-in device drivers:
 
    Tektronix     - most models (4010, 4014, 4105, 4113, 4115, 4027,
                    4662/4663), other Tektronix terminals typically
                    emulate one of these models
    REGIS         - for DEC terminals (VT-240, VT-340)
    HP-GL         - Hewlett-Packard plotters (can specify various
                    models including LaserJet III), emulated by many
                    plotter vendors
    HP 2622       - Hewlett-Packard terminal, also includes related
                    models (2623, 2647, and others)
    POSTSCRIPT    - used by many laser printers and other hard copy
                    devices
    QUIC          - used by QMS (and some Talaris) laser printers
    HP 7221       - Hewlett-Packard 7221 plotter
    GENERAL       - DATAPLOT specific metafile
    CGM           - ANSI standard Computer Graphics Metafile.
                    Currently only the clear text encoding is
                    supported.
 
Many devices provide either Tektronix, HP-GL, or Postscript emulation.
 
In addition, the following devices are available, but require some
local installation (usually linking the proper device library).
Contact your local site installer to find out if the desired device is
available.
 
    X11           - MIT windowing system, supported on most Unix based
                    workstations.  Has been tested on Sun, SGI,
                    HP-9000, VAX/ULTRIX, IBM RS-6000, Convex, and Cray.
                    Use this driver if you are running either Open Look
                    or Motif window systems.
    Sun CGI       - available on Sun only.  Uses the CGI library and
                    runs in a gfxtool or SunView window.  Sun is
                    phasing out support of CGI and going to an Open
                    Look based window system, so the X11 driver is
                    recommended even for the Sun.
    Calcomp       - uses the standard Calcomp library.  Many penplotter
                    vendors provide a Calcomp compatible library for
                    using their plotters.
    Zeta          - Zeta plotters.  Uses a slightly modified version of
                    the Calcomp library.
    IBM PC        - available for PC only.  This driver is still under
                    development, so may not be available in the public
                    PC version.  If you are simply using the PC as a
                    terminal, find a communications package that
                    provides either Tektronix or REGIS emulation.
 
DATAPLOT supports 3 devices (defined by DEVICE 1, DEVICE 2, and
DEVICE 3).  Device 1 output is sent to the screen and device 2 output
is sent to a file (DPPL1F.DAT or dppl1f.dat on most systems).  Device 3
output is also sent to a file (DPPL2F.DAT or dppl2f.dat on most
systems), but it only contains the most recent plot.  DATAPLOT supports
all 3 devices simultaneously if desired (that is, a single PLOT command
can generate both the plot on the screen and also write the plot to a
file for later printing).  The default is for device 1 to be a
Tektronix 4014 terminal, device 2 to be off, and device 3 to be a
Postscript printer.
 
The commands in this category are--
 
   TEKTRONIX                   Set device 1 to be a Tektronix device
   HP                          Set device 1 to a Hewlett-Packard device
   DEVICE-INDEPENDENT          Set device 1 to be device-independent
   GENERAL                     Set device 1 to be device-independent
   DISCRETE                    Set device 1 to be a 70 character
                               alphanumeric device
   BATCH                       Set device 1 to be a 130 character
                               alphanumeric device
   REGIS                       Set device 1 to be a Regis device
   POSTSCRIPT                  Set device 1 to be a Postscript device
   QUIC                        Set device 1 to be a QMS device
   ZETA                        Set device 1 to be a Zeta device
   CALCOMP                     Set device 1 to be a Calcomp device
   SUN                         Set device 1 to be a SUN device
   CGM                         Set device 1 to be a CGM metafile
   X11                         Set device 1 to be an X11 device
 
   CALCOMP PEN MAP             Specify the slot to color mapping for a
                               Calcomp plotter
   HPGL PEN MAP                Specify the slot to color mapping for an
                               HP-GL plotter
   ZETA PEN MAP                Specify the slot to color mapping for a
                               Zeta plotter
 
   SHOW COLORS                 List the available colors in DATAPLOT
   SHOW COLORS TEKT 4115       List the colors for a Tektronix 4115
   SHOW COLORS TEKT 4662       List the colors for a Tektronix 4662
   SHOW COLORS TEKT 4027       List the colors for a Tektronix 4027
   SHOW COLORS HP 2622         List the colors for an HP 2622
   SHOW COLORS HPGL            List the colors for an HP-GL plotter
   SHOW COLORS CALCOMP         List the colors for a Calcomp plotter
   SHOW COLORS ZETA            List the colors for a Zeta plotter
   SHOW COLORS CGM             List the colors for a CGM metafile
   SHOW COLORS SUN             List the colors for a Sun workstation
   SHOW COLORS REGIS           List the colors for a REGIS terminals
   SHOW COLORS POSTSCRIPT      List the colors for a Postscript device
   SHOW COLORS X11             List the colors for an X11 workstation
   SHOW COLORS PC              List the colors for an IBM/PC VGA device
 
   TERMINAL                    Specify the model or the power (ON/OFF)
                               for the terminal device
   CONTINUOUS                  Specify the continuity (ON/OFF) for the
                               terminal device
   PICTURE POINTS (or PP)      Specify the number of picture points for
                               the terminal device
 
   DEVICE 2 <device> <model>   Specify the manufacturer and model for
                               device 2
   DEVICE 3 <device> <model>   Specify the manufacturer and model for
                               device 3
 
   DEVICE ... POWER            Specify the device power switch (ON/OFF)
   DEVICE ... CONTINUOUS       Specify the device continuous switch
                               (ON/OFF)
   DEVICE ... PICTURE POINTS   Specify the device number of picture
                               points
   DEVICE ... COLOR            Specify the device color switch (ON/OFF)
 
   DEVICE ... OFF              Suppress plot generation on this device
                               (however the plot file remains open)
   DEVICE ... ON               Resume plot generation on this device
   DEVICE ... CLOSE            Suppress plot generation on this device
                               and close the plot file
 
The ... in some of the commands indicates user-defined options for the
command, as in
   DEVICE 1, DEVICE 2, DEVICE 3, etc.
   DEVICE 1 PICTURE POINTS, DEVICE 2 PICTURE POINTS, etc.
 
The first 4 letters of most commands will usually suffice.  Use spaces
(not commas) to separate arguments in a command.
 
For further information on a given command, enter HELP   followed by
the command name, as in
   HELP HP
   HELP BATCH
   HELP DEVICE PICTURE POINTS
 
----------------------------------------------------------
 
 
 
 
 
 
 
 
 
 
 
 
 
 
 
 
 
 
 
 
 
 
 
 
 
 
 
 
 
 
 
 
 
 
 
 
 
 
 
 
 
 
 
 
 
 
 
 
 
 
 
 
 
 
 
-------------------------  *KEYWORDS*  -------------------
 
KEYWORDS
Keywords
 
These are not commands per se but are reserved words which can appear
within a command statement to achieve an effect, such as specifying
subsets in a plot or analysis, or using predicted values and residuals
after a fit.  Examples include SUBSET, PRED and RES.  The elements in
this category are--
 
MULTI-TRACE PLOTS
   AND                  Used with plot commands for multi-trace plots
   VERSUS               Used with plot commands for multi-trace plots
 
DATA AND VARIABLE SUBSETS
   SUBSET               Qualifier denoting a subset of interest
   EXCEPT               Qualifier denoting an excepted subset
   FOR                  Qualifier denoting a variable or elements of a
                        variables of interest
   I                    A dummy index variable used by the FOR command
   TO                   Specify an interval of values within a variable
 
PRE-DEFINED PARAMETERS
   PI                   A parameter with the value 3.1415926
   INFINITY             A parameter with the value "infinity"
 
AUTOMATICALLY SAVED VARIABLES
   PRED                 A variable with the predicted values from the
                        FIT and other commands
   RES                  A variable with the residual values from the
                        FIT and other commands
   XPLOT                A variable that contains the horizontal axis
                        coordinates from the most recent plot
   YPLOT                A variable that contains the vertical axis
                        coordinates from the most recent plot
   X2PLOT               A variable that contains the second horizontal
                        axis coordinates from the most recent 3d plot
   TAGPLOT              A variable that contains the trace identifier
                        from the most recent plot
 
AUTOMATICALLY SAVED PARAMETERS
   RESSD                A parameter with the residual standard
                        deviation from the FIT and other commands
   RESDF                A parameter with the residual degrees of
                        freedom from the FIT and other commands
   REPSD                A parameter with the replication standard
                        deviation from the FIT and other commands
   REPDF                A parameter with the replication degrees of
                        freedom from the FIT and other commands
   LOFCDF               A parameter with the lack of fit cdf value from
                        the FIT and other commands
   DEMODF               A parameter with the updated complex
                        demodulation frequency
 
SETTING SWITCHES
   ON                   Set a switch to the "on" position
   OFF                  Set a switch to the "off" position
   AUTOMATIC            Set a switch to the "automatic" position
   DEFAULT              Set a switch to the "default" position
 
SPECIAL FILES
   COMMANDS             Symbolic name for DATAPLOT's commands file
   CONCLUSIONS          Symbolic name for DATAPLOT's conclusions file
   DATASETS             Symbolic name for DATAPLOT's data sets file
   DESIGNS              Symbolic name for DATAPLOT's design of
                        experiments file
   DIRECTORY            Symbolic name for DATAPLOT's directory file
   DICTIONARY           Symbolic name for DATAPLOT's dictionary file
   DISTRIBU             Symbolic name for DATAPLOT's probability
                        distributions file
   FUNCTION             Symbolic name for DATAPLOT's functions file
   MACROS               Symbolic name for DATAPLOT's macros file
   PROGRAMS             Symbolic name for DATAPLOT's programs file
   SYNTAX               Symbolic name for DATAPLOT's syntax file
 
LOGICAL OPERATORS
   NOT EXIST            Test for the existence of a variable in the IF
                        command
   =                    "Equal"; used in FIT, PRE-FIT, FOR, etc
   <>                   "Not equal to"
   <                    "Less than"
   <=                   "Less than or equal to"
   >                    "Greater than"
   >=                   "Greater than or equal to"
 
SPECIAL CHARACTERS
   ;                    The default command terminator character
   ...                  The default command continuation character
   ^                    The default substitution character
   ()                   Specify math/Greek characters in TEXT, LABEL,
                        and other commands
 
MISCELLANEOUS
   WRT                  "With respect to"; used with the LET command
                        for roots, integrals, and derivatives
   VERTICALLY [not work]Rotate contents (but not frame) of plot
 
For further information on a given keyword, enter HELP   followed by
the keyword, as in
   HELP PRED
   HELP SUBSET
   HELP DEMODF
 
----------------------------------------------------------
 
 
 
 
 
 
 
 
 
 
 
 
 
 
 
 
 
 
 
 
 
 
 
 
 
 
 
 
 
 
 
 
 
 
 
 
 
 
 
 
 
 
 
 
 
-------------------------  *FUNCTIONS*  ------------------
 
FUNCTIONS
Functions
 
DATAPLOT has an extensive library of built-in functions; these
functions find valuable application in the LET, LET FUNCTION, FIT,
PRE-FIT, PLOT, and 3D-PLOT commands.
 
The available functions fall into 3 general categories--
 
   1) General mathematical functions;
      For a list of such functions, enter HELP MATHEMATICS FUNCTIONS
      or HELP MATH FUNCTIONS   .
 
   2) Trigonometric functions;
      For a list of such functions, enter HELP TRIGONOMETRIC FUNCTIONS
      or HELP TRIG FUNCTIONS   .
 
   3) Probability functions;
      For a list of such functions, enter HELP PROBABILITY FUNCTIONS
      or HELP PROB FUNCTIONS   .
 
Library functions are distinguished from LET subcommands in the
following ways--
 
   1) Functions enclose the input value in parenthesis.  LET
      subcommands use spaces.
 
   2) Functions can accept (and return) either parameters (i.e., single
      values) or variables (i.e., an array of values) while LET
      subcommands are specific in which they accept as input and what
      they return as output.
 
   3) Functions can accept expressions while LET subcommands do not.
      For example, the following is legal:
 
         LET Y2 = ABS(Y1-INT(Y2))
 
      For LET subcommands, you typically have to do something like the
      following:
 
         LET YTEMP = Y**2 + 8
         LET A = SUM YTEMP
 
----------------------------------------------------------
 
 
 
 
-------------------------  *MATH FUNCTIONS*  -------------
 
MATHEMATICS FUNCTIONS
Mathematics Functions
 
The available (general) mathematics functions are--
 
ELEMENTARY FUNCTIONS
   ABS(X)         Compute the absolute value
   SQRT(X)        Compute the square root
   MOD(X,Y)       Compute the modulo (i.e., the remainder of x/y)
   MIN(X,Y)       Compute the minimum of two numbers
   MAX(X,Y)       Compute the maximum of two numbers
   DIM(X,Y)       Compute the positive difference (i.e., x-min(x,y))
   IND(X,Y)       Compute the mathematical indicator function
   CABS(XR,XC)    Compute the absolute value of a complex number
   CEXP(XR,XC)    Compute the real component of the exponential of a
                  complex number
   CSQRT(XR,XC)   Compute the real component of the square root of a
                  complex number
   CSQRTI(XR,XC)  Compute the complex component of the square root of a
                  complex number
 
EXPONENTIAL AND LOGARITHMIC FUNCTIONS
   EXP(X)         Compute the exponential
   LN(X)          Compute the natural logarithm of a number
   LOG(X)         Compute the natural logarithm of a number
   LOG10(X)       Compute the base 10 logarithm of a number
   LOG2(X)        Compute the base 2 logarithm of a number
   CEXPI(XR,XC)   Compute the complex component of the exponential of a
                  complex number
   CLOG(XR,XC)    Compute the real component of the logarithm of a
                  complex number
   CLOGI(XR,XC)   Compute the complex component of the logarithm of a
                  complex number
 
TYPE CONVERSION FUNCTIONS
   SIGN(X)        Compute the sign of a number
   INT(X)         Compute the integer portion of a number
   FRACT(X)       Compute the fractional portion of a number
   MSD(X)         Compute the most significant digit of a number
   ROUND(X)       Round to the closest integer of a number
 
BASE CONVERSION FUNCTIONS
   OCTDEC(X)      Perform an octal to decimal conversion
   DECOCT(X)      Perform a decimal to octal conversion
 
ERROR FUNCTIONS
   ERF(X)         Compute the error function
   ERFC(X)        Compute the complementary error function
   DAWSON(X)      Compute Dawson's integral
 
GAMMA AND BETA FUNCTIONS
   GAMMA(X)       Compute the Gamma function
   LOGGAMMA(X)    Compute the log (to the base e) Gamma function
   GAMMAI(X,A)    Compute the incomplete Gamma function
   GAMMAIP(X,A)   Compute an alternate incomplete Gamma function
   GAMMAIC(X,A)   Compute the complementary incomplete Gamma function
   GAMMAR(X)      Compute the reciprocal Gamma function
   DIGAMMA(X)     Compute the digamma (or Psi) function
   TRICOMI(X,A)   Compute Tricomi's incomplete Gamma function
   BETA(A,B)      Compute the Beta function
   BETAI(X,A,B)   Compute the incomplete Beta function
   LNBETA(A,B)    Compute the log (to the base e) Beta function
   POCH(X,A)      Compute Pchhammer's generalized symbol
   POCH1(X,A)     Compute Pchhammer's generalized symbol of the first
                  order
 
CHEBYCHEV POLYNOMIALS OF THE FIRST KIND
   CHEB0(X)       Compute the Chebychev polynomial of order 0
   CHEB1(X)       Compute the Chebychev polynomial of order 1
   CHEB2(X)       Compute the Chebychev polynomial of order 2
   CHEB3(X)       Compute the Chebychev polynomial of order 3
   CHEB4(X)       Compute the Chebychev polynomial of order 4
   CHEB5(X)       Compute the Chebychev polynomial of order 5
   CHEB6(X)       Compute the Chebychev polynomial of order 6
   CHEB7(X)       Compute the Chebychev polynomial of order 7
   CHEB8(X)       Compute the Chebychev polynomial of order 8
   CHEB9(X)       Compute the Chebychev polynomial of order 9
   CHEB10(X)      Compute the Chebychev polynomial of order 10
 
BESSEL FUNCTIONS
   BESS0(X)       Compute the Bessel function of first kind and order 0
   BESS1(X)       Compute the Bessel function of first kind and order 1
   BESSJN(X,N)    Compute the Bessel function of first kind and order n
                  (n can be fractional)
   CBESSJR(X,N)   Compute the real component of the Bessel function of
                  first kind, order n (n can be fractional), and
                  complex argument
   CBESSJI(X,N)   Compute the complex component of the Bessel function
                  of first kind, order n (n can be fractional), and
                  complex argument
   BESSY0(X)      Compute the Bessel function of second kind and order
                  0
   BESSY1(X)      Compute the Bessel function of second kind and order
                  1
   BESSYN(X,N)    Compute the Bessel function of second kind and order
                  n (n can be fractional)
   CBESSYR(X,N)   Compute the real component of the Bessel function of
                  second kind, order n (n can be fractional), and
                  complex argument
   CBESSYI(X,N)   Compute the complex component of the Bessel function
                  of second kind, order n (n can be fractional), and
                  complex argument
   BESSI0(X)      Compute the modified Bessel function of first kind
                  and order 0
   BESSI0E(X)     Compute the exponentially scaled modified Bessel`
                  function of first kind and order 0
   BESSI1(X)      Compute the modified Bessel function of first kind
                  and order 1
   BESSI1E(X)     Compute the exponentially scaled modified Bessel
                  function of first kind and order 1
   BESSIN(X,N)    Compute the modified Bessel function of first kind
                  and order n (n can be fractional)
   BESSINE(X,N)   Compute the exponentially scaled modified Bessel
                  function of first kind and order n (n can be
                  fractional)
   CBESSIR(X,N)   Compute the real component of the modified Bessel
                  function of order n (n can be fractional) and
                  complex argument
   CBESSII(X,N)   Compute the complex component of the modified Bessel
                  function of order n (n can be fractional) and
                  complex argument
   BESSK0(X)      Compute the modified Bessel function of third kind
                  and order 0
   BESSK0E(X)     Compute the exponentially scaled modified Bessel`
                  function of third kind and order 0
   BESSK1(X)      Compute the modified Bessel function of third kind
                  and order 1
   BESSK1E(X)     Compute the exponentially scaled modified Bessel
                  function of third kind and order 1
   BESSKN(X,N)    Compute the modified Bessel function of third kind
                  and order n (n can be fractional)
   BESSKNE(X,N)   Compute the exponentially scaled modified Bessel
                  function of third kind and order n (n can be
                  fractional)
   CBESSKR(X,N)   Compute the real component of the modified Bessel
                  function of the third kind, order n (n can be
                  fractional), and complex argument
   CBESSKI(X,N)   Compute the complex component of the modified Bessel
                  function of the third kind, order n (n can be
                  fractional), and complex argument
   AIRY(X)        Compute the Airy function
   BAIRY(X)       Compute the Airy function of the second kind
 
INTEGRAL FUNCTIONS
   LOGINT(X)      Compute the logarithmic integral
   EXPINT1(X)     Compute the exponential integral
   EXPINTE(X)     Compute the exponential integral
   EXPINTN(X,N)   Compute the exponential integral of integer order
   SININT(X)      Compute the sine integral
   COSINT(X)      Compute the cosine integral
   SINHINT(X)     Compute the hyperbolic sine integral
   COSHINT(X)     Compute the hyperbolic cosine integral
   SPENCE(X)      Compute the Spence dilogarithm function
   FRESNS(X)      Fresnel sine integral
   FRESNC(X)      Fresnel cosine integral
   FRESNF(X)      Fresnel auxillary function f integral
   FRESNG(X)      Fresnel auxillary function g integral
 
ELLIPTIC FUNCTIONS AND INTEGRALS
   ELLIPC1(X)     Compute the complete elliptic integral of the first
                  kind (Legendre form)
   ELLIPC2(X)     Compute the complete elliptic integral of the second
                  kind (Legendre form)
   ELLIP1(PHI,K)  Compute the elliptic integral of the first kind
                  (Legendre form)
   ELLIP2(PHI,K)  Compute the elliptic integral of the second kind
                  (Legendre form)
   ELLIP3(P,N,K)  Compute the elliptic integral of the third kind
                  (Legendre form)
   RC(X,Y)        Compute Carlson's degenerate elliptic integral
   RD(X,Y,Z)      Compute Carlson's elliptic integral of the second
                  kind
   RF(X,Y,Z)      Compute Carlson's elliptic integral of the first
                  kind
   RJ(X,Y,Z,P)    Compute Carlson's elliptic integral of the third
                  kind
   SN(X,M)        Jacobian elliptic sn function
   CN(X,M)        Jacobian elliptic cn function
   DN(X,M)        Jacobian elliptic dn function
   PEQ(XR,XI)     The real component of the Weierstrass elliptic
                  function (equianharmomic case)
   PEQI(XR,XI)    The complex component of the Weierstrass elliptic
                  function (equianharmomic case)
   PEQ1(XR,XI)    The real component of the first derivative of the
                  Weierstrass elliptic function (equianharmomic case)
   PEQ1I(XR,XI)   The complex component of the first derivative of the
                  Weierstrass elliptic function (equianharmomic case)
   PLEM(XR,XI)    The real component of the Weierstrass elliptic
                  function (lemniscatic case)
   PLEMI(XR,XI)   The complex component of the Weierstrass elliptic
                  function (lemniscatic case)
   PLEM1(XR,XI)   The real component of the first derivative of the
                  Weierstrass elliptic function (lemniscatic case)
   PLEM1I(XR,XI)  The complex component of the first derivative of the
                  Weierstrass elliptic function (lemniscatic case)
 
 
EXPERIMENT DESIGN FUNCTIONS
   BINPAT(X)      Used to generate Yates design matrices
 
MISCELLANEOUS FUNCTIONS
   JULIA(X)       Used to generate Julia sets
   CHU(X,A,B)     Compute the confluent hypergeometric function
 
For a list of available trigonometric functions, enter HELP
TRIGONOMETRIC FUNCTIONS or HELP TRIG FUNCTIONS   .
 
For a list of available probability functions, enter HELP PROBABILITY
FUNCTIONS or HELP PROB FUNCTIONS   .
 
----------------------------------------------------------
 
 
 
 
 
 
 
 
 
 
 
 
 
 
 
 
 
 
 
 
 
 
 
 
 
 
 
 
 
 
 
 
 
 
 
 
 
 
 
 
 
 
 
 
 
 
 
 
 
 
 
 
 
 
 
 
 
 
 
 
 
 
 
 
 
 
 
 
 
 
 
 
 
 
 
 
 
 
 
 
 
 
 
 
 
 
 
-------------------------  *TRIG FUNCTIONS*  -------------
 
TRIGONOMETRIC FUNCTIONS
Trigonometric Functions
 
The available trigonometric functions are--
 
TRIGONOMETRIC FUNCTIONS
   SIN(X)        Compute the sine
   COS(X)        Compute the cosine
   TAN(X)        Compute the tangent
   COT(X)        Compute the cotangent
   SEC(X)        Compute the secant
   CSC(X)        Compute the cosecant
   CSIN(XR,XC)    Compute the real component of the sine of a
                  complex number
   CSINI(XR,XC)   Compute the complex component of the sine of a
                  complex number
   CCOS(XR,XC)    Compute the real component of the cosine of a
                  complex number
   CCOSI(XR,XC)   Compute the complex component of the cosine of a
                  complex number
 
INVERSE TRIGONOMETRIC FUNCTIONS
   ARCSIN(X)     Compute the inverse sine
   ARCCOS(X)     Compute the inverse cosine
   ARCTAN(X)     Compute the inverse tangent
   ARCCOT(X)     Compute the inverse cotangent
   ARCSEC(X)     Compute the inverse secant
   ARCCSC(X)     Compute the inverse cosecant
 
HYPERBOLIC TRIGONOMETRIC FUNCTIONS
   SINH(X)       Compute the hyperbolic sine
   COSH(X)       Compute the hyperbolic cosine
   TANH(X)       Compute the hyperbolic tangent
   COTH(X)       Compute the hyperbolic cotangent
   SECH(X)       Compute the hyperbolic secant
   CSCH(X)       Compute the hyperbolic cosecant
 
INVERSE HYPERBOLIC TRIGONOMETRIC FUNCTIONS
   ARCSINH(X)    Compute the inverse hyperbolic sine
   ARCCOSH(X)    Compute the inverse hyperbolic cosine
   ARCTANH(X)    Compute the inverse hyperbolic tangent
   ARCCOTH(X)    Compute the inverse hyperbolic cotangent
   ARCSECH(X)    Compute the inverse hyperbolic secant
   ARCCSCH(X)    Compute the inverse hyperbolic cosecant
 
For a list of available mathematics functions, enter HELP MATHEMATICS
FUNCTIONS or HELP MATH FUNCTIONS   .
 
For a list of available probability functions, enter HELP PROBABILITY
FUNCTIONS or HELP PROB FUNCTIONS   .
 
----------------------------------------------------------
 
 
 
 
 
 
 
 
 
 
 
 
 
 
 
 
 
 
 
 
 
 
 
 
 
 
 
 
 
 
 
 
 
 
 
 
 
 
 
 
 
 
 
 
 
-------------------------  *PROB FUNCTIONS*  -------------
 
PROBABILITY FUNCTIONS
Probability Functions
 
Additional information on Dataplot probability functions is
available by entering the command

     LIST DISTRIBU

The available probability functions are--
 
ASYMMETRIC DOUBLE EXPONENTIAL (OR LAPLACE) DISTRIBUTION
   ADECDF(X,K)        Compute asymmetric double exponential cumulative
                      distribution function
   ADEPDF(X,K)        Compute the asymmetric double exponential
                      probability density function
   ADEPPF(P,K)        Compute the asymmetric double exponential
                      percent point function

ALPHA DISTRIBUTION
   ALPCDF(X,A,B)      Compute the alpha cumulative distribution function
   ALPCHAZ(X,A,B)     Compute the alpha cumulative hazard function
   ALPHAZ(X,A,B)      Compute the alpha hazard function
   ALPPDF(X,A,B)      Compute the alpha probability density function
   ALPPPF(X,A,B)      Compute the alpha percent point function
 
ANGLIT DISTRIBUTION
   ANGCDF(X)          Compute the anglit cumulative distribution function
   ANGPDF(X)          Compute the anglit probability density function
   ANGPPF(X)          Compute the anglit percent point function
 
ARCSIN DISTRIBUTION
   ARSCDF(X)          Compute the arcsin cumulative distribution function
   ARSPDF(X)          Compute the arcsin probability density function
   ARSPPF(X)          Compute the arcsin percent point function
 
BETA-BINOMIAL DISTRIBUTION
   BBMCDF(X,A,B,N)    Compute the Beta-binomial cumulative
                      distribution function
   BBNPDF(X,A,B,N)    Compute the Beta-binomial probability density
                      function
   BBNPPF(X,A,B,N)    Compute the Beta-binomial percent point function
 
BETA DISTRIBUTION
   BETCDF(X,A,B)      Compute the Beta cumulative distribution function
   BETPDF(X,A,B)      Compute the Beta probability density function
   BETPPF(X,A,B)      Compute the Beta percent point function
 
BIVARIATE NORMAL DISTRIBUTION
   BVNCDF(X1,X2,CORR) Compute the bivariate normal cumulative
                      distribution function
   BVNPDF(X1,X2,CORR) Compute the bivariate normal probability density
                      function

BINOMIAL DISTRIBUTION
   BINCDF(X,P,N)      Compute the binomial cumulative distribution
                      function
   BINPDF(X,P,N)      Compute the binomial probability density function
   BINPPF(X,P,N)      Compute the binomial percent point function
 
BRADFORD DISTRIBUTION
   BRACDF(X,BETA)     Compute the Bradford cumulative distribution
                      function
   BRAPDF(X,BETA)     Compute the Bradford probability density function
   BRAPPF(X,BETA)     Compute the Bradford percent point function

BI-WEIBULL DISTRIBUTION
   BWECDF(X,S1,G1,L2,S2,G2) Compute the Bi-Weibull cumulative
                            distribution function
   BWEPDF(X,S1,G1,L2,S2,G2) Compute the Bi-Weibull probability density
                            function
   BWEPPF(X,S1,G1,L2,S2,G2) Compute the Bi-Weibull percent point
                            function

CAUCHY DISTRIBUTION
   CAUCDF(X)          Compute Cauchy cumulative distribution function
   CAUPDF(X)          Compute the Cauchy probability density function
   CAUPPF(P)          Compute the Cauchy percent point function
   CAUSF(P)           Compute the Cauchy sparsity function
 
CHI DISTRIBUTION
   CHCDF(X,NU)        Compute the chi cumulative distribution
                      function
   CHPDF(X,NU)        Compute the chi probability density function
   CHPPF(P,NU)        Compute the chi percent point function
 
CHI-SQUARE DISTRIBUTION
   CHSCDF(X,NU)       Compute the chi-squared cumulative distribution
                      function
   CHSPDF(X,NU)       Compute the chi-squared probability density
                      function
   CHSPPF(P,NU)       Compute the chi-squared percent point function
 
COSINE DISTRIBUTION
   COSCDF(X)          Compute the cosine cumulative distribution
                      function
   COSPDF(X)          Compute the cosine probability density function
   COSPPF(P)          Compute the cosine percent point function
 
DOUBLE EXPONENTIAL (OR LAPLACE) DISTRIBUTION
   DEXCDF(X)          Compute double exponential cumulative
                      distribution function
   DEXPDF(X)          Compute the double exponential probability
                      density function
   DEXPPF(P)          Compute the double exponential percent point
                      function
   DEXSF(P)           Compute the double exponential sparsity function

DOUBLE GAMMA DISTRIBUTION
   DGACDF(X,GAMMA)    Compute double gamma cumulative distribution
                      function
   DGAPDF(X,GAMMA)    Compute the double gamma probability density
                      function
   DGAPPF(P,GAMMA)    Compute the double gamma percent point function

DISCRETE UNIFORM DISTRIBUTION
   DISCDF(X,N)        Compute the discrete uniform cumulative
                      distribution function
   DISPDF(X,N)        Compute the discrete uniform probability
                      density function
   DISPPF(P,N)        Compute the discrete uniform percent point
                      function

LOGARITHMIC SERIES DISTRIBUTION
   DLGCDF(X,C)        Compute the logarithmic series cumulative
                      distribution function
   DLGPDF(X,C)        Compute the logarithmic series probability
                      density function
   DLGPPF(P,C)        Compute the logarithmic series percent point
                      function

DOUBLY NON-CENTRAL F DISTRIBUTION
   DNFCDF(X,N1,N2,A,B)  Compute the doubly non-central F cumulative
                        distribution function
   DNFPPF(P,N1,N2,A,B)  Compute the doubly non-central F percent point
                        function
 
DOUBLY NON-CENTRAL T DISTRIBUTION
   DNTCDF(X,N1,A,B)   Compute the doubly non-central t cumulative
                      distribution function
   DNTPPF(P,N1,A,B)   Compute the doubly non-central t percent point
                      function
 
DOUBLE WEIBULL DISTRIBUTION
   DWECDF(X,GAMMA)    Compute double Weibull cumulative distribution
                      function
   DWEPDF(X,GAMMA)    Compute the double Weibull probability density
                      function
   DWEPPF(P,GAMMA)    Compute the double Weibull percent point function

ERROR (OR SUBBOTIN, EXPONENTIAL POWER, GENERALIZED ERROR) DISTRIBUTION
   ERRCDF(X,ALPHA)    Compute error cumulative distribution function
   ERRPDF(X,ALPHA)    Compute error probability density function
   ERRPPF(X,ALPHA)    Compute error percent point function

EXTREME VALUE TYPE I (OR GUMBEL) DISTRIBUTION
   EV1CDF(X)          Compute Gumbel cumulative distribution function
   EV1CHAZ(X)         Compute Gumbel cumulative hazard function
   EV1HAZ(X)          Compute Gumbel hazard function
   EV1PDF(X)          Compute the Gumbel probability density function
   EV1PPF(P)          Compute the Gumbel percent point function
 
EXTREME VALUE TYPE II (OR FRECHET) DISTRIBUTION
   EV2CDF(X,GAMMA)    Compute Frechet cumulative distribution function
   EV2CHAZ(X,GAMMA)   Compute Frechet cumulative hazard function
   EV2HAZ(X,GAMMA)    Compute Frechet cumulative hazard function
   EV2PDF(X,GAMMA)    Compute the Frechet probability density function
   EV2PPF(P,GAMMA)    Compute the Frechet percent point function
 
EXPONENTIATED WEIBULL DISTRIBUTION
   EWECDF(X,G,T)      Compute exponentiated Weibull cumulative
                      distribution function
   EWECHAZ(X,G,T)     Compute exponentiated Weibull cumulative
                      hazard function
   EWEHAZ(X,G,T)      Compute exponentiated Weibull hazard function
   EWEPDF(X,G,T)      Compute exponentiated Weibull probability
                      density function
   EWEPPF(P,G,T)      Compute exponentiated Weibull percent point
                      function

EXPONENTIAL DISTRIBUTION
   EXPCDF(X)          Compute exponential cumulative distribution
                      function
   EXPCHAZ(X)         Compute exponential cumulative hazard
                      function
   EXPHAZ(X)          Compute exponential hazard function
   EXPPDF(X)          Compute the exponential probability density
                      function
   EXPPPF(P)          Compute the exponential percent point function
   EXPSF(P)           Compute the exponential sparsity function
 
F DISTRIBUTION
   FCDF(X,NU1,NU2)    Compute the F cumulative distribution function
   FPDF(X,NU1,NU2)    Compute the F probability density function
   FPPF(P,NU1,NU2)    Compute the F percent point function
 
FOLDED CAUCHY DISTRIBUTION
   FCACDF(X,MU,SD)    Compute the folded Cauchy cumulative
                      distribution function
   FCAPDF(X,MU,SD)    Compute the folded Cauchy probabiity density
                      function
   FCAPPF(P,MU,SD)    Compute the folded Cauchy percent point function

FATIGUE LIFE DISTRIBUTION
   FLCDF(X,GAMMA)     Compute the Fatigue Life cumulative distribution
                      function
   FLCHAZ(X,GAMMA)    Compute the Fatigue Life cumulative hazard
                      function
   FLHAZ(X,GAMMA)     Compute the Fatigue Life hazard function
   FLPDF(X,GAMMA)     Compute the Fatigue Life probability density
                      function
   FLPPF(P,GAMMA)     Compute the Fatigue Life percent point function
 
FOLDED NORMAL DISTRIBUTION
   FNRCDF(X,MU,SD)    Compute the folded normal cumulative
                      distribution function
   FNRPDF(X,MU,SD)    Compute the folded normal probability density
                      function
   FNRPPF(P,MU,SD)    Compute the folded normal percent point function

FOLDED T DISTRIBUTION
   FTCDF(X,NU)        Compute the folded t cumulative distribution
                      function
   FTPDF(X,NU)        Compute the folded t probability density
                      function
   FTPPF(P,NU)        Compute the folded t percent point function

GAMMA DISTRIBUTION
   GAMCDF(X,GAMMA)    Compute the gamma cumulative distribution
                      function
   GAMCHAZ(X,GAMMA)   Compute the gamma cumulative hazard function
   GAMHAZ(X,GAMMA)    Compute the gamma hazard function
   GAMPDF(X,GAMMA)    Compute the gamma probability density function
   GAMPPF(P,GAMMA)    Compute the gamma percent point function
 
GEOMETRIC EXTREME EXPONENTIAL DISTRIBUTION
   GEECDF(X,GAMMA)    Compute geometric extreme exponential cumulative
                      distribution function
   GEECHAZ(X,GAMMA)   Compute geometric extreme exponential cumulative
                      hazard function
   GEEHAZ(X,GAMMA)    Compute geometric extreme exponential hazard
                      function
   GEEPDF(X,GAMMA)    Compute geometric extreme exponential
                      probability density function
   GEEPPF(P,GAMMA)    Compute geometric extreme exponential percent
                      point function
 
GENERALIZED PARETO DISTRIBUTION
   GEPCDF(X,GAMMA)    Compute the generalized Pareto cumulative
                      distribution function
   GEPCDF(X,GAMMA)    Compute the generalized Pareto probability
                      density function
   GEPCDF(P,GAMMA)    Compute the generalized Pareto percent point
                      function
 
GENERALIZED EXTREME VALUE DISTRIBUTION
   GEVCDF(X,GAMMA)    Compute the generalized extreme value
                      cumulative distribution function
   GEVPDF(X,GAMMA)    Compute the generalized extreme value
                      probability density function
   GEVPPF(P,GAMMA)    Compute the generalized extreme value
                      percent point function
 
GENERALIZED EXPONENTIAL DISTRIBUTION
   GEXCDF(X,L1,L2,S)  Compute the generalized exponential cumulative
                      distribution function
   GEXPDF(X,L1,L2,S)  Compute the generalized exponential probability
                      density function
   GEXPPF(P,L1,L2,S)  Compute the generalized exponential percent
                      point function
 
GENERALIZED GAMMA DISTRIBUTION (includes INVERTED GAMMA)
   GGDCDF(X,ALPHA,C)  Compute the generalized gamma cumulative
                      distribution function
   GGDCHAZ(X,ALPHA,C) Compute the generalized gamma cumulative
                      hazard function
   GGDHAZ(X,ALPHA,C)  Compute the generalized gamma hazard function
   GGDPDF(X,ALPHA,C)  Compute the generalized gamma probability
                      density function
   GGDPPF(P,ALPHA,C)  Compute the generalized gamma percent point
                      function
 
GEOMETRIC DISTRIBUTION
   GEOCDF(X,P)        Compute the geometric cumulative distribution
                      function
   GEOPDF(X,P)        Compute the geometric probability density
                      function
   GEOPPF(X,P)        Compute the geometric percent point function
 
G-AND-H DISTRIBUTION
   GHCDF(X,G,H)       Compute the g-and-h cumulative distribution
                      function
   GHPDF(X,G,H)       Compute the g-and-h probability density
                      function
   GHPPF(X,G,H)       Compute the g-and-h percent point function
 
GOMPERTZ DISTRIBUTION
   GOMCDF(X,C,B)      Compute the Gompertz cumulative distribution
                      function
   GOMPDF(X,C,B)      Compute the Gompertz probability density
                      function
   GOMPPF(P,C,B)      Compute the Gompertz percent point function
 
HALF-CAUCHY DISTRIBUTION
   HFCCDF(X)          Compute the half-Cauchy cumulative distribution
                      function
   HFCPDF(X)          Compute the half-Cauchy probability density
                      function
   HFCPPF(P)          Compute the half-Cauchy percent point function
 
HALF-LOGISTIC  (AND GENERALIZED HALF-LOGISTIC) DISTRIBUTION
   HFLCDF(X)          Compute the half-logistic cumulative distribution
                      function
   HFLPDF(X)          Compute the half-logistic probability density
                      function
   HFLPPF(P)          Compute the half-logistic percent point function
 
HALF-NORMAL DISTRIBUTION
   HFNCDF(X)          Compute the half-normal cumulative distribution
                      function
   HFNPDF(X)          Compute the half-normal probability density
                      function
   HFNPPF(P)          Compute the half-normal percent point function
 
HERMITE DISTRIBUTION
   HERCDF(X,A,B)      Compute the Hermite cumulative distribution
                      function
   HERPDF(X,A,B)      Compute the Hermite probability mass function
   HERPPF(X,A,B)      Compute the Hermite percent point function

HYPERBOLIC SECANT DISTRIBUTION
   HSECDF(X)          Compute the hyperbolic secant cumulative
                      distribution function
   HSEPDF(X)          Compute the hyperbolic secant probability
                      density function
   HSEPPF(P)          Compute the hyperbolic secant percent point
                      function

HYPERGEOMETRIC DISTRIBUTION
   HYPCDF(L,K,N,M)    Compute the hypergeometric cumulative
                      distribution function
   HYPPDF(L,K,N,M)    Compute the hypergeometric probability density
                      function
   HYPPPF(L,K,N,M)    Compute the hypergeometric percent point function
 
INVERTED BETA DISTRIBUTION
   IBCDF(X,A,B)       Compute the inverted beta cumulative
                      distribution function
   IBPDF(X,A,B)       Compute the inverted beta probability density
                      function
   IBPPF(P,A,B)       Compute the inverted beta percent point function
 
INVERSE GAUSSIAN DISTRIBUTION
   IGCDF(X,GAMMA)     Compute the inverse Gaussian cumulative
                      distribution function
   IGCHAZ(X,GAMMA)    Compute the inverse Gaussian cumulative
                      hazard function
   IGHAZ(X,GAMMA)     Compute the inverse Gaussian hazard function
   IGPDF(X,GAMMA)     Compute the inverse Gaussian probability density
                      function
   IGPPF(X,GAMMA)     Compute the inverse Gaussian percent point
                      function
 
INVERTED GAMMA DISTRIBUTION
   IGACDF(X,GAMMA)    Compute the inverted gamma cumulative
                      distribution function
   IGACHAZ(X,GAMMA)   Compute the inverted gamma cumulative hazard
                      function
   IGAHAZ(X,GAMMA)    Compute the inverted gamma hazard function
   IGAPDF(X,GAMMA)    Compute the inverted gamma probability density
                      function
   IGAPPF(P,GAMMA)    Compute the inverted gamma percent point
                      function
 
INVERTED WEIBULL DISTRIBUTION
   IWECDF(X,GAMMA)    Compute the inverted Weibull cumulative
                      distribution function
   IWECHAZ(X,GAMMA)   Compute the inverted Weibull cumulative
                      hazard function
   IWEHAZ(X,GAMMA)    Compute the inverted Weibull hazard function
   IWEPDF(X,GAMMA)    Compute the inverted Weibull probability density
                      function
   IWEPPF(P,GAMMA)    Compute the inverted Weibull percent point
                      function
 
JOHNSON SB DISTRIBUTION
   JSBCDF(X,A1,A2)    Compute the Johnson SB cumulative distribution
                      function
   JSBPDF(X,A1,A2)    Compute the Johnson SB probability density
                      function
   JSBPPF(P,A1,A2)    Compute the Johnson SB percent point function
 
JOHNSON SU DISTRIBUTION
   JSUCDF(X,A1,A2)    Compute the Johnson SU cumulative distribution
                      function
   JSUPDF(X,A1,A2)    Compute the Johnson SU probability density
                      function
   JSUPPF(P,A1,A2)    Compute the Johnson SU percent point function
 
MIELKE'S BETA-KAPPA DISTRIBUTION
   KAPCDF(X,K,B,T)    Compute the Mielke's beta-kappa cumulative
                      distribution function
   KAPPDF(X,K,B,T)    Compute the Mielke's beta-kappa probability
                      density function
   KAPPPF(P,K,B,T)    Compute the Mielke's beta-kappa percent point
                      function
 
TUKEY-LAMBDA DISTRIBUTION
   LAMCDF(X,LAMBDA)   Compute the Tukey-Lambda cumulative distribution
                      function
   LAMPDF(X,LAMBDA)   Compute the Tukey-Lambda probability density
                      function
   LAMPPF(P,LAMBDA)   Compute the Tukey-Lambda percent point function
   LAMSF(P,LAMBDA)    Compute the Tukey-Lambda sparsity function
 
LANDAU DISTRIBUTION
   LANCDF(X)          Compute the Landau cumulative distribution
                      function
   LANPDF(X)          Compute the Landau probability density function
   LANPPF(P)          Compute the Landau percent point function
 
LOG DOUBLE EXPONENTIAL (OR LAPLACE) DISTRIBUTION
   LDECDF(X,ALPHA)    Compute log double exponential cumulative
                      distribution function
   LDEPDF(X,ALPHA)    Compute log double exponential probability
                      density function
   LDEPPF(P,ALPHA)    Compute log double exponential percent point
                      function
 
LOG GAMMA DISTRIBUTION
   LGACDF(X,GAMMA)    Compute the log gamma cumulative distribution
                      function
   LGAPDF(X,GAMMA)    Compute the log gamma probability density
                      function
   LGAPPF(P,GAMMA)    Compute the log gamma percent point function
 
LOG-NORMAL DISTRIBUTION
   LGNCDF(X,S)        Compute the log-normal cumulative distribution
                      function
   LGNCHAZ(X,S)       Compute the log-normal cumulative hazard
                      function
   LGNHAZ(X,S)        Compute the log-normal hazard function
   LGNPDF(X,S)        Compute the log-normal probability density
                      function
   LGNPPF(P,S)        Compute the log-normal percent point function
 
LOG-LOGISTIC DISTRIBUTION
   LLGCDF(X,DELTA)    Compute the log-logistic cumulative distribution
                      function
   LLGPDF(X,DELTA)    Compute the log-logistic probability density
                      function
   LLGPPF(P,DELTA)    Compute the log-logistic percent point function
 
LOGISTIC DISTRIBUTION
   LOGCDF(X)          Compute the logistic cumulative distribution
                      function
   LOGCHAZ(X)         Compute the logistic cumulative hazard
                      function
   LOGHAZ(X)          Compute the logistic hazard function
   LOGPDF(X)          Compute the logistic probability density function
   LOGPPF(P)          Compute the logistic percent point function
   LOGSF(P)           Compute the logistic sparsity function
 
MAXWELL DISTRIBUTION
   MAXCDF(X)          Compute the Maxwell cumulative distribution
                      function
   MAXPDF(X)          Compute the Maxwell probability density function
   MAXPPF(P)          Compute the Maxwell percent point function

NEGATIVE BINOMIAL DISTRIBUTION
   NBCDF(X,P,N)       Compute the negative binomial cumulative
                      distribution function
   NBPDF(X,P,N)       Compute the negative binomial probability density
                      function
   NBPPF(X,P,N)       Compute the negative binomial percent point
                      function
 
NON-CENTRAL BETA DISTRIBUTION
   NCBCDF(X,A,B,LAM)  Compute the non-central Beta cumulative
                      distribution function
   NCBPPF(P,A,B,LAM)  Compute the non-central Beta percent point
                      function
 
NON-CENTRAL CHI-SQUARE DISTRIBUTION
   NCCCDF(X,N1,ALPHA) Compute the non-central chi-square cumulative
                      distribution function
   NCCPDF(X,N1,ALPHA) Compute the non-central chi-square probability
                      density function
   NCCPPF(P,N1,ALPHA) Compute the non-central chi-square percent point
                      function
 
NON-CENTRAL F DISTRIBUTION
   NCFCDF(X,N1,N2,A,B) Compute the non-central F cumulative
                       distribution function
   NCFPPF(P,N1,N2,A,B) Compute the non-central F percent point function
 
NON-CENTRAL T DISTRIBUTION
   NCTCDF(X,N1,ALPHA) Compute the non-central t cumulative distribution
   NCTPDF(X,N1,ALPHA) Compute the non-central t probability density
                      function
   NCTPPF(P,N1,ALPHA) Compute the non-central t percent point function
 
NORMAL DISTRIBUTION (MEAN OF ZERO, STANDARD DEVIATION OF 1)
   NORCDF(X)          Compute the normal cumulative distribution
                      function
   NORPDF(X)          Compute the normal probability density function
   NORPPF(P)          Compute the normal percent point function
   NORSF(P)           Compute the normal sparsity function
 
NORMAL MIXTURE DISTRIBUTION
   NORMXCDF(X,P,U1,S1,U2S2)  Compute the normal mixture cumulative
                             distribution function
   NORMXPDF(X,P,U1,S1,U2S2)  Compute the normal mixture probability
                             density function
   NORMXPDF(X,P,U1,S1,U2S2)  Compute the normal mixture percent point
                             function
 
PARETO DISTRIBUTION
   PARCDF(X,GAMMA)    Compute the Pareto cumulative distribution
                      function
   PARCHAZ(X,GAMMA)   Compute the Pareto cumulative hazard
                      function
   PARHAZ(X,GAMMA)    Compute the Pareto hazard function
   PARCDF(X,GAMMA)    Compute the Pareto probability density function
   PARCDF(X,GAMMA)    Compute the Pareto percent point function
 
PARETO (SECOND KIND) DISTRIBUTION
   PA2CDF(X,GAMMA)    Compute the Pareto second kind cumulative
                      distribution function
   PA2PDF(X,GAMMA)    Compute the Pareto second kind probability
                      density function
   PA2PPF(P,GAMMA)    Compute the Pareto second kind percent point
                      function
 
POWER EXPONENTIAL DISTRIBUTION
   PEXCDF(X,A,B)      Compute the power exponential cumulative
                      distribution function
   PEXCHAZ(X,A,B)     Compute the power exponential cumulative
                      hazard function
   PEXHAZ(X,A,B)      Compute the power exponential hazard function
   PEXPDF(X,A,B)      Compute the power exponential probability
                      density function
   PEXPPF(P,A,B)      Compute the power exponential percent point
                      function
 
POWER LOG-NORMAL DISTRIBUTION
   PLNCDF(X,P,SD)     Compute the power log-normal cumulative
                      distribution function
   PLNCHAZ(X,P,SD)    Compute the power log-normal cumulative
                      hazard function
   PLNHAZ(X,P,SD)     Compute the power log-normal hazard function
   PLNPDF(X,P,SD)     Compute the power log-normal probability density
                      function
   PLNPPF(P,P,SD)     Compute the power log-normal percent point
                      function
 
POWER NORMAL DISTRIBUTION
   PNRCDF(X,P,SD)     Compute the power normal cumulative distribution
                      function
   PNRCHAZ(X,P,SD)    Compute the power normal cumulative hazard
                      function
   PNRHAZ(X,P,SD)     Compute the power normal hazard function
   PNRPDF(X,P,SD)     Compute the power normal probability density
                      function
   PNRPPF(P,P,SD)     Compute the power normal percent point function
 
POISSON DISTRIBUTION
   POICDF(X,P,N)      Compute the Poisson cumulative distribution
                      function
   POIPDF(X,P,N)      Compute the Poisson probability density function
   POIPPF(X,P,N)      Compute the Poisson percent point function
 
POWER FUNCTION DISTRIBUTION
   POWCDF(X,C)        Compute the power function cumulative
                      distribution function
   POWPDF(X,C)        Compute the power function probability density
                      function
   POWPPF(P,C)        Compute the power function percent point function
 
RAYLEIGH DISTRIBUTION
   RAYCDF(X)          Compute the Rayleigh cumulative distribution
                      function
   RAYPDF(X)          Compute the Rayleigh probability density function
   RAYPPF(P)          Compute the Rayleigh percent point function

RECIPROCAL DISTRIBUTION
   RECCDF(X,B)        Compute the reciprocal cumulative distribution
                      function
   RECPDF(X,B)        Compute the reciprocal probability density
                      function
   RECPPF(P,B)        Compute the reciprocal percent point function

RECIPROCAL INVERSE GAUSSIAN DISTRIBUTION
   RIGCDF(X,GAMMA)    Compute the reciprocal inverse Gaussian
                      cumulative distribution function
   RIGCHAZ(X,GAMMA)   Compute the reciprocal inverse Gaussian
                      cumulative hazard function
   RIGHAZ(X,GAMMA)    Compute the reciprocal inverse Gaussian hazard
                      function
   RIGPDF(X,GAMMA)    Compute the reciprocal inverse Gaussian
                      probability density function
   RIGPPF(X,GAMMA)    Compute the reciprocal inverse Gaussian percent
                      point function
 
SKEW DOUBLE EXPONENTIAL (OR LAPLACE) DISTRIBUTION
   SDECDF(X,LAMBDA)   Compute the skew double exponential cumulative
                      distribution function
   SDEPDF(X,LAMBDA)   Compute the skew double exponential
                      probability density function
   SDEPPF(P,LAMBDA)   Compute the skew double exponential
                      percent point function

SEMI-CIRCULAR DISTRIBUTION
   SEMCDF(X)          Compute the semi-circular cumulative distribution
                      function
   SEMPDF(X)          Compute the semi-circular probability density
                      function
   SEMPPF(P)          Compute the semi-circular percent point function

SLASH DISTRIBUTION
   SLACDF(X)          Compute the slash cumulative distribution
                      function
   SLAPDF(X)          Compute the slash probability density function
   SLAPPF(P)          Compute the slash percent point function
 
SKEWED NORMAL DISTRIBUTION
   SNCDF(X,LAMBDA)    Compute the skewed normal cumulative
                      distribution function
   SNPDF(X,LAMBDA)    Compute the skewed normal probability density
                      function
   SNPPF(P,LAMBDA)    Compute the skewed normal percent point function

SKEWED T DISTRIBUTION
   STCDF(X,NU,LAMBDA) Compute the skewed t cumulative distribution
                      function
   STPDF(X,NU,LAMBDA) Compute the skewed t probability density
                      function
   STPPF(X,NU,LAMBDA) Compute the skewed t percent point function
 
T DISTRIBUTION
   TCDF(X,NU)         Compute the t cumulative distribution function
   TPDF(X,NU)         Compute the t probability density function
   TPPF(P,NU)         Compute the t percent point function
 
TRUNCATED EXPONENTIAL DISTRIBUTION
   TNRCDF(X,X0,M,SD)  Compute the truncated exponential cumulative
                      distribution function
   TNRPDF(X,X0,M,SD)  Compute the truncated exponential probability
                      density function
   TNRPPF(P,X0,M,SD)  Compute the truncated exponential percent point
                      function
 
TRUNCATED NORMAL DISTRIBUTION
   TNRCDF(X,A,B,M,S)  Compute the truncated normal cumulative
                      distribution function
   TNRPDF(X,A,B,M,S)  Compute the truncated normal probability density
                      function
   TNRPPF(X,A,B,M,S)  Compute the truncated normal percent point
                      function
 
TRIANGULAR DISTRIBUTION
   TRICDF(X,C)        Compute the triangular cumulative distribution
                      function
   TRIPDF(X,C)        Compute the triangular probability density
                      function
   TRIPPF(P,C)        Compute the triangular percent point function
 
TWO-SIDED POWER DISTRIBUTION
   TSPCDF(X,THETA,N)  Compute the two-sided power cumulative
                      distribution function
   TSPPDF(X,THETA,N)  Compute the two-sided power probability density
                      function
   TSPPPF(P,THETA,N)  Compute the two-sided power percent point
                      function

UNIFORM DISTRIBUTION
   UNICDF(X)          Compute the uniform cumulative distribution
                      function
   UNIPDF(X)          Compute the uniform probability density function
   UNIPPF(P)          Compute the uniform percent point function

VON MISES DISTRIBUTION
   VONCDF(X,B)        Compute the Von Mises cumulative distribution
                      function
   VONPDF(X,B)        Compute the Von Mises probability density
                      function
   VONPPF(P,B)        Compute the Von Mises percent point function

WALD DISTRIBUTION
   WALCDF(X,GAMMA)    Compute the Wald cumulative distribution function
   WALPDF(X,GAMMA)    Compute the Wald probability density function
   WALPPF(P,GAMMA)    Compute the Wald percent point function
 
WARING DISTRIBUTION
   WARCDF(X,C,A)      Compute the Waring cumulative distribution
                      function
   WARPDF(X,C,A)      Compute the Waring probability density function
   WARPPF(P,C,A)      Compute the Waring percent point function
 
WRAPPED-UP CAUCHY DISTRIBUTION
   WCACDF(X,P)        Compute the wrapped-up Cauchy cumulative
                      distribution function
   WCAPDF(X,P)        Compute the wrapped-up Cauchy probability
                      density function
   WCAPPF(X,P)        Compute the wrapped-up Cauchy percent point
                      function

WEIBULL DISTRIBUTION
   WEICDF(X,GAMMA)    Compute the Weibull cumulative distribution
                      function
   WEICHAZ(X,GAMMA)   Compute the Weibull cumulative hazard function
   WEIHAZ(X,GAMMA)    Compute the Weibull hazard function
   WEIPDF(X,GAMMA)    Compute the Weibull probability density function
   WEIPPF(P,GAMMA)    Compute the Weibull percent point function
 
YULE DISTRIBUTION
   YULCDF(X,P)        Compute the Yule cumulative distribution
                      function
   YULPDF(X,P)        Compute the Yule probability density function
   YULPPF(X,P)        Compute the Yule percent point function
 
ZIPF DISTRIBUTION
   ZIPPDF(X,ALPHA)    Compute the Zipf probability density function

For a list of available mathematics functions, enter HELP MATHEMATICS
FUNCTIONS or HELP MATH FUNCTIONS   .
 
For a list of available trigonometric functions, enter HELP
TRIGONOMETRIC FUNCTIONS or HELP TRIG FUNCTIONS   .
 
----------------------------------------------------------
 
 
 
 
 
 
 
 
 
 
 
 
 
 
 
 
 
 
 
 
 
 
 
 












































-------------------------  *LET SUBCOMMANDS*  ------------
 
LET SUBCOMMANDS
LET Subcommands
 
The LET command is the single most powerful command in DATAPLOT. The
most important capability of the LET command is carrying out function
evaluations and variable transformations.  Such evaluations and
transformations are general--any Fortran-like expression can be used.
 
In addition, the LET command can also be used by the analyst to carry
out a broad spectrum of statistical,  mathematical,  and  probabilistic
operations.  These operations are specified by inclusion of subcommands
under the LET command.  These subcommands fall into 4 general
categories--
 
   1. Computing Statistics
      For a list of available statistics, enter HELP STATISTICS
 
   2. Performing Mathematical Operations
      For a list of available operations, enter HELP MATH OPERATIONS
 
   3. Performing Matrix Operations
      For a list of available operations, enter HELP MATRIX OPERATIONS
 
   4. Generating Random Numbers
      For a list of available distributions, enter HELP RANDOM NUMBERS
 
LET subcommands are distinguished from library functions in the
following ways--
 
   1) Functions enclose the input value in parenthesis.  LET
      subcommands use spaces.
 
   2) Functions can accept (and return) either parameters (i.e., single
      values) or variables (i.e., an array of values) while LET
      subcommands are specific in which they accept as input and what
      they return as output.
 
   3) Functions can accept expressions while LET subcommands do not.
      For example, the following is legal:
 
         LET Y2 = ABS(Y1-INT(Y2))
 
      For LET subcommands, you typically have to do something like the
      following:
 
         LET YTEMP = Y**2 + 8
         LET A = SUM YTEMP
 
----------------------------------------------------------
 
 
 
 
 
 
 
 
 
 
 
 
 
 
 
 
 
 
 
 
 
 
 
 
 
 
 
 
 
 
 
 
 
 
 
 
 
 
 
 
 
 
 
 
 
 
 
 


-------------------------  *STATISTICS*  -----------------
 
STATISTICS
Statistics
 
The calculation of individual statistics is done via subcommands under
the LET command, as in
 
   LET A = MEAN X
   LET B = STANDARD DEVIATION Y
   LET C = CORRELATION X Y
 
Statistics are computed on either one, two, or three response variables
(never parameters or functions) and the computed statistic is always
stored in a parameter (never a variable or function).

Supported statistics can be used in the following commands:

   1. LET A = <stat>
   2. <stat> STATISTIC PLOT
   3. <stat> CUMULATIVE STATISTIC PLOT
   4. <stat> MOVING STATISTIC PLOT
   5. <stat> WINDOW STATISTIC PLOT
   6. CROSS TABULATE <stat> STATISTIC PLOT
   7. FLUCTUATION PLOT <stat>
   8. TABULATION PLOT <stat>
   9. <STAT> BLOCK PLOT
  10. BOOTSTRAP <STAT> PLOT
  11. JACKNIFE <STAT> PLOT
  12. DEX <STAT> PLOT
  13. <STAT> INFLUENCE CURVE
  14. CROSS TABULATE <STAT>
  15. POSITIONAL TABULATION <STAT>
  16. LET V = MATRIX COLUMN <STAT>
  17. LET V = MATRIX ROW <STAT>
  18. LET A = MATRIX GRAND <STAT>
  19. LET M = MATRIX PARTITION <STAT>
  20. LET V = CROSS TABULATE <STAT>
  21. LET V = CROSS TABULATE CUMULATIVE <STAT>
  22. LET V = SORT BY <STAT>
  23. LET YOUT = MOVING <STAT>
  24. LET YOUT = CUMULATIVE <STAT>
  25. <STAT> INTERACTION PLOT <STAT>

Some of the above commands only support the case where the
statistic is computed from a single response variable.

The available statistical subcommands are (for the specifics of a
given statistic, enter HELP <stat> where <stat> denotes one of the
statistics given here)--
 
Case 1: One Response Variable

Location Statistics:
   BIWEIGHT LOCATION
   GEOMETRIC MEAN
   <H10/H12/H15/H17/H20> LOCATION
   HARMONIC MEAN
   HODGES-LEHMAN
   JSCORE
   LP LOCATION
   MEAN
   MEDIAN
   MIDMEAN
   MIDRANGE
   SHORTEST HALF MIDMEAN
   SHORTEST HALF MIDRANGE
   STANDARD DEVIATION OF LP LOCATION
   STANDARD DEVIATION OF THE MEAN
   TRIMMED MEAN
   TRIMMED MEAN STANDARD ERROR
   VARIANCE OF THE MEAN
   VARIANCE OF LP LOCATION
   WINSORIZED MEAN

Scale Statistics:
   AAD TO MEDIAN
   AVERAGE ABSOLUTE DEVIATION (AAD)
   AVERAGE ABSOLUTE DEVIATION FROM THE MEDIAN
   BIWEIGHT MIDVARIANCE
   BIWEIGHT SCALE
   COEFFICIENT OF DISPERSION
   COEFFICIENT OF VARIATION
   GEOMETRIC STANDARD DEVIATION
   <H10/H12/H15/H17/H20> SCALE
   INDEX OF DISPERSION
   INTERQUARTILE RANGE
   LOGNORMAL COEFFICIENT OF VARIATION
   MEDIAN ABSOLUTE DEVIATION (MAD)
   NORMALIZED INTERQUARTILE RANGE
   PERCENTAGE BEND MIDVARIANCE
   QN
   Q QUANTILE RANGE
   QUARTILE COEFFICIENT OF DISPERSION
   RANGE
   RELATIVE LABORATORY PERFORMANCE (RLP)
   RELATIVE STANDARD DEVIATION
   RELATIVE VARIANCE
   RESCALED SUM
   ROBUST POOLED RANGE
   ROBUST POOLED STANDARD DEVIATION
   ROOT MEAN SQUARE ERROR (RMS)
   SCALED MEDIAN ABSOLUTE DEVIATION (MAD)
   SIGNAL TO NOISE RATIO
   SN SCALE
   STANDARD DEVIATION
   SUM OF SQUARES
   SUM OF SQUARES FROM MEAN
   TRIMMED SD
   UNBIASED COEFFICIENT OF VARIATION
   VARIANCE
   WINSORIZED STANDARD DEVIATION
   WINSORIZED VARIANCE

Higher Moments:
   EXCESS KURTOSIS
   GALTON SKEWNESS
   KURTOSIS
   PEARSON TWO SKEWNESS
   SKEWNESS

Percentile Statistics:
   ___ DECILE
   EXTREME 
   INDEX EXTREME
   INDEX MAXIMUM
   INDEX MINIMUM
   LOWER HINGE
   LOWER QUARTILE
   MINIMUM (MIN)
   MAXIMUM (MAX)
   PERCENTILE
   QUANTILE
   QUANTILE STANDARD ERROR
   UPPER HINGE
   UPPER QUARTILE

Time Series Statistics:
   AUTOCORRELATION
   AUTOCOVARIANCE
   SIN AMPLITUDE
   SIN FREQUENCY

Quality Control Statistics:
   CC
   CNP
   CNPK
   CNPM
   CNPMK
   CP
   CPK
   CPL
   CPM
   CPMK
   CPU
   EXPECTED LOSS
   PERCENT DEFECTIVE
   TAGUCHI SN+
   TAGUCHI SN-
   TAGUCHI SN0
   TAGUCHI SN00

Statistical Tests:
   A BASIS NORMAL
   A BASIS LOGNORMAL
   A BASIS WEIBULL
   A BASIS NONPARAMETRIC
   B BASIS NORMAL
   B BASIS LOGNORMAL
   B BASIS WEIBULL
   B BASIS NONPARAMETRIC
   BINOMIAL PROPORTIONS
   CHI-SQUARE SD TEST
   CHI-SQUARE SD TEST CDF
   CHI-SQUARE SD TEST PVALUE
   CHI-SQUARE SD TEST LOWER TAIL PVALUE
   CHI-SQUARE SD TEST UPPER TAIL PVALUE
   CUMULATIVE SUM FORWARD TEST
   CUMULATIVE SUM FORWARD TEST PVALUE
   CUMULATIVE SUM BACKWARD TEST
   CUMULATIVE SUM BACKWARD TEST PVALUE
   DIXON TEST
   DIXON MAXIMUM TEST
   DIXON MINIMUM TEST
   EXTREME STUDENTIZED DEVIATE
   FREQUENCY TEST
   FREQUENCY TEST CDF
   FREQUENCY WITHIN A BLOCK TEST
   FREQUENCY WITHIN A BLOCK TEST CDF
   GRUBB 
   GRUBB TEST CDF
   GRUBB TEST DIRECTION
   GRUBB TEST INDEX
   JARQUE BERA
   JARQUE BERA CDF
   JARQUE BERA PVALUE
   LOWER COEFFICIENT OF DISPERSION CONFIDENCE LIMIT
   LOWER ONESIDED COEFFICIENT OF DISPERSION CONFIDENCE LIMIT
   LOWER CONFIDENCE LIMIT
   LOWER LOGNORMAL COEFFICIENT OF VARIATION CONFIDENCE LIMIT
   LOWER PREDICTION BOUND
   LOWER PREDICTION LIMIT
   LOWER STANDARD DEVIATION CONFIDENCE LIMIT
   LOWER STANDARD DEVIATION PREDICTION LIMIT
   LJUNG BOX TEST
   MCCOOL WEIBULL LOCATION TEST
   MCCOOL WEIBULL LOCATION TEST CDF
   MCCOOL WEIBULL LOCATION TEST CV50
   MCCOOL WEIBULL LOCATION TEST CV90
   MCCOOL WEIBULL LOCATION TEST CV95
   MCCOOL WEIBULL LOCATION TEST PVALUE
   MEAN SUCCESSIVE DIFFERENCE
   MEAN SUCCESSIVE DIFFERENCE NORMALIZED
   MEAN SUCCESSIVE DIFFERENCE CDF
   MEAN SUCCESSIVE DIFFERENCE PVALUE
   NORMAL TOLERANCE K FACTOR
   NORMAL TOLERANCE LOWER LIMIT
   NORMAL TOLERANCE UPPER LIMIT
   NORMAL TOLERANCE ONE SIDED K FACTOR
   NORMAL TOLERANCE ONE SIDED LOWER LIMIT
   NORMAL TOLERANCE ONE SIDED UPPER LIMIT
   ONE SAMPLE COEFFICIENT OF VARIATION TEST
   ONE SAMPLE COEFFICIENT OF VARIATION TEST CDF
   ONE SAMPLE COEFFICIENT OF VARIATION TEST PVALUE
   ONE SAMPLE COEFFICIENT OF VARIATION LOWER PVALUE
   ONE SAMPLE COEFFICIENT OF VARIATION UPPER PVALUE
   ONE SAMPLE SIGN TEST
   ONE SAMPLE SIGN TEST CDF
   ONE SAMPLE SIGN TEST PVALUE
   ONE SAMPLE SIGN TEST LOWER TAIL PVALUE
   ONE SAMPLE SIGN TEST UPPER TAIL PVALUE
   ONE SAMPLE T-TEST
   ONE SAMPLE T-TEST CDF
   ONE SAMPLE T-TEST PVALUE
   ONE SAMPLE T-TEST LOWER TAIL PVALUE
   ONE SAMPLE T-TEST UPPER TAIL PVALUE
   ONE SAMPLE WILCOXON SIGNED RANK TEST
   ONE SAMPLE WILCOXON SIGNED RANK TEST CDF
   ONE SAMPLE WILCOXON SIGNED RANK TEST PVALUE
   ONE SAMPLE WILCOXON SIGNED RANK TEST LOWER TAIL PVALUE
   ONE SAMPLE WILCOXON SIGNED RANK TEST UPPER TAIL PVALUE
   ONE-SIDED LOWER AGRESTI-COUL
   ONE-SIDED UPPER AGRESTI-COUL
   ONE-SIDED LOWER COEFFICIENT OF VARIATION CONFIDENCE LIMIT
   ONE-SIDED UPPER COEFFICIENT OF VARIATION CONFIDENCE LIMIT
   ONE-SIDED LOWER CONFIDENCE LIMIT
   ONE-SIDED UPPER CONFIDENCE LIMIT
   ONE-SIDED LOWER EXACT BINOMIAL
   ONE-SIDED UPPER EXACT BINOMIAL
   ONE-SIDED LOWER PREDICTION BOUND
   ONE-SIDED LOWER STANDARD DEVIATION CONFIDENCE LIMIT
   ONE-SIDED LOWER STANDARD DEVIATION PREDICTION LIMIT
   ONE-SIDED UPPER PREDICTION BOUND
   ONE-SIDED LOWER PREDICTION LIMIT
   ONE-SIDED UPPER PREDICTION LIMIT
   ONE-SIDED UPPER STANDARD DEVIATION CONFIDENCE LIMIT
   ONE-SIDED UPPER STANDARD DEVIATION PREDICTION LIMIT
   POISSON DISPERSION TEST
   POISSON DISPERSION TEST CDF
   POISSON DISPERSION TEST PVALUE
   TWO-SIDED LOWER AGRESTI-COUL
   TWO-SIDED UPPER AGRESTI-COUL
   TWO-SIDED LOWER EXACT BINOMIAL
   TWO-SIDED UPPER EXACT BINOMIAL
   UPPER COEFFICIENT OF DISPERSION CONFIDENCE LIMIT
   UPPER ONESIDED COEFFICIENT OF DISPERSION CONFIDENCE LIMIT
   UPPER COEFFICIENT OF VARIATION CONFIDENCE LIMIT
   UPPER CONFIDENCE LIMIT
   UPPER LOGNORMAL COEFFICIENT OF VARIATION CONFIDENCE LIMIT
   UPPER PREDICTION BOUND
   UPPER PREDICTION LIMIT
   UPPER STANDARD DEVIATION CONFIDENCE LIMIT
   UPPER STANDARD DEVIATION PREDICTION LIMIT
   WILK SHAPIRO TEST
   WILK SHAPIRO TEST PVALUE

Spatial Data:
   RELATIVE DISPERSION INDEX
   UNIFORM CHISQUARE
   VARIATIONAL DISTANCE

Distribution:
   BOX COX NORMALITY PPCC
   BOX COX NORMALITY LAMBDA
   KAPPENMAN R
   KAPPENMAN R CUTOFF

   ANGLIT PPCC
   ANGLIT PPCC LOCATION
   ANGLIT PPCC SCALE
   ARCSINE PPCC
   ARCSINE PPCC LOCATION
   ARCSINE PPCC SCALE
   CAUCHY PPCC
   CAUCHY PPCC LOCATION
   CAUCHY PPCC SCALE
   COSINE PPCC
   COSINE PPCC LOCATION
   COSINE PPCC SCALE
   DOUBLE EXPONENTIAL PPCC
   DOUBLE EXPONENTIAL PPCC LOCATION
   DOUBLE EXPONENTIAL PPCC SCALE
   EXPONENTIAL PPCC
   EXPONENTIAL PPCC LOCATION
   EXPONENTIAL PPCC SCALE
   FATIGUE LIFE PPCC STATISTIC
   FATIGUE LIFE PPCC LOCATION
   FATIGUE LIFE PPCC SCALE
   FATIGUE LIFE PPCC SHAPE
   GAMMA PPCC STATISTIC
   GAMMA PPCC LOCATION
   GAMMA PPCC SCALE
   GAMMA PPCC SHAPE
   GENERALIZED PARETO PPCC STATISTIC
   GENERALIZED PARETO PPCC LOCATION
   GENERALIZED PARETO PPCC SCALE
   GENERALIZED PARETO PPCC SHAPE
   GH PPCC STATISTIC
   GH PPCC LOCATION
   GH PPCC SCALE
   GH PPCC SHAPE ONE
   GH PPCC SHAPE TWO
   HALF CAUCHY PPCC
   HALF CAUCHY PPCC LOCATION
   HALF CAUCHY PPCC SCALE
   HALF NORMAL PPCC
   HALF NORMAL PPCC LOCATION
   HALF NORMAL PPCC SCALE
   HYPERBOLIC SECANT PPCC
   HYPERBOLIC SECANT PPCC LOCATION
   HYPERBOLIC SECANT PPCC SCALE
   LOGISTIC PPCC
   LOGISTIC PPCC LOCATION
   LOGISTIC PPCC SCALE
   LOGNORMAL PPCC STATISTIC
   LOGNORMAL PPCC LOCATION
   LOGNORMAL PPCC SCALE
   LOGNORMAL PPCC SHAPE
   MAXWELL PPCC
   MAXWELL PPCC LOCATION
   MAXWELL PPCC SCALE
   MAXIMUM GUMBEL PPCC
   MAXIMUM GUMBEL PPCC LOCATION
   MAXIMUM GUMBEL PPCC SCALE
   MINIMUM GUMBEL PPCC
   MINIMUM GUMBEL PPCC LOCATION
   MINIMUM GUMBEL PPCC SCALE
   NORMAL PPCC
   NORMAL PPCC LOCATION
   NORMAL PPCC SCALE
   RAYLEIGH PPCC
   RAYLEIGH PPCC LOCATION
   RAYLEIGH PPCC SCALE
   SEMICIRCULAR PPCC
   SEMICIRCULAR PPCC LOCATION
   SEMICIRCULAR PPCC SCALE
   SINE PPCC
   SINE PPCC LOCATION
   SINE PPCC SCALE
   SLASH PPCC
   SLASH PPCC LOCATION
   SLASH PPCC SCALE
   TUKEY LAMBDA PPCC STATISTIC
   TUKEY LAMBDA PPCC LOCATION
   TUKEY LAMBDA PPCC SCALE
   TUKEY LAMBDA PPCC SHAPE
   WALD PPCC STATISTIC
   WALD PPCC LOCATION
   WALD PPCC SCALE
   WALD PPCC SHAPE
   WEIBULL PPCC STATISTIC
   WEIBULL PPCC LOCATION
   WEIBULL PPCC SCALE
   WEIBULL PPCC SHAPE
   UNIFORM PPCC
   UNIFORM PPCC LOCATION
   UNIFORM PPCC SCALE
   2PARAMETER WEIBULL PPCC STATISTIC
   2PARAMETER WEIBULL PPCC SCALE
   2PARAMETER WEIBULL PPCC SHAPE

   DOUBLE EXPONENTIAL ANDERSON DARLING
   DOUBLE EXPONENTIAL ANDERSON DARLING LOCATION
   DOUBLE EXPONENTIAL ANDERSON DARLING SCALE
   EXPONENTIAL        ANDERSON DARLING
   EXPONENTIAL        ANDERSON DARLING LOCATION
   EXPONENTIAL        ANDERSON DARLING SCALE
   GAMMA (2-PAR)      ANDERSON DARLING
   GAMMA (2-PAR)      ANDERSON DARLING LOCATION
   GAMMA (2-PAR)      ANDERSON DARLING SCALE
   GUMBEL             ANDERSON DARLING
   GUMBEL             ANDERSON DARLING LOCATION
   GUMBEL             ANDERSON DARLING SCALE
   LOGISTIC           ANDERSON DARLING
   LOGISTIC           ANDERSON DARLING LOCATION
   LOGISTIC           ANDERSON DARLING SCALE
   LOGNORMAL (2-PAR)  ANDERSON DARLING
   LOGNORMAL (2-PAR)  ANDERSON DARLING LOCATION
   LOGNORMAL (2-PAR)  ANDERSON DARLING SCALE
   MAXWELL            ANDERSON DARLING
   MAXWELL            ANDERSON DARLING LOCATION
   MAXWELL            ANDERSON DARLING SCALE
   NORMAL             ANDERSON DARLING
   NORMAL             ANDERSON DARLING LOCATION
   NORMAL             ANDERSON DARLING SCALE
   RAYLEIGH           ANDERSON DARLING
   RAYLEIGH           ANDERSON DARLING LOCATION
   RAYLEIGH           ANDERSON DARLING SCALE
   UNIFORM            ANDERSON DARLING
   UNIFORM            ANDERSON DARLING LOCATION
   UNIFORM            ANDERSON DARLING SCALE
   WEIBULL (2-PAR)    ANDERSON DARLING
   WEIBULL (2-PAR)    ANDERSON DARLING LOCATION
   WEIBULL (2-PAR)    ANDERSON DARLING SCALE

Miscellaneous:
   COMMON DIGITS
   INTEGRAL
   NUMBER OF COMMON DIGITS
   PRODUCT
   RAW SHANNON DIVERSITY INDEX
   RAW SIMPSON DIVERSITY INDEX
   SHANNON DIVERSITY INDEX
   SIMPSON DIVERSITY INDEX
   SIZE (or NUMBER or COUNT)
   SUM
   UNIQUE (NUMBER OF DISTINCT VALUES)
   VALUE COUNT

Following are used by  LET ... = CROSS TABULATE ...
   GROUP ONE
   GROUP TWO
   GROUP THREE
   GROUP FOUR
   GROUP FIVE
   GROUP SIX

Case 2: Two Response Variables

Group Statistics:
   COMMON COEFFICIENT OF VARIATION
   COMMON BIAS CORRECTED COEFFICIENT OF VARIATION
   LOWER COMMON COEFFICIENT OF VARIATION CONFIDENCE LIMIT
   UPPER COMMON COEFFICIENT OF VARIATION CONFIDENCE LIMIT

Weighted Statistics:
   WEIGHTED MEAN
   WEIGHTED ORDER STATISTIC MEAN
   WEIGHTED STANDARD DEVIATION
   WEIGHTED SKEWNESS
   WEIGHTED SUM
   WEIGHTED SUM OF ABSOLUTE VALUES
   WEIGHTED SUM OF SQUARES
   WEIGHTED TRIMMED MEAN
   WEIGHTED VARIANCE

Co-Relation:
   ANGULAR COSINE DISTANCE
   ANGULAR COSINE SIMILARITY
   BINARY ASYMMETRIC DICE MATCH DISSIMILARITY
   BINARY ASYMMETRIC DICE MATCH SIMILARITY
   BINARY ASYMMETRIC SOKAL MATCH DISSIMILARITY
   BINARY ASYMMETRIC SOKAL MATCH SIMILARITY
   BINARY JACCARD DISSIMILARITY
   BINARY JACCARD SIMILARITY
   BINARY MATCH DISSIMILARITY
   BINARY MATCH SIMILARITY
   BINARY ROGERS MATCH DISSIMILARITY
   BINARY ROGERS MATCH SIMILARITY
   BINARY SOKAL MATCH DISSIMILARITY
   BINARY SOKAL MATCH SIMILARITY
   BIWEIGHT MIDCORRELATION
   BIWEIGHT MIDCOVARIANCE
   COMOVEMENT
   CORRELATION
   CORRELATION ABSOLUTE VALUE
   CORRELATION CDF
   CORRELATION PVALUE
   COSINE DISTANCE
   COSINE SIMILARITY
   COVARIANCE
   DOT PRODUCT
   EUCLIDEAN DISTANCE
   EUCLIDEAN LENGTH
   GENERALZIED JACCARD COEFFICIENT
   GENERALZIED JACCARD DISTANCE
   KENDELLS TAU
   KENDALLS TAU ABSOLUTE VALUE
   KENDALLS TAU CDF
   KENDELLS TAU DISSIMILARITY
   KENDALLS TAU PVALUE
   KENDALLS TAU LOWER TAILED PVALUE
   KENDALLS TAU UPPER TAILED PVALUE
   MANHATTAN DISTANCE
   MINKOWSKI DISTANCE
   PEARSON DISSIMILARITY
   PERCENTAGE BEND CORRELATION
   RANK COMOVEMENT
   RANK CORRELATION
   RANK CORRELATION ABSOLUTE VALUE
   RANK CORRELATION CDF
   RANK CORRELATION PVALUE
   RANK CORRELATION LOWER TAILED PVALUE
   RANK CORRELATION UPPER TAILED PVALUE
   RANK COVARIANCE
   SPEARMAN DISSIMILARITY
   WINSORIZED CORRELATION
   WINSORIZED COVARIANCE

Regression/Fitting:
   LINEAR CORRELATION
   LINEAR DISTINCT X
   LINEAR INTERCEPT
   LINEAR INTERCEPT SD
   LINEAR RESSD
   LINEAR SLOPE
   LINEAR SLOPE SD
   REPEATABILITY SD
   REPRODUCABILITY SD

Categorical Data:
   CRAMER CONTINGENCY COEFFICIENT
   FALSE NEGATIVE
   FALSE POSITIVE
   LOG ODDS RATIO (BIAS CORRECTED LOG ODDS RATIO)
   NEGATIVE PREDICTIVE VALUE
   ODDS RATIO (BIAS CORRECTED ODDS RATIO)
   PEARSON CONTINGENCY COEFFICIENT
   PRECENT AGREE
   PRECENT DISAGREE
   POSITIVE PREDICTIVE VALUE
   RATIO (= SUM1/SUM2)
   RELATIVE RISK
   STANDARD ERROR ODDS RATIO (STANDARD ERROR OF THE
            BIAS CORRECTED ODDS RATIO)
   STANDARD ERROR LOG ODDS RATIO (STANDARD ERROR OF
            THE BIAS CORRECTED LOG ODDS RATIO)
   TRUE NEGATIVE
   TRUE POSITIVE
   TEST SENSITIVITY
   TEST SPECIFICITY

Difference of Location:
   DIFFERENCE OF BIWEIGHT LOCATION
   DIFFERENCE OF GEOMETRIC MEANS
   DIFFERENCE OF <H10/H12/H15/H17/H20> LOCATION
   DIFFERENCE OF HARMONIC MEANS
   DIFFERENCE OF HODGES-LEHMAN
   DIFFERENCE OF LP LOCATION
   DIFFERENCE OF MEANS
   DIFFERENCE OF MEDIANS
   DIFFERENCE OF MIDMEANS
   DIFFERENCE OF TRIMMED MEANS
   DIFFERENCE OF WINSORIZED MEANS

Difference of Scale:
   DIFFERENCE OF AAD
   DIFFERENCE OF AVERAGE ABSOLUTE DEVIATIONS FROM MEDIAN
   DIFFERENCE OF BIWEIGHT MIDVARIANCE
   DIFFERENCE OF BIWEIGHT SCALE
   DIFFERENCE OF COEFFICIENT OF VARIATION
   DIFFERENCE OF EXTREMES
   DIFFERENCE OF GEOMETRIC SD
   DIFFERENCE OF <H10/H12/H15/H17/H20> SCALE
   DIFFERENCE OF INTERQUARTILE RANGE
   DIFFERENCE OF KURTOSIS
   DIFFERENCE OF EXCESS KURTOSIS
   DIFFERENCE OF MAD
   DIFFERENCE OF MAXIMUM
   DIFFERENCE OF MIDRANGE
   DIFFERENCE OF MINIMUM
   DIFFERENCE OF NORMALIZED INTERQUARTILE RANGE
   DIFFERENCE OF PERCENTAGE BEND
   DIFFERENCE OF PRECISION
   DIFFERENCE OF QN
   DIFFERENCE OF QUANTILE
   DIFFERENCE OF RANGE
   DIFFERENCE OF RELATIVE STANDARD DEVIATION
   DIFFERENCE OF RELATIVE VARIANCE
   DIFFERENCE OF RESCALD SUM
   DIFFERENCE OF ROOT MEAN SQUARE ERROR
   DIFFERENCE OF SCALED MAD
   DIFFERENCE OF SD OF LP LOCATION
   DIFFERENCE OF SD OF MEAN
   DIFFERENCE OF SKEWNESS
   DIFFERENCE OF GALTON SKEWNESS
   DIFFERENCE OF PEARSON TWO SKEWNESS
   DIFFERENCE OF SN
   DIFFERENCE OF SNR
   DIFFERENCE OF STANDARD DEVIATIONS
   DIFFERENCE OF SUM OF SQUARES
   DIFFERENCE OF SUM OF SQUARES FROM MEAN
   DIFFERENCE OF VARIANCES
   DIFFERENCE OF VARIANCE OF LP LOCATION
   DIFFERENCE OF VARIANCE OF THE MEAN
   DIFFERENCE OF WINSORIZED SD
   DIFFERENCE OF WINSORIZED VARIANCE

Statistical Tests
   ANDERSON DARLING K SAMPLE TEST
   ANDERSON DARLING K SAMPLE TEST CRITICAL VALUE
   BINOMIAL RATIO
   BIVARIATE CRAMER VON MISES TEST
   BIVARIATE CRAMER VON MISES 95 CRITICAL VALUE
   BIVARIATE CRAMER VON MISES 05 CRITICAL VALUE
   COCHRAN VARIANCE OUTLIER TEST
   COCHRAN VARIANCE OUTLIER CV95
   COCHRAN VARIANCE OUTLIER CV99
   COCHRAN VARIANCE OUTLIER PVALUE
   COCHRAN VARIANCE OUTLIER CDF
   COCHRAN MINIMUM VARIANCE OUTLIER TEST
   COCHRAN MINIMUM VARIANCE OUTLIER CV95
   COCHRAN MINIMUM VARIANCE OUTLIER CV99
   COCHRAN MINIMUM VARIANCE OUTLIER PVALUE
   COCHRAN MINIMUM VARIANCE OUTLIER CDF
   DIFFERENCE OF BINOMIAL PROPORTIONS
   DIFFERENCE OF BINOMIAL PROPORTIONS LOWER CONFIDENCE LIMIT
   DIFFERENCE OF BINOMIAL PROPORTIONS UPPER CONFIDENCE LIMIT
   F TEST
   F TEST CDF
   F TEST PVALUE
   FISHER TWO SAMPLE RANDOMIZATION TEST
   FISHER TWO SAMPLE RANDOMIZATION TEST PVALUE
   FISHER TWO SAMPLE RANDOMIZATION LOWER TAIL PVALUE
   GROUPED POISSON DISPERSION TEST
   GROUPED POISSON DISPERSION TEST CDF
   GROUPED POISSON DISPERSION TEST PVALUE
   KLOTZ TEST
   KLOTZ TEST CDF
   KLOTZ TEST PVALUE
   KLOTZ TEST LOWER TAILED PVALUE
   KLOTZ TEST UPPER TAILED PVALUE
   KRUSKALL WALLIS TEST
   KRUSKALL WALLIS TEST CDF
   KRUSKALL WALLIS TEST PVALUE
   MANN WHITNEY RANK SUM TEST
   MANN WHITNEY RANK SUM TEST CDF
   MANN WHITNEY RANK SUM TEST PVALUE
   MANN WHITNEY RANK SUM LOWER TAIL PVALUE
   MANN WHITNEY RANK SUM UPPER TAIL PVALUE
   MANN WHITNEY U STATISTIC
   MEAN NEAREST NEIGHBOR DISTANCE CDF
   MEAN NEAREST NEIGHBOR DISTANCE PVALUE
   MEAN NEAREST NEIGHBOR DISTANCE TEST
   MEDIAN TEST
   MEDIAN TEST CDF
   MEDIAN TEST PVALUE
   ONE SAMPLE COEFFICIENT OF VARIATION TEST
   ONE SAMPLE COEFFICIENT OF VARIATION TEST CDF
   ONE SAMPLE COEFFICIENT OF VARIATION TEST PVALUE
   ONE SAMPLE COEFFICIENT OF VARIATION LOWER PVALUE
   ONE SAMPLE COEFFICIENT OF VARIATION UPPER PVALUE
   POLLARD ONE CDF
   POLLARD ONE PVALUE
   POLLARD ONE TEST
   POLLARD TWO CDF
   POLLARD TWO PVALUE
   POLLARD TWO TEST
   POLLARD THREE CDF
   POLLARD THREE PVALUE
   POLLARD THREE TEST
   POLLARD FOUR CDF
   POLLARD FOUR PVALUE
   POLLARD FOUR TEST
   POLLARD FIVE CDF
   POLLARD FIVE PVALUE
   POLLARD FIVE TEST
   SQUARED RANK TEST
   SQUARED RANK TEST CDF
   SQUARED RANK TEST PVALUE
   SQUARED RANK TEST LOWER TAILED PVALUE
   SQUARED RANK TEST UPPER TAILED PVALUE
   SUMMARY COEFFICIENT OF VARIATION
   SUMMARY LOWER SD CONFIDENCE LIMITS
   SUMMARY LOWER SD PREDICTION LIMITS
   SUMMARY ONE SIDED LOWER SD CONFIDENCE LIMITS
   SUMMARY ONE SIDED LOWER SD PREDICTION LIMITS
   SUMMARY ONE SIDED UPPER SD CONFIDENCE LIMITS
   SUMMARY ONE SIDED UPPER SD PREDICTION LIMITS
   SUMMARY UPPER SD CONFIDENCE LIMITS
   SUMMARY UPPER SD PREDICTION LIMITS
   TWO SAMPLE CHI-SQUARE TEST
   TWO SAMPLE CHI-SQUARE TEST CDF
   TWO SAMPLE CHI-SQUARE TEST PVALUE
   TWO SAMPLE COEFFICIENT OF VARIATION TEST
   TWO SAMPLE COEFFICIENT OF VARIATION TEST CDF
   TWO SAMPLE COEFFICIENT OF VARIATION TEST PVALUE
   TWO SAMPLE COEFFICIENT OF VARIATION LOWER PVALUE
   TWO SAMPLE COEFFICIENT OF VARIATION UPPER PVALUE
   TWO SAMPLE KOLMOGOROV SMIRNOV TEST
   TWO SAMPLE KOLMOGOROV SMIRNOV CRITICAL VALUE
   TWO SAMPLE PAIRED T-TEST
   TWO SAMPLE PAIRED T-TEST CDF
   TWO SAMPLE PAIRED T-TEST PVALUE
   TWO SAMPLE PAIRED T-TEST LOWER TAIL PVALUE
   TWO SAMPLE PAIRED T-TEST UPPER TAIL PVALUE
   TWO SAMPLE SIGN TEST
   TWO SAMPLE SIGN TEST CDF
   TWO SAMPLE SIGN TEST PVALUE
   TWO SAMPLE SIGN TEST LOWER TAIL PVALUE
   TWO SAMPLE SIGN TEST UPPER TAIL PVALUE
   TWO SAMPLE T-TEST
   TWO SAMPLE T-TEST CDF
   TWO SAMPLE T-TEST PVALUE
   TWO SAMPLE T-TEST LOWER TAIL PVALUE
   TWO SAMPLE T-TEST UPPER TAIL PVALUE
   TWO SAMPLE WILCOXON SIGNED RANK TEST
   TWO SAMPLE WILCOXON SIGNED RANK TEST CDF
   TWO SAMPLE WILCOXON SIGNED RANK TEST PVALUE
   TWO SAMPLE WILCOXON SIGNED RANK TEST LOWER TAIL PVALUE
   TWO SAMPLE WILCOXON SIGNED RANK TEST UPPER TAIL PVALUE

Distribution
   COMMON WEIBULL SHAPE TEST
   COMMON WEIBULL SHAPE TEST CDF
   COMMON WEIBULL SHAPE TEST PVALUE
   COMMON WEIBULL SHAPE TEST CV90
   COMMON WEIBULL SHAPE TEST CV95
   COMMON WEIBULL SHAPE TEST CV99

Consensus Means
   DERSIMONIAN LAIRD
   DERSIMONIAN LAIRD STANDARD ERROR
   DERSIMONIAN LAIRD HHD
   DERSIMONIAN LAIRD MINMAX
   MANDEL PAULE
   MANDEL PAULE STANDARD ERROR
   MODIFIED MANDEL PAULE
   MODIFIED MANDEL PAULE STANDARD ERROR
   VANGEL RUKHIN
   VANGEL RUKHIN STANDARD ERROR
   GENERALIZED CONFIDENCE INTERVAL
   GENERALIZED CONFIDENCE INTERVAL STANDARD ERROR
   BOB
   BOB STANDARD ERROR
   BCP
   BCP STANDARD ERROR
   MEAN OF MEANS
   MEAN OF MEANS STANDARD ERROR
   FAIRWEATHER
   FAIRWEATHER STANDARD ERROR
   SCHILLER-EBERHARDT
   SCHILLER-EBERHARDT STANDARD ERROR
   GRAYBILL DEAL
   GRAYBILL DEAL SINHA STANDARD ERROR
   GRAYBILL DEAL NAIVE STANDARD ERROR
   GRAYBILL DEAL ZHANG ONE STANDARD ERROR
   GRAYBILL DEAL ZHANG TWO STANDARD ERROR

Miscellaneous:
   DIFFERENCE OF BINOMIAL PROPORTIONS
   DIFFERENCE OF COUNTS
   DIFFERENCE OF INTEGRALS
   DIFFERENCE OF PRODUCTS
   DIFFERENCE OF SUMS
   INDEX FIRST MATCH
   INDEX LAST  MATCH
   INDEX FIRST NOT MATCH
   INDEX LAST  NOT MATCH
   PERCENTAGE DIFFERENCE OF THE MEAN

Case 3: Three Response Variables

Fit/Correlation:
   EQUAL SLOPES
   EQUAL SLOPES CDF
   EQUAL SLOPES CRITICAL VALUE
   PARTIAL CORRELATION
   PARTIAL CORRELATION ABSOLUTE VALUE
   PARTIAL CORRELATION CDF
   PARTIAL CORRELATION PVALUE
   PARTIAL KENDALL TAU CORRELATION
   PARTIAL KENDALL TAU CORRELATION ABSOLUTE VALUE
   PARTIAL RANK CORRELATION
   PARTIAL RANK CORRELATION ABSOLUTE VALUE

Statistical Tests
   FRIEDMAN TEST
   FRIEDMAN TEST CDF
   FRIEDMAN TEST PVALUE
   QUADE TEST
   QUADE TEST CDF
   QUADE TEST PVALUE
   SUMMARY LOWER COEFFICIENT OF VARIATION CONFIDENCE LIMITS
   SUMMARY LOWER CONFIDENCE LIMITS
   SUMMARY LOWER PREDICTION BOUNDS
   SUMMARY LOWER PREDICTION LIMITS
   SUMMARY NORMAL TOLERANCE K FACTOR
   SUMMARY NORMAL TOLERANCE LOWER LIMIT
   SUMMARY NORMAL TOLERANCE UPPER LIMIT
   SUMMARY NORMAL TOLERANCE ONE SIDED K FACTOR
   SUMMARY NORMAL TOLERANCE ONE SIDED LOWER LIMIT
   SUMMARY NORMAL TOLERANCE ONE SIDED UPPER LIMIT
   SUMMARY ONE SAMPLE COEFFICIENT OF VARIATION CDF
   SUMMARY ONE SAMPLE COEFFICIENT OF VARIATION PVALUE
   SUMMARY ONE SAMPLE COEFFICIENT OF VARIATION TEST
   SUMMARY ONE SIDED LOWER CONFIDENCE LIMITS
   SUMMARY ONE SIDED LOWER PREDICTION BOUNDS
   SUMMARY ONE SIDED LOWER PREDICTION LIMITS
   SUMMARY ONE SIDED UPPER CONFIDENCE LIMITS
   SUMMARY ONE SIDED UPPER PREDICTION BOUNDS
   SUMMARY ONE SIDED UPPER PREDICTION LIMITS
   SUMMARY UPPER COEFFICIENT OF VARIATION CONFIDENCE LIMITS
   SUMMARY UPPER CONFIDENCE LIMITS
   SUMMARY UPPER PREDICTION BOUNDS
   SUMMARY UPPER PREDICTION LIMITS

Consensus Means
   SUMMARY DERSIMONIAN LAIRD
   SUMMARY DERSIMONIAN LAIRD STANDARD ERROR
   SUMMARY DERSIMONIAN LAIRD HHD
   SUMMARY DERSIMONIAN LAIRD MINMAX
   SUMMARY MANDEL PAULE
   SUMMARY MANDEL PAULE STANDARD ERROR
   SUMMARY MODIFIED MANDEL PAULE
   SUMMARY MODIFIED MANDEL PAULE STANDARD ERROR
   SUMMARY VANGEL RUKHIN
   SUMMARY VANGEL RUKHIN STANDARD ERROR
   SUMMARY GENERALIZED CONFIDENCE INTERVAL
   SUMMARY GENERALIZED CONFIDENCE INTERVAL STANDARD ERROR
   SUMMARY BOB
   SUMMARY BOB STANDARD ERROR
   SUMMARY BCP
   SUMMARY BCP STANDARD ERROR
   SUMMARY MEAN OF MEANS
   SUMMARY MEAN OF MEANS STANDARD ERROR
   SUMMARY FAIRWEATHER
   SUMMARY FAIRWEATHER STANDARD ERROR
   SUMMARY SCHILLER-EBERHARDT
   SUMMARY SCHILLER-EBERHARDT STANDARD ERROR
   SUMMARY GRAYBILL DEAL
   SUMMARY GRAYBILL DEAL SINHA STANDARD ERROR
   SUMMARY GRAYBILL DEAL NAIVE STANDARD ERROR
   SUMMARY GRAYBILL DEAL ZHANG ONE STANDARD ERROR
   SUMMARY GRAYBILL DEAL ZHANG TWO STANDARD ERROR

----------------------------------------------------------































































-------------------------  *MATH OPERATIONS*  ------------
 
MATHEMATICS OPERATIONS
Mathematics Operations
 
The execution of a variety of mathematical operations is done via
subcommands under the LET command, as in
 
   LET A = SUM X
   LET B = INTEGRAL F WRT X FOR X = 0 TO 10
   LET C = SORT X
 
The math operations are of 5 types--
 
   1) the operation is applied to a variable and the result is a
      parameter.
   2) the operation is applied to a variable and the result is a
      variable.
   3) the operation is applied to a function.
   4) the operation generates sequences or patterns.
   5) the operation is applied to a matrix (these are under MATRIX
      OPERATIONS).
 
1. The available mathematical subcommands which operate on a variable
   and result in a parameter are as follows--
 
      SUM                    Compute the sum of elements in a variable
      PRODUCT                Compute the product of elements in a
                             variable
      INTEGRAL               Compute the integral of elements in a
                             variable
 
2. The available mathematical subcommands which operate on a variable
   and result in a variable are as follows--
 
      CUMULATIVE SUM         Compute the cumulative sums of elements in
                             a variable
      CUMULATIVE PRODUCT     Compute the cumulative products of
                             elements in a variable
      CUMULATIVE INTEGRAL    Compute the cumulative integrals of
                             elements in a variable
 
      SEQUENTIAL DIFFERENCE  Compute the sequential differences of
                             elements in a variable
 
      SORT                   Sort the elements in a variable
      SORTC                  Sort one variable and carry another
      RANK                   Rank the elements in a variable
      CODE                   Code the elements in a variable
      CODE2                  Binary code the elements in a variable
      CODE4                  Quartile code the elements in a variable
      CODEH                  Hinge code the elements in a variable
      CODE8                  Octal code the elements in a variable
      CODE<N>                Decile code the elements in a variable
      COCODE                 Code one variable by another variable
      COCOPY                 Code one variable by another variable
      DISTINCT               Extract the distinct elements from a
                             variable
      FREQUENCY              Compute the frequencies of distinct values
 
      CONVOLUTION            Compute the convolution of the elements in
                             2 variables
      DECONVOLUTION          Compute the deconvolution of the elements
                             in 2 variables
 
      RUNGE-KUTTA            Solve an ordinary first or second order
                             differential equation
 
      INTERPOLATION          Perform a cubic interpolation
      LINEAR INTERPOLATION   Perform a linear interpolation
      BILINEAR INTERPOLATION Perform bivariate linear interpolation
      BIVARIATE INTERPOLAT   Perform bivariate interpolation starting
                             from a regular grid
      2D INTERPOLATION       Perform bivariate interpolation from
                             irregular points to a grid
 
      FOURIER TRANSFORM      Compute the Fourier transform
      INVERSE FOURIER TRANS  Compute the inverse Fourier transform
      FFT                    Compute the fast Fourier transform
      INVERSE FFT            Compute the inverse fast Fourier transform
      COSINE TRANSFORM       Compute the cosine transform
      SINE TRANSFORM         Compute the sine transform
 
      COMPLEX ADDITION       Perform a complex addition
      COMPLEX CONJUGATES     Calculate a complex conjugates
      COMPLEX DIVISION       Perform a complex division
      COMPLEX EXPONENTIATION Perform a complex exponentiation
      COMPLEX MULTIPLICATION Perform a complex multiplication
      COMPLEX ROOTS          Compute the complex roots
      COMPLEX SQUARE ROOTS   Compute the complex square roots
      COMPLEX SUBTRACTION    Perform a complex subtraction
 
      POLYNOMIAL ADDITION    Perform a polynomial addition
      POLYNOMIAL DIVISION    Perform a polynomial division
      POLYNOMIAL EVALUATION  Perform a polynomial evaluation
      POLYNOMIAL MULT        Perform a polynomial multiplication
      POLYNOMIAL SQUARE      Perform a polynomial square
      POLYNOMIAL SUBTRACTION Perform a polynomial subtraction
 
      VECTOR ADDITION        Perform a vector addition
      VECTOR ANGLE           Compute the vector angle
      VECTOR DISTANCE        Compute the vector distance
      VECTOR DOT PRODUCT     Compute the vector dot product
      VECTOR LENGTH          Compute the vector length
      VECTOR SUBTRACTION     Perform a vector subtraction
 
      SET CARDINALITY        Compute the set cardinality
      SET CARTESIAN PRODUCT  Perform a set cartesian product
      SET COMPLEMENT         Perform a set complement
      SET DISTINCT           Extract the distinct elements of a set
      SET INTERSECTION       Perform a set intersection
      SET UNION              Perform a set union
 
      LOGICAL AND            Perform a logical and
      LOGICAL IFF            Perform a logical iff
      LOGICAL IFTHEN         Perform a logical ifthen
      LOGICAL NAND           Perform a logical nand
      LOGICAL NOR            Perform a logical nor
      LOGICAL NOT            Perform a logical not
      LOGICAL OR             Perform a logical or
      LOGICAL XOR            Perform a logical xor
 
 
3. The available mathematical subcommands which operate on a function
   are as follows--
 
      ROOTS                  Compute the real roots of a function
      DERIVATIVE             Compute the symbolic derivative of a
                             function
      INTEGRAL               Compute the definite integral of a
                             function
      OPTIMIZE               Perform unconstrained optimization of a
                             univariate function
 
 
4. The available mathematical subcommands which generate sequences and
   patterns--
 
      SEQUENCE               Generate a sequence within a variable
      PATTERN                Generate a patterned sequence within a
                             variable
      PRIME NUMBERS          Generate prime numbers
      FIBONNACCI NUMBERS     Generate Fibonnacci numbers
      DATA                   Place numbers in a variable
      CANTOR NUMBERS         Generate Cantor numbers
      JULIA                  Generate Julia numbers
 
----------------------------------------------------------
 
 
 
 
 
 
 
 
 
 
 
 
 
 
 
 
 
 
 
 
 
 
 
 
 
 
 
 
 
 
 
 
 
 
 
 
 
 
 
 
 
 
 
 
 
 
 
 
 
 
 
 
-------------------------  *MATRIX OPERATIONS*  ------------
 
MATRIX OPERATIONS
Matrix Operations
 
The execution of a variety of matrix operations is done via
subcommands under the LET command, as in
 
   LET X = MATRIX SOLVE M B
   LET M = DIAGONAL MATIX V
   LET MINV = MATRIX INVERSE M
 
The following is a list of the matrix commands.
 
 
      CHOLESKY DECOMP        Perform a Cholesky decomposition
      CORRELATION MATRIX     Compute the correlation matrix of a matrix
      DIAGONAL MATRIX        Generate a diagonal matrix from a vector
      MATRIX ADDITION        Perform a matrix addition
      MATRIX ADJOINT         Compute the adjoint of a matrix
      MATRIX AUGMENT         Add columns to a current matrix
      MATRIX COFACTOR        Compute the matrix cofactors
      MATRIX DIAGONAL        Extract the diagonal elements of a matrix
      MATRIX DEFINITION      Set a matrix definition
      MATRIX DETERMINANT     Compute the matrix determinant
      MATRIX EIGENVALUES     Compute the matrix eigenvalues
      MATRIX EIGENVECTORS    Compute the matrix eigenvectors
      MATRIX ELEMENT         Extract a specific element of a matrix
      MATRIX EUCLIDEAN NORM  Compute the matrix euclidean norm
      MATRIX INVERSE         Compute the matrix inverse
      MATRIX ITERATIVE SOLU  Solve a linear system of equations and
                             apply iterative refinement
      MATRIX MINOR           Compute the matrix minor
      MATRIX MULTIPLICATION  Perform a matrix multiplication
      MATRIX NUMB OF COLUMNS Compute the matrix number of columns
      MATRIX NUMBER OF ROWS  Compute the matrix number of rows
      MATRIX RANK            Compute the rank of a matrix
      MATRIX REPLACE ELEMENT Replace a specific element of a matrix
      MATRIX REPLACE ROW     Replace a row of a matrix
      MATRIX ROW             Extract a row of a matrix
      MATRIX SIMP SOLUTION   Compute the matrix simplex solution
      MATRIX SOLUTION        Solve a system of linear equations
      MATRIX SPECTRAL NORM   Compute the matrix spectral norm
      MATRIX SPECTRAL RADIUS Compute the matrix spectral radius
      MATRIX SUBMATRIX       Define the matrix submatrix
      MATRIX SUBTRACTION     Perform a matrix subtraction
      MATRIX TRACE           Compute the matrix trace
      MATRIX TRANSPOSE       Compute the matrix transpose
      PRINCIPLE COMPONENTS   Generate a matrix of principle components
      PRIN COMP EIGENVECTORS Generate a matrix of principle components
                             eigenvectors
      PRIN COMP EIGENVALUES  Generate a matrix of principle components
                             eigenvalues
      ... PRINCIPLE COMP     Generate a specific principle component
      ... PRIN COMP EIGENVEC Generate a specific principle component
                             eigenvector
      ... PRIN COMP EIGENVAL Generate a specific principle component
                             eigenvalue
      SINGULAR VALUES        Compute the singular values of a matrix
      SINGULAR VALUE DECOMP  Compute the singular value decomposition
                             of a matrix
      SINGULAR VALUE FACTOR  Compute the singular value factorization
                             of a matrix
      TRIANGULAR INVERSE     Compute the inverse of a triangular matrix
      TRIANGULAR SOLVE       Solve a triangular system of equations
      TRIDIAGONAL SOLVE      Solve a tridiagonal system of equations
      VARIANCE-COVA MATRIX   Compute the variance-covariance matrix of
                             a matrix
 
----------------------------------------------------------
 
 
 
 
 
 
 
 
 
 
 
 
 
 
 
 
 
 
 
 
 
 
 
 
 
 
 
 

 
-------------------------  *RANDOM NUMBERS*  -------------
 
RANDOM NUMBERS
Random Numbers
 
The generation of random numbers is done via subcommands under the LET
command, as in
 
   LET X = UNIFORM RANDOM NUMBERS FOR I = 1 1 25
 
   LET Y = NORMAL RUNDOM NUMBERS FOR I = 1 1 100
 
   LET GAMMA = 2.5
   LET Z = WEIBULL RUNDOM NUMBERS FOR I = 1 1 100
 
The output from the random number generation is always a variable
(never a parameter or function).  Random numbers can be generated from
a variety of distributions.  Some distributions represent a family of
distributions.  In this case, one or more parameters need to be
specified (via the LET command) before generating the random numbers.
 
The SEED command is used to specify the seed for the random number
generator.
 
The available random number generators are--
 
DISTRIBUTIONS WITH NO PARAMETERS
   NORMAL RANDOM NUMBERS          Generate standard normal (N(0,1))
                                  random numbers
   UNIFORM RANDOM NUMBERS         Generate uniform random numbers in
                                  the interval (0,1)
   LOGISTIC RANDOM NUMBERS        Generate logistic random numbers
   DOUBLE EXPON RANDOM NUMBERS    Generate double exponential random
                                  numbers
   CAUCHY RANDOM NUMBERS          Generate Cauchy random numbers
   SEMI-CIRCULAR RANDOM NUMBERS   Generate semi-circular random numbers
   TRIANGULAR RANDOM NUMBERS      Generate triangular random numbers
   LOGNORMAL RANDOM NUMBERS       Generate lognormal random numbers
   HALFNORMAL RANDOM NUMBERS      Generate halfnormal random numbers
   EXPONENTIAL RANDOM NUMBERS     Generate exponential random numbers
   EXTREME VALUE TYPE 1 RAND NUMB Generate extreme value type 1 random
                                  numbers
   FRECHET RANDOM NUMBERS         Generate extreme value type 1 random
                                  numbers
   HALF CAUCHY RANDOM NUMBERS     Generate half Cauchy random numbers
   COSINE RANDOM NUMBERS          Generate cosine random numbers
   ANGLIT RANDOM NUMBERS          Generate anglit random numbers
   ARCSIN RANDOM NUMBERS          Generate arcsin random numbers
   HYPERBOLIC SECANT RANDOM NUMB  Generate hyperbolic secant random
                                  numbers
   HALF-LOGISTIC RANDOM NUMBERS   Generate half-logistic random
                                  numbers
   SLASH RANDOM NUMBERS           Generate slash random numbers
   LANDAU RANDOM NUMBERS          Generate Landau random numbers
   RAYLEIGH RANDOM NUMBERS        Generate Rayleigh random numbers
   MAXWELL RANDOM NUMBERS         Generate Maxwell random numbers
 
DISTRIBUTIONS REQUIRING THE PARAMETER N
   DISCRETE UNIFORM RANDOM NUMBER Generate discrete uniform random
                                  numbers
 
DISTRIBUTIONS REQUIRING THE PARAMETER NU (DEGREES OF FREEDOM)
   T RANDOM NUMBERS               Generate t random numbers
   CHI-SQUARED RANDOM NUMBERS     Generate chi-squared random numbers
   FOLDED T RANDOM NUMBERS        Generate folded t random numbers
 
DISTRIBUTIONS REQUIRING THE PARAMETER LAMBDA (SHAPE PARAMETER)
   TUKEY LAMBDA RANDOM NUMBERS    Generate Tukey lambda random numbers
   POISSON RANDOM NUMBERS         Generate Poisson random numbers
   SKEWED NORMAL RANDOM NUMBERS   Generate skewed normal random
                                  numbers
   SKEWED DOUBLE EXPO NUMBERS     Generate skewed double exponential
                                  random numbers

DISTRIBUTIONS REQUIRING THE PARAMETERS NU1, NU2 (DEGREES OF FREEDOM)
   F RANDOM NUMBERS               Generate F random numbers

DISTRIBUTIONS REQUIRING THE PARAMETERS ALPHA, BETA (SHAPE PARAMETERS)
   BETA RANDOM NUMBERS            Generate beta random numbers
   POWER LAW RANDOM NUMBERS       Generate power law random numbers
                                  (i.e., failure times that follow a
                                  non-homogeneous Poisson process)
   ALPHA RANDOM NUMBERS           Generate alpha random numbers
   POWER EXPONENTIAL RANDOM NUMB  Generate power exponential random
                                  numbers
   INVERTED BETA RANDOM NUMBERS   Generate inverted beta random numbers
   HERMITE RANDOM NUMBERS         Generate Hermite random numbers
 
DISTRIBUTIONS REQUIRING THE PARAMETER GAMMA (SHAPE PARAMETER)
   GAMMA RANDOM NUMBERS           Generate gamma random numbers
   DOUBLE GAMMA RANDOM NUMBERS    Generate double gamma random numbers
   INVERTED GAMMA RANDOM NUMBERS  Generate inverted gamma random
   LOG GAMMA RANDOM NUMBERS       Generate log gamma random numbers
   WEIBULL RANDOM NUMBERS         Generate Weibull random numbers
   DOUBLE WEIBULL RANDOM NUMBERS  Generate double Weibull random
                                  numbers
   EXTREME VALUE TYPE 2 RAND NUMB Generate extreme value type 2 random
                                  numbers
   FRECHET RANDOM NUMBERS         Generate extreme value type 2 random
                                  numbers
   GENERALZIED EXTREME VALUE RAND NUMB Generate generalized extreme
                                       value random numbers
   PARETO RANDOM NUMBERS          Generate Pareto random numbers
   INVERSE GAUSSIAN               Generate inverse gaussian random
                                  numbers
   REVERSE INVERSE GAUSSIAN       Generate reverse inverse gaussian
                                  random numbers
   FATIGUE LIFE                   Generate fatigue life random numbers
   WALD                           Generate Wald random numbers
   INVERTED WEIBULL               Generate inverted Weibull random
                                  numbers
   GENERALIZED PARETO RAND NUMB   Generate generalized Pareto random
                                  numbers
   GEOMETRIC EXTREME EXPONENTIAL  Generate geometric estreme
                                  exponential random numbers
   GENERALIZED LOGISTIC RAND NUMB        Generate generalized logistic
                                         random numbers
   GENERALIZED HALF LOGISTIC RAND NUMB   Generate generalized half
                                         logistic random numbers
 
DISTRIBUTIONS REQUIRING THE PARAMETER C (SHAPE PARAMETER)
   POWER RANDOM NUMBERS           Generate power random numbers

DISTRIBUTIONS REQUIRING THE PARAMETERS K
   ASYMMETRIC DOUBLE EXPO RANDOM NUMBER Generate asymmetric double
                                        exponential random numbers

DISTRIBUTIONS REQUIRING THE PARAMETER MU, SD (SHAPE PARAMETER)
   FOLDED NORMAL RANDOM NUMBERS   Generate folded normal random numbers

DISTRIBUTIONS REQUIRING THE PARAMETER MU1, SD1, MU2, SD2, P (SHAPE PARAMETER)
   NORMAL MIXTURE RANDOM NUMBERS  Generate normal mixture random numbers

DISTRIBUTIONS REQUIRING THE PARAMETER NU, LAMBDA (SHAPE PARAMETER)
   NON-CENTRAL CHI-SQUARE RAND NUMB  Generate non-central Chi-square
                                     random numbers
   NON-CENTRAL T RAND NUMB           Generate non-central t random
                                     numbers
   SKEWED T RANDOM NUMBERS           Generate skewed t random numbers

DISTRIBUTIONS REQUIRING THE PARAMETERS NU, LAMBDA1 and LAMBDA2 (SHAPE PARAMETER)
   DOUBLY NON-CENTRAL T RAND NUMB    Generate doubly non-central t
                                     random numbers

DISTRIBUTIONS REQUIRING THE PARAMETER NU1, NU2, LAMBDA (SHAPE PARAMETER)
   NON-CENTRAL F RANDOM NUMBER      Generate non-central F random
                                    numbers
   NON-CENTRAL BETA RANDOM NUMBER   Generate non-central beta random
                                    numbers

DISTRIBUTIONS REQUIRING THE PARAMETER NU1, NU2, LAMBDA1, LAMBDA2 (SHAPE PARAMETER)
   DOUBLY NON-CENTRAL F RAND NUMB Generate doubly non-central F random
                                  numbers

DISTRIBUTIONS REQUIRING THE PARAMETER LAMBDA3, LAMBDA4 (SHAPE PARAMETER)
   GENERALIZED TUKEY-LAMBDA RAND NUMB  Generate generalized
                                  Tukey-Lambda random numbers

DISTRIBUTIONS REQUIRING THE PARAMETER P
   GEOMETRIC RANDOM NUMBERS       Generate geometric random numbers
   POWER NORMAL RANDOM NUMBERS    Generate power normal random
                                  numbers
   YULE RANDOM NUMBERS            Generate Yule random numbers
 
DISTRIBUTIONS REQUIRING THE PARAMETERS P, N
   BINOMIAL RANDOM NUMBERS        Generate binomial random numbers
 
DISTRIBUTIONS REQUIRING THE PARAMETERS P, K
   NEGATIVE BINOMIAL RANDOM NUMBER Generate negative binomial random
                                   numbers
 
DISTRIBUTIONS REQUIRING THE PARAMETERS L, K, N, M
   NEGATIVE BINOMIAL RANDOM NUMBER Generate hypergeometric random
                                   numbers

DISTRIBUTIONS REQUIRING THE PARAMETER DELTA
   LOG-LOGISTIC RANDOM NUMBER      Generate log-logistic random
                                   numbers

DISTRIBUTIONS REQUIRING THE PARAMETER ALPHA
   LOG DOUBLE EXPO RANDOM NUMBER   Generate log double exponential
                                   random numbers
   ERROR RANDOM NUMBER             Generate error (=exponential power
                                   or Subbotuin) random numbers
   ZIPF RANDOM NUMBERS             Generate Zipf random numbers

DISTRIBUTIONS REQUIRING THE PARAMETER BETA
   BRADFORD RANDOM NUMBER          Generate Bradford random numbers

DISTRIBUTIONS REQUIRING THE PARAMETER B
   RECIPROCAL RANDOM NUMBER        Generate reciprocal random numbers

DISTRIBUTIONS REQUIRING THE PARAMETERS C AND B
   GOMPERTZ RANDOM NUMBER          Generate Gompertz random numbers

DISTRIBUTIONS REQUIRING THE PARAMETER THETA
   LOGARITHMIC SERIES RAND NUMB    Generate logarithmic series random
                                   numbers

DISTRIBUTIONS REQUIRING THE PARAMETERS GAMMA AND THETA
   EXPONENTIATED WEIBULL RAND NUMB Generate exponentiated Weibull
                                   random numbers

DISTRIBUTIONS REQUIRING THE PARAMETERS THETA AND N
   TWO-SIDED POWER RAND NUMB       Generate two-sided power
                                   random numbers

DISTRIBUTIONS REQUIRING THE PARAMETERS G AND H
   G AND H RANDOM NUMBERS          Generate g-and-h random numbers

DISTRIBUTIONS REQUIRING THE PARAMETERS XI, LAMBDA, AND THETA
   GOMPERTZ-MAKEHAM RAND NUMB      Generate Gompertz-Makeham random
                                   numbers

DISTRIBUTIONS REQUIRING THE PARAMETERS SCALE1, GAMMA1, LOC2, SCALE2,
   AND GAMMA2
   BIWEIBULL RAND NUMB             Generate Bi-Weibull random numbers

DISTRIBUTIONS REQUIRING THE PARAMETERS A, B, C, and D
   TRAPEZOID RAND NUMB             Generate trapezoid random numbers

DISTRIBUTIONS REQUIRING THE PARAMETERS A, B, C, D, NU1, NU3, and ALPHA
   GENERALIZED TRAPEZOID RAND NUMB Generate generalized trapezoid
                                   random numbers

DISTRIBUTIONS REQUIRING THE PARAMETERS A AND C
   WARING RANDOM NUMBER            Generate Waring random numbers

Several distributions generate a matrix, as oppossed to a vector,
of random numbers.

Multivariate normal distribution:

   LET MU = DATA <list of p means>
   READ MATRIX SIGMA
      <pxp set of values>
   END OF DATA
   LET N = <value>
   LET M = MULTIVARIATE NORMAL RANDOM NUMBERS MU SIGMA N

   Note that M will be an NxP matrix.  N is the number of rows
   generated for each component and their are P components to
   the multivariate normal.  SIGMA is the pxp variance-covariance
   matrix of the multivariate normal.  SIGMA will be checked to
   ensure that it is a positive definite matrix.  MU is a vector
   specifying the means of the p components.

Multivariate t distribution:

   LET MU = DATA <list of p means>
   LET NU = DATA <list of p degrees of freedom>
   READ MATRIX SIGMA
      <pxp set of values>
   END OF DATA
   LET N = <value>
   LET M = MULTIVARIATE T RANDOM NUMBERS MU SIGMA NU N

   The variables are the same as for the multivariate normal random
   numbers with the exception that there is an additional vector,
   NU, that specifies the degrees of freedom for the p components.

Uniform distribution:

   LET U = INDEPENDENT UNIFORM RANDOM NUMBERSS LOWL UPPL NP
   LET U = MULTIVARIATE UNIFORM RANDOM NUMBERSS SIGMA N

   The first syntax generates independent uniform random numbers
   with LOWL and UPPL denoting vectors that contain the lower and
   upper limits for the uniform distributions, respectively.  The
   scalar NP denotes the number of rows to generate.

   The second syntax generates correlated uniform random numbers.
   The matrix SIGMA is the variance-covariance matrix of a
   multivariate normal distribution and N denotes the number of
   rows to generate.

Multinomial distribution:

   LET P = DATA <list of probabilities that sum to 1>
   LET N = <value>
   LET NEVENTS = <value>
   LET M = MULTINOMIAL RANDOM NUMBERS P N NEVENTS

   The P variable defines the probabilities for each of the
   outcomes, N defines the number of trials, and NEVENTS
   defines the number of multinomial experiments to simulate.
   The returned M will be a matrix with NEVENTS rows and
   the number of columns equal to the number of rows in P.

Dirichlet distribution:

   LET ALPHA = DATA <list of shape parameters>
   LET N = <value>
   LET D = DIRICHLET RANDOM NUMBERS ALPHA N

   The ALPHA variable contains the shape parameters of the
   Dirichlet distribution and N denotes the number of rows to
   generate.

Wishart distribution:

   LET P = DATA <list of probabilities that sum to 1>
   LET N = <value>
   LET NEVENTS = <value>
   LET W = WISHART RANDOM NUMBERS MU SIGMA N

   Note that W will be a PxP matrix.  N is a scalar that specifies
   the sample size.  SIGMA is the pxp variance-covariance
   matrix of the multivariate normal.  SIGMA will be checked to
   ensure that it is a positive definite matrix.  MU is a vector
   specifying the means of the p components.


Also, as of the May, 2002 version, support is provided for several
different uniform random number generators.  Enter
HELP RANDOM NUMBER GENERATOR for details.

----------------------------------------------------------


















































































-------------------------  *TEXT SUBCOMMANDS*-------------
 
TEXT SUBCOMMANDS
Text Subcommands
 
An important feature of the TEXT, TITLE, LABEL, and LEGEND commands is
the ability to use within-text subcommands to specify the following--
 
   1) To temporarily change the case (upper versus lower) in mid-text.
      For example, leading characters of words can be upper case and
      trailing characters can be lower case.
 
   2) To shift to subscripts and superscripts in mid-text.
 
   3) To generate Greek letters.
 
   4) To generate mathematical symbols (for example, integral sign,
      partial derivative sign, etc.).
 
   5) To generate other special symbols (for example, brackets,
      arrows, carats, daggers, etc.).
 
The above may be done whenever the Hershey fonts (simplex, duplex,
triplex, triplex italic, complex, simplex script, and complex script)
have been specified (see the FONT command).  The only special symbols
recognized with hardware fonts are the in-line case shifts (i.e., upper
and lower case) and the space character.
 
Within-text subcommands (indicators) are used to specify the desired
text operations.  For example--
 
   UC()   to shift to capital letters;
   SUB()  to shift to subscript mode;
   ALPH() to draw a Greek alpha;
   INTE() to draw an integral;
   RBRA() to draw a right bracket.
 
The within-text subcommands are all distinguished by an appended ().
The () is a flag to DATAPLOT that the previous character sub-string is
not to be printed literally but rather should be converted and acted
upon in a special fashion.
 
Enter   --HELP CAPITALIZATION to list case (lower/upper) information.
          HELP SUBSCIPTS to list sub/super-script information.
          HELP GREEK SYMBOLS to list Greek characters.
          HELP MATH SYMBOLS to list mathematics symbols.
          HELP MISC SYMBOLS to list miscellaneous symbols.
 
----------------------------------------------------------
 
 
 
 
 
 
 
 
 
 
 
 
 
 
 
 
 
 
 
 
 
 
 
 
 
 
 
 
 
 
 
 
 
 
 
 
 
 
 
 
 
 
 
 






 
-------------------------  *CAPITALIZATION*  -------------
 
CAPITALIZATION
Capitalization
 
DATAPLOT by default prints all text in upper case.  Simply entering
the text in lower case is not sufficient to have it printed in lower
case.
 
Shifts between capitalized letters and non-capitalized letters can be
carried out within text strings generated by any TEXT, TITLE, LABEL, or
LEGEND command.  Case shifts are recognized for both hardware and
software generated text.
 
To shift to upper case, enter CAPS(), CAP(), or UC() followed by the
desired text sub-string.  To shift to lower case, enter LC() followed
by the desired text sub-string.
 
The within-text case shifting overrides the setting from the CASE
command.  If the within-text case shift takes place mid-line, then the
first part of the text string follows whatever the current setting is
as given by the CASE command.  At the end of a text line with a case
shift, the current CASE command setting takes effect again.
 
The capitalization indicators are--
 
   upper case        UC(), CAP(), CAPS()
   lower case        LC()
 
If all characters on a text line are to have the same case, either all
upper or all lower, then it is easier to set the case globally with the
CASE command than using within-text case shifts.  For example,
 
   CASE UPPER
   CASE LOWER
 
Example --Go to the middle of screen, and write out "DATAPLOT is from
          NBS" with all symbols in simplex font--
             CASE UPPER
             FONT SIMPLEX
             MOVE 50 50
             TEXT DATAPLOT LC()IS FROM UC()NBS
Example --Go to the middle of screen, and write out "Future Goals"
          in triplex font--
             FONT TRIPLEX
             MOVE 50 50
             TEXT UC()FLC()UTURE UC()GLC()OALS
 
Enter   --HELP CASE to list information about the CASE command.
          HELP SUBSCIPTS to list sub/super-script information.
          HELP GREEK SYMBOLS to list Greek characters.
          HELP MATH SYMBOLS to list mathematics symbols.
          HELP MISC SYMBOLS to list miscellaneous symbols.
 
Note    --Upper and lower case characters can now be entered without
          using UC() and LC() shifts.  The various CASE commands
          (CASE, LABEL CASE, TITLE CASE, LEGEND CASE) accept an ASIS
          clause in addition to UPPER and LOWER.  The ASIS clause
          specifies that the case will be preserved as entered on the
          command line.  For example,
              CASE ASIS
              TEXT Mix UPPER and lower case Characters
 
----------------------------------------------------------
 
 
 
 
 
 
 
 
 
 
 
 
 
 
 
 
 
 
 
 
 
 
 
 
 
 
 
 
 
 
 
 
 
 
 
 
-------------------------  *SUBSCRIPTS*  -----------------
 
SUBSCRIPTS
Subscripts
 
Subscripts and superscripts can be generated within any TEXT, TITLE,
LABEL, or LEGEND command whenever the Hershey fonts (simplex, duplex,
triplex, triplex italic, complex, simplex script, and complex script)
have been specified (see the FONT command).
 
To shift to subscript mode, simply enter SUB() followed by the desired
subscript.  To terminate subscript mode, enter UNSB() and continue on
with the desired text.  Similarly, SUP() shifts into superscript mode,
and UNSP() shifts out of superscipt mode.  The () is a flag to DATAPLOT
that the previous character sub-string is not to be printed literally
but rather should be converted and acted upon in a special fashion.  It
is an indicator that is used not only for sub/super-scripting, but also
for Greek symbols, mathematics symbols, and other special symbols.
 
Subscript and superscript strings can be of any length.  Nested
subscripts and superscripts are permitted 7 deep.  The size of a
sub/super-script is always half the size of the previous level.
 
The sub/super-script indicators are--
 
   subscript           SUB()
   un-subscript        UNSB()
   superscript         SUP()
   un-superscript      UNSP()
 
Example --Go to the middle of screen, and write out
             e = mc squared      (Einstein's classic equation)
          with all symbols in lower case simplex font--
             CASE LOWER
             FONT SIMPLEX
             MOVE 50 50
             TEXT E = MCSUP()2
Example --Go to the middle of screen, and write out
             T (with superscript *) = e (with superscript integral f)
          in lower case triplex font--
             CASE LOWER
             FONT TRIPLEX
             MOVE 50 50
             TEXT TSUP()*UNSP() = ESUP()INTE()F
 
Enter   --HELP CAPITALIZATION to list case (lower/upper) information.
          HELP GREEK SYMBOLS to list Greek characters.
          HELP MATH SYMBOLS to list mathematics symbols.
          HELP MISC SYMBOLS to list miscellaneous symbols.
 
----------------------------------------------------------
 
 
 
 
 
 
 
 
 
 
 
 
 
 
 
 
 
 
 
 
 
 
 
 
 
 
 
 
 
 
 
 
 
 
 
 
 
 
 
 
 
 
 
 
 
 
 
 
 
-------------------------  *GREEK SYMBOLS*  --------------
 
GREEK SYMBOLS
Greek Symbols
 
Greek symbols can be generated within any TEXT, TITLE, LABEL, or LEGEND
command whenever the Hershey fonts (simplex, duplex, triplex, triplex
italic, complex, simplex script, and complex script) have been
specified (see the FONT command).  Both lower case and upper case Greek
symbols are available (see the CASE command).
 
To indicate that a Greek symbol should appear in some text string,
simply enter the English name of the desired Greek letter and append an
open and closed parenthesis after the name, as in PI(), RHO(), and
TAU().  The () is a flag to DATAPLOT that the previous character
sub-string is not to be printed literally but rather should be
converted and drawn as a special symbol.  Greek names longer than 4
letters should be truncated to 4 letters, as in ALPH(), GAMM(), and
OMEG().
 
The Greek symbols are--
 
   alpha          ALPH()
   beta           BETA()
   gamma          GAMM()
   delta          DELT()
   epsilon        EPSI()
   zeta           ZETA()
   eta            ETA()
   theta          THET()
   iota           IOTA()
   kappa          KAPP()
   lambda         LAMB()
   mu             MU()
   nu             NU()
   xi             XI()
   omicon         OMIC()
   pi             PI()
   rho            RHO()
   sigma          SIGM()
   tau            TAU()
   upsilon        UPSI()
   phi            PHI()
   chi            CHI()
   psi            PSI()
   omega          OMEG()
 
Example --Go to the middle of screen, and write out the first 3 Greek
          letters in lower case simplex font--
             CASE LOWER
             FONT SIMPLEX
             MOVE 50 50
             TEXT ALPH()BETA()GAMM()
Example --Go to the middle of screen, and write out the value of
          pi = 3.1415926 in lower case triplex font--
             CASE LOWER
             FONT TRIPLEX
             MOVE 50 50
             TEXT THE VALUE OF PI() = 3.1415926
 
Enter   --HELP CAPITALIZATION to list case (lower/upper) information.
          HELP SUBSCIPTS to list sub/super-script information.
          HELP MATH SYMBOLS to list mathematics symbols.
          HELP MISC SYMBOLS to list miscellaneous symbols.
 
----------------------------------------------------------
 
 
 
 
 
 
 
 
 
 
 
 
 
 
 
 
 
 
 
 
 
 
 
 
 
 
 
 
 
 
 
 
 
 
-------------------------  *MATH SYMBOLS*  ---------------
 
MATHEMATICS SYMBOLS
Mathematics Symbols
 
Mathematics symbols can be generated within any TEXT, TITLE, LABEL, or
LEGEND command whenever the Hershey fonts (simplex, duplex, triplex,
triplex italic, complex, simplex script, and complex script) have been
specified (see the FONT command).
 
To indicate that a mathematics symbol should appear in some text
string, simply enter the abbreviated (never more than 4 characters)
name from the list below and append an open and closed parenthesis
after the name, as in INTE(), SUMM(), and DOTP().  The () is a flag to
DATAPLOT that the previous character sub-string is not to be printed
literally but rather should be converted and drawn as a special symbol.
 
The mathematics symbols are--
 
   partial derivative       PART()
   integral                 INTE()
   circular integral        CINT()
   summation                SUMM()
   product                  PROD()
   infinity                 INFI()
   + or -                   +-()
   - or +                   -+()
   times                    TIME()
   dot product              DOTP()
   vector product           DEL()
   division                 DIVI()
   less than                LT()
   greater than             GT()
   less than or equal to    LTEQ()
   greater than or equal to GTEQ()
   not equal                NOT=()
   approximately equal to   APPR()
   equivalence              EQUI()
   varies                   VARI()
   tilda                    TILD()
   carat                    CARA()
   prime                    PRIM()
   radical                  RADI()
   large radical            LRAD()
   larger radical           BRAD()
   subset                   SUBS()
   superset                 SUPE()
   un-subset                UNSB()
   un-superset              UNSP()
   union                    UNIO()
   intersection             INTR()
   is an element of         ELEM()
   there exists             THEX()
   therefore                THFO()
 
Example --Go to the middle of screen, and draw out summation,
          integration, and infinity symbols in simplex font--
             FONT SIMPLEX
             MOVE 50 50
             TEXT SUMM()INTE()INFI()
Example --Go to the middle of screen, and write out
             A union B
          (a set theory notation) in upper case triplex font--
             CASE UPPER
             FONT TRIPLEX
             MOVE 50 50
             TEXT AUNIO()B
 
Enter   --HELP CAPITALIZATION to list case (lower/upper) information.
          HELP SUBSCIPTS to list sub/super-script information.
          HELP GREEK SYMBOLS to list Greek characters.
          HELP MISC SYMBOLS to list miscellaneous symbols.
 
----------------------------------------------------------
 
 
 
 
 
 
 
 
 
 
 
 
 
 
 
 
 
 
 
 
 
 
 
 
 
 
-------------------------  *MISC SYMBOLS*  ---------------
 
MISCELLANEOUS SYMBOLS
Miscellaneous Symbols
 
Miscellaneous symbols can be generated within any TEXT, TITLE, LABEL,
or LEGEND command whenever the Hershey fonts (simplex, duplex, triplex,
triplex italic, complex, simplex script, and complex script) have been
specified (see the FONT command).
 
To indicate that a special symbol should appear in some text string,
simply enter the abbreviated (never more than 4 characters) name from
the list below and append an open and closed parenthesis after the
name, as in LAPO(), LBRA(), and RBRA().  The () is a flag to DATAPLOT
that the previous character sub-string is not to be printed literally
but rather should be converted and drawn as a special symbol.
 
The miscellaneous symbols are--
 
   space                    SP()
   carriage return          CR()
   left apostrophe          LAPO()
   right apostrophe         RAPO()
   left bracket             LBRA()
   right bracket            RBRA()
   left curly bracket       LCBR()
   right curly bracket      RCBR()
   left elbow               LELB()
   right elbow              RELB()
   right accent             RACC()
   left accent              LACC()
   breve                    BREV()
   right quote              RQUO()
   left quote               LQUO()
   nasp                     NASP()
   inverted nasp            IASP()
   right arrow              RARR()
   left arrow               LARR()
   up arrow                 UARR()
   down arrow               DARR()
   paragraph                PARA()
   dagger                   DAGG()
   double dagger            DDAG()
   vertical bar             VBAR()
   double vertical bar      DVBA()
   long vertical bar        LVBA()
   horizontal bar           HBAR()
   long horizontal bar      LHBA()
   bar                      BAR()
   degree                   DEGR()
 
The SP() and CR() can be used with hardware fonts.  The SP()
is useful as a placeholder (e.g., for alphabetic tic mark labels,
it can be used for an empty group).  The CR() can be used for
multiline text.  For example, you can create a multiline title.

EXAMPLE --Go to the middle of screen, and draw out ABC surrounded by
          curly brackets with ABC in upper case simplex font--
             CASE UPPER
             FONT SIMPLEX
             MOVE 50 50
             TEXT LCBR()ABCRCBR()
EXAMPLE --Go to the middle of screen, and write out x surrounded by 2
          vertical bars (a mathematics notation for the absolute value
          of x) where X is in lower case triplex font--
             CASE LOWER
             FONT TRIPLEX
             MOVE 50 50
             TEXT VBAR()XVBAR()
 
Enter   --HELP CAPITALIZATION to list case (lower/upper) information.
          HELP SUBSCIPTS to list sub/super-script information.
          HELP GREEK SYMBOLS to list Greek characters.
          HELP MATH SYMBOLS to list mathematics symbols.
 
----------------------------------------------------------
 
 
 
 
 
 
 
 
 
 
 
 
 
 
 
 
 
 
 
 
 
 
 
 
-------------------------  *CHARACTER TYPES*  ------------
 
CHARACTER TYPES
Character Types
 
The available character types are from 5 categories--
 
   1) common plotting characters;
   2) any ascii keyboard character;
   3) Greek symbols;
   4) mathematics symbols;
   5) miscellaneous symbols.
 
The case (upper versus lower) and font (simplex, duplex, triplex, etc.)
of the plot character follows the current setting of the CASE and FONT
commands.  The Greek, math, and miscellaneous symbols are available
only when one of the Hershey fonts have been specified via the FONT or
CHARACTER FONT commands.
 
The common plot characters include--
 
   blank                BLANK or NONE or BL or NO
   circle               CIRCLE or O or CI
   square               SQUARE or BOX or SQ
   diamond              DIAMOND or DI
   triangle             TRIANGLE or TR
   reverse triangle     REVTRI or TRIREV or RT
   star                 STAR or ST
   arrow up             ARROWU or AU
   arrow down           ARROWD or AD
   vertical bar         VERTICAL BAR or VB
   pyramid              PYRAMID
   cube                 CUBE
   arrow                ARROW or ARRH
   vector               VECTOR
 
The ascii keyboard characters include
 
   0 to 9               0 to 9
   A to Z               A to Z
   period               .  or PERIOD or POINT or DOT
   bar                  -  or BAR or BARS or HYPHEN
   plus                 +  or PLUS or CROSS
   asterisk             *  or ASTERISK
   left bracket         [
   right bracket        ]
   left brace           {
   right brace          }
   exclamation point    !  or EXCLAMATION
   double quote         "  or QUOTE2
   number               #  or NUMBER
   dollar               $  or DOLLAR
   percent              %  or PERCENT
   ampersand            &  or AMPERSAND
   single quote         '  or QUOTE1
   left parenthesis     (  or LEFTPA
   right parenthesis    )  or RIGHTP
   colon                :  or COLON
   semi-colon           ;  or SEMICO
   comma                ,  or COMMA
   question mark        ?  or QUESTION
   apostrophe           '  or APOSTROPHE
   underscore           -  or UNDERSCORE
   at sign              @  or AT
   slash                /  or SLASH or DIAGONAL
   reverse slash         `  or REVSLASH
   equal sign           =  or EQUAL
   greater than sign    >  or GREATER
   less than sign       <  or LESS
   vertical bar         \  or VBAR
   tilda                ~  or TILDA
   carat                ^  or CARAT
 
The available Greek symbols are listed by entering HELP GREEK SYMBOLS.
 
The available mathematics symbols are listed by entering HELP MATH
SYMBOLS.
 
The available miscellaneous symbols are listed by entering HELP MISC
SYMBOLS.
 
When special math, greek, or miscellaneous symbols are used, the
trailing () is left off.
 
----------------------------------------------------------
 
 
 
 
 
 
 
 
 
 
 
 
 
 
 
-------------------------  *LINE TYPES*  -----------------
 
LINE TYPES
Line Types
 
The available line types are--
 
   no line         BLANK or NONE or BL or NO
   solid           SOLID or SO
   dotted          DOT or DOTTED or DO
   dashed          DASH or DASHED or DA
   dashed type 1   DASH1 or DA1
   dashed type 2   DASH2 or DA2
   dashed type 3   DASH3 or DA3
   dashed type 4   DASH4 or DA4
 
The short designations (e.g., BL for BLANK, SO for SOLID, DA3 for
DASH3) allow for the specification of a large number of line types on a
single command line, as in
 
   LINES SO SO SO SO SO DO DO DO DO DO DA DA DA DA DA
 
DATAPLOT does all dash patterns in hardware, so dashed lines may have a
somewhat different appearance on different devices.
 
----------------------------------------------------------
 
 
 
 
 
 
 
 
 
 
 
 
 
 
 
 
 
 
 
 
 
 
 
 
 
 
 
 
 
 
 
 
 
 
 
 
 
 
 
 
 
 
 
 
 
 
 
 
 
 
 
 
 
 
 
 
 
 
 
 
 
 
 
 
 
 
 
 
 
 
 
 


-------------------------  *COLOR TYPES*  ----------------
 
COLOR TYPES
Color Types
 
The graphics devices that DATAPLOT supports vary widely in the degree
to which they support color.  However, for the sake of device
independence, all devices will recognize the same set of color names
and color indices.  If a given device does not support a requested
color, DATAPLOT maps it to the closest available supported color
(closest is somewhat arbitrary, although we tried to be reasonably
consistent).
 
DATAPLOT borrowed its color scheme from Release 3 of X11 with a few
additions from Release 4.  It also uses the RGB values from Release 4
for those devices that support direct RGB specification (currently
Postscript and CGM).  Although these values should be reasonably
robust, different devices will generate different colors from them.
 
The following is the list of colors that DATAPLOT recognizes.  Only the
first 4 characters of the color name are significant and colors can
also be specified by an index.
 
 
                                    DATAPLOT
        COLOR             INDEX     NAME
        =====             =====     ========
        WHITE               0       WHIT
        BLACK               1       BLAC
        RED                 2       RED
        BLUE                3       BLUE
        GREEN               4       GREE
        MAGENTA             5       MAGE
        ORANGE              6       ORAN
        CYAN                7       CYAN
        YELLOW              8       YELL
        YELLOW GREEN        9       YGRE
        DARK GREEN         10       DGRE
        LIGHT BLUE         11       LBLU
        BLUE VIOLET        12       VBLU
        VIOLET RED         13       VRED
        DARK SLATE GRAY    14       DGRA,DGRY
        LIGHT GRAY         15       LGRA,LGRY
        AQUAMARINE         16       AQUA
        BROWN              17       BROW
        CADET BLUE         18       CABL
        CORAL              19       CORA
        CORNFLOWER BLUE    20       CBLU
        DARK OLIVE GREEN   21       DOGR
        DARK ORCHID        22       DORC
        DARK SLATE BLUE    23       DSBL
        DARK TURQUOISE     24       DTUR
        FIREBRICK          25       FIRE
        FOREST GREEN       26       FGRE
        GOLD               27       GOLD
        GOLDENROD          28       GLDR
        GRAY               29       GRAY, GREY
        INDIAN RED         30       IRED
        KHAKI              31       KHAK
        DIM GRAY           32       DMGR
        LIGHT STEEL BLUE   33       LSBL
        LIME GREEN         34       LGRE
        MAROON             35       MARO
        MEDIUM AQUAMARINE  36       MAQU
        MEDIUM BLUE        37       MBLU
        MEDIUM FOREST GREEN38       MFGR
        LIGHT GOLDENROD YEL39       MGLD
        MEDIUM ORCHID      40       MORC
        MEDIUM SEA GREEN   41       MSGR
        MEDIUM SLATE BLUE  42       MSBL
        MEDIUM SPRING GREEN43       MSPG
        MEDIUM TURQUOISE   44       MTUR
        MEDIUM VIOLET RED  45       MVRD
        MIDNIGHT BLUE      46       MDBL
        NAVY BLUE          47       NAVY
        ORANGE RED         48       ORED
        ORCHID             49       ORCH
        PALE GREEN         50       PGRE
        PINK               51       PINK
        PLUM               52       PLUM
        PURPLE             53       PURP
        SALMON             54       SALM
        SEA GREEN          55       SGRE
        SIENNA             56       SIEN
        SKY BLUE           57       SKBL, SKYB
        SLATE BLUE         58       SBLU
        SPRING GREEN       59       SPGR
        STEEL BLUE         60       STBL
        TAN                61       TAN
        THISTLE            62       THIS
        TURQUOISE          63       TURQ
        VIOLET             64       VIOL
        WHEAT              65       WHEA
        GREEN YELLOW       66       GYEL
        LIGHT CYAN         67       LCYA
        BLUE2              68       BLU2
        BLUE3              69       BLU3
        BLUE4              70       BLU4
        CYAN2              71       CYA2
        CYAN3              72       CYA3
        CYAN4              73       CYA4
        GREEN2             74       GRE2
        GREEN3             75       GRE3
        GREEN4             76       GRE4
        YELLOW2            77       YEL2
        YELLOW3            78       YEL3
        YELLOW4            79       YEL4
        ORANGE2            80       ORA2
        ORANGE3            81       ORA3
        ORANGE4            82       ORA4
        RED2               83       RED2, LRED
        RED3               84       RED3
        RED4               85       RED4
        MAGENTA2           86       MAG2, LMAG
        MAGENTA3           87       MAG3
        MAGENTA4           88       MAG4
 
 
In addition, gray scale can be specified with the following scheme:
 
        G0     = BLACK
        G1-G99 = GRAY SCALE FROM BLACK TO WHITE
        G100   = WHITE
 
Gray scale values can also be specified with negative indices (that is,
-1 through -100).
 
Currently, Postscript and X11 support gray scale.  Other devices will
map gray scale to either black or white.
 
Penplotters no longer automatically map an index to the corresponding
slot.  DATAPLOT assumes the following slot to color mapping:
 
        4 PENS            8 PENS:
        ======            =======
        BLACK             BLACK
        RED               RED
        BLUE              BLUE
        GREEN             GREEN
                          MAGENTA
                          ORANGE
                          CYAN
                          YELLOW
 
You can use the <HPGL/CALCOMP/ZETA> PEN MAP command to specify a
different slot to color mapping for HP-GL, Calcomp, and Zeta plotters
respectively.
 
The following command shows the available colors.
 
         SHOW COLORS
 
The following commands show the colors available on the various color
devices that DATAPLOT supports.  That is, they show the color you
actually get with the requested DATAPLOT color for that device.
 
         SHOW COLORS TEKT 4115
         SHOW COLORS TEKT 4662
         SHOW COLORS TEKT 4027
         SHOW COLORS HP 2622
         SHOW COLORS CALCOMP
         SHOW COLORS ZETA
         SHOW COLORS CGM
         SHOW COLORS GENERAL
         SHOW COLORS SUN
         SHOW COLORS REGIS
         SHOW COLORS POSTSCRIPT
         SHOW COLORS X11
         SHOW COLORS PC
 
For some color display terminals (e.g., Tektronix 4105/7/9/15), the
color can be altered locally after the plot has been generated on the
screen.  This gives the analyst a "second chance" if the original
color choices do not mix well.
 
Finally, be wary of the idiosyncracies of color hardcopy devices.  They
rarely capture the same color hues as on the screen (e.g., the
Tektronix 4662 ink jet plotter maps a brilliant blue on the 4105/7/9/15
screen into a drab purple on the hardcopy).  Also note that it is
common for color hardcopies to map screen white into hardcopy black
and vice versa.

Note that the Postscript device is set to black and white by
default.  To activate color for Postscript devices, do the
following:

     DEVICE 2 POSTSCRIPT
     DEVICE 2 COLOR ON

Note that the order of these commands is relevant.

----------------------------------------------------------








-------------------------  *ASCII FILES*  ----------------
 
ASCII FILES
Ascii Files
 
This section provides guidance on reading ASCII data files in
Dataplot.  This includes discussion of some commands added to
the 1/2004 version of Dataplot.  In particular, discussion is
included for ASCII files created by the Excel program.

Dataplot has limited support for binary data files.  Currently,
only binary files created using Fortran unformatted WRITE are
supported.  Enter HELP SET READ FORMAT for details.

Also, Dataplot does not currently support directly reading files
from other statistical/spreadsheet programs or database files.
Some support may be provided in future releases, but for now
you need to save the data from these programs in an ASCII file
in order to read them into Dataplot.  XML based data files are
becoming increasingly popular as well.  At this time, Dataplot does
not support XML based data files, although we anticipate looking
at this issue for subsequent releases.


IDEAL CASE

By default, Dataplot assumes rectangular data files containing
numeric data where the data columns are separated by one or
more spaces.  Commas or tabs may be used as delimiters as well.  

In this case, you can read the file with a command like the
following:

    READ  FILE.DAT  Y X1 X2

The first argument after the READ is the name of the ASCII file.
The remaining arguments identify the variable names.  Variable
names can be up to eight characters long and should be limited
to alphabetic (A-Z) and numeric (0-9) characters.  Although
other characters can in fact be used, this is discouraged
since their use can cause problems in some contexts.  Variable
names are not case sensitive (Dataplot converts all alphabetic
characters to upper case).

Dataplot recognizes the first argument as a file name if it
finds a "." in the name.  If no "." is found, Dataplot assumes
the first argument is a variable name and it tries to read
from the keyboard rather than the file.

The remainder of this section discusses various issues that
may cause problems when reading ASCII files and provides
suggestions on how to deal with  these issues.  The following
topics are discussed:

   1. Viewing ASCII files within Dataplot
   2. Header lines/restricted rows or columns
   3. Long data records
   4. Automatic variable names
   5. Reading fixed columns
   6. Reading variables with unequal lengths (empty fields)
   7. Reading character data
   8. Reading row oriented data
   9. Comment lines in data files
  10. Reading Excel files
  11. File name restrictions
  12. Comma as decimal point
  13. Missing values and undefined numbers
  14. Reading date and time fields
  15. Reading IP addresses
  16. Reading monetary data (e.g., $23,461,58)
  17. Reading numeric values with trailing "+" or "-"
  18. Commas withing character fields
  19. Reading binary data
  20. Reading image data
  21. What if all of the data will not fit into memory?

If you create the ASCII file yourself, it is recommended that
you create it with variables of equal length (pick some numeric
value to signify missing data) and with data items separated by one
or more spaces.  Inclusion of a header giving a description of
the data file is optional, but we find it helpful (Dataplot
can skip over the header lines).  When the ASCII files are created
by another program (e.g., Excel), then you may have less control
over the format of the file.  Hopefully, most ASCII files you
encounter can be handled using the commands discussed below.


VIEWING THE ASCII FILE WITHIN DATAPLOT

In order to identify some of the issues discussed below, it is
often helpful to view the ASCII file before trying to read it into
Dataplot.  You can do this with the command

    LIST FILE.DAT

This will list the file 20 lines (you can change the number
of lines with the SET LIST LINES command) at a time.  You
can then enter a carriage return to view the next 20 lines or
a "no" to stop viewing the file.

For some of the commands given below, you need to either know
approriate line numbers or column numbers.

To view the file with line numbers, enter the command

   NLIST  FILE.DAT

To identify appropriate columns, enter the command

   RULER

This will identify the first 80 columns.


HEADER LINES/RESTRICTED ROWS OR COLUMNS

Many data files contain header lines at the beginning of the
file that provide a description of the file.  In order to
skip over these lines, enter the command

    SKIP N

where N identifies how many lines to skip.

Most of the sample data files that are distributed with Dataplot
contain a line starting with hyphens ("---").  You can use the
command 

    SKIP AUTOMATIC

for these files.  Dataplot will skip all lines until a line
starting with three or more hypens is encoutered.
 
In a related issue, if you want to restrict the read to certain
rows in the file, you can enter the command

    ROW LIMITS  N1  N2

with N1 and N2 denoting the first and last rows to read,
respectively. 

You can also restrict the read to certain columns of the file
using the command

    COLUMN LIMITS  C1  C2

with C1 denoting the first column to read and C2 the last column
to read.


LONG DATA RECORDS

When reading from the keyboard, Dataplot restricts a single record
to a maximum of 80 columns.

When reading from a file, Dataplot previously restricted a single
record to a maximum of 132 columns.  The March, 2003 version raised
the default limit to 255 characters.  In addition, the following
command was added:

   SET MAXIMUM RECORD LENGTH  N

with N denoting the size of the largest record to be read.

Dataplot accepts values of N up to 9999.  However, be aware
that some Fortran compilers may impose their own limit.  These
limits tend not to be well documented, but with modern compilers
they tend to be sufficiently large that this should not be a
problem in practice.

If you specify a SET READ FORMAT command (discussed below), you
do not need to specify the maximum record length.


AUTOMATIC VARIABLE NAMES

Dataplot normally reads variable names on the READ command.
However, many ASCII files will have the name of the variables
given directly in the file or Dataplot can assign the variable
names automatically.

Specific methods include the following.

  1. Many of the sample files provided in the Dataplot 
     installation use a syntax like

         Y     X1   X2
        ----------------
           <data values>

     For these files, you can enter the commands

         SKIP AUTOMATIC
         READ FILE.DAT

     In this case, Dataplot will skip all lines until a line
     starting with three or more hypens is encountered.  It
     will then backspace to the previous line and read the
     variable names from that line.

  2. Many ASCII data files will have the variable names on
     the first line of the file.  For these files, you can
     enter the commands

        SET VARIABLE LABELS ON
        READ FILE.DAT

  3. If you would like Dataplot to simply assign the variable
     names, enter the command

         READ FILE.DAT

     Dataplot will read the first line of the file to determine
     the number of variables.  It will then assign the names
     X1, X2, and so on to the variable names.

Note that Dataplot's usual rules for variable names still apply.
That is, a maximum of eight characters will be used and spaces will
delimit variable names.  The use of special (i.e., not a number and
not an alphabetic character) characters is discouraged.  You may
need to edit the file if the variable names do not follow these
rules (more than eight characters will simply be ignored, so the
issue is more one of duplicate variable names in the first eight
characters).


READING FIXED COLUMNS

By default, Dataplot performs free format reads.  That is,
you do not need to line up the columns neatly.  You do need
to provide one or more spaces (tabs, commas, colons, semi-colons,
parenthesis, or brackets can be used as well) between data fields.

Many data files will contain fixed fields.  There are several reasons
you may want or need to take advantage of these fixed fields rather
than using a free format read.

   1. If your data fields do not contain spaces (or some other
      delimiter) between data columns, you need to tell
      Dataplot how to interpret the columns.

   2. In some cases, you may only want to read selected
      variables in the data file.

   3. Using a formatted read can significantly speed up the reading
      of the data.  If you have small or moderate size data files (say
      500 rows or fewer), this is really not an issue.  However, if you
      are reading 50,000 rows, you can significantly speed up the read
      by specifying the format.

   4. If the data fields have unequal lengths, Dataplot will not
      interpret the data file correctly with a free format read.
      It assigns the data items in the order they are encountered
      to the variable names in the order they are given.  Dataplot
      does not try to guess if a data item is missing based on the
      columns.

      The issue of unequal lengths is discussed in detail in the
      next section.

There are two basic cases for fixed fields.

   1. The data fields are justified by the decimal point.

      In this case, you can use the 

          SET READ FORMAT  <format>

      command to specify a Fortran-like format to read the file.
      Enter HELP READ FORMAT for details.

      Using a formatted read is significantly faster than a
      free format read.

   2. Many programs will write ASCII files with fixed columns,
      but the data fields will be either left or right justified
      rather than lined up by the decimal point.

      In this case, you can use a special form of the
      COLUMN LIMITS command that was introduced with the
      January, 2004 version.  Normally, the first and last columns
      to read are specified.  However, you can now enter variables for
      the lower and upper limits as in the following example:

         LET LOWER = DATA  1  21   41
         LET UPPER = DATA 10  30   50
         COLUMN LIMITS LOWER UPPER

      That is, if variables rather than parameters are specified,
      separate column limits are specified for each data field.
      In this case, the first data field is between columns
      1 and 10, the second field is between columns 21 and 30, and
      the third field is between 41 and 50.

      When this syntax is used, only one variable is read for
      each specified field.  If the field is blank, then this is
      interpreted as a missing value.


READING VARIABLES OF UNEQUAL LENGTH (EMPTY FIELDS)

Dataplot typically expects all variables to be of equal length.  That is,
the data is rectangular with no empty fields.

Performing free format reads with space delimited data files when there
are empty fields is problematic.  Dataplot reads the file one row at a
time.  When reading a row, Dataplot will assign the first value read to
the first variable name, the second value to second variable and so on.
By default, the row with smallest number of values defines the number of
variables that will be read.  For example, if you requested four
variables be read, but one row of the data file only has two values, then
only two variables will be read into Dataplot.

If you have a data file where the columns have unequal lengths (i.e.,
empty fields), you can try one of the following things.

    1. Pick some value to represent a missing value and fill
       in missing data points with that value.  After reading
       the data, you can use a RETAIN command to remove them.
       For example, if you use -99 to signify a missing value,
       you can enter

          RETAIN Y SUBSET Y > -99

       Alternatively, you can use a SUBSET clause on subsequent
       plot and analysis commands.

       There are two SET commands that pertain to missing values.

            i. SET DATA MISSING VALUE <value> specifies a character
               string that will be interpreted as a missing value in the
               data file (this character string can be a numeric value).

           ii. SET READ MISSING VALUE <value> specifies the numeric value
               that will be saved to the Dataplot variable when a missing
               value (as defined by the SET DATA MISSING VALUE) is
               encountered.

       When feasible, this is the recommended solution.

    2. If your data file has consistent formats for the rows,
       then there are two possible solutions.

         i. If the fields are justified by the decimal point so that a
            Fortran format statement can be applied, then you can use the
            SET READ FORMAT command.  In this case, empty fields are read
            as zero.  If zero can be a valid data value for one or more of
            your variables, then it can be ambiguous whether a zero in your
            variable denotes a valid data point or a missing value.  The
            SET READ MISSING VALUE setting does not apply when the
            SET READ FORMAT is used.

        ii. Many spreadsheets have an option for saving data to a "fixed
            width" ASCII text file.  In these cases, the fields are
            typically either right or left justified.  However, the column
            for the decimal point will not be consistent so that the
            SET READ FORMAT command cannot be used.  In this case, you can
            use the variable form of the COLUMN LIMITS command as
            described above.  By default, when a blank field is
            encountered, it is set to zero.  You can specify the
            value to use by entering the command

              SET READ MISSING VALUE  <value>

    3. If your data has both columns of unequal length and
       inconsistent columns for given data fields, an alternative
       is to use a comma delimited data file.  If there is no data data
       between successive commas, this is treated as a missing value.
       The default is to assign a value of zero.  Alternatively, you
       can use the SET READ MISSING VALUE command described above.

       You can specify a delimiter other than a comma with the
       command

          SET READ DELIMITER <character>

    4. You can use the following command

           SET READ PAD MISSING COLUMNS ON

       When this command is used, if the number of values read on a row
       is less than the number of variables specified, then the
       values from the row are padded with missing values (as
       specified by the SET READ MISSING VALUE).  For example, if
       you entered

           READ FILE.DAT X1 X2 X3 X4

       and a particular row only had two values, then the first value
       will be assigned to X1 and the second value to X2.  X3 and X4
       will be assigned the missing value for that row.

       This works if the empty fields are at the end.  However, if
       the empty fields are not at the end, then the assignment of
       the data to the variables will not be what is expected.  In
       this case, it is recommended that empty fields be coded with
       a missing value code.

       NOTE: The default (SET READ PAD MISSING COLUMNS OFF) action
             was modified 2019/04.  Previously, if this was off,
             the number of variables read was truncated to the
             number of values on the row(s) with the smallest number
             of values.  This was changed so that the behavior of
             the OFF setting is similar to the ON setting.  The
             difference is that for OFF, a warning message will be
             printed for rows that have fewer than the expected
             number of values.

The variable form of the COLUMN LIMITS, the SET READ MISSING VALUE, and
the SET READ DELIMITER commands were introduced in the January, 2004
version.  The interpretation of successive commas as a missing value was
also introduced in the January, 2004 version.


READING DATA WITH CHARACTER FIELDS

Dataplot has not previously supported character data.  The one
execption is that you could read row labels with the READ ROW LABEL
command (enter HELP READ ROW LABEL for details).  If encountered,
Dataplot would generate an error message and not read the data file
correctly. 

With the January 2004 version, we have introduced some limited
support for character data.  Specifically, we have added the command

     SET CONVERT CHARACTER <ON/IGNORE/ERROR>

Setting this to ERROR will continue the current Dataplot action of
reporting character data as an error.  This is recommended for the
case when a file is suppossed to contain only numeric data and the
presence of character data is in fact indicative of an error in the
data file.

Setting this to IGNORE will instruct Dataplot to simply ignore any
fields containing character data.  This can be useful if you simply
want to extract the numeric data fields in the file without
entering COLUMN LIMITS or SET READ FORMAT commands.

Setting this to ON will read character fields and write them to the
file "dpzchf.dat".  Note that Dataplot saves numeric data
"in memory" for fast access.  Since character data has limited
use in Dataplot, we have decided to save character data 
externally to minimize memory requirements.  Dataplot keeps a
separate name table for the character data fields (the names for
character variables are stored in the file "dpzchf.dat").

NOTE 2018/10: The CATEGORICAL option was added.  This option works
              similarly to ON.  However, in addition to creating the
              character variable in "dpzchf.dat", it also creates
              numerical variables automatically from the character
              data.

There are some restrictions on when Dataplot will try to
read character data:

   1) This only applies to the variable read case.  That
      is, READ PARAMETER and READ MATRIX will ignore
      character fields or treat them as an error.

   2) Dataplot will only try to read character data from
      a file.  When reading from the keyboard (i.e., when
      READ is specified with no file name), character data
      will be ignored when a SET CONVERT CHARACTER ON is
      specified.

   3) This capability is not supported for the SERIAL READ
      case.

   4) The SET READ FORMAT command does not accept the
      "A" format specification for reading character
      fields.

   5) A maximum of 20 character variables will be saved.

   6) A maximum of 24 characters for each character variable
      will be saved.

   7) The character variables from at most one data file
      will be saved in a given session.

Some of these restrictions may be addressed in subsequent
releases of Dataplot.

Currently, Dataplot has limited support for character variables.
Specifically,

   1) The row label can be used for the plot character by
      entering the command

        CHARACTER ROWLABEL

   2) You can convert a character variable to a coded numeric
      variable with the command

        LET Y = CHARACTER CODE IX
        LET Y = ALPHABETIC CHARACTER CODE IX

      with IX denoting the name of the character variable.  These
      command assigns a numeric value for each unique name in
      the character variable.

      For the CHARACTER CODE case, the coding is from 1 to K where
      K is the number of unique values.  The order is based on
      the order these values are found in the file.

      For the ALPHABETIC CHARACTER CODE case, the coding is from
      1 to K where K is the number of unique values.  The order is
      performed in alpabetical order.

We anticipate additional use of character variables in subsequent
releases of Dataplot.

If your character fields contain non-numeric/non-alphabetic characters,
then it is recommended that the character fields be enclosed in
quotes.  When Dataplot encounters a quote (either a single or double
quote), it interprets everything until a matching quote is found
as part of that character field.  If the quotes are not used,
then spaces, tabs, parenthesis, brackets, colons, and semi-colons
are interpreted as delimiters that signify the end of that data item.


READING ROW ORIENTED DATA

Dataplot assumes a column oriented format.  That is, a row of
data represents a single record (or case) and a column of data 
represents a variable.  If a data file has a row orientation, then
this is reversed.  A row of data represents a variable and a column
of data represents a record (or case).

The following example shows one way of correctly reading the data
into Dataplot.  Suppose that your data file contains five rows with
each row corresponding to a single variable.  You can do the following:

   LOOP FOR K = 1 1 5
      ROW LIMITS K K
      SERIAL READ FILE.DAT  X^K
   END OF LOOP

NOTE 2018/10: Dataplot added a READ ROW command that will read each row
              into a separate column.  This command assumes all of the
              data in a given row are numeric.  It does not assume that
              all rows must contain the same number of elements.

COMMENT LINES IN DATA FILES

It is sometimes convenient to include comments in data files.
If these comments are contained at the beginning of the file, then
the SKIP command can be used.  To have Dataplot check for comment
lines in the data file, enter the command

    COMMENT CHECK ON

The default comment character is a ".".  That is, any line starting
with a ". " is treted as a comment line and ignored.  To specify
a different comment character, enter the command

    COMMENT CHARACTER  <char>

with <char> denoting the desired comment character.


EXCEL FILES

At the current time (1/2004), Dataplot does not support the
direct reading of Excel data files.  We are planning to add
this capability in a future release of Dataplot.  Until that
time, you need to save the data in Excel to an ASCII file and
read that ASCII file into Dataplot.
 
Excel provides the following options for writing ASCII data
files:

  1. Formatted text (space delimited) (.PRN extension)

     This format will use consistent columns for the data fields.
     The variable form of the COLUMN LIMITS command can be used
     when the data columns have unequal length.

     Character fields will often not have the separating space.  The
     variable form of the COLUMN LIMITS command can be used in this
     case as well.

  2. CSV (Comma delimited) (.CSV extension)

     This format will separate data fields with a single comma.
     Missing data is represented with successive commas.  Dataplot
     can now (as of the January 2004 version) handle this correctly.

  3. Text (Tab delimited) (.TXT extension)
     Text (MS-DOS) (.TXT extension)

     These files will separate data fields with a tab character.
     Note that Dataplot converts all non-printing characters
     (including tabs) to a single space character.

     This format is not appropriate for data containing variables
     with unequal lengths since it will not generate consistent
     columns for the data fields.  Use either the space delimited
     or comma delimited file for that case.

The 2014/12 version of Dataplot added the capability of reading
and writing to the system clipboard under Windows.  Using the
"copy" function and Excel and then using the READ CLIPBOARD command
in Dataplot will in many cases be the easiest way to retrieve
data from Excel files.  Enter HELP CLIPBOARD for details.

FILE NAME RESTRICTIONS

A few comments on file names.

  1. File names are limited to 80 characters or less (this includes
     the path name if given).

  2. If the file name contains either spaces or hypens, it
     should be enclosed in double quotes.  For example,

       READ "C:\My Documents\SAMPLE.DAT"  Y X1 X2

  3. The file name should be a valid file name on the local
     operating system.

  4. The file name must contain a period "." in the file name itself
     or as a trailing character.  Dataplot strips off trailing periods
     on those systems where it is appropriate to do so.  On systems
     where trailing periods can be a valid file name (e.g., Unix),
     Dataplot opens the file with the trailing period.
 
  5. On systems where file names are case sensitive (i.e., Unix),
     Dataplot first tries to open the file name as given.  If the
     file is not found, it then tries to match the file name
     after converting the name to all upper case characters.  If
     it is still not found, it will convert the file name to all
     lower case characters

     If your file name contains a mixture of upper and lower case
     characters, then you need to enter the case for the file name
     correctly on the READ command.


COMMA AS DECIMAL POINT

Dataplot follows the United States convention where the decimal
point is the period ".".  Some locales may use a different
character to denote the decimal point.  In particular, some
countries use the comma ",".

To allow Dataplot to read files that use a character other than
the "." for the decimal point, enter the command

     SET DECIMAL POINT <value>

where <value> denotes the character that specifies the decimal
point.

Note this support is fairly limited.  Specifically, it applies
to free-format reads (i.e., no SET READ FORMAT command has been
entered).  In addition,

   1. This option is not supported for the WRITE command.  WRITE
      will always use a period for the decimal point.

   2. Dataplot alphanumeric output (e.g., the output from the FIT
      command) is generated with the period as the decimal point.

   3. As mentioned above, if you read your data with a 
      SET READ FORMAT command, the data must use the period
      for the decimal point.


MISSING VALUES AND UNDEFINED VALUES

Some software programs will have special characters to denote
missing values or undefined values (e.g., the result of trying
to divide by 0).

In particular, Unix/Linux software often uses "nan" to denote an
undefined number.  If Dataplot encounters an "nan" in a numeric
field, it will convert it to the Dataplot "missing value".  The "nan"
search is not case sensitive (i.e., it will check for "NAN", "NaN",
etc.).  You can specify what Dataplot will use for the missing value
by entering the command

    SET READ MISSING VALUE  <value>

where <value> is a numeric value.

Missing value flags are specific to individual programs.  You can
specify a character string that denotes a missing value with the
command

    SET DATA MISSING VALUE <value>

where <value> is a string with 1 to 4 characters.  If Dataplot
encounters <value> in a numeric field, it will convert it to the
Dataplot "missing value".  The missing value string is not case
sensitive.  You can specify what Dataplot will use for the missing
value by entering the command

    SET READ MISSING VALUE  <value>

where <value> is a numeric value.

READING DATE AND TIME FIELDS

Date and time fields will typically have syntax like

   2016/06/22
   12:43:08

Dataplot treats the "/" and ":" as indicating character fields
(based on the SET CHARACTER CONVERT command, this will either cause
an error, result in this field being ignored, or the field being
read as a character variable).

The following commands were added (2016/06) to help deal with date and
time fields.

   SET DATE DELIMITER <character>
   SET TIME DELIMITER <character>

Although Dataplot does not have explicit date or time variables,
these commands allow the components of date and time fields to
be read as separate numeric variables.  For example,

   SET DATE DELIMITER /
   SET TIME DELIMITER :
   READ YEAR MONTH DAY HOUR MIN SEC
   2016/06/22  23:19:03
   END OF DATA

READING IP ADDRESSES

IP addresses typically have a syntax like

   129.6.37.209

By default, Dataplot will generate an error when trying to read a
field of this type.  To address this, you can enter the command

   SET READ IP ADDRESSES ON

If this switch is ON, Dataplot will scan the line and if a field is
encountered that conains more than one period ".", Dataplot will
convert these periods to spaces before parsing the line.

The default is OFF since this adds additional processing time to
the READ and most data sets do not contain IP addresses.

READING MONETARY DATA

Monetary data may sometimes be given as

   $11,456.12  $1,021,111.10

The "$" and "," in these numeric fields will cause problems.  The
"$" will be treated as a non-numeric value (depending on other
SET commands, this will be treated as an error or the numeric field
will be read as a character field).  The comma is typically treated
as a field delimiter.  If you have this kind of data, enter the
commands

    set read dollar sign ignore on
    set read comma ignore on

To reset the defaults, enter

    set read dollar sign ignore off
    set read comma ignore off

Note that if you enter the SET READ COMMA IGNORE ON command, the
comma will no longer be treated as the delimiter.  Dataplot cannot
currently handle the case where the comma is used both for monetary
data and also as a field delimiter.

READING NUMERIC VALUES WITH TRAILING "+" OR "-"

On occassion, numeric fields may have a trailing "+" or a
trailing "-".  The "+" is typically used to indicate that the
value is greater than or equal to the entered value.  Likewise, the
"-" is used to indicate that the value is less than or equal to the
entered value.  This may be used when data is truncated at a high
or low value.  If you have data that uses this convention, enter

    set read trailing plus minus ignore on

Dataplot does not have any convention for indicating that a number
in fact means "greater than" or "less than", so it will read the
numeric value and simply ignore the "+" or "-".

To reset the defualt, enter

    set read trailing plus minus ignore off

COMMAS WITHIN CHARACTER FIELDS

If you are reading data that may contain character fields, you can
specify whether you want commas in the character fields to be
treated as part of the character field or as a delimiter.

To have the comma treated as a delimiter, enter

    set character field comma delimiter on

To have the comma not be interpreted as a delimiter (i.e., it
will simply be another character in the character field), enter
 
    set character field comma delimiter off

The default is OFF.

READING BINARY DATA

Currently, the only types of binary data that Dataplot currently
supports are:

  1) A few types of image files can be read on some platforms.
     This is discussed in the next section.

  2) Dataplot may be able to read some files created using Fortran
     unformatted data files.  Dataplot is most likely to have success
     reading unformatted Fortran files that contain only numeric data
     and use a consistent record structure.  Unformatted Fortran
     files that contain a mixture of character and numeric data
     will not be read successfully.

Support for other types of binary files may be added in future
releases.  However, this support will probably be for specific
types of binary files as oppossed to arbitrary binary files.

The advantage of using unformatted Fortran files is that file sizes
may be significantly smaller and reading the data can be significantly
faster.  One potential use of unformatted Fortran files is to save
a large data file that you will read many times in Dataplot.

The disadvantages of using unformatted Fortran files are that they
are not human readable, they cannot be edited or modified using an
ASCII editor, and, most importantly, they are not portable between
operating systems and compilers.  That is, unformatted Fortran files
typically need to be read using the same operating system and compiler
that was used to create them.

For details on using unformatted Fortran files, enter

     HELP SET READ FORMAT


READING IMAGE DATA

If Dataplot was built with support for the GD library, Dataplot
can read image data in PNG, JPEG, or GIF format.  If you have
image data in another format, you may be able to use an image
conversion program (e.g., NetPMB or ImageMagick) to convert it
to one of the supported formats.

For further information, enter

    HELP READ IMAGE


WHAT IF ALL THE DATA WILL NOT FIT INTO MEMORY?

Dataplot was designed primarily for interactive usage.  For this reason,
it reads all data into memory.  The current default is to have a
workspace that accomodates 10 columns with 1,500,000 rows (you can
re-dimension to obtain more columns at the expense of fewer rows, however
you cannot increase the maximum number of rows).

With the advent of "big data", there are more data files that cannot be
read into Dataplot's available memory.  For these data files, there are
several things that can potentially be done

   1. For some platforms, if you have a large amount of memory you may
      be able to build a version of Dataplot that raises the maximum
      number of rows.  For example, on a Linux system with 64MB of RAM,
      we were able to build a version that supports a maximum of
      10,000,000 rows.  Contact Alan Heckert if you need assistance
      with this.

   2. The STREAM READ command was added.  This command uses one pass
      algorithms to do a number of things.

      a. You can create a new file that uses SET WRITE FORMAT.  This
         is typically done once so that you can use SET READ FORMAT on
         subsequent reading of the data file (this can substantially
         speed up processing of these large files).

      b. You can generate various summary statistics either for the full
         data set or for groups in the data.

      c. You can generate cross tabulation statistics (up to 4
         cross tabulation variables may be specified).

      d. You can create various types of distance (e.g., Euclidean
         distances, correlation distances) matrices either for the full
         data set or for cross tabulations of the data.
 
         Distance matrices are often used for various types of
         multivariate analysis.

      e. You can generate approximate percentiles either for the full
         data set or for cross tabulations of the data.  Based on this,
         you can perform distributional modeling for a single variable
         or distributional comparisons between variables (e.g.,
         quantile quantile plots, bihistograms, two sample KS tests, and
         so on).

      The STREAM READ command can allow you to do a fair bit of
      exploratory analyses on these large data sets.  

----------------------------------------------------------


























































































-------------------------  *SYSTEM LIMITS*  ----------------
 
SYSTEM LIMITS
SYSTEM LIMITS
 
This section documents some relevant limits when using Dataplot.
Note that many of these limits can be set (in the file DPCOPA.INC)
before Dataplot is compiled on a specific system.  The limits below
are the default values, but the limits for your specific system may
be set differently.


  1) MAXOBV   = 1,500,000

     Defines the maximum number of observations for a single
     variable.  This is the parameter most likely to be modified
     before Dataplot is compiled.  It will typically be in the
     range 100,000 to 1,500,000.

  2) MAXOBW   = 10*MAXOBV

     Defines the total size of the data work space.  Note that
     the DIMENSION command can be used to re-allocate between
     rows and columns with the restriction that the maximum
     number of rows cannot exceed MAXOBV.

  3) MAXPOP   = 2*MAXOBV

     Defines the maximum number of points on a plot.
  
  4) MAXNME   = 50000

     Define the maximum number of names (variables, parameters,
     strings, matrices).

  5) MAXSTR   = 255

     Defines the maximum number of characters in a single command
     line.

  6) MAXEDC   = 24*MAXOBV

     Defines the maximum number of characters that the EDIT/FED
     command can accomodate.

  7) MAXEDL   = 25000

     Defines the maximum number of lines for the EDIT/FED command.

  8) MAXCMP   = 35

     Defines the maximum number of coefficients in multilinear
     regression.

  9) MAXLIS   = 200
     MAXCIS   = 255

     Define the maximum number of lines/columns in the LIST/SAVE
     table.

 10) MAXLIL   = 20000
     MAXCIL   = 255

     Define the maximum number of lines/columns in the LOOP table.

 11) MAXLIP   = 20

     Define the maximum number of lines in the REPLOT table.

 12) MAXPM    = 200

     Define the maximum number of pixmaps that can be saved.

 13) MAXTOM   = 46*MAXOBV/3

     Define the maximum size (rows times columns) for a matrix.
     Note that the DIMENSION MATRIX command can be used to
     specify the allocation between rows and columns in the matrix.

 14) Define the following plot control component dimensions:

     MAXTC    = 100    = the maximum number of tic marks on an axis.
     MAXLG    = 100    = the maximum number of legends.
     MAXBX    = 100    = the maximum number of boxes.
     MAXAR    = 100    = the maximum number of arrows.
     MAXSG    = 100    = the maximum number of segments.
     MAXLN    = 100    = the maximum number of line traces.
     MAXCH2   = 100    = the maximum number of character traces.
     MAXFL    = 100    = the maximum number of region fills.
     MAXPT    = 100    = the maximum number of patterns.
     MAXSP    = 100    = the maximum number of spikes.
     MAXBA    = 100    = the maximum number of bars.
     MAXRE    = 100    = the maximum number of regions.
     MAXTX    = 100    = the maximum number for the text command.
     MAXSUB   = 10     = the maximum number of subregions.
     MAXGRP   = 5      = the maximum number of group label variables.
     MAXGR2   = 40     = the maximum number of characters for a group
                         label.
     MAXGLA   = MAXOBV/100  = the maximum number of levels for a
                              group label.
     MAXCNL   = 100    = the maximum number of contour labels.

 15) MAXCH    = 200

     Define the maximum number of characters in a text string for a
     title or label.

 16) MAXLG2   = 200

     Define the maximum number of characters in a legend.

 17) MAXF1    = 50,000

     Define the total number of characters for all strings/functions.

 18) MAXF2    = 500

     Define the maximum number of strings/functions.

 19) MAXF3    = 500

     Define the maximum number of characters printed for the last
     model fitted.

 20) MAXRCL   = 9999

     Define the maximum number of characters that can be read from a
     single record in a data file.






----------------------------------------------------------

































































-------------------------  *DISTRIBUTIONS*  -----------------
 
PROBABILITY DISTRIBUTIONS
Probability Distributions
 
The following commands operate on distributions:

    <dist> PROBABILITY PLOT
    <dist> <PPCC/ANDERSON DARLING/KOLM SMIR/CHI-SQUARE> PLOT
    <dist> <ANDERSON DARLING/KOLMOGOROV SMIRNOV/CHI-SQUARE/PPCCC>
           GOODNESS OF FIT
    LET Y = <dist> RANDOM NUMBERS FOR I = 1 1 N
    BOOTSTRAP <dist> <MLE/PPCC/ANDERSON DARLING/KOLM SMIR> PLOT
    JACKNIFE <dist> <MLE/PPCC/ANDERSON DARLING/KOLM SMIR> PLOT

For these commands, you may need to enter value of one or more
shape parameters and/or values for location and scale parameters.
For example,

   LET GAMMA = 2.5
   WEIBULL PROBABILITY PLOT Y

More specifically:

   1) For the RANDOM NUMBERS command, you need to specify the values
      of any shape parameters.  This command does not utilize location
      or scale parameters.  However, you can transform the random
      numbers using the relation

          Y = LOC + SCALE*Y

      For example,

          LET GAMMA = 2.5
          LET LOC = 10
          LET SCALE = 5
          LET Y = WEIBULL RANDOM NUMBERS FOR I = 1 1 N
          LET Y = LOC + SCALE*Y

   2) For the PROBABILITY PLOT command, you need to specify the
      values for any shape parameters.  For example,

          LET GAMMA = 2.5
          WEIBULL PROBABILITY PLOT Y

      You can optionally specify location and scale parameters with
      the commands

          LET PPLOC = <value>
          LET PPSCALE = <value>

      Note that the probability plot is invariant to location and
      scale (i.e., the linearity of the probability plot does not
      depend on the values of the location and scale parameters).
      PPLOC and PPSCALE are typically used when a non-PPCC method
      is used to estimate the location/scale parameters.
      
   3) For the PPCC PLOT, ANDERSON DARLING PLOT, KOLMOGOROV SMIRNOV PLOT
      and CHI-SQUARE PLOT commands, you can optionally specify the
      range for the shape parameter(s) (default ranges will be used if
      they are not specified).  For example,

          LET GAMMA1 = 0.5
          LET GAMMA2 = 5
          WEIBULL PPCC PLOT Y

      That is, you append a 1 (for the lower limit) and a 2 (for the
      upper limit) to the shape parameter name.

      For the ANDERSON DARLING, KOLMOGOROV SMIRNOV, and CHI-SQUARE
      variants, you can optionally fix the values of the location/scale
      parameters with the commands

          LET KSLOC = <value>
          LET KSSCALE = <value>

   4) For the GOODNESS OF FIT and the BOOTSTRAP/JACKNIFE PLOT commands,
      you need to specify the values for any shape parameters.

      In addition, you can specify the values for the location/scale
      parameters with the commands (these will default to 0 and 1
      if these commands are not given)

          LET KSLOC = <value>
          LET KSSCALE = <value>

      Distributions that are bounded both above and below specify
      the lower and upper limits (rather than the location/scale)
      with the commands

          LET A = <value>
          LET B = <value>

      Distributions that use A and B rather than KSLOC/KSSCALE will
      be denoted by the phrase "bounded distribution" in the tables
      below.

      An example of using these commands:

          LET GAMMA = 2.5
          LET KSLOC = 5
          LET KSSCALE = 10
          WEIBULL ANDERSON DARLING GOODNESS OF FIT Y
          BOOTSTRAP WEIBULL ANDERSON DARLING PLOT Y

The extreme value type 1 (Gumbel), extreme value type 2 (Frechet),
generalized Pareto, generalized extreme value and the Weibull
support "minimum" and "maximum" forms of the distribution.  You
can specify the minimum form with either of the following commands

    SET MINMAX 1
    SET MINMAX MINIMUM

You can specify the maximum form with either of the following commands

    SET MINMAX 2
    SET MINMAX MAXIMUM

The default is the "minimum" for the Weibull and "maximum" for the
others.

This section documents the values you need to enter for the distributions
supported in Dataplot.

CONTINUOUS DISTRIBUTIONS:

Location/Scale Distributions:
      1) NORMAL
      2) UNIFORM - bounded distribution
      3) LOGISTIC
      4) DOUBLE EXPONENTIAL
      5) CAUCHY
      6) SEMI-CIRCULAR
      7) COSINE
      8) ANGLIT
      9) HYPERBOLIC SECANT
     10) HALF-NORMAL
     11) ARCSIN
     12) EXPONENTIAL
     13) EXTREME VALUE TYPE I (GUMBEL)
     14) HALF-CAUCHY
     15) SLASH
     16) RAYLEIGH
     17) MAXWELL
     18) LANDAU

One Shape Parameter Distributions - name of shape parameter(s) listed:
      1) ALPHA                          - ALPHA
      2) ASYMMETRIC DOUBLE EXPONENTIAL  - K (or MU)
      3) BRADFORD                       - BETA
      4) BURR TYPE 2                    - R
      5) BURR TYPE 7                    - R
      6) BURR TYPE 8                    - R
      7) BURR TYPE 10                   - R
      8) BURR TYPE 11                   - R
      9) CHI                            - NU
     10) CHI-SQUARED                    - NU
     11) DOUBLE GAMMA                   - GAMMA
     12) DOUBLE WEIBULL                 - GAMMA
     13) ERROR (SUBBOTIN)               - ALPHA
     14) EXPONENTIAL POWER              - BETA
     15) EXTREME VALUE TYPE 2 (FRECHET) - GAMMA
     16) FATIGUE LIFE                   - GAMMA
     17) FOLDED T                       - NU
     18) GAMMA                          - GAMMA
     19) GENERALIZED EXTREME VALUE      - GAMMA
     20) GENERALIZED HALF LOGISTIC      - GAMMA
     21) GENERALIZED LOGISTIC           - ALPHA
     22) GENERALIZED LOGISTIC TYPE 2    - ALPHA
     23) GENERALIZED LOGISTIC TYPE 3    - ALPHA
     24) GENERALIZED LOGISTIC TYPE 5    - ALPHA
     25) GENERALIZED PARETO             - GAMMA
     26) GEOMETRIC EXTREME  EXPONENTIAL - GAMMA
     27) INVERTED GAMMA                 - GAMMA
     28) INVERTED WEIBULL               - GAMMA
     29) LOG DOUBLE EXPONENTIAL         - ALPHA
     30) LOG GAMMA                      - GAMMA
     31) LOGISTIC-EXPONENTIAL           - BETA
     32) LOG LOGISTIC                   - DELTA
     33) LOGNORMAL                      - SIGMA
     34) MCLEISH                        - ALPHA
     35) MUTH                           - BETA
     36) OGIVE                          - N
     37) PEARSON TYPE 3                 - GAMMA
     38) POWER FUNCTION                 - C
     39) POWER NORMAL                   - P, bounded distribution
     40) RECIPROCAL                     - B
     41) REFLECTED POWER                - C, bounded distribution
     42) SKEW DOUBLE EXPONENTIAL        - LAMBDA
     43) SKEW NORMAL                    - LAMBDA
     44) SLOPE                          - ALPHA, bounded distribution
     45) T                              - NU
     46) TOPP AND LEONE                 - BETA, bounded distributin
     47) TRIANGULAR                     - C, bounded distribution
     48) TUKEY LAMBDA                   - LAMBDA
     49) VON MISES                      - B
     50) WALD                           - GAMMA
     51) WEIBULL                        - GAMMA
     52) WRAPPED CAUCHY                 - P
    
Two Shape Parameter Distributions:
      1) ASYMMETRIC LOG DOUBLE EXPO     - ALPHA, BETA
      2) BETA                           - ALPHA, BETA,
                                          bounded distribution
      3) BETA NORMAL                    - ALPHA, BETA
      4) BURR TYPE 3                    - R, K
      5) BURR TYPE 4                    - R, C
      6) BURR TYPE 5                    - R, K
      7) BURR TYPE 6                    - R, K
      8) BURR TYPE 9                    - R, K
      9) BURR TYPE 12                   - C, K
     10) DOUBLY PARETO UNIFORM          - M, N
     11) EXPONENTIATED WEIBULL          - GAMMA, THETA
     12) F                              - NU1, NU2
     13) FOLDED CAUCHY                  - LOC, SCALE
     14) FOLDED NORMAL                  - MU, SD
     15) G-AND-H                        - G, H
     16) GENERALIZED ASYMMETRIC LAPLACE - K, TAU or K, MU
     17) GENERALIZED GAMMA              - ALPHA, C
     18) GENERALZIED INVERSE GAUSSIAN   - LAMBDA, OMEGA
     19) GENERALIZED LOGISTIC TYPE 4    - P, Q
     20) GENERALIZED MCLEISH            - ALPHA, A
     21) GENERALIZED TOPP AND LEONE     - ALPHA, BETA,
                                          bounded distribution
     22) GENERALIZED TUKEY LAMBDA       - LAMBDA3, LAMBDA4
     23) GOMPERTZ                       - C, B or ALPHA, K
     24) GOMPERTZ-MAKEHAM               - ETA, ZETA
                                          (Meeker parameterization)
     25) INVERSE GAUSSIAN               - GAMMA, MU
     26) INVERTED BETA                  - ALPHA, BETA
     27) JOHNSON SB                     - ALPHA1, ALPHA2
     28) JOHNSON SU                     - ALPHA1, ALPHA2
     29) KAPPA                          - K, H
     30) KUMARASWAMY                    - ALPHA, BETA
                                          bounded distribution
     31) LOG-SKEW-NORMAL                - LAMBDA, SD
     32) MIELKE'S BETA-KAPPA            - THETA, K
     33) NON-CENTRAL T                  - NU, LAMBDA
     34) NON-CENTRAL CHI-SQUARE         - NU, LAMBDA
     35) PARETO                         - GAMMA, A (A defaults to 1
                                          if not specified)
     36) PARETO SECOND KIND             - GAMMA, A (A defaults to 1
                                          if not specified)
     37) POWER LOGNORMAL                - P, SD
     38) RECIPROCAL INVERSE GAUSSIAN    - GAMMA, NU
     39) REFLECTED GENERALIZED TOPP LEONE - ALPHA, BETA
                                          bounded distribution
     40) TWO-SIDED OGIVE                - THETA, N
                                          bounded distribution
     41) TWO-SIDED POWER                - THETA, N
                                          bounded distribution
     42) TWO-SIDED SLOPE                - THETA, ALPHA
                                          bounded distribution
     43) SKEW T                         - LAMBDA, NU
    
Three or More Shape Parameter Distributions:
      1) BESSEL I-FUNCTION              - SIGMA1SQ, SIGMA2SQ, NU or
                                          B, C, M
      2) BESSEL K-FUNCTION              - SIGMA1SQ, SIGMA2SQ, NU or
                                          B, C, M
      3) BI-WEIBULL                     - GAMMA1, GAMMA2, SCALE1,
                                          SCALE2, LOC2
      4) BRITTLE FRACTURE               - ALPHA, BETA, R
      5) DOUBLY NON-CENTRAL BETA        - ALPHA, BETA, LAMBDA1, LAMBDA2
      6) DOUBLY NON-CENTRAL F           - NU1, NU2, LAMBDA1, LAMBDA2
      7) DOUBLY NON-CENTRAL T           - NU, LAMBDA1, LAMBDA2
      8) GENERALIZED EXPONENTIAL        - LAMBDA1, LAMBDA12, S
      9) GENERALZIED TRAPEZOID          - A, B, C, D, ALPHA, NU1, NU3
     10) GOMPERTZ-MAKEHAM               - CHI, LAMBDA, THETA or
                                          GAMMA, LAMBDA, K
     11) LOG BETA                       - ALPHA, BETA, C, D
     12) LOG-SKEW-T                     - NU, LAMBDA, SD
     13) NON-CENTRAL BETA               - ALPHA, BETA, LAMBDA
     14) NON-CENTRAL F                  - NU1, NU2, LAMBDA
     15) NORMAL MIXTURE                 - U1, SD1, U2, SD2, P
     16) TRAPEZOID                      - A, B, C, D
     17) TRUNCATED EXPONENTIAL          - X0, M, SD
                                          (X0 assumed known for PPCC)
     18) TRUNCATED NORMAL               - MU, SD, A, B
     19) TRUNCATED PARETO               - GAMMA, A, NU
     20) UNEVEN TWO-SIDED POWER         - ALPHA, NU1, NU3, D
                                          bounded distribution
     21) WAKEBY                         - GAMMA, BETA, DELTA, ALPHA, CHI
                                          (CHI and ALPHA are the location
                                          and scale parameters)

DISCRETE DISTRIBUTIONS:
      1) BETA-BINOMIAL                  - ALPHA, BETA, N
      2) BETA GEOMETRIC                 - ALPHA, BETA
      3) BETA NEGATIVE BINOMIAL         - ALPHA, BETA, K
      4) BINOMIAL                       - P, N
      5) BOREL-TANNER                   - LAMBDA, K
      6) CONSUL (GENERALIZED GEOMTRIC)  - THETA, BETA or MU, BETA
      7) DISCRETE UNIFORM               - N
      8) DISCRETE WEIBULL               - Q, BETA
      9) GEETA                          - THETA, BETA or MU, BETA
     10) GENERALIZED LOGARITHMIC SERIES - THETA, BETA
     11) GENERALIZED LOST GAMES         - P, J, A
     12) GENERALIZED NEGATIVE BINOMIALS - THETA, BETA, M
     13) GEOMETRIC                      - P
     14) HERMITE                        - ALPHA, BETA
     15) HYPERGEOMETRIC                 - L, K, N, M
     16) KATZ                           - ALPHA, BETA
     17) LAGRANGE-POISSON               - LAMBDA, THETA
     18) LEADS IN COIN TOSSING          - N
     19) LOGARITHMIC SERIES             - THETA
     20) LOST GAMES                     - P, R
     21) MATCHING                       - K
     22) NEGATIVE BIONOMIAL             - P, N
     23) POISSON                        - LAMBDA
     24) POLYA-AEPPLI                   - THETA, P
     25) QUASI BINOMIAL TYPE I          - P, PHI
     26) TRUNCATED GENE NEGATIVE BINOM  - THETA, BETA, B, N
     27) WARING                         - C, A
     28) YULE                           - P
     29) ZETA                           - ALPHA
     30) ZIPF                           - ALPHA, N


----------------------------------------------------------
















































































-END -----*-----      ----------------ZZZZZ------
