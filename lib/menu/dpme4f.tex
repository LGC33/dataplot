MENU FILE #4 (FUNCTIONS)
 
0                200         0
Built-in Functions
1                200        19
Sign, Integer, Fraction, Round, Mod, & MSD Functions
1.1              200        30
Sign Function SIGN(X)
1.2              200        48
Integer Function INT(X)
1.3              200        66
Fraction Function FRACT(X)
1.4              200        84
Round Function ROUND(X,NDP)
1.5              200       106
Modulus Function MOD(X,BASE)
1.6              200       124
Most Significant Digit Function MSD(X)
2                200       143
Min, Max, & Positive Difference Functions
2.1              200       151
Minimum Function MIN(X,Y)
2.2              200       170
Maximum Function MAX(X,Y)
2.3              200       189
Positive Difference Function DIM(X,Y)
3                200       208
Absolute Value, Square Root, Exponential, & Log Functions
3.1              200       217
Absolute Value Function ABS(X)
3.2              200       235
Square Root Function SQRT(X)
3.3              200       253
Exponential Function EXP(X)
3.4              200       271
Logarithmic Functions LOG(X), LOG10(X), LOG2(X)
4                200       294
Error & Gamma Functions
4.1              200       303
Error Function ERF(X)
4.2              200       323
Complementary Error Function ERFC(X)
4.3              200       343
Gamma Function GAMMA(X)
4.4              200       363
Log (to base e) Gamma Function LOGGAMMA(X)
5                200       383
Chebychev & Bessel Functions
5.1              200       390
Chebychev Functions CHEB0(X) to CHEB10(X)
5.2              200       412
Bessel Functions BESS0(X) AND BESS1(X)
6                200       433
Trig Functions (Circular)
7                200       458
Inverse Trig Functions (Circular)
8                200       484
Trig Functions (Hyperbolic)
9                200       503
Inverse Trig Functions (Hyperbolic)
10               200       522
Octal/Decimal Conversion Functions
11               200       540
Probability Functions
11.1             200       554
Normal Distribution
11.2             200       578
t Distribution
11.3             200       600
Chi-squared Distribution
11.4             200       622
F Distribution
11.5             200       645
Weibull Distribution
11.6             200       667
Inverse Gaussian Distribution
11.7             200       689
Wald Distribution
11.8             200       711
Reciprocal Inverse Gaussian Distribution
11.9             200       734
Fatigue Life Distribution
12               200       756
Indicator Function IND(X,TAG)
13               200       776
Binary Pattern Function BINPAT(X,K)
 
 
 
 
 
 
 
 
 
 
 
 
 
 
 
 
 
 
 
 
 
 
 
 
 
 
 
 
 
 
 
 
 
 
 
 
 
 
 
 
 
 
 
 
 
 
 
 
 
 
 
 
 
 
 
 
 
 
 
 
 
 
 
 
 
 
 
 
 
 
 
 
 
 
 
 
 
 
 
 
 
 
 
 
 
 
 
 
 
 
 
 
 
 
 
 
 
 
 
 
 
 
 
 
 
 
 
 
 
 
 
 
 
----------  *BUILT-IN FUNCTIONS*  ------------------------
 
0
Built-in Functions
   1. Sign, Integer, Fraction, Round, Mod, & MSD Functions
   2. Min, Max, & Positive Difference Functions
   3. Absolute Value, Square Root, Exponential, & Log Functions
   4. Error & Gamma Functions
   5. Chebychev & Bessel Functions
   6. Trig Functions (Circular)
   7. Inverse Trig Functions (Circular)
   8. Trig Functions (Hyperbolic)
   9. Inverse Trig Functions (Hyperbolic)
  10. Octal/Decimal Conversion Functions
 
  11. Probability Functions
  12. Indicator Function IND(X,TAG)
  13. Binary Pattern Function BINPAT(X,K)
 
----------------------------------------
 
1
Sign, Integer, Fraction, Round, Mod, & MSD Functions
   1. Sign Function SIGN(X)
   2. Integer Function INT(X)
   3. Fraction Function FRACT(X)
   4. Round Function ROUND(X,NDP)
   5. Modulus Function MODULUS(X,BASE)
   6. Most Significan Digit Function MSD(X)
 
----------------------------------------
 
1.1
Sign Function SIGN(X)
      LET A = -7.2
      LET B = SIGN(A)
      LET C = 8.6
      LET D = SIGN(D)
      WRITE A B C D
      .
      LET X = DATA -2 -5 3 15 -6
      LET Y = SIGN(X)
      WRITE X Y
      SET WRITE DECIMALS 1
      WRITE X Y
      .
      PLOT SIGN(X) FOR X = -10 1 10
 
----------------------------------------
 
1.2
Integer Function INT(X)
      LET A = -7.265
      LET B = INT(A)
      LET C = 8.61
      LET D = INT(D)
      WRITE A B C D
      SET WRITE DECIMALS 4
      WRITE A B C D
      .
      LET X = DATA 3.1415 2.718 -1.77 2.9979
      LET Y = INT(X)
      WRITE X Y
      .
      PLOT INT(X) FOR X = -3 .1 3
 
----------------------------------------
 
1.3
Fraction Function FRACT(X)
      LET A = -7.265
      LET B = FRACT(A)
      LET C = 8.61
      LET D = FRACT(D)
      WRITE A B C D
      SET WRITE DECIMALS 4
      WRITE A B C D
      .
      LET X = DATA 3.1415 2.718 -1.77 2.9979
      LET Y = FRACT(X)
      WRITE X Y
      .
      PLOT FRACT(X) FOR X = -3 .1 3
 
----------------------------------------
 
1.4
Round Function ROUND(X,NDP)
      LET A = 3.141592
      LET B0 = ROUND(A,0)
      LET B1 = ROUND(A,1)
      LET B2 = ROUND(A,2)
      LET B3 = ROUND(A,3)
      LET B4 = ROUND(A,4)
      WRITE A B0 B1 B2 B3 B4
      SET WRITE DECIMALS 5
      WRITE A B0 B1 B2 B3 B4
      WRITE "THE VALUE OF PI IS ^A "
      WRITE "THE VALUE OF PI TO 4 DECIMAL PLACES IS ^B4 "
      .
      LET X = NORMAL RANDOM NUMBERS FOR I = 1 1 10
      LET Y = ROUND(X,2)
      WRITE X Y
      .
      PLOT ROUND(X,1) FOR X = -1 .01 1
 
----------------------------------------
 
1.5
Modulus Function MOD(X,BASE)
      LET A = 4
      LET B2 = MOD(A,2)
      LET B3 = MOD(A,3)
      LET B4 = MOD(A,4)
      WRITE A B2 B3 B4 B5
      SET WRITE DECIMALS 0
      WRITE A B2 B3 B4 B5
      .
      LET X = SEQUENCE 1 1 10
      LET Y = MOD(X,8)
      WRITE X Y
      .
      PLOT MOD(X,2) FOR X = -5 .1 5
 
----------------------------------------
 
1.6
Most Significant Digit Function MSD(X)
      LET A = 38.2
      LET B = MSD(A)
      LET C = -123.456
      LET D = MSD(D)
      SET WRITE DECIMALS 3
      WRITE A B C D E F
      .
      LET X = UNIFORM RANDOM NUMBERS FOR I = 1 1 100
      LET Y = MSD(X)
      SET WRITE DECIMALS 3
      WRITE X Y
      TABULATE Y
      .
      PLOT MSD(X) FOR X = -5 .1 5
 
----------------------------------------
 
2
Min, Max, & Positive Difference Functions
   1. Minimum Function MIN(X,Y)
   2. Maximum Function MAX(X,Y)
   3. Positive Difference Function DIM(X,Y)
 
----------------------------------------
 
2.1
Minimum Function MIN(X,Y)
 
LET A = 3
LET B = 5
LET C = MIN(A,B)
WRITE A B C
SET WRITE DECIMALS 0
WRITE A B C
.
LET X1 = NORMAL RANDOM NUMBERS FOR I = 1 1 10
LET X2 = NORMAL RANDOM NUMBERS FOR I = 1 1 10
LET Y = MIN(X1,X2)
WRITE X1 X2 Y
.
PLOT MIN(X,1) FOR X = -5 1 5
 
----------------------------------------
 
2.2
Maximum Function MAX(X,Y)
 
LET A = 3
LET B = 5
LET C = MAX(A,B)
WRITE A B C
SET WRITE DECIMALS 0
WRITE A B C
.
LET X1 = NORMAL RANDOM NUMBERS FOR I = 1 1 10
LET X2 = NORMAL RANDOM NUMBERS FOR I = 1 1 10
LET Y = MAX(X1,X2)
WRITE X1 X2 Y
.
PLOT MAX(X,1) FOR X = -5 1 5
 
----------------------------------------
 
2.3
Positive Difference Function DIM(X,Y)
Note--The Positive Difference of x and y = x - min(x,y)
      LET A = 3
      LET B = 5
      LET C = DIM(A,B)
      WRITE A B C
      SET WRITE DECIMALS 0
      WRITE A B C
      .
      LET X1 = NORMAL RANDOM NUMBERS FOR I = 1 1 10
      LET X2 = NORMAL RANDOM NUMBERS FOR I = 1 1 10
      LET Y = DIM(X1,X2)
      WRITE X1 X2 Y
      .
      PLOT DIM(X,1) FOR X = -5 1 5
 
----------------------------------------
 
3
Absolute Value, Square Root, Exponential, & Log Functions
   1. Absolute Value Function ABS(X)
   2. Square Root Function SQRT(X)
   3. Exponential Function EXP(X)
   4. Logarithmic Functions LOG(X), LOG10(X), LOG2(X)
 
----------------------------------------
 
3.1
Absolute Value Function ABS(X)
      LET A = -7.2
      LET B = ABS(A)
      LET C = 8.6
      LET D = ABS(D)
      WRITE A B C D
      .
      LET X = DATA -2 -5 3 15 -6
      LET Y = ABS(X)
      WRITE X Y
      SET WRITE DECIMALS 1
      WRITE X Y
      .
      PLOT ABS(X) FOR X = -10 1 10
 
----------------------------------------
 
3.2
Square Root Function SQRT(X)
      LET A = 2
      LET B = SQRT(A)
      LET C = -2
      LET D = SQRT(C)
      WRITE A B C D
      .
      LET X = SEQUENCE 1 1 10
      LET Y = SQRT(X)
      WRITE X Y
      SET WRITE DECIMALS 3
      WRITE X Y
      .
      PLOT SQRT(X) FOR X = 0 .1 10
 
----------------------------------------
 
3.3
Exponential Function EXP(X)
      LET A = 2
      LET B = EXP(A)
      LET C = -2
      LET D = EXP(C)
      WRITE A B C D
      .
      LET X = SEQUENCE 1 .1 3
      LET Y = EXP(X)
      WRITE X Y
      SET WRITE DECIMALS 3
      WRITE X Y
      .
      PLOT EXP(X) FOR X = -5 .1 5
 
----------------------------------------
 
3.4
Logarithmic Functions LOG(X), LOG10(X), LOG2(X)
Note--LOG(X) is the natural logarithm.
      LN(X) and LOGE(X) are synonyms of LOG(X).
    --LOG10(X) is the logarithm to the base 10.
    --LOG2(X)  is the logarithm to the base 2.
 
      LET A = 2
      LET LE = LOG(A)
      LET L10 = LOG10(A)
      LET L2 = LOG2(A)
      WRITE A LE L10 L2
      .
      LET X = SEQUENCE 1 1 10
      LET YE = LOG(X)
      LET Y10 = LOG10(X)
      LET Y2 = LOG2(X)
      WRITE X YE Y10 Y2
      .
      PLOT LOG(X) FOR X = .1 .1 10
 
----------------------------------------
 
4
Error & Gamma Functions
   1. Error Function ERF(X)
   2. Complementary Error Function ERFC(X)
   3. Gamma Function GAMMA(X)
   4. Log (to base e) Gamma Function LOGGAMMA(X)
 
----------------------------------------
 
4.1
Error Function ERF(X)
      LET A = 1.5
      LET B = ERF(A)
      LET C = -1.5
      LET D = ERF(C)
      WRITE A B C D
      .
      LET X = SEQUENCE -2 .1 2
      LET Y = ERF(X)
      WRITE X Y
      SET WRITE DECIMALS 3
      WRITE X Y
      .
      LINES SOLID DOTTED
      PLOT ERF(X) FOR X = -3 .1 3 AND
      PLOT ERFC(X) FOR X = -3 .1 3
 
----------------------------------------
 
4.2
Complementary Error Function ERFC(X)
      LET A = 1.5
      LET B = ERFC(A)
      LET C = -1.5
      LET D = ERFC(C)
      WRITE A B C D
      .
      LET X = SEQUENCE -2 .1 2
      LET Y = ERFC(X)
      WRITE X Y
      SET WRITE DECIMALS 3
      WRITE X Y
      .
      LINES SOLID DOTTED
      PLOT ERF(X) FOR X = -3 .1 3 AND
      PLOT ERFC(X) FOR X = -3 .1 3
 
----------------------------------------
 
4.3
Gamma Function GAMMA(X)
      LET A = 1.5
      LET B = GAMMA(A)
      LET C = -1.5
      LET D = GAMMA(C)
      WRITE A B C D
      .
      LET X = SEQUENCE .1 .1 4
      LET Y = GAMMA(X)
      WRITE X Y
      SET WRITE DECIMALS 3
      WRITE X Y
      .
      LINES SOLID DOTTED
      PLOT GAMMA(X) FOR X = .1 .1 4 AND
      PLOT LOGGAMMA(X) FOR X = .1 .1 4
 
----------------------------------------
 
4.4
Log (to base e) Gamma Function LOGGAMMA(X)
      LET A = 1.5
      LET B = LOGGAMMA(A)
      LET C = -1.5
      LET D = LOGGAMMA(C)
      WRITE A B C D
      .
      LET X = SEQUENCE .1 .1 4
      LET Y = LOGGAMMA(X)
      WRITE X Y
      SET WRITE DECIMALS 3
      WRITE X Y
      .
      LINES SOLID DOTTED
      PLOT GAMMA(X) FOR X = .1 .1 4 AND
      PLOT LOGGAMMA(X) FOR X = .1 .1 4
 
----------------------------------------
 
5
Chebychev & Bessel Functions
   1. Chebychev Functions CHEB0(X) to CHEB10(X)
   2. Bessel Functions BESS0(X) and BESS1(X)
 
----------------------------------------
 
5.1
Chebychev Functions CHEB0(X) to CHEB10(X)
Note--These are Chebychev Polynomials of First Kind--T0(x) to T10(x)
      LET A = 1.5
      LET C0 = CHEB0(A)
      LET C1 = CHEB1(A)
      LET C6 = CHEB6(A)
      WRITE A C0 C1 C6
      .
      LET X = SEQUENCE .1 .1 4
      LET Y0 = CHEB0(X)
      LET Y2 = CHEB2(X)
      LET Y7 = CHEB7(X)
      WRITE X Y0 Y2 Y7
      .
      LINES SOLID DASHED DOTTED
      PLOT CHEB0(X) FOR X = -1 .01 1 AND
      PLOT CHEB1(X) FOR X = -1 .01 1 AND
      PLOT CHEB2(X) FOR X = -1 .01 1
 
----------------------------------------
 
5.2
Bessel Functions BESS0(X) AND BESS1(X)
Note--These are the Bessel Functions of order 0--J0(x) and J1(x)
      LET A = 1.5
      LET B0 = BESS0(A)
      LET B1 = BESS1(A)
      WRITE A B0 B1
      .
      LET X = SEQUENCE .1 .1 4
      LET J0 = BESS0(X)
      LET J1 = BESS1(X)
      LET NU = 1
      LET J2 = (2*NU/X)*J1-J0
      WRITE X J0 J1 J2
      .
      LINES SOLID DASHED DOTTED
      PLOT BESS0(X) FOR X = .1 .1 10 AND
      PLOT BESS1(X) FOR X = .1 .1 10
 
----------------------------------------
 
6
Trig Functions (Circular)
      RADIANS;. (set the trig units to radians (= the default))
      LET A = PI/6
      LET B1 = SIN(A); LET B2 = COS(A); LET B3 = TAN(A)
      LET B4 = COT(A); LET B5 = SEC(A); LET B6 = CSC(A)
      DEGREES;. (change the trig units to degrees)
      LET C = 30
      LET D1 = SIN(C); LET D2 = COS(C); LET D3 = TAN(C)
      LET D4 = COT(C); LET D5 = SEC(C); LET D6 = CSC(C)
      WRITE A B1 B2 B3 B4 B5 B6
      WRITE C D1 D2 D3 D4 D5 D6
      .
      DEGREES; LET THETA = SEQUENCE 0 10 1440
      LET X = THETA*COS(THETA); LET Y = THETA*SIN(THETA)
      PRE-SORT OFF; PLOT Y X
      LET X2 = SIN(THETA); LET Y2 = SIN(1.5*THETA); PLOT Y2 X2
      .
      RADIANS;. (change the trig units back to radians)
      PLOT SIN(X)*EXP(-X/10) FOR X = 0 .1 10
      SPIKE ON
      PLOT SIN(X)*EXP(-X/10) FOR X = 0 .1 10
 
----------------------------------------
 
7
Inverse Trig Functions (Circular)
      RADIANS;. (set the trig units to radians (= the default))
      LET A = 0.5
      LET B1 = ARCSIN(A); LET B2 = ARCCOS(A); LET B3 = ARCTAN(A)
      LET B4 = ARCCOT(A); LET B5 = ARCSEC(A); LET B6 = ARCCSC(A)
      DEGREES;. (change the trig units to degrees)
      LET C1 = ARCSIN(A); LET C2 = ARCCOS(A); LET C3 = ARCTAN(A)
      LET C4 = ARCCOT(A); LET C5 = ARCSEC(A); LET C6 = ARCCSC(A)
      WRITE A B1 B2 B3 B4 B5 B6
      WRITE A C1 C2 C3 C4 C5 C6
      .
      RADIANS;. (change the trig units back to radians)
      LET X = SEQUENCE -1 .1 1
      LET Y = ARCSIN(X); WRITE X Y
      .
      LINES SOLID DASHED DOTTED
      PLOT ARCSIN(X) FOR X = -1 .1 1 AND
      PLOT ARCCOS(X) FOR X = -1 .1 1 AND
      PLOT ARCTAN(X) FOR X = -3 .1 3
      .
     . THE FOLLOWING 2-ARGUMENT ARCTAN FUNCTION IS FROM FORTRAN--
      LET E1 = ATAN2(1,.5); LET E2 = ATAN2(1,1); WRITE E1 E2
 
----------------------------------------
 
8
Trig Functions (Hyperbolic)
      LET A = 1.5
      LET B1 = SINH(A); LET B2 = COSH(A); LET B3 = TANH(A)
      LET B4 = COTH(A); LET B5 = SECH(A); LET B6 = CSCH(A)
      WRITE A B1 B2 B3 B4 B5 B6
      .
      LET X = SEQUENCE -2 .1 2
      LET Y1 = SINH(X)
      LET Y2 = COSH(X)
      WRITE X Y1 Y2
      .
      LINES SOLID DASHED DOTTED
      PLOT SINH(X) FOR X = -3 .1 3 AND
      PLOT COSH(X) FOR X = -3 .1 3 AND
      PLOT TANH(X) FOR X = -3 .1 3
 
----------------------------------------
 
9
Inverse Trig Functions (Hyperbolic)
      LET A = 0.5
      LET B1 = ARCSINH(A); LET B2 = ARCCOSH(A); LET B3 = ARCTANH(A)
      LET B4 = ARCCOTH(A); LET B5 = ARCSECH(A); LET B6 = ARCCSCH(A)
      WRITE A B1 B2 B3 B4 B5 B6
      .
      LET X = SEQUENCE -2 .1 2
      LET Y = ARCSINH(X)
      WRITE X Y
      .
      LINES SOLID DASHED DASHED DOTTED
      PLOT ARCSINH(X) FOR X = -3 .1 3 AND
      PLOT ARCCOSH(X) FOR X = -3 .1 1 AND
      PLOT ARCCOSH(X) FOR X = 1 .1 3 AND
      PLOT ARCTANH(X) FOR X = -.95 .1 .95
 
----------------------------------------
 
10
Octal/Decimal Conversion Functions
      LET A = 10
      LET B = OCTDEC(A)
      LET C = DECOCT(A)
      SET WRITE DECIMALS 0
      WRITE A B C
      .
      LET D = SEQUENCE 1 1 20
      LET O = DECOCT(D)
      WRITE D O
      .
      LET O = DATA 1 2 3 4 5 6 7 10 11 12 13 14 15 16 17 20 21
      LET D = OCTDEC(O)
      WRITE O D
 
----------------------------------------
 
11
Probability Functions
   1. Normal Distribution
   2. t Distribution
   3. Chi-squared Distribution
   4. F Distribution
   5. Weibull Distribution
   6. Inverse Gaussian Distribution
   7. Wald Distribution
   8. Reciprocal Inverse Gaussian Distribution
    Fatigue Life Distribution
 
----------------------------------------
 
11.1
Normal Distribution
      LET A = 2
      LET B = NORPPF(2); LET C = NORCDF(A)
      LET C = NORPPF(.975)
      .
      LET X = -2 .1 2
      LET PDF = NORPDF(X)
      LET CDF = NORCDF(X)
      LET P = .01 .01 .99
      LET PPF = NORPPF(P)
 
      PLOT NORPDF(X) FOR X = -2 .1 2 AND
      PLOT NORCDF(X) FOR X = -2 .1 2
      PLOT NORPPF(P) FOR P = .01 .01.99
      .
      READ MARSHAK.DAT Y
      LET XBAR = MEAN Y; LET S = STANDARD DEVIATION Y
      XLIMITS 21000 24000; YLIMITS 0 .4; PRE-ERASE OFF; ERASE
      HISTOGRAM Y
      PLOT NORPDF((X-XBAR)/S) FOR X = 21000 10 24000
 
----------------------------------------
 
11.2
t Distribution
      LET A = TPDF(2,5)
      LET B = TCDF(2,5)
      LET C = TPPF(.975,5)
      LET D = TPPF(.975,100)
      .
      LET X = SEQUENCE -5 .1 5
      LET Y1 = TPDF(X,1)
      LET Y2 = TPDF(X,2)
      LET Y3 = TPDF(X,3)
      PLOT Y1 Y2 Y3 VERSUS X
      .
      LET NU = SEQUENCE 1 1 30
      LET Y = TPPF(.975,NU)
      PLOT Y NU
      .
      PLOT TPDF(X,3) FOR X = -5 .1 5
      PLOT TPPF(.975,NU) FOR NU = 1 1 30
 
----------------------------------------
 
11.3
Chi-squared Distribution
      LET A = CHSPDF(2,5)
      LET B = CHSCDF(2,5)
      LET C = CHSPPF(.95,5)
      LET D = CHSPPF(.95,100)
      .
      LET X = SEQUENCE -5 .1 5
      LET Y1 = CHSPDF(X,1)
      LET Y2 = CHSPDF(X,2)
      LET Y3 = CHSPDF(X,3)
      PLOT Y1 Y2 Y3 VERSUS X
      .
      LET NU = SEQUENCE 1 1 30
      LET Y = CHSPPF(.95,NU)
      PLOT Y NU
      .
      PLOT CHSPDF(X,3) FOR X = -5 .1 5
      PLOT CHSPPF(.95,NU) FOR NU = 1 1 30
 
----------------------------------------
 
11.4
F Distribution
      LET A = FPDF(2,5,7)
      LET B = FCDF(2,5,7)
      LET C = FPPF(.95,5,7)
      LET D = FPPF(.95,100,7)
      .
      LET X = SEQUENCE -5 .1 5
      LET Y1 = FPDF(X,1,4)
      LET Y2 = FPDF(X,2,4)
      LET Y3 = FPDF(X,3,4)
      PLOT Y1 Y2 Y3 VERSUS X
      .
      LET NU1 = SEQUENCE 1 1 30
      LET NU2 = 5
      LET Y = FPPF(.95,NU1,NU2)
      PLOT Y NU1
      .
      PLOT FPDF(X,3,5) FOR X = -5 .1 5
      PLOT FPPF(.95,NU1,10) FOR NU1 = 1 1 30
 
----------------------------------------
 
11.5
Weibull Distribution
      LET A = WEIPDF(2,5); LET B = WEICDF(2,5)
      LET C = WEIPPF(.95,5); LET D = WEIPPF(.95,100)
      .
      LET X = SEQUENCE .1 .1 5
      LET Y1 = WEIPDF(X,1)
      LET Y2 = WEIPDF(X,2)
      LET Y3 = WEIPDF(X,3)
      PLOT Y1 Y2 Y3 VERSUS X
      .
      LET GAMMA = SEQUENCE 1 1 30
      LET Y = WEIPPF(.95,GAMMA)
      PLOT Y GAMMA
      .
      PLOT WEIPDF(X,3) FOR X = .1 .1 5
      PLOT WEIPPF(.95,GAMMA) FOR GAMMA = 1 1 30
      LET T0 = 50; LET SCALE = 10
      PLOT T0+SCALE*WEIPPF(P,3) FOR X = 50 1 100
 
----------------------------------------
 
11.6
Inverse Gaussian Distribution
      LET A = IGPDF(2,5); LET B = IGCDF(2,5)
      LET C = IGPPF(.95,5); LET D = IGPPF(.95,100)
      .
      LET X = SEQUENCE .1 .1 5
      LET Y1 = IGPDF(X,1)
      LET Y2 = IGPDF(X,2)
      LET Y3 = IGPDF(X,3)
      PLOT Y1 Y2 Y3 VERSUS X
      .
      LET GAMMA = SEQUENCE 1 1 30
      LET Y = IGPPF(.95,GAMMA)
      PLOT Y GAMMA
      .
      PLOT IGPDF(X,3) FOR X = .1 .1 5
      PLOT IGPPF(.95,GAMMA) FOR GAMMA = 1 1 30
      LET T0 = 50; LET SCALE = 10
      PLOT T0+SCALE*IGPPF(P,3) FOR X = 50 1 100
 
----------------------------------------
 
11.7
Wald Distribution
      LET A = WALPDF(2,5); LET B = WALCDF(2,5)
      LET C = WALPPF(.95,5); LET D = WALPPF(.95,100)
      .
      LET X = SEQUENCE .1 .1 5
      LET Y1 = WALPDF(X,1)
      LET Y2 = WALPDF(X,2)
      LET Y3 = WALPDF(X,3)
      PLOT Y1 Y2 Y3 VERSUS X
      .
      LET GAMMA = SEQUENCE 1 1 30
      LET Y = WALPPF(.95,GAMMA)
      PLOT Y GAMMA
      .
      PLOT WALPDF(X,3) FOR X = .1 .1 5
      PLOT WALPPF(.95,GAMMA) FOR GAMMA = 1 1 30
      LET T0 = 50; LET SCALE = 10
      PLOT T0+SCALE*WALPPF(P,3) FOR X = 50 1 100
 
----------------------------------------
 
11.8
Reciprocal Inverse Gaussian Distribution
      LET A = RIGPDF(2,5); LET B = RIGCDF(2,5)
      LET C = RIGPPF(.95,5); LET D = RIGPPF(.95,100)
      .
      LET X = SEQUENCE .1 .1 5
      LET Y1 = RIGPDF(X,1)
      LET Y2 = RIGPDF(X,2)
      LET Y3 = RIGPDF(X,3)
      PLOT Y1 Y2 Y3 VERSUS X
      .
      LET GAMMA = SEQUENCE 1 1 30
      LET Y = RIGPPF(.95,GAMMA)
      PLOT Y GAMMA
      .
      PLOT RIGPDF(X,3) FOR X = .1 .1 5
      PLOT RIGPPF(.95,GAMMA) FOR GAMMA = 1 1 30
      LET T0 = 50; LET SCALE = 10
      PLOT T0+SCALE*RIGPPF(P,3) FOR X = 50 1 100
 
 
----------------------------------------
 
11.9
Fatigue Life Distribution
      LET A = FLPDF(2,5); LET B = FLCDF(2,5)
      LET C = FLPPF(.95,5); LET D = FLPPF(.95,100)
      .
      LET X = SEQUENCE .1 .1 5
      LET Y1 = FLPDF(X,1)
      LET Y2 = FLPDF(X,2)
      LET Y3 = FLPDF(X,3)
      PLOT Y1 Y2 Y3 VERSUS X
      .
      LET GAMMA = SEQUENCE 1 1 30
      LET Y = FLPPF(.95,GAMMA)
      PLOT Y GAMMA
      .
      PLOT FLPDF(X,3) FOR X = .1 .1 5
      PLOT FLPPF(.95,GAMMA) FOR GAMMA = 1 1 30
      LET T0 = 50; LET SCALE = 10
      PLOT T0+SCALE*FLPPF(P,3) FOR X = 50 1 100
 
----------------------------------------
 
12
Indicator Function IND(X,TAG)
Note--IND(X,TAG) is useful for fitting simultaneous equations
      when 1) wish to have a common fitted additive constant
      across all equations, and 2) not all fitted equations
      have the same number of data points being fitted.
 
      LET X   = DATA 1   2   3   4   5      1   2   3
      LET Y   = DATA 1.0 2.1 2.9 4.1 5.0    0.5 1.0 1.5
      LET TAG = DATA 1   1   1   1   1       2   2   2
      CHARACTERS 1 2
      PLOT Y X TAG
      LET X1 = X*IND(TAG,1)
      LET X2 = X*IND(TAG,2)
      FIT Y X1 X2
      PLOT Y X TAG AND
      PLOT PRED X TAG
 
----------------------------------------
 
13
Binary Pattern Function BINPAT(X,K)
Note--This function may be used in the generation of 2**k
      factorial designs after run sequence randomization.
 
      LET SEQ = SEQUENCE 1 1 16
      LET X1 = BINPAT(SEQ,1)
      LET X2 = BINPAT(SEQ,2)
      LET X3 = BINPAT(SEQ,3)
      LET X4 = BINPAT(SEQ,4)
      SET WRITE DECIMALS 0
      WRITE SEQ X1 X2 X3 X4
      .
      LET SEQ = RANDOM PERMUTATION FOR I = 1 1 16
      LET X1 = BINPAT(SEQ,1)
      LET X2 = BINPAT(SEQ,2)
      LET X3 = BINPAT(SEQ,3)
      LET X4 = BINPAT(SEQ,4)
      WRITE SEQ X1 X2 X3 X4
 
----------------------------------------
 
 
 
 
 
 
 
 
 
 
 
 
 
 
 
 
 
 
 
 
 
 
 
 
 
 
 
 
 
 
 
 
 
 
 
 
 
 
 
 
 
 
 
 
 
 
 
 
 
 
 
 
 
 
 
 
 
 
 
 
 
 
 
 
 
 
 
 
 
 
 
 
 
 
 
 
 
 
 
 
 
 
 
 
 
 
 
 
 
 
 
 
 
 
 
 
 
 
 
 
