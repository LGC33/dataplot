. This is (jjf's) Dataplot login file dplogf.tex
. (Customized for Jim Filliben running on a Windows 2000 PC  2/3/05)
. (Updated: 9/14/12:  Alan's checksys.dp)
. (Updated: 9/25/12): Change the   set fatal error   setting from the default (ignore) to prompt
. (Updated: 1/7/16):  For NIST DEX Class
.
feedback off
.
.  User Login File
.
.  The following options are available as command line switches:
.
.  -true    - use if your video display is set to "true color" mode
.             (if no text shows up in the text window)
.  -svga    - set size of frame window based on assumption that you
.             are running in Super VGA mode (i.e., 1024x768)
.  -large   - set size of frame window based on assumption that you
.             are running with high resolution (e.g, 1180x1024)
.  -tile    - have text and graphics window come up in "tiled", i.e.,
.             no overlap, mode.
.  (JJF Preference: -tile and -large)
.
. ---Step 1: Define Your Graphics Output Devices------------
.      Device 1 is your computer screen.
.      Devices 2 and 3 are (graphics) output files on your computer.
.
.         1. SET DEVICE 1 (SCREEN) TO QWIN (I..E., FOR THE COMMAND LINE
.            VERSION (GUI VERSION WILL AUTOMATICALLY OVERRIDE THE CHOICE
.            GIVEN HERE).
.
.         2. FOR DEVICE 2 AND DEVICE 3 (file devices), THERE ARE 3 CASES:
.
.              i) You want to access a Postscript printer
.             ii) You want to access a printer with HP-GL emulation
.                 (e.g., most HP LaserJet printers)
.            iii) You want to access a generic printer that has neither
.                 Postscript or HPGL capability.
.
.            Set IDEV to 1, 2, or 3 based on the above.  Also, you may
.            need to define the name of your network printer (the actual
.            name will need to be obtained from your local site
.            system administrator).
.
DEVICE 1 QWIN;. (since I am running on a PC)
.
. DEVICE 1 FONT SIMPLEX   ;. this may be commented out to speed up character plots 1/16/14 (but bad side effects)
DEVICE 1 FONT TEKTRONIX;. due to an Alan update that sped up graphics by factor of 40! (but has char offsets) 5/9/16
CHAR OFFSET -.05 .35;. use this when use DEVICE 1 FONT TEKTRONIX
.
. SET QWIN FONT VERDONNA;.  due to an Alan update that sped up graphics by a factor of 40   5/9/16
.          in fact commented out 5/9/16 due to Alan's update on font support
SET QWIN COLOR DIRECT;. (To activate on-screen multiple colors)
. DEVICE 3 CLOSE
. DEVICE 3 HPGL LASER
.
SET POSTSCRIPT DEFAULT COLOR ON;. inserted 9/1/11 to allow dynamic color viewing of dppl2f.dat
.
LET IDEV = 1
.
IF IDEV = 1
   . Postscript Case.  Set printer to PRN: for a local printer, to
   . the appropriate network id for a network printer.
   SET PRINTER PRN:
   . SET PRINTER "\\DCIS-NT3\HP8_PS"  12/1/04
   DEVICE 2 POSTSCRIPT
ELSE IF IDEV = 2
   . HP-GL Case.  Set printer to PRN: for a local printer, to
   . the appropriate network id for a network printer.
   . SET PRINTER PRN:
   SET PRINTER "\\DCIS-NT3\HP8"
   DEVICE 2 HP-GL LJET
ELSE
  . Generic printer case.  Use the Ghostscript/Ghostview "GSPRINT" command.
  . This assumes that Ghostscript (at least version 6.5) and Ghostview have
  . been installed.  If the path for Ghostview is not "C:\Ghostgum\Gsview",
  . Then set the correct path with the "SET GHOSTVIEW PATH" command.
  . If you do not want to access your default printer, then use the
  . SET PRINTER command to specify the printer id.
  . SET PRINTER "\\DCIS-NT3\HP8 - HP 8000 Series PCL 5e"
  . SET PRINTER \\DCIS-NT3\HP8_PS - HP LaserJet 8000 Series PS
  SET GHOSTSCRIPT PRINTER ON
  SET POSTSCRIPT PATH F:\GHOSTGUM\GSVIEW\
  . SET GHOSTSCRIPT PATH "C:\Program Files\gs\gs8.54";. changed 9/1/11 at Alan's direction
  . SET GHOSTVIEW PATH C:\GHOSTGUM\GSVIEW\;.            changed 9/1/11 at Alan's direction
END IF
.
. ---Step 2: Define Your Browser (optional) (updated by Alan 10/09/14)---------------------
. -----        Note: this is needed only by the    WEB   command which spins off a separate job to view a web page-----
.
. SET BROWSER "C:\program files\netscape\navigator\program\netscape.exe"
.
. Define the browser (Uncomment the relevant line)
.
. LET IBROWSE = 1 ; . Internet Explorer (the default)
LET IBROWSE = 2 ; . Mozilla Firefox;. This setting is for Jim Filliben at NIST (64-bit Dell)
. LET IBROWSE = 3 ; . Mozilla Seamonkey
. LET IBROWSE = 4 ; . Google Chrome
. LET IBROWSE = 5 ; . Opera
.
. LET IOPS = 1 ; . Windows XP/7/8 32-bit
LET IOPS = 2 ; . Windows XP/7/8 64-bit;. This setting is for Jim Filliben at NIST (64-bit Dell)
.
IF IBROWSE = 1  ; . Internet Explorer
   IF IOPS = 1  ; . 32-bit
     SET BROWSER "C:\program files\Internet Explorer\iexplore.exe"
   ELSE         ; . 64-bit
     SET BROWSER "C:\program files (x86)\Internet Explorer\iexplore.exe"
   END OF IF
END OF IF
.
IF IBROWSE = 2  ; . Mozilla Firefox
   IF IOPS = 1  ; . 32-bit
     SET BROWSER "C:\program files\Mozilla Firefox\firefox.exe"
   ELSE         ; . 64-bit
     SET BROWSER "C:\program files (x86)\Internet Explorer\iexplore.exe"
   END OF IF
END OF IF
IF IBROWSE = 3  ; . Mozilla Seamonkey
   IF IOPS = 1  ; . 32-bit
     SET BROWSER "C:\program files\SeaMonkey\seamonkey.exe"
   ELSE         ; . 64-bit
     SET BROWSER "C:\program files (x86)\SeaMonkey\seamonkey.exe"
   END OF IF
END OF IF
IF IBROWSE = 4  ; . Google Chrome
   IF IOPS = 1  ; . 32-bit
     SET BROWSER "C:\program files\Google\Chrome\Application\chrome.exe" -disk-cache-size=1 -media-cache-size=1
   ELSE         ; . 64-bit
     SET BROWSER "C:\program files (x86)\Google\Chrome\Application\chrome.exe" -disk-cache-size=1 -media-cache-size=1
   END OF IF
END OF IF
IF IBROWSE = 5  ; . Opera
   IF IOPS = 1  ; . 32-bit
     SET BROWSER "C:\program files\Opera\opera.exe"
   ELSE         ; . 64-bit
     SET BROWSER "C:\program files (x86)\Opera\opera.exe"
   END OF IF
END OF IF
.
. ---Step 3: Define Dataplot and NIST Handbook URL's (optional)----
.
. Dataplot has a web site.
. The NIST e-Handbook of Statistical Methods (a useful 3000-page
. companion resource) also has a web site.
. By default, the following Dataplot and e-Handbook URL's
. are for the NIST facility.
. If you have installed (free via CD) the Dataplot and e-Handbook
. onto your local site, then change the following URL's:
.
SET DATAPLOT URL http://www.itl.nist.gov/div898/software/dataplot/
SET HANDBOOK URL http://www.itl.nist.gov/div898/handbook/
.
. ---Step 4: Define Any Dataplot Personal Preferences (optional)-----
.
. ---Preference 1: Plot Appearance:
.
. The following sets default foreground and background colors.
. Set to suit your own taste.
. Some prefer yellow foreground on a blue background.
. I (Jim Filliben) prefer a black foreground on a white background--
. it is better for me for printing, and for day-in/day-out data analysis.
.
     . background color blue
     . let string fore = yellow
     background color white
     let string fore = black
     .
     frame color ^fore
     line color ^fore all
     character color ^fore all
     spike color ^fore all
     bar color ^fore all
     bar border color ^fore all
     tic mark color ^fore
     tic mark label color ^fore
     label color ^fore
     title color ^fore
     legend color ^fore
     bar fill color ^fore
     region fill color ^fore
     segment color ^fore
     color ^fore
     grid color ^fore
     box color ^fore
.
. ---Preference 2: Data Workspace Number of Variables:
.    I like 40 variables (rather than the 10 variable default)
.    for the data workspace:
.
.    dimension 40 variables
.    dimension 50 variables;. changed from 40 to 50 due to dexplot.dp
     dimension 100 variables;. changed from 50 to 100 due to dex10stepanalysis.dp
.
. ---Preference 3: Print Formatting:
.    I like decimal (rather than exponential) default for printing:
.
     set write decimals 4
.
. ---Preference 4: Plot Appearance:
.    I like margins (spacings) within the frames on my plots:
.
     tic offset units screen
     tic offset 5 5
.
. ---Preference 5: Plot Appearance:
.    I like upper and lower case text on my plots:
.
     case asis
     title case asis
     label case asis
     tic label case asis
     character case asis all
     legend case asis
.
. ---Preference 6: Command Execution Feedback:
.    I like automatic feedback for executed commands:
.
. ---Preference 7: Title Positioning
.    I like the title to be closer to the top frame line
.
     title displacement 2
.
. ---Preference 8: Box Appearance
.    I like the box to have no shadow
.
     box shadow hw 0 0
.
. ---Preference 9: Auto-Annotation of Plot Command
.    I like the command which generates a plot to be printed
.    immediately below the plot (in x3label)
.
     x3label automatic
.
. ---Preference 10: Resolution of Output Graphics
.    I like output graphics to be 600 dpi rather than the default 300 dpi
.
     set postscript dpi 600
.
. ---Preference 11: Color Graphics (7/27/10)-----
.    I like plots with blue lines, characters, frame lines, labels, etc.
.
     . call jjfcolor.dp
     device 2 color on
     set postscript font helvetica bold;. versus arial or verdanna or . . .
     let string foreground = blue
     let string background = white
     .
     . call color.dp
     if foreground not exist; let string foreground = white; end if
     if background not exist; let string background = blue; end if
     .
     background color ^background
     .
     char color ^foreground all
     line color ^foreground all
     bar border color ^foreground all
     spike color ^foreground all
     frame color ^foreground
     tic color ^foreground
     tic label color ^foreground
     label color ^foreground
     title color ^foreground
     legend color ^foreground
     color ^foreground
     grid color ^foreground
     .
     let string darkgree = 76
     let string olivegre = 10
     .
     let string darkgree = 76
     let string olivegre = 10
.
. ---Preference 12: Header Size-----
.    I like the 3 header lines above a plot to be larger than the default of     hw 2 1
.
let string headhw1 = 2.6 1.3
let string headhw2 = 2.6 1.3
let string headhw3 = 2.6 1.3
.
. ---Preference 13: Label and tic label size-----
.    I like the plot labels and tic labels to be larger than the default of    height = 2
.
label size 2.6
.
. ---Preference 14 (9/2/11): I like to be able to see ghostview's version of t
.
. let string show = system gvdp2.bat
let string show = system c:\ghostgum\gsview\gsview32.exe dppl2f.dat
.
. ---Preference 15 (9/14/12): Add Alan's checksys.dp to
.                             1. determine if running in batch mode or GUI
.                             2. appropriatly sets device 1 to X11 (if linux) or QWIN (if Windows)
.                             3. checks device 2--if not open, then open as Postscript
.                                                 if already open, then do nothing
. call checksys.dp
.
. ---Preference 15: Set Fatal Error---
.    When Dataplot encounters a fatal error, I like Dataplot's response to be a prompt message
.    (asking as to whether to continue or to exit),
.    as opposed to the default Dataplot response (= ignore and execute onward)
.
set fatal error prompt
.
. ---Preference 16: Print out the date for this version of dataplot-----
.
. print " "
print "Dataplot Version Date: 04/18/13
.
. ---Preference 17: Print out note about lost cursor-----
.
. print " "
. print "Executing c:\0\dpx\folder.tex
. print " "
. print "Note: If cursor gets 'lost' off bottom of the text"
. print "      (= left) screen, hit  Page Down  key to restore."
.
. ---Preference 18: Print out note about all graphics being captured in dppl1f.dat (modified 1/7/16)-----
.
. print " "
print "      After a Dataplot run, postscript versions of"
print "      all graphics are captured in dataplot's"
print "      graphics output file     dppl1f.dat"
print "      for viewing (& conversion to pdf) by ghostscript"
print "      via e.g.    gvdp.bat"
print " "
. print "      If   dppl1f.dat   incorrectly contains only 1"
. print "      plot (or no plots), then fix by inserting"
. print "      a   device 2 postscript    command"
. print "      into your dataplot login file   dplogf.tex  ."
.
. ---Preference 19: Show version-----
.
. version
.
. ---Preference 20: Allow 4-plot output to be controlled by the    multiplot corner coordinates    command---
.
.    When the 4-plot command is used, the default output is
.    sitting a bit high on the page to the point that there is not as much
.    room to write annotations on top of the page.
.    The following command will solve that problem.
.    Note: in the program, it is recommended that the user
.          enter (immediately before the 4plot command):
.             multiplot 5 10 95 90
.
set 4plot multiplot on
.
. ---Preference 21 (Temporary--for use with GUI and folder c:\0\dpx): link to c:\0\---
.                   where many of my data files and macros exist     11/21/14---
.
set search directory c:\0\
.
. ---Preference 22: Define (and print) a string    version   which defines
.                   the version ID for this dataplot exec utable---
.
let string versiond = Executable: 10/17/18
. print " "
print "      ^versiond"
.
. ---Preference 23: Define a string command    g3    which will display device 3
.                   (= the postscript version of the last plot generated)          6/16/16
.    (Note: this assumes that a gvdp.bat batch file exists in your path,
.    and that file calls ghostscript to render the contents of    dppl12.dat
.    and example of gvdp.bat:
.       c:\progra~1\ghostgum\gsview\gsview64.exe dppl2f.dat
.    To use the    g3   "command" within dataplot,
.    then after a lower-resolution plot has been generated on the screen in the usual fashion
.    (for example, via    plot x for x = 1 1 10),
.    enter   ^g3   to generate a higher-resolution of the same plot in a separate window.
.
. ---Preference 24: Define path on my PC to ghostview (used by the    psview    command)
.
set ghostscript version 64
let string stpath = c:\progra~1\gs\gs9.07\bin\
set ghostscript path ^stpath
.
. here
. ---Preference 25 (4/9/18): Note the question-mark-based search commands---
.
let printsw = 0
if printsw = 1
   print " "
   print "      ? = General  search of Dataplot Reference Manual, as in:"
   print "          ? distributions   ? plot   ? fit   ? let"
   print "          ? lognormal distribution   ? block plot   ? gallery   etc."
   print " "
   print "     ?? = Specific search of Dataplot Reference Manual, as in:"
   print "          ?? distributions   ?? plot   ?? fit   ?? let"
   print "          ?? lognormal distribution    ?? block plot   ?? gallery   etc."
   print " "
   print "    ??? = General  search of NIST/SEMATECH Handbook, as in:"
   print "          ??? gallery   ??? fit   ??? 10 step"
   print "          ??? main effects plot   ??? effects plot"
   print "          ??? halfnormal plot     ??? fitting   ??? 4plot"
   print "          ??? lognormal   ??? distributions   ??? spc"
   print "          ??? block plot  ??? gallery"
   print " "
   print "   ???? = Specific search of NIST/SEMATECH Handbook, as in:"
   print "          ???? gallery   ???? fit   ???? 10 step"
   print "          ???? main effects plot   ???? effects plot"
   print "          ???? halfnormal plot     ???? fitting   ???? 4plot"
   print "          ???? lognormal   ???? distributions   ???? spc"
   print "          ???? block plot   ???? gallery"
   print " "
   print "  ????? = Specific search of internet via Google, as in:"
   print "          ????? gallery   ????? fit   ????? 10 step"
   print "          ????? main effects plot   ????? effects plot"
   print "          ????? halfnormal plot     ????? fitting   ????? 4plot"
   print "          ????? lognormal   ????? distributions   ????? spc"
   print "          ????? block plot   ????? gallery"
end if
.
. -----All done----
.
feedback on
print " "
. echo on
