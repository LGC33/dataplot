9.
                                                     April     2020
This is the DATAPLOT News file DPNEWF.TEX.  This NEWS file contains a
list of DATAPLOT enhancements.  Some commands may be documented here
that are not yet available in the standard documentation.

The complete news file can be viewed at

    http://www.itl.nist.gov/div898/software/dataplot/news.htm

-----------------------------------------------------------------------
The following enhancements were made to DATAPLOT
July 2020 - August 2020.
-----------------------------------------------------------------------

 1) Made the following updates to string functions.

    a) The LET ... = STRING SPLIT command now saves the number of new
       strings generated in the parameter NUMBWORD.

    b) The LET ... = NUMBER OF WORDS command now supports the 
       SET WORD DELIMITER option.  Previously, words were delineated by
       spaces (or any non-printing character).  The SET WORD DELIMITER
       option lets you specificy the character that will be treated as
       the delimiter.  For example, SET WORD DELIMITER , will treat
       the "," as the word delimiter.

 2) Many data files contain the variable names as the first line of the
    file.  Dataplot has the ability to extract these variable names by
    doing something like

        SET READ VARIABLE LABEL ON
        READ FILE.DAT

    A few tweaks were added to this capability.

    a) Previously, only the first 255 characters of the variable names
       line were read.  This has been updated to support the number
       of characters specified by the MAXIMUM RECORD LENGTH command.

    b) Dataplot will now automatically strip spaces and other special
       characters out of the variable names.  Specifically, only
       alphabetic characters (A-Z), numbers (0-9), and underscores are
       retained.

    c) Dataplot only supports eight characters for variable names.  This
       can lead to duplicate file names.  To reduce the possibility of
       duplicate names, Dataplot does the following if a duplicate name
       is found.

          i) If the name has less than eight characters, a "Z" is appended
             to the end of one of the names.  The right most name will
             be modified.

         ii) If the name has eight characters exactly, the right most
             name will change the last character to a Z (or if that
             character is already a Z, then to a X).

       If blank names are encountered, these will be changed to
       Zxxx where "xxx" is a sequence number (i.e., if there are three
       blank names encountered, they wiil be set to Z1, Z2, and Z3).

  3) The Dataplot GUI is somewhat out of date.  We will be updating the
     GUI in several stages.

     a) The first stage is to update the contents of the menu.  This
        update has started, but not completed, that process.

     b) After the menu contents are updated, we will begin revising
        the Tcl/Tk scripts.

  4) Added the command

         SET CHARACTER TABULATION PLOT DIGITS <value>

  5) Added the commands

         HALF NORMAL MAXIMUM LIKELIHOOD Y
         HALF LOGISTIC MAXIMUM LIKELIHOOD Y

     Note that the half-logistic case actually computes a method
     of moments estimate.

  6) A number of bug fixes were made.

-----------------------------------------------------------------------
The following enhancements were made to DATAPLOT
May 2020 - June 2020.
-----------------------------------------------------------------------

 1) The Dataplot web site now supports installation of ".rpm" files
    for CentOS 7, CentOS 8, Fedora 30, Fedora 31 and Fedora 32.

 2) Added the following graphics commands.

    a) Added the following plot

          TOTAL TIME ON TEST PLOT Y CENSOR

 3) Added the following Math LET sub-commands.

      LET Y = TOTAL TIME ON TEST        X CENSOR
      LET Y = SCALED TOTAL TIME ON TEST X CENSOR

 4) Corrected an issue where commas inside a character field enclosed
    in quotes were still being interpreted as field delimiters.  Commas
    inside a quoted field are now treated as part of the character
    field and not a delimiter.

 5) Made a number of corrections to address warning messages generated
    by version 10 of the gfortran/gcc compilers (Fedora 32).

 6) Support for accessing the clipboard has been extended to Linux
    (previously this was only supported on Windows platforms).  On
    Linux, access to the clipboard is through the "xclip" utility.
    Note that xclip is not installed by default on most Linux systems,
    so you may need to install it for CLIPBOARD commands to work under
    Linux.  The following CLIPBOARD commands are currently supported

         LIST CLIPBOARD
         COPY CLIPBOARD <file>
         COPY <file> CLIPBOARD
         CLEAR CLIPBOARD

         READ CLIPBOARD  <var-list>
         WRITE CLIPBOARD <var-list>

         CALL CLIPBOARD  <file>

-----------------------------------------------------------------------
The following enhancements were made to DATAPLOT
March 2020 - April 2020.
-----------------------------------------------------------------------

 1) The SET SEARCH DIRECTORY command has been enhanced to allow five
    additional directories that can added to Dataplot's list of directories
    to search for file names.

       SET SEARCH2 DIRECTORY <directory-name>
       SET SEARCH3 DIRECTORY <directory-name>
       SET SEARCH4 DIRECTORY <directory-name>
       SET SEARCH5 DIRECTORY <directory-name>
       SET SEARCH6 DIRECTORY <directory-name>

 2) Added the command

       OUTPUT  <name>

    This command executes the following command

       DEVICE 2 CLOSE
       SET IPL1NA <name>.ps
       DEVICE 2 POSTSCRIPT

 3) Added the command

       SET WRITE CSV  <ON/OFF>

    If this is set to ON, the WRITE command will use commas rather
    than spaces as the delimiter between fields.  This is primarily
    intended for importing into other programs that may require
    comma separated variables (CSV) for ASCII files.

 4) For the commands BEST DISTRIBUTIONAL FIT and DISTRIBUTIONAL FIT PLOT,
    you can now specify which distributions to include.  Enter
    HELP BEST DISTRIBUTIONAL FIT command for details.

 5) The READ EXCEL command was updated so that you can specify the
    first and last rows of the Excel file to read.

        SET EXCEL START ROW <value>
        SET EXCEL STOP  ROW <value>

    One use of this is to skip over header lines in the Excel file.

 6) The maximum number of characters for the command line was
    increased from 255 to 1024.  The maximum number of characters
    for a file name was increased from 80 to 256.

 7) A large number of changes were made to remove warning messages
    when using a more strigent level of warning messages for the
    gfortran and Intel compilers.  These changes do not change
    any Dataplot commands, but it did correct several potential
    bugs.

-----------------------------------------------------------------------
The following enhancements were made to DATAPLOT
September 2019 - February 2020.
-----------------------------------------------------------------------

 1) The following operating system dependent commands were added

      RM      <file-list>    - remove one or more files
      RMDIR   <file-list>    - remove one or more directories
      MKDIR   <file-list>    - make a new directory
      CAT     <file>         - list the contents of a file
      DIR     <file-list>    - list files
      GREP <string> <file-list> - perform an operating system based
                                  matching of a string to one or more
                                  files
      RSCRIPT <file-name>    - run an R script
      PYTHON <file-name>     - run a Python script

    Enter HELP RM, HELP MKDIR, HELP CAT, HELP DIR, HELP GREP,
    HELP RSCRIPT or HELP PYTHON for details.

 2) The following updates were made to the READ and WRITE commands.

    a) Added the commands

          WRITE EXCEL <excel-file>  <var-list>
          READ  EXCEL <excel-file>  <var-list>

       Note that these commands invoke Python scripts to read/write the
       Excel files.  So a requirement for using these commands is that
       version 3.x of Python is already installed on your system and that
       the Pandas and xlsxwriter packages are also installed.

       For additional information on using these commands, enter
       HELP WRITE EXCEL or HELP READ EXCEL.

    b) Reading character data from the terminal is now permitted.
       For example,

          SET CONVERT CHARACTER ON
          READ IMONTH VALUE
          January    21205
          February   19867
          March      24991
          April      16523
          May        17341
          June       27912
          July       29105
          August     28766
          September  23332
          October    20211
          November   18298
          December   13112
          END OF DATA

    c) When reading a data file that contains character variables,
       Dataplot saves the character data to the file "dpzchf.dat".
       If a subsequent READ command also contains character variables,
       the contents of dpzchf.dat are overwritten.

       To specify that new character variables should be appended to
       the current dpzchf.dat file, enter the command

           SET CHARACTER VARIABLE APPEND

       To reset the default, enter

           SET CHARACTER VARIABLE OVERWRITE

       If you specify APPEND, it is recommended that you delete the
       current dpzchf.dat file at the beginning of the Dataplot session
       with the command

           RM dpzchf.dat

    d) Some ASCII files may include a percent sign at the end of a
       numeric field to indicate the value is a percentage.  To ignore
       these percent signs, enter the command

           SET READ PERCENT SIGN IGNORE ON

    e) For the READ CLIPBOARD command on Windows platforms, made two
       corrections.

         i. If the last column contained an empty field, this missing
            value was not added to the data read.  This caused a
            misalingment of the data.  This was corrected.

        ii. If the number of variables on the READ command did not
            match the number of columns in the clipboard, this caused a
            misalignment of the data.  This has been corrected.

            If the number of variables on the READ CLIPBOARD command is
            greater than the number of columns in the clipboard, the
            extra columns will be set to the missing value.  This value
            can be specified with the command

                SET READ MISSING VALUE <value>

 3) Added the following graphics commands.

    a) Added the following option to the I PLOT command

          RATIO OF MEANS CONFIDENCE LIMIT PLOT Y1 Y2 X

 4) Added the following analysis commands.

    a) Generate a confidence interval for the ratio of two means
       (i.e., E(Y)/E(X)) for the case of paired data where both X and Y
       are approximately normal

          RATIO OF MEANS CONFIDENCE LIMITS Y X

       Note that numerous methods have been proposed for this problem.
       Dataplot currently supports the Fieller method, the large sample
       approximation method, and the log ratio method.  To specify which
       method is used, enter

          SET RATIO OF MEANS METHOD FIELLER
          SET RATIO OF MEANS METHOD LARGE SAMPLE
          SET RATIO OF MEANS METHOD LOG RATIO

       The default is the Fieller method.  

    b) Added the following outlier commands

         DAVID TEST Y
         SKEWNESS OUTLIER TEST Y
         KURTOSIS OUTLIER TEST Y

       In addition, the GRUBS TEST now supports the case where the
       standard deviation is available from previous data.  For
       this case, enter the commands

            SET GRUBB STANDARD DEVIATION <value>
            SET GRUBB DEGREES OF FREEDOM <value>

       If the specified standard deviation is positive, Dataplot
       uses the formulas based on an independent estimate of the
       standard deviation.  The independent standard deviation also
       has an associated degrees of freedom (typically the sample
       size used to compute that standard deviation).  If the
       degrees of freedom is not specified, a value of 10,000 will
       be used.  Essentially, any value greater than 120 is
       effectively treated as a "known" standard deviation.

       These new outlier commands were added to support the
       2016 edition of the ASTM-178 standard for outliers.

 5) Added the following Statistics LET sub-commands.

      LET A = RATIO OF MEANS Y X
      LET A = RATIO OF MEANS LOWER CONFIDENCE LIMIT Y X
      LET A = RATIO OF MEANS UPPER CONFIDENCE LIMIT Y X

      LET A = DAVID TEST Y
      LET A = DAVID TEST CDF Y
      LET A = DAVID TEST PVALUE Y
      LET A = DAVID TEST MINIMUM INDEX Y
      LET A = DAVID TEST MAXIMUM INDEX Y

      LET A = SKEWNESS OUTLIER TEST Y
      LET A = SKEWNESS OUTLIER TEST CDF Y
      LET A = SKEWNESS OUTLIER TEST PVALUE Y
      LET A = SKEWNESS OUTLIER TEST CRITICAL VALUE Y
      LET A = SKEWNESS OUTLIER TEST INDEX Y

      LET A = KURTOSIS OUTLIER TEST Y
      LET A = KURTOSIS OUTLIER TEST CDF Y
      LET A = KURTOSIS OUTLIER TEST PVALUE Y
      LET A = KURTOSIS OUTLIER TEST CRITICAL VALUE Y
      LET A = KURTOSIS OUTLIER TEST INDEX Y

 6) Added the following Math LET sub-commands.

      LET Y3 = INSERT Y1 Y2 NLOC

 7) Added the following commands for strings.

    a) The LET ... = STRING COMBINE ... command is used to concatenate 
       two or more strings.  By default, this command puts a space
       between the concatenated strings.  To specify a different 
       separator character, enter the command

           SET STRING COMBINE SEPARATOR <string>

    b) Added the command

          let ix = string variable s1 s3 s3

       where s1, s2 and s3 are previously defined strings.  This adds
       "ix" to the list of character variables in the character variable
       file dpzchf.dat.

 8) Added the following miscellanous commands.

    a) Added the following options to the LIST command

          LIST HEAD <file>
          LIST TAIL <file>

       LIST HEAD will list the first 10 lines of the file and LIST TAIL
       will list the last 10 lines of the file.

       To modify the number of lines the HEAD and TAIL options print,
       enter

          SET HEAD LINES <value>
          SET TAIL LINES <value>

       Also added the option

          LIST NEW WINDOW <file>

       This will open up the file in a new window.  For Windows, the
       default application is Wordpad.  For Linux, a gnome terminal window
       will be opened and the file will be displayed in the vi editor.

       To change the default application, enter

          SET LIST VIEWER <name>

       Note that Dataplot does no error checking on this name.  If
       an invalid application is given, the file will not be displayed.

       To make opening a new window the default for the LIST command,
       enter

          SET LIST NEW WINDOW ON

       To reset the default of listing the contents in the Dataplot
       window, enter

          SET LIST NEW WINDOW OFF

       Similarly, you can open the output from the HELP command in a
       new window by entering

          SET HELP NEW WINDOW ON

       To reset the default enter

          SET HELP NEW WINDOW OFF

       For Linux systems, you can specify which command is used
       to launch the new window (for either the LIST or HELP
       command) with the command

          SET LIST LAUNCHER  "gnome-terminal -e"

       This is the default.  Other choices include

          SET LIST LAUNCHER  "xterm -e"
          SET LIST LAUNCHER  "konsole -e"

       There are many desktops available on various types of Linux
       systems and each of these desktops may have its own
       command for initiating a new terminal window.  We have
       explicitly tested xterm, konsole, and gnome-terminal on
       CentOS and Fedora.  However, depending on what desktop you
       use, these options may not be available on your local system.

       In addition to LIST NEW WINDOW, the following are also available

          LIST EXCEL <file-name>
          LIST WORD <file-name>
          LIST POWER POINT <file-name>

       On Windows systems, the default is equivalent to entering
       the following command

             SYSTEM "<file-name>"

       On Linux systems, the default is the equivalent to entering

             SYSTEM xdg-open "<file-name>"

       On MacOS systems, the default is the equivalent to entering

             SYSTEM open "<file-name>"

       The "xdg-open" and "open" commands under Linux and MacOS
       will select the application based on the file name extension.
       Typically there will be a file name association defined for
       many file name extensions.  However, there may be file name
       extensions for which no association has been defined.  In this
       case, or if you simply want to be explicit about what application
       to use, you can specify which application will be used with the
       following commands

             SET EXCEL VIEWER        "<application-name>"
             SET WORD VIEWER         "<application-name>"
             SET POWER POINT VIEWER  "<application-name>"

       Dataplot does no error checking to see if <application-name>
       is in fact installed on your system.

       For example, to explicitly use libreoffice applications
       under Linux, you could enter

             SET EXCEL VIEWER       "libreoffice --calc"
             SET WORD  VIEWER       "libreoffice --writer"
             SET POWER POINT VIEWER "libreoffice --impress"

       These commands are not restricted to Microsoft Office
       applications.

       A few comments on this.

           i. Dataplot does not check the file name extension.
              You need to explicitly use LIST EXCEL, LIST WORD,
              or LIST POWER POINT to invoke the application.

          ii. Once the application is invoked, control returns
              to the Dataplot window.  So you can view the
              spreadsheet or document while still entering Dataplot
              commands.

         iii. There are a large number of spreadsheet and word
              processing programs each which tends to have their
              own file extensions.  Although the Microsoft extensions
              (.xls, .xlsx, .doc, .docx, .ppt, .pptx) are likely to
              have file associations defined on most systems, this is
              less likely to be true for other spreadsheet or word
              processing programs.  In this case, you can either
              create the file association or use the SET EXCEL VIEWER,
              SET WORD VIEWER, or SET POWER POINT VIEWER commands to
              specify the desired application.

    b) The PSVIEW command was updated to view PDF and image files
       in addition to Postscript files.  Enter HELP PSVIEW for
       details.

    c) Added the following commands

          HEAD <var-list>
          TAIL <var-list>

       HEAD and TAIL are synonyms for the WRITE command.  However, only
       the first (or last) 10 lines will be printed.

       To modify the number of lines the HEAD and TAIL options print,
       enter

          SET HEAD LINES <value>
          SET TAIL LINES <value>

    d) Added the following command

          SET OUTPUT LINE NUMBERS <ON/OFF>

       If this switch is ON, the alphanumeric output will
       contain line numbers.

    e) Fixed a bug where the ONE SAMPLE T TEST was not
       interpreted correctly.

    f) Corrected several of the random number generators to
       properly reset the sequence when a new SEED value is
       entered.

    g) If you enter a command without arguments and the command
       is not matched, then Dataplot will try adding  PRINT to the
       beginning of the command.  That is, the command

           Y

       willl be interpreted as 

           PRINT Y

       However, be aware that a check is made to see if it is
       a legitimate command first.  For example, R, REPEAT, X,
       S, SAVE, L, and LIST are valid commands without arguments.
       So if you have a variable called R, you need to enter
       PRINT R rather than just R.

    h) The TIC MARK LABEL FORMAT VARIABLE option now supports
       character variables as well as numeric variables.

    i) The following command returns the current value of the
       seed for various random number generators

           PROBE SEED

    j) The following commands return the values for certain
       operating system defined enviornment variables.

           PROBE HOME  (or PROBE USER PROFILE)
           PROBE USER  (or PROBE USER NAME)
           PROBE HOST  (or PROBE HOST NAME or PROBE COMPUTER NAME)
           PROBE DEFAULT PRINTER (Linux only)
           PROBE PROGRAM FILES X86 (64-bit Windows only)
           PROBE PROGRAM FILES     (Windows only)
           PROBE WINDOW BITS       (Windows only)

      Specifically, PROBE HOME returns the user's home directory,
      PROBE USER returns the user name, PROBE HOST returns the computer
      name, PROBE DEFAULT PRINTER returns the name of the default
      printer, PROBE PROGRAM FILES X86 returns the location of
      64-bit applications under Windows, PROBE PROGRAM FILES returns
      the location of 32-bit applications under Windows, and
      PROBE WINDOW BITS returns "32" if you are running on a
      32-bit machine and "64" if you are running on a 64-bit
      machine.

      Remember that you can define a string after the PROBE.  For
      example,

          PROBE USER
          LET STRING USERNAME = PROBESTR

    k) To include your home directory in the list of directories that
       are searched when looking for a file enter

          SET HOME PATH ON

       To reset the default of not including your home directory, enter

          SET HOME PATH OFF

       To see what your home directory is, enter

          PROBE HOME

    l) Previously if a command is recognized as a file name, Dataplot
       would interpret this as a CALL command.  For example

           test.dp

       would be equivalent to

           call test.dp

       This has been expanded to recognize certain file extensions.
       Specifically, if <file-name> is the command

           i) If the file has a ".ps", ".PS", ".eps" or ".EPS" extension,
              the following will be done

                  PSVIEW  <file-name>

              This command will view the file using the Postscript
              viewer.  The SET POSTSCRIPT VIEWER command can be used to
              specify what application will be used to view the
              Postscript file.

          ii) If the file has a ".pdf" or ".PDF" extension, the following
              will be done

                  PSVIEW  <file-name>

              This command will view the file using the PDF viewer.  The
              SET PDF VIEWER command can be used to specify what
              application will be used to view the PDF file.

         iii) If the file has a ".jpg", ".JPG", ".jpeg", ".JPEG", ".png",
              ".PNG", ".gif", ".GIF", ".tif", ".TIF", ".tiff",  or ".TIFF"
              extension, the following will be done

                  PSVIEW  <file-name>

              This command will view the file using the image viewer.  The
              SET IMAGE VIEWER command can be used to specify what
              application will be used to view the image file.

          iv) If the file has a ".dat", ".DAT", ".csv", ".CSV", ".out",
              or ".OUT" extension, the following will be done

                  LIST NEW WINDOW  <file-name>

              The command SET LIST VIEWER can be used to specify what
              application is used to view ASCII files.

           v) If the file has a ".xls", ".XLS", ".xlsx", or ".XLSX"
              extension, the following will be done

                  LIST EXCEL  <file-name>

              If the file has a ".doc", ".DOC", ".docx", or ".DOCX"
              extension, the following will be done

                  LIST WORD  <file-name>

              If the file has a ".ppt", ".PPT", ".pptx", or ".PPTX"
              extension, the following will be done

                  LIST POWER POINT  <file-name>

              The following commands can be used to set the viewers
              for these types of files

                  SET EXCEL VIEWER
                  SET WORD VIEWER
                  SET POWER POINT VIEWER

-----------------------------------------------------------------------
The following enhancements were made to DATAPLOT
May 2019 - August 2019.
-----------------------------------------------------------------------

 1) The Dataplot source code is now available on the following
    github site

        https://github.com/usnistgov/dataplot

    The build from source code has been simplified for Linux and
    MacOS systems.

 2) Numerous changes were made to make more efficient use of scratch
    space.  This allowed us to increase the default maximum number of
    rows from 1,500,000 to 2,000,000.

 3) Made the following changes to the Graphics commands.

      a. For the CONSENSUS MEANS PLOT, the default is to put the method
         results on the left side of the plot and the lab data results
         on the right side of the plot.  To reverse this (i.e., data
         results first, then method results), enter the command

            SET CONSENSUS MEAN PLOT DATA LEFT

         You can also specify RIGHT, but this is equivalent to the
         default of ON.

      b. Added the following option for the BOX PLOT command

            SET BOXPLOT FENCE SKEWNESS <OFF/GALTON/KIMBER>

         For skewed datasets, the default box plot algorithm for
         identifying outliers may identify an excessive number of
         outliers when the FENCES switch is turned on.  Several
         authors have suggested alternative algorithms to address
         this.  Enter HELP BOX PLOT for details.

 4) Made the following changes to the Analysis commands.

      a. Added the following option for the ONE SAMPLE PROFICIENCY TEST
         command

            SET ONE SAMPLE PROFICIENCY TEST IDENTIFY LAB
                <DEFAULT/UNUSUAL/EXTREMELY UNUSUAL>

         For Table 2, if there are multiple labs with the same values,
         only one lab-id is given (there is a column for the number of
         occurrences for that value).  For outliers, it can be useful
         to identify all the labs.  If UNUSUAL is specified, all lab-id's
         are given for the Unusual and Extremely Unusual categories.  If
         EXTREMELY UNUSUAL is specified, all lab-id's are given for the
         Extremeley Unusual category (but not the Unusual category).

 5) Added the following Math LET sub-commands.

      LET Y = BREAK LOCATIONS X
      LET Y = FRAGMENT LOCATIONS X
      LET Y = FRAGMENT LENGTHS X

 6) Added the following Statistics LET sub-commands.

      LET A = LOWER SEMI-INTERQUARTILE RANGE Y
      LET A = UPPER SEMI-INTERQUARTILE RANGE Y

      LET A = KENDALLS TAU A  Y1 Y2
      LET A = KENDALLS TAU B  Y1 Y2
      LET A = YULES Y  Y1 Y2

      LET A = CORRELATION RATIO Y X
      LET A = INTRACLASS CORRELATION RATIO Y X

 7) Miscellaneous Updates

    a. Added the following option to the SET POSTSCRIPT CONVERT
       command

          SET POSTSCRIPT CONVERT PS2PDF

       The SET POSTSCRIPT CONVERT command is used to automatically
       convert Dataplot's Postscript output to another format (most
       commonly PDF, but not limited to this).  Previously, this
       command supported "GHOSTSCRIPT" and "CONVERT".  GHOSTSCRIPT
       uses Ghostscript to perform the conversion and CONVERT uses
       the "convert" command from ImageMagick.

       The PS2PDF option uses the ps2pdf program to perform the
       conversion.  The ps2pdf program is a batch script file that
       is typically installed as part of the Ghostscript installation.
       The GHOSTSCRIPT option generates a Ghostscript command to
       perform the conversion, so the GHOSTSCRIPT and PS2PDF options
       perform essentially the same conversion.  The advantage of the
       PS2PDF option is that it has a simpler syntax.  The advantage
       of the GHOSTSCRIPT option is that it is not limited to
       converting the Postscript to PDF format (it also supports
       JPEG, TIFF, PBM, PBM, PGM, PPM, and PNM).

-----------------------------------------------------------------------
The following enhancements were made to DATAPLOT
April 2019 - April (18) 2019.
-----------------------------------------------------------------------

 1) Made the following changes to the Analysis commands.

      i. For linear fits (i.e., FIT Y X, QUADRATIC FIT Y X, and so on), a
         header line is now written to the dpst1f.dat, dpst2f.dat, and
         dpst4f.dat files (dpst3f.dat already had a header line).

         For non-linear fits, a header line is now written to the dpst1f.dat
         and dpst2f.dat files.

 2) I/O updates

      i. If you are reading a space delimited file in free format mode
         and a row of the file has fewer values than expected, the
         default behavior was modified.  Previously, the number of
         variables read would be equal to the minimum number of values
         for a row.  For example, if you entered

             READ FILE.DAT Y X1 X2 X3

         and a row was encountered that had only two values, then only
         Y and X1 would be read.  This was changed so that when the number
         of values for a row is less than the expected number of values,
         the missing value number (as specified by the SET READ MISSING
         VALUE command) will be inserted.  So in the above example, the
         two values will be inserted into Y and X1 while X2 and X3 will be
         assigned the missing value code.

         Note that this can still be problematic if the missing fields are
         not the end columns (values will be assigned to variables in the
         order specified).  This is the reason for printing the warning
         message.

         This new behavior is equivalent to

              SET READ PAD MISSING COLUMNS ON

         The one difference is that SET READ PAD MISSING COLUMNS ON will
         not print the warning message.

     ii. Added the following SET command

            SET READ ASTERISK IGNORE <ON/OFF>

         You may on occassion need to read a data file where an asterisk
         ("*") is used to identify certain points.  For example, this may
         indicate a statistic that is above a critical value or it may
         be used to identify points that are of particular interest.  In
         this case, you can set the value to ON to ignore leading or
         trailing asterisks.

    iii. Fixed a bug with the vector form of the COLUMN LIMITS command.

 3) Miscellaneous

      i. Added the following command

           SET AUXILIARY FILES DECIMAL POINTS <value>

         Certain commands write information to the files dpst1f.dat,
         dpst2f.dat, dpst3f.dat, dpst4f.dat and dpst5f.dat.  Typically,
         numbers are written in E15.7 format.  This command allows you to
         specify the number of significant digits to use (currently,
         between 1 and 15 are allowed).  For example, if you enter

           SET AUXILIARY FILES DECIMAL POINTS 9

         then an E17.9 format will be used.

         This option will be implemented incrementally.  The initial
         implementation was for the fit commands, but support for
         additional commands will be added in an arbitraty order.
         Currently the following commands support this option

           FIT, ORTHOGONAL DISTANCE FIT, ARMA (for dpst1f.dat, but not
           dpst2f.dat), ODDS RATIO CHI-SQUARE TEST, ODDS RATIO
           INDEPENDENCE TEST, OPTIMIZE, PAGE TEST, CAPABILITY ANALYSIS,
           and CHI-SQUARE INDEPENDENCE TEST

         Output to these files that does not use E15.7 format will not
         use this option.

-----------------------------------------------------------------------
The following enhancements were made to DATAPLOT
August 2018 - March 2019.
-----------------------------------------------------------------------

 1) Made the following updates to the graphics commands

      i) Added the following command

           CLASSIFICATION SCATTER PLOT Y X1 ... XK
           CLASSIFICATION <stat> PLOT  Y X1 ... XK

         Enter HELP CLASSIFICATION SCATTER PLOT or
         HELP CLASSIFICATION STATISTIC PLOT for details.

     ii) For the DEX <stat> PLOT, added the command

            SET DEX STATISTIC PLOT INTERACTION <NONE/2/3>

         By default, the DEX <stat> PLOT command shows the main effects.
         If you specify "2" for this command, all 2-term interactions
         will be added to the plot and if you specify "3", all 2-term
         and 3-term interactions will be added to the plot.

         This option is only intended for 2-level full and fractional
         factorial designs.

    iii) For the BLOCK <statistic> PLOT, if the CHARACTER MAPPING
         is set to EXACT, suppress the coding of the level id's.

         This can be useful when there are a large number of levels
         for the primary factor where these levels are basically
         an index but that index has missing values.

 2) Added the following analysis commands

      HEDGES G CONFIDENCE LIMIT Y1 Y2

 3) Added the following LET statistic subcommands

      LET A  = HEDGES G STANDARD ERROR Y1 Y2
      LET A  = HEDGES G LOWER CONFIDENCE LIMIT Y1 Y2
      LET A  = HEDGES G UPPER CONFIDENCE LIMIT Y1 Y2

      LET A  = HAMMING DISTANCE Y1 Y2
      LET A  = CANBERRA DISTANCE Y1 Y2
      LET A  = GROUPED CORRELATION Y1 Y2 W
      LET A  = WEIGHTED COSINE DISTANCE Y1 Y2 W
      LET A  = WEIGHTED COSINE SIMILARITY Y1 Y2 W

      LET LOWLIMIT = value
      LET UPPLIMIT = value
      LET A = INTERVAL COUNT Y

      LET A = PYTHON MEAN Y
      LET A = YOUDEN INDEX Y1 Y2

 4) Added the following LET matrix subcommands

      LET MOUT = COSINE <COLUMN/ROW> DISTANCE M
      LET MOUT = COSINE <COLUMN/ROW> SIMILARITY M
      LET MOUT = ANGULAR COSINE <COLUMN/ROW> DISTANCE M
      LET MOUT = ANGULAR COSINE <COLUMN/ROW> SIMILARITY M
      LET MOUT = JACCARD <COLUMN/ROW> DISTANCE M
      LET MOUT = JACCARD <COLUMN/ROW> SIMILARITY M
      LET MOUT = PEARSON <COLUMN/ROW> DISTANCE M
      LET MOUT = PEARSON <COLUMN/ROW> SIMILARITY M
      LET MOUT = HAMMING <COLUMN/ROW> DISTANCE M
      LET MOUT = CANBERRA <COLUMN/ROW> DISTANCE M

 5) Added the following STRING subcommands

      LET SOUT = STRING SPLIT SORG
      LET SOUT = STRING REMOVE PUNCTUATION SORG
      LET SOUT = STRING REMOVE WHITESPACE SORG
      LET SOUT = STRING EXPAND WHITESPACE SORG
      LET SOUT = STRING DELETE SORG SDEL
      LET SOUT = STRING RIGHT INDEX SORG  SCHAR
      LET SOUT = STRING <LEFT/CENTER/RIGHT> JUSTIFY SORG NLEN
      LET SOUT = SWAP CASE SORG

      LET IFLAG = STRING STARTS WITH SORG SMATCH
      LET IFLAG = STRING ENDS   WITH SORG SMATCH
      LET IFLAG = STRING CONTAIN SORG SMATCH
      LET NC    = STRING SUBSET COUNT SORG SMATCH

      LET D = STRING HAMMING DISTANCE S1 S2

 6) Added the following LET math subcommands

      LET Y2 = CELL MATCH X VALUE

      LET Y2 = LARGEST Y NVAL
      LET Y2 = SMALLEST Y NVAL

      LET Y = <YTIC/Y1TIC/Y2TIC/XTIC/X1TIC/X2TIC>  <DATA/SCREEN> COORDINATES

      LET TAG = DEX CHECK CENTER POINTS X1 ... XK
      LET XNEW = CODE DEX 2-LEVEL X

 7) Made the following updates to the I/O commands

    a) Made the following updates to the STREAM READ command.

         i) Added the additional distance options

               STREAM READ CANBERRA DISTANCE <file>  <var-list>
               STREAM READ HAMMING DISTANCE <file>  <var-list>

        ii) Added the distance cross-tabulation options

               STREAM READ CROSS TABULATE EUCLIDEAN DISTANCE <file>  <var-list>
               STREAM READ CROSS TABULATE MANHATTAN DISTANCE <file> <var-list>
               STREAM READ CROSS TABULATE CHEBYCHEV DISTANCE <file> <var-list>
               STREAM READ CROSS TABULATE COSINE DISTANCE <file> <var-list>
               STREAM READ CROSS TABULATE COSINE SIMILARITY <file> <var-list>
               STREAM READ CROSS TABULATE ANGULAR COSINE DISTANCE <file> <var-list>
               STREAM READ CROSS TABULATE ANGULAR COSINE SIMILARITY <file> <var-list>
               STREAM READ CROSS TABULATE CANBERRA DISTANCE <file> <var-list>
               STREAM READ CROSS TABULATE HAMMING DISTANCE <file> <var-list>
               STREAM READ CROSS TABULATE CORRELATION <file>  <var-list>
               STREAM READ CROSS TABULATE COVARIANCE <file>  <var-list>

       iii) Added the percentiles option

               STREAM READ PERCENTILES <file>  <var-list>

        iv) Added the percentiles cross-tabulation options

               STREAM READ CROSS TABULATE PERCENTILES <file>  <var-list>

    b) Dataplot is a column oriented program.  That is, columns
       denote variables whiles rows denote observations.

       Sometimes you may encounter data files that are row
       oriented, that is rows denote variables while columns
       denote observations.  This is often the case when the
       number of variables is significantly greater than the
       number of observations.

       To better accomodate these types of data files, the
       following command was added

           READ ROW <file.dat>  Y

       In this case, the file is read one row at a time
       and each row is added as a Dataplot variable.  Only
       a single variable name is listed.  Note that this
       variable name serves as "base name".  So if Y is
       the variable name and there are 25 rows of data,
       variables Y1, Y2, ..., Y25 will be created by the
       READ ROW command.

       Currently, READ ROW is only supported for numeric
       data.  The rows do not need to contain the same
       number of elements.

       If the maximum number of available columns is reached,
       the READ ROW command will be terminated.  However,
       any rows that have already been successfully read
       will still be retained.  If there in error in reading
       a specific row, that row will be skipped and Dataplot
       will go to the next row.

       Similarly, the following command has been added for
       writing variables in a row-wise fashion

           WRITE ROW <file.dat>  <var-list>

       where <var-list> is the list of variables to write.

       The primary reason for adding the WRITE ROW command
       is to make it easy to create a version of the
       row oriented file that can be read using the
       SET READ FORMAT command.  This can significantly
       speed up the READ ROW command at the expense of
       creating larger data files.

    c) Added the following commands

          WRITE1 <file-name> <var-list>
          WRITE2 <file-name> <var-list>
          WRITE3 <file-name> <var-list>

       These commands allow writing to three distinct files in
       append mode while still making normal use of the standard
       WRITE command.  Enter HELP WRITE for details on the usage
       of these commands.

 8) The IF commands now supports the following for strings

        IF S1 <  S2
        IF S1 <= S2
        IF S1 >  S2
        IF S1 >= S2

     where S1 and S2 are pre-defined strings.  The comparison is based
     on the ASCII collating sequence (so "A" is less than "a" and "C" is
     less than  "b").  The comparison is performed left to right.  If
     one string is shorter than the other, the shorter string will
     return 0 for the ASCII code when its length has been exceeded.

 9) Added the following plot control command

        ...LABEL COORDINATES XCOOR YCOOR

10) Added the following SET commands

       SET WRITE FEEDBACK <ON/OFF>
       SET WORD DELIMITER <VALUE>
       SET COMMAND SUBSTITUTION <ON/OFF>
       SET SUBSTITUTION FORMAT <STRING>
       SET CARRIAGE RETURN GAP <VALUE>
       SET CLIPBOARD RUN CLEAR <ON/OFF>

11) Made the following updates to the SYSTEM command.

    For the Windows 7/8/10 platforms, the SYSTEM command works
    as follows:

      i) A new terminal window is opened.
     ii) The requested command is executed.
    iii) When the requested command is completed, the new window is
         closed and control returns to the Dataplot session.

    Several SET commands have been added to control this behavior.
    Specifically,

      i) In some cases, it is desirable to leave the new Window up.
         For example, you may need to view the results from the
         SYSTEM command.  To specify that the new command window
         should remain, enter the command

              SET SYSTEM PERSIST ON

         To reset the default, enter

              SET SYSTEM PERSIST OFF

         Note that control does not return to the Dataplot session
         until the new terminal window is closed.

         This command has no effect on Unix/Linux and MacOS platforms.

     ii) In some cases, you may want the SYSTEM command to operate in
         the background and not open a new terminal window.  To specify
         this, enter the command

              SET SYSTEM HIDDEN ON

         For Windows, Dataplot normally executes the SYSTEM command with
         the SYSTEMQQ system call in the Intel Fortran library.  If
         HIDDEN is set to ON, Dataplot will use the EXECUTE_COMMAND_LINE
         routine that was added in the Fortran 2008 standard.  If you
         compile Dataplot for Windows using an older version of the Intel
         compiler or a non-Intel compiler, the EXECUTE_COMMAND_LINE
         subroutine may or may not be available.  If not, Dataplot will
         revert to using SYSTEMQQ (and HIDDEN ON will have no effect).

         To reset the default, enter

              SET SYSTEM HIDDEN OFF

         This command has no effect on Unix/Linux and MacOS platforms.

    iii) By default, Dataplot uses "CALL SYSTEM" for Linux/Unix and
         MacOS platforms and "CALL SYSTEMQQ" for Windows platforms.

         You can request that Dataplot use the EXECUTE_COMMAND_LINE
         subroutine instead by entering the command

              SET QWIN SYSTEM EXECUTE COMMAND LINE

         Be aware that this is relatively new addition to the Fortran
         standard and may not be available in all Fortran compilers (or
         in older versions of compilers).  Specifically, it is not
         supported in gfortran (Linux, MacOS) until version 5.x and
         version 16 of the Intel compiler.  So this option may not be
         available on all platforms.  The Windows executable for 2019/03
         that can be downloaded from the Dataplot web site uses version 17
         of the Intel compiler, so this feature is available.  If you are
         running Dataplot on a platform where EXECUTE_COMMAND_LINE is not
         available, Dataplot will revert to "CALL SYSTEM" or
         "CALL SYSTEMQQ".

         One advantage of using EXECUTE_COMMAND_LINE is that it supports
         either synchronous or asynchronous execution.  By synchronous, we
         mean that control does not return to the Dataplot session until
         the SYSTEM command completes execution.  By asynchronous, we
         mean that control returns to the Dataplot session after the
         SYSTEM command is initated (but not neccessarily completed).

         To specify asynchronous, enter

             SET COMMAND LINE EXECUTE WAIT ON

         To restore the default of synchronous, enter

             SET COMMAND LINE EXECUTE WAIT OFF

         Note that this command only applies if EXECUTE_COMMAND_LINE is
         used to implement the SYSTEM command.

12) Added the command

      PRINTFILE <filename>

    This command can be used to print an ASCII file from within a
    Dataplot session.

    Also, added the command

       COPY SYSTEM <file1> <file2>

    Without the SYSTEM option, the COPY command works by reading lines
    from <file1> and writing them to <file2>.  With the SYSTEM option,
    the COPY is implemented by using an appropriate operating system
    command ("COPY" for Windows platforms and "cp" for Linux/Unix and
    MacOS platforms).

13) Miscellaneous

      i) If you enter a command like

            Y = NORMAL RANDOM NUMBERS FOR I = 1 1 100

         Dataplot would previously return an error.  Dataplot was
         updated so that if a command is not matched, the command
         does not start with LET, and the command contains an
         "=" character, Dataplot will insert a "LET " at the
         beginning of the command string and try to match the command
         again.

         Note that using "LET" to start the command is still the
         preferred syntax since this new syntax is only attempted if
         the first word of the command line does not match a
         Dataplot command.  For example, "X", "R", "S", and "W"
         are short-cuts to existing commands, so the following
         will not work

             X = X + 1
             W = W + 1
             R = R + 1
             S = S + 1

     ii) Added the command

            SET TAB EXPAND <VALUE>

         By default, when Dataplot parses a command line it converts
         all non-printing characters to a single space.  This command
         allows you to specify how tab characters should be handled.

         If <VALUE> is 0, then the tab character is left as is.
         If <VALUE> is 1, then the tab character is replaced with
         a single space (i.e., the default Dataplot behavior).  If
         <VALUE> is a positive integer greater than 1, then the tab
         character will be replaced with <VALUE> spaces.  If <VALUE>
         is negative, it will be set to 1.  If <VALUE> is greater than
         20, it will be set to 20.

         In most cases, the default behavior should be preferred.  This
         command is most likely to be useful when processing tabs
         contained within strings.

    iii) When writing text on a plot (e.g., the TEXT, LEGEND, TITLE,
         LABEL commands), support was added for "tabs".  Note that
         tabs are denoted by a "TAB()" sequence, not a hard tab
         character.  This is supported with the following SET commands

            SET TAB HORIZONTAL POSITION  <INDEX> <VALUE>
            SET TAB VERTICAL   POSITION  <INDEX> <VALUE>
            SET TAB COLOR                <INDEX> <VALUE>
            SET TAB JUSTIFICATION        <INDEX> <VALUE>
            SET TAB FONT                 <INDEX> <VALUE>
            SET TAB UNITS                <INDEX> <VALUE>
            SET TAB VERTICAL UNITS       <INDEX> <VALUE>
            SET TAB SIZE                 <INDEX> <VALUE>
            SET TAB WIDTH                <INDEX> <VALUE>

         This feature is most useful in the context of creating
         table type text on a plot.

     iv) The SET POSTSCRIPT CONVERT option was updated so that a
         DEVICE 2 CLOSE (or DEVICE 3 CLOSE) command is no longer required
         before exiting Dataplot.

      v) Dataplot supports a built-in editor with the FED (or EDIT)
         command.  The built-in editor is a line-mode editor.  You can
         now specify the name of an editor of your choice by entering
         the command

             SET EDITOR <name>

         For example, on Windows platforms you can use

             SET EDITOR NOTEPAD
             SET EDITOR WORDPAD

         On Linux, you can use

             SET EDITOR vi
             SET EDITOR emacs

         If your desired editor is not in the default search path, then
         you need to include the complete path.  For example,

             SET EDITOR "C:\Program Files (x86)\notepad++\notepad++.exe"

     vi) Changed the default seed for random numbers from 305 to 3005.

    vii) Added CALL CLIPBOARD and CB as synonyms for CLIPBOARD RUN.

   viii) For Linux/Unix platforms, changed the default browser to use
         "xdg-open".  This will use the default browser on the local
         platform.

-----------------------------------------------------------------------
The following enhancements were made to DATAPLOT
June 2018 - July 2018.
-----------------------------------------------------------------------

 1) Made the following changes to the graphics commands.

    a) Added the following options to the KERNEL DENSITY PLOT command

          SET KERNEL DENSITY PROBABILITY FUNCTION <pdf/cdf/ppf>
          SET KERNEL DENSITY RANDOM NUMBERS <value>

       The KERNEL DENSITY PLOT estimates the underlying probability
       density function.  However, it can also be used to estimate
       the cumulative distribution function (cdf) or the percent point
       function (ppf).  To estimate the cdf, the cumulative integral of
       the kernel density plot is computed.  The ppf is inverse of the
       cdf, so the role of the x and y values from the estimated cdf are
       switched to obtain an estimate of the ppf function.

       To plot the estimated cdf, enter
 
          SET KERNEL DENSITY PROBABILITY FUNCTION CDF

       To plot the estimated ppf, enter
 
          SET KERNEL DENSITY PROBABILITY FUNCTION PPF

    Given that we can estimate the ppf function, we can use this to
    generate random numbers based on the kernel density plot.  If you
    would like to generate random numbers, enter a value between 1 and
    the maximum number of rows for the SET KERNEL DENSITY RANDOM NUMBERS
    command (if this value is set to 0 or a negative value, no random
    numbers will be generated).

    Specifically, the following procedure is used:

        i) Generate uniform random numbers (the uniform random numbers
           correspond to x-axis values on the ppf version of the kernel
           density plot).

       ii) From the ppf version of the kernel density plot, determine
           the y-axis value on the kernel density curve that corresponds
           to the x-axis value.  Cubic spline interpolation is used to
           estimate the y-axis value.  That is, at the points defined
           by the uniform random numbers, we find interpolated values
           based on the (x,y) coordinates of the kernel density curve.

      iii) The random numbers are written to the file dpst1f.dat.

 2) Made the following changes to the analysis commands.

    a) Updated the LIMITS OF DETECTION command to accept negative values.
       Negative data is handled by adding a constant to make the data all
       positive.  This constant is then subtracted off for the computed
       critical value.

 3) Added the following LET math subcommands

      LET YRANK = MEAN RANK Y XSEQ X1 ... XK

    Y is a response variable, X1 ... XK are group-id variables (up to
    six supported), and XSEQ is a sequence number variable.

    This command performs cross-tabulations based on the group-id
    variables and then ranks the Y values for each cell of the
    cross-tabulation.  For a given sequence number, the average rank over
    all cross-tabulation cells is returned.

    The sequence variable is used so that not all cross-tabulation cells
    need have the same number of values and also so that it is not
    required that the Y values be "ordered" (the sequence number defines the
    ordering) within a cell.

 4) Added the following LET statistic subcommands

      LET YCOV  = WEIGHTED COVARIANCE  Y1 Y2 W
      LET YCORR = WEIGHTED CORRELATION Y1 Y2 W

    Also, modified the definition of ANGULAR COSINE DISTANCE and
    ANGULAR COSINE SIMILARITY.  Specifically, the corrected formula is

      Angular Cosine Distance = AFACT*ARCCOS((COSINE SIMILARITY)/PI
      Angular Cosine Similarity = 1 - Angular Cosine Distance

    where AFACT is 2 if there are no negative values and 1 if there
    are negative values.  ARCCOS is the arccosine function.

 5) Added the following LET string subcommans

       LET IX = REFERENCE CHARACTER CODE IX IG

    This syntax allows you to specify the character string to numeric
    value mapping when converting character fields to numeric values.

 6) The following updates were made to the IO commands

    a) The PRINT command now supports group labels, row labels, and
       character variables.

    b) SET PRINT FORMAT is now an alias to SET WRITE FORMAT.

    c) Added the option

         SET CONVERT CHARACTER CATEGORICAL

       This is similar to SET CONVERT CHARACTER ON.  However, in
       addition to creating the character variable, it will create
       a numeric variable that converts the unique values of the
       character field to an integer code.  Currently, the code values
       are assigned in the order that the unique values are detected
       in the file.  Up to 1,000 levels are supported (if more than
       1,000 levels are required, the remaining levels are all set
       to "-1").

    d) Added the command

         SET ROW LABEL COLUMN <ival>

       If the SET CONVERT CHARACTER option is set either ON or
       CATEGORICAL, this command allows you to specify a column to be
       treated as a row label.  Row labels are typically the first
       column, but you are not restricted to that.  If the specified
       column is a numeric column, this command has no effect.

    e) Made the following updates to the STREAM READ command.

         i) Added the cross-tabulation option

               STREAM READ CROSS TABULATE <file>  <var-list>

            You can specify from one to four cross-tabulation variables
            with the commands

               SET STREAM READ CROSS TABULATE VARIABLE ONE <name>
               SET STREAM READ CROSS TABULATE VARIABLE TWO <name>
               SET STREAM READ CROSS TABULATE VARIABLE THREE <name>
               SET STREAM READ CROSS TABULATE VARIABLE FOUR <name>

            With this syntax, a set of nine statistics will be computed
            for each cross-tabulation cell (count, minimum, maximum,
            range, mean, standard deviation, skewness, kurtosis, number
            of missing values).

        ii) For the WRITE and CROSS TABULATE cases, character variables
            will be automatically converted to categorical variables.  For
            the GROUP STATISITC, DEFAULT STATISTIC, and FULL STATISTIC
            cases, character variables will still be ignored.

            If you would like to save the character strings from the
            character variables as group labels, enter the command

                SET STREAM READ GROUP LABEL ON

            If you have specified a column to be the row label variable
            using the SET ROW LABEL COLUMN command, this variable will
            not be converted to a numeric variable and group labels will
            not be created.

       iii) When computing statistics, missing values (as specified by the
            SET READ MISSING VALUE command) will now be omitted.

        iv) Added the options

                STREAM READ EUCILDEAN DISTANCE <file> <var-list>
                STREAM READ MANHATTAN DISTANCE <file> <var-list>
                STREAM READ CHEBYCHEV DISTANCE <file> <var-list>
                STREAM READ COSINE DISTANCE <file> <var-list>
                STREAM READ COSINE SIMILARITY <file> <var-list>
                STREAM READ ANGULAR COSINE DISTANCE <file> <var-list>
                STREAM READ ANGULAR COSINE SIMILARITY <file> <var-list>

                STREAM READ COVARIANCE  <file> <var-list>
                STREAM READ CORRELATION <file> <var-list>

             With this option, the STREAM READ will return a distance,
             covariance, or correlation matrix.  The raw data is not
             saved.  For example,

                 STREAM READ CORRELATION FILE.DAT Y1 Y2 Y3

             will return the 3 variables Y1, Y2, and Y3 where each of
             these variables will contain 3 rows.  For example, Y2(3)
             contains the correlation between the second response column
             and the third response column.

             This syntax will ignore character fields.  If you do not
             want some fields in the file to be included, you can do
             something like

                 LET ITYPE = DATA 1 1 1 0 1
                 SET STREAM READ VARIABLE TYPE ITYPE

             These commands specify that fields 1, 2, 3, and 5 will be
             included while field 4 will be excluded.  This can be useful
             if some of the fields are categorical variables where
             distance and covariance/correlation do not make sense.
 
    f) For comma delimited files, the READ command will no longer treat
       spaces as a delimiters (i.e., spaces in character fields should be
       read correctly even if the field is not contained in quotes).

       In order to accomodate this change, the comma is no longer the
       default read delimiter.  If you have a comma delimited file, you
       should enter the command

           SET READ DELIMITER ,

       before reading the data file.  If you subsequently need to read
       a space delimited data file, it is recommended that you enter the
       command

           SET READ DELIMITER

       This resets the read delimiter to the space character.

 7) The following miscellaneous changes were made

    a) Made the following tweaks to the STATUS command

        i) STATUS V now prints the number of variables currently
           assigned and the maximum number of variables allowed.

       ii) STATUS F <str> prints the function/string names starting with
           <str>.  For example, STATUS F ST will print all function/string
           names starting with ST.

      iii) For Linux/Unix platforms, the DATE (or TIME) command now
           uses the Fortran 90 standard DATE_AND_TIME subroutine
           (previously it used the Unix specific "fdate" function).
           This changes makes the DATE command consistent across
           Windows and Linux/Unix platforms.

 8) Fixed some bugs.

-----------------------------------------------------------------------
The following enhancements were made to DATAPLOT
October 2017 - May 2018 (includes updates for the 2018/06/12 version).
-----------------------------------------------------------------------

 1) The following device driver was added

        device <1/2/3> CAIRO <device>

    Cairo is a general purpose 2D vector graphics library.  Currently,
    Dataplot supports the X11, Postscript, encapsulated Postscript,
    PDF, Scalable Vector Graphics (SVG), and PNG devices through
    the Cairo device driver.  Enter HELP CAIRO for details.

    This driver is still somewhat experimental.

 2) The following changes were made to the Graphics commands

    a) Added the following options to the I PLOT command

         COEFFICIENT OF VARIAITON CONFIDENCE LIMIT PLOT Y X
         COEFFICIENT OF DISPERSION CONFIDENCE LIMIT PLOT Y X
         COEFFICIENT OF QUARTILE DISPERSION CONFIDENCE LIMIT PLOT Y X

         DIFFERENCE OF MEANS CONFIDENCE LIMIT PLOT Y1 Y2 X
         DIFFERENCE OF PROPORTIONS CONFIDENCE LIMIT PLOT Y1 Y2 X

         CORRELATION CONFIDENCE INTERVAL PLOT Y1 Y2 X

    b) For the fluctuation plot, uncertainty intervals are now
       supported for the difference of means and the difference
       of binomial proportions statistics.

    c) Added the following option to the STATISTIC PLOT command

          <stat> TAG PLOT Y X TAG

       The X variable is used to identify the groups for the
       purpose of computing the statistic.  The TAG variable can
       be used to give different plot attributes to different
       groups.  For example, you may want to identify "outlying"
       groups.

    d) Added the command

         DEX ORDER PLOT Y X1 ... XK

       This command is primarily used by the DEXODP.DP macro that is
       part of the 10-step analysis for 2-level full and fractional
       factorial designs.

 3) The following changes were made to the Analysis commands

    a) Modified the methods for the PROPORTION CONFIDENCE LIMITS
       command.  Enter HELP PROPORTION CONFIDENCE LIMITS for details.

    b) Modified the methods for the DIFFERENCE OF PROPORTION CONFIDENCE
       LIMITS command.  Enter HELP DIFFERENCE OF PROPORTION CONFIDENCE
       LIMITS for details.

    c) For the CALIBRATION command, added two columns to the output to
       show the expanded error and the coverage factor in addition to
       the standard error.

    d) Added the commands

           COEFFICIENT OF DISPERSION CONFIDENCE LIMITS Y
           COEFFICIENT OF QUARTILE DISPERSION CONFIDENCE LIMITS Y

    e) Added an option to the SD CONFIDENCE LIMITS command to
       support Bonett's intervals for non-normal data.

    f) A few enhancements were made to the normal tolerance limits.

          i) For two-sided intervals, adde the command

                SET TOLERANCE LIMITS METHOD <HOWE/WALD WOLFOWITZ>

             The HOWE option uses the Howe approximation.  The
             WALD WOLFOWITZ option uses an approximation given
             by Gardiner to the Beatty method (which implements
             the method suggested by Wald and Wolfowitz).

             The default is HOWE.  Note that prior versions of
             Dataplot are based on the Gardiner approximation.

         ii) Guenther suggested a correction to the Howe method.
             To apply the Guenther correction, use the command

                SET GUENTHER CORRECTION ON

        iii) For one-sided intervals, added the command

                SET TOLERANCE LIMITS ONE SIDED METHOD
                    <NONCENTRAL T/NORMAL/DEFAULT>

             Using NONCENTRAL T uses an approximation based on the
             non-central t percent point function while NORMAL uses an
             approximation that only requires the normal percent point
             function.  Although the non-central t approximation is
             considered more accurate, the non-central t percent point
             function can lose accuracy for large sample sizes.  The
             DEFAULT option will use the non-central t approximation
             for sample sizes of 100 or less and the normal approximation
             for sample sizes greater than 100.  Previous versions of
             Dataplot used the non-central t approximation.

 4) The following enhancement was made to the DRAW command.

    a) The DRAW command now supports variable arguments.  That is,

           DRAW X1 Y1 X2 Y2

       where X1, Y1, X2, and Y2 are variables rather than parameters.
       Note that the DRAW command can accept a mix of parameter and
       variable names.  However, all variable names must be of the same
       length.

       If there are N rows in the variables, then N separate lines are
       drawn (i.e., each row of the variable is drawn as a separate line).
       When parameter names are mixed with variable names, the parameter
       values will be used for all N lines.

    b) Added the following command

           DRAW SYMBOL XPOS YPOS TAG

       where XPOS, YPOS, and TAG are variables.  XPOS and YPOS define
       the x and y coordinates.  The TAG variable contains index values
       (from 1 to 100) for the CHARACTER and associated character
       attribute commands.  For example,

           LET XPOS = DATA 35 55
           LET YPOS = DATA 40 55
           LET TAG  = DATA  1  2
           CHARACTER  - +
           CHARACTER COLOR RED BLUE
           DRAW SYMBOL XPOS YPOS TAG

        will draw a "-" symbol in red at position (35,55) and a "+" symbol
        at position (40,55) in blue.

        If one of the arguments is a parameter rather than a variable, the
        parameter value will be used for all rows.

        You can specify screen or data coordinates in the standard way
        (e.g., DRAWDATA, DRAWSDSD).

    These commands were motivated to provide performance improvements to
    the 10-step macros for 2-level full and fractional factorial designs.
    However, they may have use outside of that context.  Specifically,
    if you have a number of symbols to add to a plot, using the
    DRAW SYMBOL command may be significantly faster than a series of
    MOVE and TEXT commands.

 5) Added the following LET Statistics commands

       LET A = DIFF OF BINOMIAL PROPORTIONS LOWER CONFIDENCE LIMIT Y1 Y2
       LET A = DIFF OF BINOMIAL PROPORTIONS UPPER CONFIDENCE LIMIT Y1 Y2

       LET A = LOWER COEFFICIENT OF DISPERSION CONFIDENCE LIMIT Y
       LET A = UPPER COEFFICIENT OF DISPERSION CONFIDENCE LIMIT Y
       LET A = LOWER ONESIDED COEFFICIENT DISPERSION CONFIDENCE LIMIT Y
       LET A = UPPER ONESIDED COEFFICIENT DISPERSION CONFIDENCE LIMIT Y

       LET A = LOWER COEFFICIENT OF QUARTILE DISPERSION CONFIDENCE LIMIT Y
       LET A = UPPER COEFFICIENT OF QUARTILE DISPERSION CONFIDENCE LIMIT Y

 6) The following new LET subcommands were added:

       LET Y2 = HERMITE DERIVATIVE Y X X2

       LET XNEW = CODE DEX X
       LET COREFAC = DEX CORE X1 X2 ... XK
       LET CONFTAG1 CONFTAG2 = DEX CONFOUND X1 X2 ... XK
       LET IFLAG = DEX CHECK CLASSIC X1 X2 ... XK

    The above four commands are primarily used by the macros for the
    10-step analysis of full and fractional 2-level factorial designs.

 7) The following new string commands were added

       LET SOUT = STRING COMBINE S1 ... SK
       LET SOUT = STRING COMPARE AND REPLACE SOLD SREPLACE SC1 TO SCK
       LET SOUT = STRING INTERACTION J1 ... JK

    The STRING COMBINE command is similar to the STRING CONCATENATE, but
    there are a few differences.  Enter HELP STRING COMBINE for details.

    The STRING COMPARE AND REPLACE command is used by the EST.DP macro
    (part of the 10-step macros for 2-level full and fractional factorial
    designs).  This command was added for performance reasons.  It is not
    anticipated that this command will be used outside of the EST.DP macro.

    The STRING INTERACTION command was also added with the EST.DP macro
    in mind.  However, it did not in fact improve performance so it was
    ultimately not used there.

 8) The following new plot control subcommands were added:

    a) Added the command

         CHARACTER UNITS <val> 

       where <val> can be one of DD, DS, SD, or SS.  The D means data
       units of the current plot and S means 0 to 100 screen units.
       The first character refers to the x-axis coordinate and the
       second character refers to the y-axis coordinate.

 9) The following enhancments were made to the SEARCH, WEB HELP,
    and WEB HANDBOOK commands.

    a) Added the options

         SEARCH REFERENCE <string>
         SEARCH HANDBOOK  <string>

       SEARCH REFERENCE will search the file refman.tex (this is the
       file used by the WEB HELP command).  Likewise, SEARCH HANDBOOK
       searches the file handbk.tex (this is the file used by the
       WEB HANDBOOK command).

    b) The SEARCH command was updated to support more than one
       word matches.  For example, previously
       SEARCH <file> MEAN PLOT would print all lines containing the
       word MEAN.  It will now only list lines that contain the
       words MEAN PLOT.  Note that words MEAN PLOT must appear
       contiguously (i.e., as a single phrase) on the line.
       It does not do a separate search for MEAN and then for
       PLOT.  For example, "MEAN OF THE PLOT" would not be a
       match for "MEAN PLOT".

    c) The following synonyms were added.

          ? is a synonym for SEARCH REFERENCE
          SEARCH RM is a synonym for SEARCH REFERENCE
          SEARCH is a synonym for SEARCH REFERENCE

          ?? is a synonym for WEB HELP

          ??? is a synonym for SEARCH HANDBOOK
          SEARCH HB is a synonym for SEARCH HANDBOOK
          SEARCH HANDBK is a synonym for SEARCH HANDBOOK

          ???? is a synonym for WEB HANDBOOK
          HANDBOOK is a synonym for WEB HANDBOOK
          HB is a synonym for WEB HANDBOOK
          WHB is a synonym for WEB HANDBOOK

          W is a synonym for WEB

          ????? is a synonym for WEB SEARCH
          WS is a synonym for WEB SEARCH

          SEARCH DIR is a synonym for SEARCH DIRECTORY
          SEARCH DIRE is a synonym for SEARCH DIRECTORY
          SEARCH DIC is a synonym for SEARCH DICTIONARY
          SEARCH DICT is a synonym for SEARCH DICTIONARY

    d) The following SET command was added

          SET WEB SEARCH DATAPLOT ON

       When this switch is set to ON, the keyword DATAPLOT will be added
       to the search.  This is useful if you are primarily using the
       WEB SEARCH command to locate Dataplot documentation.

       The default is OFF.

10) The macros for performing the 10-step analysis for 2-level full and
    fractional factorial designs were extensively rewritten to improve
    the performance.

    Note that the new macros require Dataplot executables built with the
    May, 2018 source code as they incorporate several of the new
    commands described above.

11) The following updates were made to the READ command.

    a) Corrected the TO syntax when character variables are being
       read.

    b) If an error is encountered when reading a line, terminate the
       READ immediately rather than continuing to additional lines.

12) The STATUS command was updated in the following ways.

    a) For variables, print the number of assigned variables and the
       maximum number of variables alllowed.

    b) Add row labels, group labels, and character variables.

13) The following miscellaneous changes were made.

    a. SHOW was added as a synonym for PSVIEW.

    b. LIST now uses SET LIST LINES rather SET HELP LINES to specify the
       number of lines to list before prompting to continue.

       The number of columns for the list was increased from a maximum of
       80 to 240.

    c. For Windows platforms, when using SET POSTSCRIPT CONVERT PDF
       within a CAPTURE HTML, the PDF file will now use the "embed"
       tag rather than providing a link to the PDF file.

    d. The method for passing arguments with the CALL command has been
       updated.  Previously, both positional and named arguments were
       supported.  The following enhancements were added:

           i. The argument list can be enclosed in parentheis.  Note that
              the use of parenthesis is optional.  You can optionally
              include one or more spaces between the arguments and the
              parenthesis.

          ii. For named arguments with quoted values, you can include
              only the value of named argument in quotes.  That is,

                  file="c:\my data\test.txt"

              Previously, this had to be entered as

                  "file=c:\my data\test.txt"


              If the argument name is not inside the quotes, then you
              cannot have spaces around the equal sign.  If the argument
              name is inside the quotes, then spaces around the equal sign
              are optional.

         iii. Commas can optionally be used as an argument delimiter.  You
              can mix the use of spaces and commas as the delimiter.  For
              example

                   call test.dp  zx=x,zy=y  zz=z

              Although this is allowed, it is recommended that if you use
              commas that you do so consistently.  That is,

                   call test.dp  zx=x,zy=y,zz=z

          iv. Previously, calling a macro without arguments would clear
              the current command arguments.  This was changed so that
              a CALL command without arguments will not modify the current
              command list arguments.

       Specifically, the following are all acceptable ways to enter the
       same argument list.

           Previously supported:

              call test.dp  y "for i = 1 1 50" x
              call test.dp  zy=y "target=for i = 1 1 50" x

           New formats:

              call test.dp  y,"for i = 1 1 50",x
              call test.dp  zy=y,"target=for i = 1 1 50",x
              call test.dp  (y,"for i = 1 1 50",x)
              call test.dp  ( y, "for i = 1 1 50",x )
              call test.dp  (zy=y, "target=for i = 1 1 50",zx=x)
              call test.dp  ( zy=y, "target=for i = 1 1 50",zx=x )
              call test.dp  (zy=y, target="for i = 1 1 50",zx=x)
              call test.dp  ( zy=y, target="for i = 1 1 50",zx=x )

       The choice of which syntax to use is primarily a matter of personal
       preference.

    e. The following commands were added

          SET HYPHEN COMMAND LINE <ON/OFF>
          SET COMMA COMMAND LINE  <ON/OFF>
          SET EQUAL COMMAND LINE  <ON/OFF>

       These commands specify whether hyphens, commas, and equal signs
       are treated as word delimiters when parsing Dataplot commands.
       These settings are typically used internally by Dataplot for
       certain commands (specifically, they are used to support the
       various syntax for arguments to the CALL command), but they may
       occasionally be useful for your own use.

       For example, you can do something like

          SET COMMA COMMAND LINE ON
          PLOT Y,X

    f. Several tweaks were made to the IF command.

          i. You can now have up to 10 AND, OR, or XOR clauses on a
             single IF command.

         ii. When A does not exist for the IF A = ... command, set the IF
             status to false and do not signal an error condition.

        iii. The following syntax is now supported

                IF 3 > 2
                IF 3 > A

              That is, the left hand side of the logical operator can
              now be a number as well as a parameter.

          iv. When testing strings, if the left hand side of the logical
              operator is enclosed in quotes, treat this is a literal
              string.

           v. Added the IF COMMAND LINE ARGUMENT ... NOT EXIST command
              (the EXIST version was added 2016/10).

          vi. Added better feedback for some of the special cases.

    g. Changed the default random number generator from FIBONACCI to
       FIBONACCI CONGRUENTIAL.

    h. The PRINT command was updated to accomodate group labels.  You
       can use PRINT GROUP LABELS to print all group labels or you
       can include group labels in the variable list for the PRINT
       command.  Group labels are created with the LET ... = GROUP LABEL
       command.

       The PRINT command was also updated to handle row labels and
       character variables.  To print row labels, use ROWLABEL as the
       variable name.  Character variables and row labels will print the
       first 24 characters and the field width will be 25.

    i. The maximum number of group label variables was increased from 5
       to 20.

-----------------------------------------------------------------------
The following enhancements were made to DATAPLOT
October 2016 - September 2017.
-----------------------------------------------------------------------

 1) Updates to the graphics commands

    a) Added the commands

         EMPIRICAL QUANTILE PLOT Y
         QUANTILE BOX PLOT Y
         TRUNCATED INFORMATIVE QUANTILE PLOT Y

       These commands were added to support the MIL-Handbook 17 standard.

    b) Added the command

         BLAND ALTMAN PLOT

    c) Added the command

         POINCARE PLOT

    d) Added the command

         NORMAL KERNEL DENSITY MIXTURE PLOT

    e) Added the command

          SET BLOCK PLOT JITTER  value

       This command can be used to add some random jitter to the
       x-coordinate of the plot character of the block plot.  This can
       be useful when the plot characters overlap vertically.

    f) The PSVIEW command allows you to view the current plot (i.e.,
       the contents of the DEVICE 3 file) without exiting Dataplot.

       This command was updated so that you can view the contents of
       the DEVICE 2 file.  You can also view an arbitrary Postscript
       file.  Enter HELP PSVIEW for details.

    g) Added the following commands

          SET CONSENSUS MEAN PLOT OMIT LABS <list of lab-id's>

          SET CONSENSUS MEAN PLOT OMIT METHOD ONE   <method>
          SET CONSENSUS MEAN PLOT OMIT METHOD TWO   <method>
          SET CONSENSUS MEAN PLOT OMIT METHOD THREE <method>

       The SET CONSENSUS MEAN PLOT OMIT LABS command allows you to
       specify from one to ten labs that will be omitted from the
       consensus mean plot.  Note that the omitted labs will still be
       included in the computation of the consensus means and
       uncertainties, they just will not be included in the plot of the
       lab data.  This can be useful when there are extreme outliers in
       the lab data as the outlying lab can make it difficult to see
       differences in the non-outlying labs.

       Similarly, the SET CONSENSUS MEAN PLOT OMIT METHOD command
       allows you to specify a method (e.g., VANGEL RUKHIN) that
       will be omitted from the consensus mean plot.  The method will
       be included in the computation of the consensus means and
       uncertainties, but it will not be displayed on the plot.  As
       with omitting labs, this can be helpful in the case of extreme
       outliers where the results from a particular method might
       result in poor resolution for the plot.

 2) Updates to the analysis commands

    a) Added the following commands for cluster analysis

          K MEANS Y1 ... YK
          NORMAL MIXTURE CLUSTER Y1 ... Yk

          K MEDOID Y1 ... YK

          FANNY Y1 ... YK

          AGNES Y1 ... YK
          DIANA Y1 ... YK

       The K MEANS command performs a k-means cluster analysis and the
       NORMAL MIXTURE CLUSTER command performs a clustering based on
       Hartigan's mixture of normal distributions.

       The K MEDOID command performs a k-medoids cluster analysis based
       on the CLARA and PAM programs of Rousseeuw and Kauffman.

       The AGNES command performs hierarchial clustering using
       agglomerative nesting methods.  The DIANA command performs
       hierarchial clustering using divisive analysis.

    b) The CALIBRATION command was updated to include propogation of error
       methods as defined in the NIST/SEMATECH e-Handbook of Statistical
       Methods for both the linear and quadratic calibration cases.

    c) Corrected the tolerance limit factor for the Weibull ABASIS
       command.

    d) Added the commands

           COEFFICIENT OF VARIATION CONFIDENCE LIMITS Y
           LOGNORMAL COEFFICIENT OF VARIATION CONFIDENCE LIMITS Y
           COMMON COEFFICIENT OF VARIATION CONFIDENCE LIMITS Y X
           ONE SAMPLE COEFFICIENT OF VARIATION TEST Y X GAMMA0
           TWO SAMPLE COEFFICIENT OF VARIATION TEST Y1 X1 Y2 X2

       Enter HELP COEFFICIENT OF VARIATION CONFIDENCE LIMITS for
       details.

    e) Added the command

           LOGNORMAL CONFIDENCE LIMITS Y

 3) The following new statistic LET subcommands were added:

       LET A = UNBIASED COEFFICIENT OF VARIATION Y
       LET A = LOGNORMAL COEFFICIENT OF VARIATION Y

       LET A = LOWER COEFFICIENT OF VARIATION CONFIDENCE LIMIT Y
       LET A = UPPER COEFFICIENT OF VARIATION CONFIDENCE LIMIT Y
       LET A = LOWER ONESIDED COEFFICIENT OF VARIATION CONFIDENCE LIMIT Y
       LET A = UPPER ONESIDED COEFFICIENT OF VARIATION CONFIDENCE LIMIT Y
       LET A = LOWER LOGNORMAL COEFFICIENT OF VARIATION CONFIDENCE LIMIT Y
       LET A = UPPER LOGNORMAL COEFFICIENT OF VARIATION CONFIDENCE LIMIT Y

       LET A = SUMMARY COEFFICIENT OF VARIATION YMEAN YSD N
       LET A = SUMMARY LOWER COEFFICIENT OF VARIATION CONFIDENCE LIMIT
               YMEAN YSD N
       LET A = SUMMARY UPPER COEFFICIENT OF VARIATION CONFIDENCE LIMIT
               YMEAN YSD N

       LET A = COMMON COEFFICIENT OF VARIATION Y X
       LET A = COMMON BIAS CORRECTED COEFFICIENT OF VARIATION Y X
       LET A = LOWER COMMON COEFFICIENT OF VARIATION CONFIDENCE LIMIT Y X
       LET A = UPPER COMMON COEFFICIENT OF VARIATION CONFIDENCE LIMIT Y X

       LET A = SIGNAL TO NOISE RATIO Y
       LET A = PRECISION Y
       LET A = COEFFICIENT OF DISPERSION Y
       LET A = INDEX OF DISPERSION Y
       LET A = QUARTILE COEFFICIENT OF DISPERSION Y
       LET A = AAD TO MEDIAN Y

       LET A = DIFFERENCE OF PRECISION Y1 Y2
       LET A = DIFFERENCE OF SIGNAL TO NOISE RATIO Y1 Y2
       LET A = DIFFERENCE OF COEFFICIENT OF DISPERSION Y
       LET A = DIFFERENCE OF INDEX OF DISPERSION Y
       LET A = DIFFERENCE OF QUARTILE COEFFICIENT OF DISPERSION Y
       LET A = DIFFERENCE OF AAD TO MEDIAN Y

       LET A = SHORTEST HALF MIDMEAN Y
       LET A = SHORTEST HALF MIDRANGE Y
       LET A = MIDHINGE Y
       LET A = TRIMEAN Y

       LET A = DIFFERENCE OF SHORTEST HALF MIDMEAN  Y1 Y2
       LET A = DIFFERENCE OF SHORTEST HALF MIDRANGE Y1 Y2
       LET A = DIFFERENCE OF MIDHINGE Y1 Y2
       LET A = DIFFERENCE OF TRIMEAN Y1 Y2

       LET A = COSINE DISTANCE Y1 Y2
       LET A = COSINE SIMILARITY Y1 Y2
       LET A = ANGULAR COSINE DISTANCE Y1 Y2
       LET A = ANGULAR COSINE SIMILARITY Y1 Y2
       LET A = EUCLIDEAN DISTANCE Y1 Y2
       LET A = EUCLIDEAN LENGTH Y1
       LET A = DOT PRODUCT Y1 Y2
       LET A = MANHATTAN DISTANCE Y1 Y2
       LET A = CHEBYSHEV DISTANCE Y1 Y2

       LET P = <value>
       LET A = MINKOWSKI DISTANCE Y1 Y2

       LET A = BINARY MATCH DISSIMILARITY Y1 Y2
       LET A = BINARY MATCH SIMILARITY Y1 Y2
       LET A = BINARY ROGERS MATCH DISSIMILARITY Y1 Y2
       LET A = BINARY ROGERS MATCH SIMILARITY Y1 Y2
       LET A = BINARY SOKAL MATCH DISSIMILARITY Y1 Y2
       LET A = BINARY SOKAL MATCH SIMILARITY Y1 Y2
       LET A = BINARY JACCARD DISSIMILARITY Y1 Y2
       LET A = BINARY JACCARD SIMILARITY Y1 Y2
       LET A = BINARY ASYMMETRIC SOKAL MATCH DISSIMILARITY Y1 Y2
       LET A = BINARY ASYMMETRIC SOKAL MATCH SIMILARITY Y1 Y2
       LET A = BINARY ASYMMETRIC DICE MATCH DISSIMILARITY Y1 Y2
       LET A = BINARY ASYMMETRIC DICE MATCH SIMILARITY Y1 Y2
       LET A = YULES Q Y1 Y2

       LET A = GENERALIZED JACCARD COEFFICIENT Y1 Y2
       LET A = GENERALIZED JACCARD DISTANCE Y1 Y2

       LET A = PEARSON DISSIMILARITY Y1 Y2
       LET A = SPEARMAN DISSIMILARITY Y1 Y2
       LET A = KENDALL TAU DISSIMILARITY Y1 Y2

       LET A = HEDGES G Y1 Y2
       LET A = BIAS CORRECTED HEDGES G Y1 Y2
       LET A = GLASS G Y1 Y2
       LET A = COHEN D Y1 Y2

 4) The following new LET subcommands were added:

       LET M = GENERATE MATRIX <stat> Y1 ... YK

       LET Y2 = HERMITE INTERPOLATION Y X X2
       LET A  = HERMITE INTEGRAL Y X

       LET Y TAG = SAMPLE RANDOM PERMUTATION NPOP NKEEP P NITER

       LET WSDF POOLSD = VARIANCES WELCH SATTERTHWAITE YVAR YDF
       LET WSDF        = GUM       WELCH SATTERTHWAITE YSD  YDF  AUNC

       LET Y X = NORMAL KERNEL DENSITY MIXTURE YMEAN YSD

       LET Y = EMPIRICAL QUANTILE FUNCTION X
       LET Y = INFORMATIVE QUANTILE FUNCTION X
       LET Y = TRUNCATED INFORMATIVE QUANTILE FUNCTION X

       LET Y = CODEX X

 5) The following commands were added to aid in reading certain
    types of ASCII files.

        SET CHARACTER FIELD COMMA DELIMITER <ON/OFF>
        SET READ CHARACTER MISSING VALUE <STRING>

        SET READ TRAILING PLUS MINUS IGNORE <ON/OFF>
        SET READ DOLLAR SIGN IGNORE <ON/OFF>
        SET READ COMMA IGNORE <ON/OFF>

 6) Miscellaneous updates

    a) Command line arguments for macros have been updated to support
       named arguments (previously, only ordered arguments were
       supported).

          Enter HELP MACRO SUBSTITUTION CHARACTER for details.

    b) Updated the IF command to support the syntax

          IF NOT <expression>

       If <expression> is true, the IF command returns FALSE and
       if <expression> is false, the IF command returns TRUE.

       In addition, the following are now supported

          IF <expr1>  AND  <expr2>
          IF <expr1>  OR   <expr2>
          IF <expr1>  XOR  <expr2>

       Currently, only one AND, OR, or XOR clause may be included on
       an IF command.

    c) Previously, the following did not work with the SET command

          LET DP = 3
          SET WRITE DECIMALS DP

       You needed to do the following

          LET DP = 3
          SET WRITE DECIMALS ^DP

       The SET command was updated so that the last argument will
       be checked to see if it is a parameter or a string.  If so,
       the last argument will be replaced with the value of the
       parameter or string.  So the "^" is no longer required,
       although that syntax will still work.

       Currently for strings, only the first 8 characters will be
       used.  This means there a few limitations that should be noted.

          i) SET commands that expect a file name or a path name
             are not yet supported.

         ii) Only the last argument is checked.  So if the SET
             command needs multiple word arguments, this update
             will not help.

    d) Corrected the CAPTURE SCRIPT command so that it will work within
       loops.  Also addded the command

          SET CAPTURE SCRIPT LOOP SUBSTITUTION <ON/OFF>

       If ON, then substitutions denoted by the "^" character will be
       performed before writing the line to the CAPTURE file.  If OFF,
       no substitution will be performed.  ON is the default.  The OFF
       is typically only needed if the "^" character is used and has
       a specific meaning you want to preserve (e.g., "^" is the
       exponentiation symbol in many scripting languages).

    e) Added the command

          SET HYPHEN WORD SEPARATOR <ON/OFF>

-----------------------------------------------------------------------
The following enhancements were made to DATAPLOT
December 2015 - September 2016.
-----------------------------------------------------------------------

 1) Made the following updates to graphics commands.

    a) Added the command

          LET NSIZE = <value>
          <stat> WINDOW STATISTIC PLOT Y

       This is similar to the

          <stat> PLOT Y X

       command.  However, there is a distinction in how groups are
       formed.

       The <stat> PLOT command forms groups based on the unique values of
       the group-id variable.  The <stat> WINDOW STATISTIC PLOT creates
       groups of contiguous rows of the response variable.  The size of
       these groups is determined by the LET NSIZE command that is
       entered before the <stat> WINDOW STATISTIC PLOT command.

       This command can be useful for very large data sets where it may
       be impractical to plot all the individual points.  As an
       alternative, you can plot various summary statistics of blocks
       of data.

    b) The QUANTILE QUANTILE PLOT was updated in several ways.

          i) A line is fit to the points on the quantile-quantile (qq)
             plot.  The attributes of this line are controlled by the
             settings for trace 3 (trace 1 is the plot points and trace 2
             is the 45 degree reference line).

         ii) The parameters PPA0, PPA1, and PPCC are automatically saved
             from this fitted line.  These parameters define the intercept,
             the slope and the correlation coefficient of the fitted line,
             respectively.  For exact fit (e.g.,
             QUANTILE QUANTILE PLOT Y Y), these values would be 0, 1, and
             0.

        iii) The following command was added

                SET QUANTILE QUANTILE PLOT NUMBER OF PERCENTILES <value>

             By default, the quantile quantile plot generates the plot
             points corresponding to percentiles of the smaller of N1
             and N2 where N1 and N2 are the number of observations for
             the two samples.

             This SET command allows you to specify an arbitrary number
             of percentiles.  This is intended for the case where there
             are a large number of data points.  For example, suppose
             the two columns being compared each have a million or more
             points.  This results in a time consuming plot and a very
             large Postscript file (which you may have a hard time
             viewing or printing).  By setting the number of percentiles
             to something like a 1,000 or 10,000, you can generate a
             quantile quantile plot quickly with a reasonably sized
             Postscript file without sacrificing too much information.
             That is, the basic message of the quantile-quantile plot
             should still be clear even with the reduced number of
             points plotted.

    c) Made several updates to the HISTOGRAM command.

       When dealing with pathological data sets (e.g., Cauchy distributed
       data), there is an issue with generating a class size that is
       appropriate for central bulk of the data while still being able
       to generate the histogram in an efficient fashion.  The following
       commands provide some methods for addressing this.

       When you have data where there are a small percentage of points
       that are quite far from the bulk of the data, you might want to
       use the command (this already existed, enter HELP HISTOGRAM CLASS
       WIDTH for details)

            SET HISTOGRAM CLASS WIDTH IQ RANGE

       This bases the bin width for the histogram on the interquartile
       range rather than the standard deviation as the other class width
       algorithms do.  This can result in more reasonable class widths
       for the center of the data when there are extreme outliers in the
       data.  Also, these commands are typically used when the

            SET HISTOGRAM OUTLIERS ON

       command is also given (this command extends the bins to cover all
       outliers).

         i) The following command can be used to specify the maximum
            number of classes for the histogram.

                SET HISTOGRAM MAXIMUM CLASSES <value>

            If this command is entered, then the class width is initially
            computed in the standard way.  If the number of bins needed to
            cover the outliers is greater than the value given here, then
            the class width is recomputed so that the number of bins is
            equal to the value given here.

        ii) The following command can be used to specify that outliers
            be drawn as individual points rather than extending the bins
            to cover them.

               SET HISTOGRAM OUTLIER POINTS ON/OFF

    d) Made several updates to the BOX PLOT command.

          i) It is no longer considered an error to only have a single
             point for the box plot.  Although box plots are typically
             not drawn for a small number of points, when automating an
             analysis for a large data set it can be useful to have the
             box plot drawn for degenerate cases.

         ii) To have a horizontal bars drawn at the 1%, 5%, 10%, 90%,
             95%, and 99% points of the distribution, enter

                SET BOX PLOT EXTREME PERCENTILES ON

             This option may be useful for large data sets.

             If the FENCES switch is OFF, then the CHARACTER and LINE
             settings for traces 21 through 26 will be used to draw these
             percentiles.  If the FENCES switch is ON, then the CHARACTER
             and LINE settings for traces 25 through 30 will be used to
             draw these percentiles.  Currently, the LINES BOX PLOT and
             CHARACTER BOX PLOT commands do not set these.  You can use
             something like the following to set these switches.

                LET INDX = DATA 21 22 23 24 25 26
                LET PLOT CHARACTER INDX = BLANK
                LET PLOT LINE      INDX = SOLID

        iii) If you use the following syntax

                MULTIPLE BOX PLOT Y1 Y2 ... Y5

             Dataplot will internally create a stacked Y X set of data.
             Dataplot was modified so that if there are four or fewer
             response variables, then Dataplot will not stack the data
             to generate the box plot.  Although this has no effect on
             the appearance of the plot, it can be useful when generating
             box plots for large data sets in that it may avoid exceeding
             Dataplot's maximum number of rows.

 2) Made the following updates to analysis commands.

    a) The KOLMOGOROV SMIRNOV TWO SAMPLE TEST was updated to use the
       following command

           SET TWO SAMPLE TEST NUMBER OF PERCENTILES <value>

       By default, the Kolmogorov-Smirnov test is generated using all
       the points.  When the number of points gets large, this can result
       in this command taking a very long time.  Computing this test for
       a specified number of percentiles of the data allows this command
       to be executed quickly without sacrificing too much information.

 3) The following new statistic LET subcommands were added:

       LET A = NORMALIZED IQR Y
       LET A = SCALED MAD Y

       LET A = DIFFERENCE OF NORMALIZED IQR Y1 Y2
       LET A = DIFFERENCE OF SCALED MAD Y1 Y2

       LET A = 2PARAMETER WEIBULL PPCC Y
       LET A = 2PARAMETER WEIBULL PPCC SHAPE Y
       LET A = 2PARAMETER WEIBULL PPCC SCALE Y

 4) Although Dataplot has a large number of built-in statistics,
    there may be cases where you need a statistic not directly
    supported by Dataplot.

    The STATISTIC BLOCK command was added to allow you to define
    your own statistic.  The power in this command is not the
    generation of the statistic itself (this could be accomplished
    using a Dataplot macro), but in the ability to use the
    statistic with 20+ Dataplot commands.  For details, enter
    HELP STATISTIC BLOCK.

 5) The following new LET subcommands were added:

       LET YMIN1 YMAX1 = YFRAME Y
       LET XMIN1 XMAX1 = XFRAME X

       LET Y2 = SEQUENTIAL SUM Y
       LET Y2 = SEQUENTIAL MEAN Y
       LET Y2 = SEQUENTIAL MINIMUM Y
       LET Y2 = SEQUENTIAL MAXIMUM Y
       LET Y2 = SEQUENTIAL PRODUCT Y
       LET Y2 = SEQUENTIAL LOWER Y
       LET Y2 = SEQUENTIAL UPPER Y

       LET Y2 GROUP2 = SEQUENTIAL SUM Y GROUPID
       LET Y2 GROUP2 = SEQUENTIAL MEAN Y GROUPID
       LET Y2 GROUP2 = SEQUENTIAL MINIMUM Y GROUPID
       LET Y2 GROUP2 = SEQUENTIAL MAXIMUM Y GROUPID
       LET Y2 GROUP2 = SEQUENTIAL PRODUCT Y GROUPID
       LET Y2 GROUP2 = SEQUENTIAL LOWER Y GROUPID
       LET Y2 GROUP2 = SEQUENTIAL UPPER Y GROUPID

       LET DIPERC = ISO 13528 DIPERC Y XREF
       LET PA     = ISO 13528 PA     Y XREF DELTAE

       LET Y = EXECUTE <file-name> X
       LET IFLAG = INQUIRE <file-name>

       LET Y = WINDOW <stat> X
       LET Y = VECTOR PERCENTILE X NPERC
       LET Y = CODEZ X

       LET Z = UNSTACK Y X

 6) The following updates were made to the READ command.

    a) Date and time fields will typically have syntax like

         2016/06/22
         12:43:08

       Dataplot treats the "/" and ":" as indicating character fields
       (based on the SET CHARACTER CONVERT command, this will either cause
       an error, result in this field being ignored, or the field being
       read as a character variable).

       The following commands were added to help deal with date and
       time fields.

          SET DATE DELIMITER <character>
          SET TIME DELIMITER <character>

       Although Dataplot does not have explicit date or time variables,
       these commands allow the components of date and time fields to
       be read as separate numeric variables.  For example,

          SET DATE DELIMITER /
          SET TIME DELIMITER :
          READ YEAR MONTH DAY HOUR MIN SEC
          2016/06/22  23:19:03
          END OF DATA

    b) IP addresses typically have a syntax like

         129.6.37.209

       By default, Dataplot will generate an error when trying to read a
       field of this type.  To address this, you can enter the command

          SET READ IP ADDRESSES ON

       If this switch is ON, Dataplot will scan the line and if a field is
       encountered that contains more than one period ".", Dataplot will
       convert these periods to spaces before parsing the line.

       The default is OFF since this adds additional processing time to
       the READ and most data sets do not contain IP addresses.

    c) Added the command STREAM READ.  This command can be useful
       in processing large data sets (particularly for data sets
       exceeding Dataplot's maximum number of rows).  There are a
       number of variants of this command.  Specifically,

          i) It can be used to create a file that can be read
             using a Fortran like format (i.e., the SET READ
             FORMAT command).

         ii) It can be used to compute 23 statistics for groups
             of the data (either for groups of a specified number
             of rows or for when a specific group-id variable
             changes value).

        iii) Compute eight summary statistics for the full data
             set.

 7) Miscellaneous updates

    a) Added the command

          SET STANDARD INPUT <FNAME>

       This command is used to set standard input to an external
       file.  Enter HELP STANDARD INPUT for details.

    b) Added the following SET commands

          SET DEVICE 3 <AUTOMATIC/USER>
          SET DEVICE 2 SPLIT <ON/OFF>
          SET DEVICE 3 NAME COUNTER <ON/OFF>

       For details on these commands, enter HELP DEVICE 3,
       HELP DEVICE 2 SPLIT, and HELP DEVICE 3 NAME COUNTER.

          SET CAPTURE SPLIT <ON/OFF>
          SET CAPTURE CUMULATIVE <ON/OFF>

       For details on the above two commands, enter HELP CAPTURE.

    c) Added the command

          WEB SEARCH <string>

       This command will perform a web search.  The desired search
       engine can be specified with the command

          SET SEARCH ENGINE <GOOGLE/BING/DUCK/WOW/YAHOO>

   d) Updated the QWIN device driver for Windows to better support
      hardware fonts.  Enter HELP QWIN for details.

   e) Added the command

          SET MACRO QUOTES STRIP <ON/OFF>

      Enter HELP MACRO SUBSTITUTION CHARACTER for details.

   f) Updated the RESET command to allow you to specify names
      that will not be reset by the RESET DATA, RESET PARAMETERS,
      RESET VARIABLES, RESET FUNCTIONS and RESET MATRICES commands.

      The syntax is

         RESET NO RESET <name>

      Enter the above command for each name that you want the
      RESET command to ignore.  Up to 30 names can be specified.
      To remove a name from the list, enter

         RESET NO RESET <name> OFF

  g) Added the command

         RESET COMMAND LINE ARGUMENTS

     This command clears any command line arguments (i.e., $0, $1, ...).

     Alternatively, using

         CALL file.dat  NULL

     will also clear the command line arguments.

  h) Added the following SET commands that apply to the CORRELATION MATRIX
     command

         SET CORRELATION ABSOLUTE VALUE <ON/OFF>
         SET CORRELATION PERCENTAGE VALUE <ON/OFF>
         SET CORRELATION DIGITS <VALUE>

     These commands are typically used when plotting the correlation values.
     Specifically, the first command allows you to specify the absolute
     value of the correlation (useful when you are trying to identify
     significant correlation regardless of whether it is a positive or a
     negative correlation).  The second command specifies the correlation
     as a percentage value (e.g., a correlation of 0.91 would be given as
     91.0).  The third command specifies how many digits to store for the
     correlation.

-----------------------------------------------------------------------
The following enhancements were made to DATAPLOT
January 2014 - November 2015.
-----------------------------------------------------------------------

 1) The following updates were made to to the Graphics commands.

    a. Added the command

         DISTRIBUTIONAL FIT PLOT Y X

       This command is a graphical represention of the output from the
       BEST DISTRIBUTIONAL FIT command for multiple groups as a tabulation
       plot.  It is intended as a screening tool for identifying good
       candidate distributional models.

    b. Added the commands

        H CONSISTENCY PLOT    Y LABID MATID
        K CONSISTENCY PLOT    Y LABID MATID
        COCHRAN VARIANCE PLOT Y LABID MATID

    c. Added the commands

        TWO FACTOR PLOT    Y LABID MATID

    d. Added the commands

        TWO WAY ROW PLOT     Y LABID MATID
        TWO WAY COLUMN PLOT  Y LABID MATID

    e. Added the commands

         <stat> CUMULATIVE STATISTIC PLOT Y
         <stat> CUMULATIVE STATISTIC PLOT Y X
         <stat> MOVING STATISTIC PLOT Y
         <stat> MOVING STATISTIC PLOT Y X

       where <stat> is one of Dataplot's built-in statistics.

    f. Added the command

         LORENZ CURVE Y

    g. Added the commands

         EMBED <ON/OFF>
         EMBED CORNER COORDINATES <xlow> <ylow> <xhigh> <yhigh>

       The EMBED command is an alternative approach to generating
       multiplot plots per pages.  Enter HELP EMBED for details.

 2) The following updates were made to to the Analysis commands.

    a. The following commands

           LET Y = ROOTS ...
           LET Y = OPTIMIZE ...
           PLOT ...
           3-D PLOT ...
           LET A = INTERGRAL ...
           LET A = NUMERICAL DERIVATIVE ...

       can work on functions.  One limitation in Dataplot's function
       definitions is the library functions are computed "row by row".
       For many statistical applications, we need to define vector
       functions (e.g., various sums and sums of squares).  The above
       commands now support a "FUNCTION BLOCK" capability that allows
       much greater flexibility in defining functions.

       Enter HELP FUNCTION BLOCK for details on how to define and use
       function blocks.

    b. The BEST CP command was updated to print the BIC statistic
       in addition to the Cp statistic.  Note that the selection
       of regressions is still based on the Cp statistic.

    c. Added the command

         COMPLETE SPATIAL RANDOMNESS Y X

       This command tests for complete spatial randomness in the 2D case.
       It implements the following three tests:

          i) A bivariate Cramer Von Mises test for uniformity

         ii) The mean nearest neighbors distance test.

        iii) Pollard's statistic for distance indices 1, 2, 3, 4, 5.

    d. Added the command

         COMMON WEIBULL SHAPE TEST Y X

    e. Added the command

         <dist1> AND <dist2> DISTRIBUTIONAL LIKELIHOOD RATIO TEST Y

    f. Added the command

         EQUAL SLOPES TEST Y X TAG

       for testing the equality of slopes for two linear regression lines.

    f. Made the following updates to the distributional maximum
       likelihood commands

          i) added 3-parameter lognormal
         ii) added 3-parameter gamma
        iii) made some updates to the 3-parameter Weibull
         iv) added 3-parameter inverse gaussian (also added confidence
             limits for parameter estimates)
          v) added parameter confidence intervals and percentile
             confidence intervals for several additional cases

    g. The CROSS TABULATE command was updated to support up to eight
       group-id variables (the previous limit was six).

    h. The TOLERANCE LIMITS command was updated to support the following
       options

          LOGNORMAL TOLERANCE LIMITS Y
          BOX COX TOLERANCE LIMITS Y

       The LOGNORMAL case takes the log of the response data, generates
       normal-based tolerance limits, and then takes the exponent of the
       normal-based limits.  Similarly, the BOX COX case performs a
       Box-Cox transformation to normalize the data and then generates
       normal-based tolerance limits.  These normal-based limits are
       then transformed back.  These options can be useful for data that
       does not follow a normal distribution. They can give more
       efficient limits than non-parametric tolerance limits for many
       non-normal datasets.

    i. The PREDICTION LIMITS command was updated to support the
       following options

          LOGNORMAL PREDICTION LIMITS Y
          BOX COX PREDICTION LIMITS Y

       The LOGNORMAL case takes the log of the response data, generates
       normal-based prediction limits, and then takes the exponent of the
       normal-based limits.  Similarly, the BOX COX case performs a
       Box-Cox transformation to normalize the data and then generates
       normal-based prediction limits.  These normal-based limits are
       then transformed back.  These options can be useful for data that
       does not follow a normal distribution.

    j. The CAPABILITY ANALYSIS command was updated to include several
       additional capability statistics.  The output was also reformatted.

    k. Several updates were made to the E691 INTERLAB command.

    l. Added the command

           COCHRAN VARIANCE OUTLIER TEST Y X

       This command performs the Cochran variance outlier test for the
       maximum variance.  It includes the extensions of 'r Lam to handle
       unequal group sizes and to handle the minimum variance case.

 3) The following new statistic LET subcommands were added:

          LET A = VARIATIONAL DISTANCE Y
          LET A = RELATIVE DISPERSION INDEX Y
          LET A = UNIFORM CHI-SQUARE Y

          LET A = DECILE RATIO Y

          LET XVALUE = <value>
          LET A = VALUE COUNT Y

          LET A = GALTON SKEWNESS Y
          LET A = PEARSON TWO SKEWNESS Y
          LET A = DIFFERENCE OF GALTON SKEWNESS Y
          LET A = DIFFERENCE OF PEARSON TWO SKEWNESS Y
          LET A = WEIGHTED SKEWNESS Y W

          LET A = BOX COX NORMALITY PPCC Y
          LET A = BOX COX NORMALITY LAMBDA Y

          LET A = AVERAGE ABSOLUTE DEVIATION FROM THE MEDIAN Y
          LET A = DIFFERENCE OF AVERAGE ABSOLUTE DEVIATION FROM MEDIAN Y1 Y2
                  Note: The definition of average absolute deviation
                        was corrected to compute differences from the
                        mean rather than differences from the median.

          LET A = COMMON WEIBULL SHAPE TEST Y X
          LET A = COMMON WEIBULL SHAPE TEST CDF Y X
          LET A = COMMON WEIBULL SHAPE TEST PVALUE Y X
          LET A = COMMON WEIBULL SHAPE TEST CV90 Y X
          LET A = COMMON WEIBULL SHAPE TEST CV95 Y X
          LET A = COMMON WEIBULL SHAPE TEST CV99 Y X

          LET A = KAPPENMAN R Y
          LET A = KAPPENMAN R CUTOFF  Y

          LET A = BIVARIATE CRAMER VON MISES TEST X Y
          LET A = BIVARIATE CRAMER VON MISES TEST CV95 X Y
          LET A = BIVARIATE CRAMER VON MISES TEST CV05 X Y

          LET A = MEAN NEAREST NEIGHBOR DISTANCE TEST X Y
          LET A = MEAN NEAREST NEIGHBOR DISTANCE CDF X Y
          LET A = MEAN NEAREST NEIGHBOR DISTANCE PVALUE X Y
          
          LET A = POLLARD <ONE/TWO/THREE/FOUR/FIVE> TEST X Y
          LET A = POLLARD <ONE/TWO/THREE/FOUR/FIVE> CDF X Y
          LET A = POLLARD <ONE/TWO/THREE/FOUR/FIVE> PVALUE X Y

          LET A = <dist> ANDERSON DARLING Y
          (HELP STATISTIC ANDERSON DARLING for details)

          LET A = CPMK  Y
          LET A = CNP   Y
          LET A = CNPM  Y
          LET A = CNPMK Y

 4) The following new math LET subcommands were added:

          LET Y = LOGNORMAL MOMENT ESTIMATES X
          LET Y = GAMMA MOMENT ESTIMATES X
          LET Y = INVERSE GAUSSIAN MOMENT ESTIMATES X

          LET YNEW = SHUFFLE GROUPS Y X INDEX

          LET YOUT = RANDOM ERROR QUANTITY X Y

          LET Y = DIGITS A

 5) The following new string LET subcommands were added:

          LET STYPE = TYPE <name>
          LET CON COP = CONFOUND K N
          LET S = DIGITS TO STRING
          LET S = STRING REMOVE SPACES
          LET S = NUMBER TO STRING

 6) The following new LET subcommands were added:

          LET IFLAG = CHECK NAMES
          LET IFLAG = CHECK EQUAL LENGTH
          LET IFLAG = CHECK TYPE

 7) The following new library functions were added

      i) For a number of distributions commonly used in
         reliability/lifetime applications, survival and
         inverse survival functions were added.  The
         survival function is:  SURV(X) = 1 - CDF(X). The
         inverse survival function is: ISURV(P) = PPF(1-P).

          LET A = EXPSURV(X,LOC,SCALE)
          LET A = EXPISURV(P,LOC,SCALE)

          LET A = EWESURV(X,SHAPE1,SHAPE2,LOC,SCALE)
          LET A = EWEISURV(P,SHAPE1,SHAPE2,LOC,SCALE)

          LET A = FLSURV(X,SHAPE1,LOC,SCALE)
          LET A = FLISURV(P,SHAPE1,LOC,SCALE)

          LET A = GAMSURV(X,SHAPE1,LOC,SCALE)
          LET A = GAMISURV(P,SHAPE1,LOC,SCALE)

          LET A = GEESURV(X,SHAPE1,LOC,SCALE)
          LET A = GEEISURV(P,SHAPE1,LOC,SCALE)

          LET A = GEVSURV(X,SHAPE1,LOC,SCALE)
          LET A = GEVISURV(P,SHAPE1,LOC,SCALE)

          LET A = GOMSURV(X,SHAPE1,SHAPE2,LOC,SCALE)
          LET A = GOMISURV(P,SHAPE1,SHAPE2,LOC,SCALE)

          LET A = IGSURV(X,SHAPE1,LOC,SCALE)
          LET A = IGISURV(P,SHAPE1,LOC,SCALE)

          LET A = IGASURV(X,SHAPE1,LOC,SCALE)
          LET A = IGAISURV(P,SHAPE1,LOC,SCALE)

          LET A = IWESURV(X,SHAPE1,LOC,SCALE)
          LET A = IWEISURV(P,SHAPE1,LOC,SCALE)

          LET A = LGNSURV(X,SHAPE1,LOC,SCALE)
          LET A = LGNISURV(P,SHAPE1,LOC,SCALE)

          LET A = NORSURV(X,LOC,SCALE)
          LET A = NORISURV(P,LOC,SCALE)

          LET A = PLNSURV(X,SHAPE1,SHAPE2,LOC,SCALE)
          LET A = PLNISURV(P,SHAPE1,SHAPE2,LOC,SCALE)

          LET A = PNRSURV(X,SHAPE1,LOC,SCALE)
          LET A = PNRISURV(P,SHAPE1,LOC,SCALE)

          LET A = RIGSURV(X,SHAPE1,LOC,SCALE)
          LET A = RIGISURV(P,SHAPE1,LOC,SCALE)

          LET A = WALSURV(X,SHAPE1,LOC,SCALE)
          LET A = WALISURV(P,SHAPE1,LOC,SCALE)

          LET A = WEISURV(X,SHAPE1,LOC,SCALE)
          LET A = WEIISURV(P,SHAPE1,LOC,SCALE)

          LET A = UNISURV(X,LOC,SCALE)
          LET A = UNIISURV(P,LOC,SCALE)

     ii) The following miscellaneous functions were added.

          LET A = TRIGAMMA(X)

          LET A = NORPPCV(N,ALPHA)

 8) Plot control updates

    The Dataplot CHARACTER, LINE, SPIKE, BAR, and REGION commands
    and the associated attribute setting commands (e.g., CHARACTER
    SIZE) support up to 100 settings.

    In most applications, only the first few settings need to be made.
    However, there are occassions where we would like to be able
    to change the setting for a high trace number without entering
    the values for all the lower trace numbers.  This can now be
    done with commands of the form

         LET PLOT CHARACTER      24 = CIRCLE
         LET PLOT CHARACTER FILL 24 = ON

    In this syntax, the number 24 is the index of the trace being
    set.  The attribute being set ("CHARACTER" and "CHARACTER FILL"
    in the above examples) is given on the left hand side of the
    equal sign.  The assigned value is given on the right hand side
    of the equal sign.

    You can set several values at once as follows

        LET IINDEX = DATA 21 22 23 24
        LET PLOT CHARACTER FILL IINDEX = ON

        LET IINDEX = DATA 21 22 23 24
        LET PLOT CHARACTER FILL IINDEX = ON OFF ON OFF

    In the first example, traces 21, 22, 23, and 24 will all be
    set to ON.  In the second example, trace 21 will be set to ON,
    trace 22 will be set to OFF, trace 23 will be set to ON and
    trace 24 will be set to OFF.

 9) Miscellaneous updates

    a) When the command 

         SET FATAL ERROR PROMPT

       is given, Dataplot will now print a trace of all macros called
       at the time of the error.

    b) The LET ... = CROSS TABULATE .... command will now accept six
       cross tabulation factors (up from four).

    c) Dataplot uses the GD graphics library to generate plots
       in JPEG, PNG, and GIF format.  In addition, images in these
       formats can be read into Dataplot (as numerical variables).

       We updated Dataplot to use the newest version of GD (2.1).
       In doing so, we added support for several additional
       image formats:

           i) BMP  - a common format in the Windows environment.
          ii) WBMP - a black and white format intended for mobile/
                     wireless applications.  Not commonly used
                     now.
         iii) TGA  - Targa format.
          iv) TIFF - the widely used TIFF format.  Note that this
                     requires the LIBTIFF library to be installed,
                     so it may not be available on some installations.
           v) WEBP - this is a new format being championed by Google
                     for displaying videos on the web.  This requires
                     the VPX library to be installed.  This will
                     probably not be available on most installations
                     by default.

    d) Added the command

          SET FIT AUXILLARY FILES <ON/OFF>

       By default, the FIT command writes information to the temporary
       files dpst1f.dat, dpst2f.dat, dpst3f.dat, dpst4f.data, and
       dpst5f.dat.  If you set this switch to OFF, you can suppress this
       writing to files.  Note that although this switch was added for
       internal Dataplot usage, it can also be entered explicitly.

    e) Added the command

          SET NORMAL PLOT AXES <DEFAULT/REVERSE>

       This reverses the role of the x- and y-axis on the NORMAL PLOT.

    f) Added the command

          SET LATEX RESIZE <ON/OFF>

       This command will add the Latex line

          \resizebox{\linewidth}{!} {

       to the beginning of LaTex tables.  This is useful for tables with a
       large number of columns (it automatically resizes the text size if
       needed).

    g) Added the command

          SET CIRCLE CORRECTION <ON/OFF>

       By default, Dataplot applies a correction factor to circles based
       on the ratio of vertical pixels to horizontal pixels.  When the
       number of pixels in each direction is not equal, this has the
       advantage of maintaining the circular appearance of the circle.
       However, the coordinates of the circle may not be as expected.

       If this switch is set to OFF, the coordinates will be as
       expected.  However, it is possible that the circle will have an
       elliptic rather than a circular appearance.

    h) Added the command

         SET SEARCH DIRECTORY  <directory name>

       This adds a directory that will be searched for file names.
       For example, you might want to create a directory to store
       commonly called macros.

       When Dataplot encounters a file name, it will first try to
       find it in the current directory.  If it is not found there,
       it will then search the directory, if any, specified by the
       SET SEARCH DIRECTORY command.  If the file is still not
       found, Dataplot will then search the Dataplot auxillary
       directories.

    i) Made the following updates to the READ command.

          i) After a READ, the following parameters are saved:

               ISKIP  - the number of header lines skipped
               NUMLRD - the number of data lines read
               NUMVRD - the number of variables read

             In addition, the variable names read are saved in the
             strings ZZZV1, ZZZV2, ZZZV3, and so on.  Note that these
             parameters and strings are saved each time a READ is
             performed.

         ii) If you read from a file without specifying a list of
             variables, Dataplot would previously do the following.

             If a SKIP AUTOMATIC was in effect, Dataplot would search
             for a line starting with four dashes ("----").  It would
             then assume that the line preceeding this contained the
             list of variable names.

             On the other hand, if a SKIP N was in effect, Dataplot
             would read the first line after the header to determine
             the number of variables.  It would then create variable
             names of the form X1, X2, X3, and so on.

             This has been modified. Now, for either the
             SKIP AUTOMATIC or the SKIP N cases, you can specify
             whether to create the variable names automatically
             or to read them from the data file with the command

                SET VARIABLE NAME <AUTOMATIC/FILE>

             The AUTOMATIC option specifies that automatic variable
             names will be created.  If the FILE option is specified,
             Dataplot will try to read the variable names from the
             the file.  If SKIP N is in effect, Dataplot will check
             the last line of the header to see if it starts with
             4 dashes.  If so, it will assume the variable names
             are in the preceeding line.  If not, it will assume the
             variable lines are the last line in the header.

             Also, the default for automatic variable names has been
             changed from X1, X2, and so on to COL1, COL2, and so on.
             You can specify the base (e.g., X or COL) for the variable
             names with the command

                 SET AUTOMATIC VARIABLE BASE NAME <value>

    j) Dataplot has added a number of commands for accessing the system
       clipboard.  This capability is operating system and compiler
       dependent.  It is currently supported under Windows for the
       Intel Fortran compiler (support for Linux and Mac OS X is
       still under development).

       Enter HELP CLIPBOARD for more information.

    k) Added the commands

         CALL EXIT
         CALL EXIT ALL

       The CALL EXIT command will exit the currently executing macro
       and the CALL EXIT ALL command will exit all currently running
       macros and return control to the keyboard.  These commands can be
       useful for general purpose macros where an error is detected.

    l) Added the commands

         WEB SEARCH <key1> ... <keyn>

10) Fixed a number of bugs.

-----------------------------------------------------------------------
The following enhancements were made to DATAPLOT
December 2010 - December 2013.
-----------------------------------------------------------------------

 1) The following library functions were added:

        LET YOUT = MERGE(Y1,Y2,TAG)
        LET YOUT = MERGE3(Y1,Y2,Y3,TAG)
        LET YOUT = RELDIF(Y1,Y2)
        LET YOUT = RELERR(Y1,Y2)
        LET YOUT = PERCDIF(Y1,Y2)
        LET YOUT = PERCERR(Y1,Y2)
        LET YOUT = ANGRAD(X1,Y1,X2,Y2,X3,Y3)
        LET YOUT = DPNTLINE(X1,Y1,X2,Y2,SLOPE)
        LET YOUT = SLOPE(X1,Y1,X2,Y2)
        LET YOUT = LININTER(X1,Y1,X2,Y2,X3)

    In addition, the MIN and MAX functions were updated to handle up to
    eight input arguments (previously limited to two arguments).

 2) The following string commands were added:

    a) LET IVAL = STRING COMPARE A B

       IVAL will be set to 1 if strings A and B are identical and
       set to 0 if they are not identical.

    b) LET SBASE = GROUP LABEL TO STRINGS IG

       This command will convert the group labels in IG to
       the strings SBASE1, SBASE2, ..., SBASEN (where N is the
       number of group labels in IG).

    c) Several enhancements were made to the use of row labels.

       The command 

           LET ROWLABEL = <ix>

       previously would convert a character variable, <ix>,
       found in the character data file (dpzchf.dat) to row
       labels.  This command was expanded to include the
       numeric variables as well.  One example where this can
       be useful is when lab-id's are used to label plot points.
       Note that the dpzchf.dat file is searched first.  If no
       match is found there, then Dataplot will check the list
       of currently defined numeric variables.

       The command

          LET ROWLABEL = STRING TO ROW LABEL <irow> <s>

       was added.  This will set the <irow>-th row label to <s>.
       If <s> is a previously defined string, then the contents
       of that string will be used.  If no previously defined string
       is found, then <s> is treated as literal text.

       The command

           LET ROWLABEL <ivalue> = <string>

       will define the <ivalue>-th row of the row labels to <string>.

       The commands

           LET ROWLABEL = SHIFT LEFT  <ivalue>
           LET ROWLABEL = SHIFT RIGHT <ivalue>

       will shift the row label by <ivalue> rows either left
       (= down) or right (up).  The vacated rows will be set to
       blank.

       The command

           LET ROWLABEL = DELETE

       will re-initialize all row labels to blank.

 3) The following enhancements were made to the LET subcommands.

    a) The following new statistic LET subcommands were added:

          LET A = SHANNON DIVERSITY INDEX Y
          LET A = SIMPSON DIVERSITY INDEX Y
          LET A = ROBUST POOLED STANDARD DEVIATION Y
          LET A = ROBUST POOLED RANGE Y
          LET A = UNIQUE X
          LET A = EXCESS KURTOSIS Y
          LET A = SUM OF SQUARES Y
          LET A = DIFFERENCE OF SUM OF SQUARES Y1 Y2
          LET A = SUM OF SQUARES FROM MEAN Y
          LET A = DIFFERENCE OF SUM OF SQUARES FROM MEAN Y1 Y2
          LET A = RESCALED SUM Y
          LET A = RLP Y

          LET QUANT = <value>
          LET A = Q QUANTILE RANGE Y

          LET A = PERCENT AGREE Y1 Y2
          LET A = PERCENT DISAGREE Y1 Y2
          LET A = GROUPED POISSON DISPERSION TEST        Y X
          LET A = GROUPED POISSON DISPERSION TEST CDF    Y X
          LET A = GROUPED POISSON DISPERSION TEST PVALUE Y X
          LET A = CORRELATION PVALUE Y1 Y2
          LET A = CORRELATION CDF Y1 Y2
          LET A = CORRELATION ABSOLUTE VALUE Y1 Y2
          LET A = RANK CORRELATION ABSOLUTE VALUE Y1 Y2
          LET A = RANK CORRELATION CDF Y1 Y2
          LET A = RANK CORRELATION PVALUE Y1 Y2
          LET A = RANK CORRELATION LOWER TAILED PVALUE Y1 Y2
          LET A = RANK CORRELATION UPPER TAILED PVALUE Y1 Y2
          LET A = KENDALL TAU ABSOLUTE VALUE Y1 Y2
          LET A = KENDALL TAU CDF Y1 Y2
          LET A = KENDALL TAU PVALUE Y1 Y2
          LET A = KENDALL TAU LOWER TAILED PVALUE Y1 Y2
          LET A = KENDALL TAU UPPER TAILED PVALUE Y1 Y2
          LET A = PARTIAL CORRELATION Y1 Y2 Y3
          LET A = PARTIAL CORRELATION PVALUE Y1 Y2 Y3
          LET A = PARTIAL CORRELATION CDF Y1 Y2 Y3
          LET A = PARTIAL CORRELATION ABSOLUTE VALUE Y1 Y2 Y3
          LET A = PARTIAL RANK CORRELATION Y1 Y2 Y3
          LET A = PARTIAL RANK CORRELATION ABSOLUTE VALUE Y1 Y2 Y3
          LET A = PARTIAL KENDALL TAU CORRELATION Y1 Y2 Y3
          LET A = PARTIAL KENDALL TAU ABSOLUTE VALUE Y1 Y2 Y3
          LET A = INDEX FIRST MATCH Y1 Y2
          LET A = INDEX LAST  MATCH Y1 Y2
          LET A = INDEX FIRST NOT MATCH Y1 Y2
          LET A = INDEX LAST  NOT MATCH Y1 Y2

          LET A = WEIGHTED ORDER STATISTIC MEAN  Y W
          LET A = WEIGHTED SUM   Y W
          LET A = WEIGHTED SUM OF SQUARES   Y W
          LET A = WEIGHTED SUM OF ABSOLUTE VALUES   Y W
          LET A = WEIGHTED AVERAGE OF ABSOLUTE VALUES   Y W
          LET A = WEIGHTED SUM OF DEVIATIONS FROM THE MEAN   Y W
          LET A = WEIGHTED SUM OF SQUARED DEVIATIONS FROM THE MEAN   Y W

          LET A = A BASIS NORMAL                                Y
          LET A = A BASIS LOGNORMAL                             Y
          LET A = A BASIS WEIBULL                               Y
          LET A = A BASIS NONPARAMETRIC                         Y
          LET A = B BASIS NORMAL                                Y
          LET A = B BASIS LOGNORMAL                             Y
          LET A = B BASIS WEIBULL                               Y
          LET A = B BASIS NONPARAMETRIC                         Y
          LET A = LOWER CONFIDENCE LIMIT                        Y
          LET A = UPPER CONFIDENCE LIMIT                        Y
          LET A = ONE SIDED LOWER CONFIDENCE LIMIT              Y
          LET A = ONE SIDED UPPER CONFIDENCE LIMIT              Y
          LET A = LOWER PREDICTION LIMIT                        Y
          LET A = UPPER PREDICTION LIMIT                        Y
          LET A = ONE SIDED LOWER PREDICTION LIMIT              Y
          LET A = ONE SIDED UPPER PREDICTION LIMIT              Y
          LET A = LOWER PREDICTION BOUND                        Y
          LET A = UPPER PREDICTION BOUND                        Y
          LET A = ONE SIDED LOWER PREDICTION BOUND              Y
          LET A = ONE SIDED UPPER PREDICTION BOUND              Y
          LET A = LOWER SD CONFIDENCE LIMIT                     Y
          LET A = UPPER SD CONFIDENCE LIMIT                     Y
          LET A = ONE SIDED LOWER SD CONFIDENCE LIMIT           Y
          LET A = ONE SIDED UPPER SD CONFIDENCE LIMIT           Y
          LET A = LOWER SD PREDICTION LIMIT                     Y
          LET A = UPPER SD PREDICTION LIMIT                     Y
          LET A = ONE SIDED LOWER SD PREDICTION LIMIT           Y
          LET A = ONE SIDED UPPER SD PREDICTION LIMIT           Y
          LET A = CUMULATIVE SUM FORWARD TEST                   Y
          LET A = CUMULATIVE SUM FORWARD TEST PVALUE            Y
          LET A = CUMULATIVE SUM BACKWARD TEST                  Y
          LET A = CUMULATIVE SUM BACKWARD TEST PVALUE           Y
          LET A = DIXON TEST                                    Y
          LET A = DIXON MINIMUM TEST                            Y
          LET A = DIXON MAXIMUM TEST                            Y
          LET A = EXTREME STUDENTIZED DEVIATE TEST              Y
          LET A = FREQUENCY TEST CDF                            Y
          LET A = FREQUENCY TEST                                Y
          LET A = FREQUENCY WITHIN A BLOCK TEST CDF             Y
          LET A = FREQUENCY WITHIN A BLOCK TEST                 Y
          LET A = GRUBB TEST                                    Y
          LET A = GRUBB TEST CDF                                Y
          LET A = GRUBB TEST DIRECTION                          Y
          LET A = GRUBB TEST INDEX                              Y
          LET A = JARQUE BERA TEST                              Y
          LET A = JARQUE BERA TEST CDF                          Y
          LET A = JARQUE BERA TEST PVALUE                       Y
          LET A = MEAN SUCCESSIVE DIFFERENCE TEST               Y
          LET A = MEAN SUCCESSIVE DIFFERENCE TEST NORMALIZED    Y
          LET A = MEAN SUCCESSIVE DIFFERENCE TEST CDF           Y
          LET A = MEAN SUCCESSIVE DIFFERENCE TEST PVALUE        Y
          LET A = NORMAL TOLERANCE K FACTOR                     Y
          LET A = NORMAL TOLERANCE LOWER LIMIT                  Y
          LET A = NORMAL TOLERANCE UPPER LIMIT                  Y
          LET A = NORMAL TOLERANCE ONE SIDED K FACTOR           Y
          LET A = NORMAL TOLERANCE ONE SIDED LOWER LIMIT        Y
          LET A = NORMAL TOLERANCE ONE SIDED UPPER LIMIT        Y
          LET A = ONE SAMPLE SIGN TEST                          Y
          LET A = ONE SAMPLE SIGN TEST CDF                      Y
          LET A = ONE SAMPLE SIGN TEST PVALUE                   Y
          LET A = ONE SAMPLE SIGN TEST LOWER TAIL PVALUE        Y
          LET A = ONE SAMPLE SIGN TEST UPPER TAIL PVALUE        Y
          LET A = ONE SAMPLE T TEST                             Y
          LET A = ONE SAMPLE T TEST CDF                         Y
          LET A = ONE SAMPLE T TEST PVALUE                      Y
          LET A = ONE SAMPLE T TEST LOWER TAIL PVALUE           Y
          LET A = ONE SAMPLE T TEST UPPER TAIL PVALUE           Y
          LET A = ONE SAMPLE WILCOXON SIGNED RANK TEST          Y
          LET A = ONE SAMPLE WILCOXON SIGNED RANK TEST CDF      Y
          LET A = ONE SAMPLE WILCOXON SIGNED RANK TEST PVALUE   Y
          LET A = ONE SAMPLE WILCOXON TEST LOWER TAILED PVALUE  Y
          LET A = ONE SAMPLE WILCOXON TEST UPPER TAILED PVALUE  Y
          LET A = POISSON DISPERSION TEST                       Y
          LET A = POISSON DISPERSION TEST CDF                   Y
          LET A = POISSON DISPERSION TEST PVALUE                Y
          LET A = SUMMARY NORMAL TOLERANCE K FACTOR             MEAN SD N
          LET A = SUMMARY NORMAL TOLERANCE LOWER LIMIT          MEAN SD N
          LET A = SUMMARY NORMAL TOLERANCE UPPER LIMIT          MEAN SD N
          LET A = SUMMARY NORMAL TOLERANCE ONE SIDED K FACTOR   MEAN SD N
          LET A = SUMMARY NORMAL TOLERANCE ONE SIDED LOWER LIMI MEAN SD N
          LET A = SUMMARY NORMAL TOLERANCE ONE SIDED UPPER LIMI MEAN SD N
          LET A = SUMMARY LOWER PREDICTION LIMIT                MEAN SD N
          LET A = SUMMARY UPPER PREDICTION LIMIT                MEAN SD N
          LET A = SUMMARY ONE SIDED LOWER PREDICTION LIMIT      MEAN SD N
          LET A = SUMMARY ONE SIDED UPPER PREDICTION LIMIT      MEAN SD N
          LET A = SUMMARY LOWER PREDICTION BOUND                MEAN SD N
          LET A = SUMMARY UPPER PREDICTION BOUND                MEAN SD N
          LET A = SUMMARY ONE SIDED LOWER PREDICTION BOUND      MEAN SD N
          LET A = SUMMARY ONE SIDED UPPER PREDICTION BOUND      MEAN SD N
          LET A = SUMMARY LOWER SD CONFIDENCE LIMIT             SD N
          LET A = SUMMARY UPPER SD CONFIDENCE LIMIT             SD N
          LET A = SUMMARY ONE SIDED LOWER SD CONFIDENCE LIMIT   SD N
          LET A = SUMMARY ONE SIDED UPPER SD CONFIDENCE LIMIT   SD N
          LET A = SUMMARY LOWER SD PREDICTION LIMIT             SD N
          LET A = SUMMARY UPPER SD PREDICTION LIMIT             SD N
          LET A = SUMMARY ONE SIDED LOWER SD PREDICTION LIMIT   SD N
          LET A = SUMMARY ONE SIDED UPPER SD PREDICTION LIMIT   SD N
          LET A = TIETJEN MOORE TEST                            Y
          LET A = TIETJEN MOORE MINIMUM TEST                    Y
          LET A = TIETJEN MOORE MAXIMUM TEST                    Y
          LET A = WILK SHAPIRO TEST                             Y
          LET A = WILK SHAPIRO TEST PVALUE                      Y

          LET A = ANGLIT PPCC                       Y
          LET A = ANGLIT PPCC LOCATION              Y
          LET A = ANGLIT PPCC SCALE                 Y
          LET A = ARCSINE PPCC                      Y
          LET A = ARCSINE PPCC LOCATION             Y
          LET A = ARCSINE PPCC SCALE                Y
          LET A = CAUCHY PPCC                       Y
          LET A = CAUCHY PPCC LOCATION              Y
          LET A = CAUCHY PPCC SCALE                 Y
          LET A = COSINE PPCC                       Y
          LET A = COSINE PPCC LOCATION              Y
          LET A = COSINE PPCC SCALE                 Y
          LET A = DOUBLE EXPONENTIAL PPCC           Y
          LET A = DOUBLE EXPONENTIAL PPCC LOCATION  Y
          LET A = DOUBLE EXPONENTIAL PPCC SCALE     Y
          LET A = FATIGUE LIFE PPCC STATISTIC       Y
          LET A = FATIGUE LIFE PPCC LOCATION        Y
          LET A = FATIGUE LIFE PPCC SCALE           Y
          LET A = FATIGUE LIFE PPCC SHAPE           Y
          LET A = GAMMA PPCC STATISTIC              Y
          LET A = GAMMA PPCC LOCATION               Y
          LET A = GAMMA PPCC SCALE                  Y
          LET A = GAMMA PPCC SHAPE                  Y
          LET A = GH PPCC STATISTIC                 Y
          LET A = GH PPCC LOCATION                  Y
          LET A = GH PPCC SCALE                     Y
          LET A = GH PPCC SHAPE ONE                 Y
          LET A = GH PPCC SHAPE TWO                 Y
          LET A = GENERALIZED PARETO PPCC STATISTIC Y
          LET A = GENERALIZED PARETO PPCC LOCATION  Y
          LET A = GENERALIZED PARETO PPCC SCALE     Y
          LET A = GENERALIZED PARETO PPCC SHAPE     Y
          LET A = EXPONENTIAL PPCC                  Y
          LET A = EXPONENTIAL PPCC LOCATION         Y
          LET A = EXPONENTIAL PPCC SCALE            Y
          LET A = HALF CAUCHY PPCC                  Y
          LET A = HALF CAUCHY PPCC LOCATION         Y
          LET A = HALF CAUCHY PPCC SCALE            Y
          LET A = HALF NORMAL PPCC                  Y
          LET A = HALF NORMAL PPCC LOCATION         Y
          LET A = HALF NORMAL PPCC SCALE            Y
          LET A = HYPERBOLIC SECANT PPCC            Y
          LET A = HYPERBOLIC SECANT PPCC LOCATION   Y
          LET A = HYPERBOLIC SECANT PPCC SCALE      Y
          LET A = INVERTED WEIBULL PPCC STATISTIC   Y
          LET A = INVERTED WEIBULL PPCC LOCATION    Y
          LET A = INVERTED WEIBULL PPCC SCALE       Y
          LET A = INVERTED WEIBULL PPCC SHAPE       Y
          LET A = LOGISTIC PPCC                     Y
          LET A = LOGISTIC PPCC LOCATION            Y
          LET A = LOGISTIC PPCC SCALE               Y
          LET A = LOGNORMAL PPCC STATISTIC          Y
          LET A = LOGNORMAL PPCC LOCATION           Y
          LET A = LOGNORMAL PPCC SCALE              Y
          LET A = LOGNORMAL PPCC SHAPE              Y
          LET A = MAXWELL PPCC                      Y
          LET A = MAXWELL PPCC LOCATION             Y
          LET A = MAXWELL PPCC SCALE                Y
          LET A = MAXIMUM GUMBEL PPCC               Y
          LET A = MAXIMUM GUMBEL PPCC LOCATION      Y
          LET A = MAXIMUM GUMBEL PPCC SCALE         Y
          LET A = MINIMUM GUMBEL PPCC               Y
          LET A = MINIMUM GUMBEL PPCC LOCATION      Y
          LET A = MINIMUM GUMBEL PPCC SCALE         Y
          LET A = NORMAL PPCC LOCATION              Y
          LET A = NORMAL PPCC SCALE                 Y
          LET A = RAYLEIGH PPCC                     Y
          LET A = RAYLEIGH PPCC LOCATION            Y
          LET A = RAYLEIGH PPCC SCALE               Y
          LET A = SEMICIRCULAR PPCC                 Y
          LET A = SEMICIRCULAR PPCC LOCATION        Y
          LET A = SEMICIRCULAR PPCC SCALE           Y
          LET A = SLASH PPCC                        Y
          LET A = SLASH PPCC LOCATION               Y
          LET A = SLASH PPCC SCALE                  Y
          LET A = TUKEY LAMBDA PPCC STATISTIC       Y
          LET A = TUKEY LAMBDA PPCC LOCATION        Y
          LET A = TUKEY LAMBDA PPCC SCALE           Y
          LET A = TUKEY LAMBDA PPCC SHAPE           Y
          LET A = UNIFORM PPCC LOCATION             Y
          LET A = UNIFORM PPCC SCALE                Y
          LET A = WALD PPCC STATISTIC               Y
          LET A = WALD PPCC LOCATION                Y
          LET A = WALD PPCC SCALE                   Y
          LET A = WALD PPCC SHAPE                   Y
          LET A = WEIBULL PPCC STATISTIC            Y
          LET A = WEIBULL PPCC LOCATION             Y
          LET A = WEIBULL PPCC SCALE                Y
          LET A = WEIBULL PPCC SHAPE                Y

          LET SIGMA = <VALUE>
          LET A = CHI-SQUARE SD TEST                                 Y
          LET A = CHI-SQUARE SD TEST CDF                             Y
          LET A = CHI-SQUARE SD TEST PVALUE                          Y
          LET A = CHI-SQUARE SD TEST LOWER TAIL PVALUE               Y
          LET A = CHI-SQUARE SD TEST UPPER TAIL PVALUE               Y

          LET A = F TEST                                             Y1 Y2
          LET A = F TEST CDF                                         Y1 Y2
          LET A = F TEST PVALUE                                      Y1 Y2
          LET A = KLOTZ TEST                                         Y1 Y2
          LET A = KLOTZ TEST CDF                                     Y1 Y2
          LET A = KLOTZ TEST PVALUE                                  Y1 Y2
          LET A = KLOTZ TEST LOWER TAILED PVALUE                     Y1 Y2
          LET A = KLOTZ TEST UPPER TAILED PVALUE                     Y1 Y2
          LET A = KRUSKAL WALLIS TEST                                Y  X
          LET A = KRUSKAL WALLIS TEST CDF                            Y  X
          LET A = KRUSKAL WALLIS TEST PVALUE                         Y  X
          LET A = MANN WHITNEY RANK SUM TEST                         Y1 Y2
          LET A = MANN WHITNEY RANK SUM TEST CDF                     Y1 Y2
          LET A = MANN WHITNEY RANK SUM TEST PVALUE                  Y1 Y2
          LET A = MANN WHITNEY RANK SUM LOWER TAIL PVALUE            Y1 Y2
          LET A = MANN WHITNEY RANK SUM UPPER TAIL PVALUE            Y1 Y2
          LET A = MANN WHITNEY U STATISTIC                           Y1 Y2
          LET A = TWO SAMPLE CHI SQUARE TEST                         Y1 Y2
          LET A = TWO SAMPLE CHI SQUARE TEST CDF                     Y1 Y2
          LET A = TWO SAMPLE CHI SQUARE TEST PVALUE                  Y1 Y2
          LET A = TWO SAMPLE KOLMOGOROV SMIRNOV TEST                 Y1 Y2
          LET A = TWO SAMPLE KOLMOGOROV SMIRNOV TEST CRITICAL VALUE  Y1 Y2
          LET A = TWO SAMPLE SIGN TEST                               Y1 Y2
          LET A = TWO SAMPLE SIGN TEST CDF                           Y1 Y2
          LET A = TWO SAMPLE SIGN TEST PVALUE                        Y1 Y2
          LET A = TWO SAMPLE SIGN TEST LOWER TAIL PVALUE             Y1 Y2
          LET A = TWO SAMPLE SIGN TEST UPPER TAIL PVALUE             Y1 Y2
          LET A = TWO SAMPLE T TEST                                  Y1 Y2
          LET A = TWO SAMPLE T TEST CDF                              Y1 Y2
          LET A = TWO SAMPLE T TEST PVALUE                           Y1 Y2
          LET A = TWO SAMPLE T TEST LOWER TAILED PVALUE              Y1 Y2
          LET A = TWO SAMPLE T TEST UPPER TAILED PVALUE              Y1 Y2
          LET A = TWO SAMPLE WILCOXON SIGNED RANK TEST               Y1 Y2
          LET A = TWO SAMPLE WILCOXON SIGNED RANK TEST CDF           Y1 Y2
          LET A = TWO SAMPLE WILCOXON SIGNED RANK TEST PVALUE        Y1 Y2
          LET A = TWO SAMPLE WILCOXON TEST LOWER TAILED PVALUE       Y1 Y2
          LET A = TWO SAMPLE WILCOXON TEST UPPER TAILED PVALUE       Y1 Y2

          LET A = ANDERSON DARLING K SAMPLE TEST                     Y X
          LET A = ANDERSON DARLING K SAMPLE TEST CRITICAL VALUE      Y X
          LET A = MEDIAN TEST                                        Y X
          LET A = MEDIAN TEST CDF                                    Y X
          LET A = MEDIAN TEST PVALUE                                 Y X
          LET A = SQUARED RANK TEST                                  Y X
          LET A = SQUARED RANK TEST CDF                              Y X
          LET A = SQUARED RANK TEST PVALUE                           Y X
          LET A = SQUARED RANK TEST LOWER TAILED PVALUE              Y X
          LET A = SQUARED RANK TEST UPPER TAILED PVALUE              Y X

    b) The statistic LET subcommands now support matrix arguments.
       For example,

            LET A = MEAN M
            LET A = DIFFERENCE OF MEANS M N

       where M and N are matrices.  Note that the matrix will be converted
       to a variable (in a columnwise order) before applying the command.
       This means that the number of rows times the number of columns must
       be less than or equal to the maximum number of rows per variable
       (this is set to 1,000,000 on most current implementations).

       Be aware that Dataplot distinguishes between "statistic" and "math"
       LET subcommands.  The statistic LET subcommands work with variables
       on the right hand side and always return a parameter (i.e., scalar)
       value.  The math LET subcommands may have a mix of parameters and
       variables on both the left and right hand sides.  At the current
       time, only those math LET subcommands that explicitly work with
       matrices support matrix arguments.  Volume II of the online Reference
       Manual provides separate chapters for the "statistic", "math", and
       "matrix" LET subcommands.  It is only those commands in the
       "statistic" chapter that are affected by this update.

    c) The following new math LET subcommands were added:

          LET X FREQ CDF = MANN WHITNEY U STATISTIC FREQUENCY N1 N2
          LET TAG = KEEP X XKEEP
          LET TAG = OMIT X XOMIT
          LET Y2 TAG = THRESHOLD MINIMUM Y TVAL
          LET Y2 TAG = THRESHOLD MAXIMUM Y TVAL

          LET Y = CUMULATIVE <STAT> Y
          LET Y = CROSS TABULATE CUMULATIVE <STAT> Y X

          LET Y = WEIBULL MOMENT ESTIMATORS X

          LET Y = PERCENTAGE RANK X
          LET Y = EXPAND XLAB XVAL

          LET Y2 = JSCORE Z ROUND

          LET Y2 = ISO 13528 ZSCORE         Y XREF SIGMA
          LET Y2 = ISO 13528 ZPRIME         Y XREF SIGMA
          LET Y2 = ISO 13528 EN SCORE       Y ULAB XREF UREF
          LET Y2 = ISO 13528 ZETA SCORE     Y ULAB XREF UREF
          LET Y2 = ISO 13528 EZMINUS SCORE  Y ULAB XREF UREF
          LET Y2 = ISO 13528 EZPLUS  SCORE  Y ULAB XREF UREF

          LET MOUT = MATRIX COMBINE COLUMNS M N
          LET MOUT = MATRIX COMBINE ROWS    M N

          LET MOUT = PARTIAL CORRELATION MATRIX  M
          LET MOUT = PARTIAL CORRELATION CDF MATRIX  M
          LET MOUT = PARTIAL CORRELATION PVALUE MATRIX  M
          LET MOUT = CORRELATION CDF MATRIX  M
          LET MOUT = CORRELATION PVALUE MATRIX  M

          LET YOUT = LOW PASS FILTER Y
          LET YOUT = HIGH PASS FILTER Y

          LET TAG = POINTS IN POLYGON XVAL YVAL XPOLY YPOLY
          LET Y2 X2 = TRANSFORM POINTS Y X TX TY SX SY THETA
          LET Y2 X2 = EXTREME POINTS Y X
          LET Y2 X2 = LINE INTERSECTIONS X1 Y1 X2 Y2 X3 Y3 X4 Y4
          LET Y2 X2 = PARALLEL LINES X1 Y1 X2 Y2 X3 Y3
          LET Y2 X2 = PERPINDICULAR LINES X1 Y1 X2 Y2 X3 Y3
          LET YINDEX = NEAREST NEIGHBOR INDEX Y X
          LET YDIST  = NEAREST NEIGHBOR DISTANCE Y X
          LET YINDEX YDIST  = NEAREST NEIGHBOR Y X
          LET Y2 X2 TAG2 = NEAREST NEIGHBOR JOIN Y1 X1 YINDEX
          LET Y3 X3 DIST = FIRST NEAREST NEIGHBOR Y1 X1 Y2 X2
          LET Y3 X3 DIST TAG1 TAG2 = ALL NEAREST NEIGHBORS Y1 X1 Y2 X2

          LET Y2 X2 YCODED = BINNED CODED Y

       The INTEGRAL command was updated to allow indefinite integrals
       (i.e., either the lower limit or the upper limit is infinity).
       If you specify the lower limit as CPUMIN or -INFINIY or you
       specify the upper limit as CPUMAX or INFINITY, then the
       indefinite integration code will automatically be invoked.
       You do not have to define CPUMIN/CPUMAX/INIFINITY (Dataplot
       checks for the literal text, not the value of any parameter
       that may be defined these strings).

 4) The following enhancements were made to the graphics commands.

    a) The following graphics commands were added

           ISO 13528 PLOT Y Z ROUND LABID LAB

           ISO 13528 ZSCORE PLOT Z MATID ROUNDID
           ISO 13528 JSCORE PLOT Z MATID ROUNDID

           ISO 13528 RLP PLOT Z LABID MATID

    b) The following updates were made to the HOMOSCEDASTICITY PLOT:

          i) Added support for the MULTIPLE option.
         ii) Added support for the SUBSET (or HIGHLIGHT) option.
        iii) Allow more than one group-id variable.
         iv) Allow alternate measures for location and scale.
          v) Added support for summary data.
         vi) Added support for the "circle technique" to identify
             non-homogeneous labs.
        vii) Added support for the TO syntax.

       Enter HELP HOMOSCEDASTICITY PLOT for details.

    c) Added several options to the BLOCK PLOT.  Enter HELP BLOCK PLOT
       for details.

    d) Added the command 

          FRECHET PLOT Y

       This is similar to a Weibull plot, but it fits a 2-parameter
       Frechet (maximum case) rather than the 2-parameter Weibull.

    e) Several enhancements were made to the I PLOT command.

        i) Added the options

              MEAN I PLOT Y X
              MIDMEAN I PLOT Y X
              MIDRANGE I PLOT Y X
              TRIMMED MEAN I PLOT Y X
              BIWEIGHT I PLOT Y X

              MEAN CONFIDENCE LIMIT PLOT Y X
              MEDIAN CONFIDENCE LIMIT PLOT Y X
              QUANTILE CONFIDENCE LIMIT PLOT Y X
              BIWEIGHT CONFIDENCE LIMIT PLOT Y X
              TRIMMED MEAN PLOT CONFIDENCE LIMIT PLOT Y X
              ONE STANDARD ERROR PLOT Y X
              TWO STANDARD ERROR PLOT Y X
              ONE STANDARD DEVIATION PLOT Y X
              TWO STANDARD DEVIATION PLOT Y X
              NORMAL TOLERANCE LIMIT PLOT Y X
              NORMAL PREDICTION LIMIT PLOT Y X
              AGRESTI COULL LIMIT PLOT Y X

       ii) Added the REPLICATION option

              REPLICATED I PLOT Y X1 .... XK

           where there can be from one to six replication variables.

           In addition, there is a special replication syntax when
           there are exactly two replication variables

               I PLOT Y X1 X2

           (that is, omit the REPLICATED keyword).

           Enter HELP I PLOT for a description of using replication
           variables.

    f) Continued to add support for the MULTIPLE, REPLICATION, and
       HIGHLIGHT options and support for MATRIX arguments.  Specifically,

          i) Support for MULTIPLE option:

             ANOP PLOT, I PLOT, INFLUENCE CURVE, PERCENT POINT PLOT,
             RUN SEQUENCE PLOT, VIOLIN PLOT

         ii) Support for REPLICATION option:

             I PLOT, PERCENT POINT PLOT, RUN SEQUENCE PLOT, VIOLIN PLOT

        iii) Support for HIGHLIGHT option:

             BIHISTOGRAM, LAG PLOT, NORMAL PLOT, PERCENT POINT PLOT,
             QUANTILE-QUANTILE PLOT, RUN SEQUENCE PLOT, SHIFT PLOT,
             TUKEY MEAN DIFFERENCE PLOT, WEIBULL PLOT, 4-PLOT

         iv) Support for matrix arguments (not supported when the
             REPLICATION option is used):

             ANOP PLOT, BIHISTOGRAM, COMPLEX DEMODULATION, DUANE PLOT,
             I PLOT, INFLUENCE CURVE, LAG PLOT, NORMAL PLOT, 
             PARALLEL COORDINATES PLOT, PERCENT POINT PLOT,
             QUANTILE-QUANTILE PLOT, RUN SEQUENCE PLOT, SHIFT PLOT,
             STEM AND LEAF PLOT, TUKEY MEAN DIFFERENCE PLOT,
             VIOLIN PLOT, WEIBULL PLOT, 4-PLOT

             Note that matrix arguments will be converted to a variable
             in a column-wise fashion.  So the number of rows times the
             number of columns must be less than the maximum number of
             rows for a variable (this is set to 1,000,000 on most
             systems, but it may vary from this).

 5) The following updates were made to the Analysis commands.

    a) The following non-parametric tests were added:

           COX STUART TEST Y         - sign test for trend
           KLOTZ TEST Y1 Y2          - two-sample test for equal variances
           MEDIAN TEST Y X           - test for equal medians for k groups
           SQUARED RANKS TEST Y X    - test for equal variances for k groups
           FISHER TWO SAMPLE RAND    - two sample Fisher randomization test
                  TEST Y1 Y2           for equal location
           PAGE TEST Y X1 X2         - Page test for two factor ANOVA
           QUADE TEST Y X1 X2        - Quade test for two factor ANOVA

           KENDALL TAU INDEPENDENCE TEST Y1 Y2       - two sample
                                                       independence test
           RANK CORRELATION INDEPENDENCE TEST Y1 Y2  - two sample
                                                       independence test

    b) Added the commands

           PREDICTION LIMITS               - prediction limits for the
                                             mean of new observations
           PREDICTION BOUNDS               - prediction limits to cover
                                             all new observations

           SD CONFIDENCE LIMITS            - confidence limits for the
                                             standard deviation
           SD PREDICTION LIMITS            - prediction limits for the
                                             standard deviation

           CORRELATION CONFIDENCE LIMITS   - confidence limits for the
                                             correlation coefficient based on
                                             Fisher's normal approximation
    c) Added the commands

           JARQUE BERA NORMALITY TEST  - perform a Jarque-Bera test for
                                         normality

           POISSON DISPERSION TEST     - perform the Poisson dispersion
                                         test for Poissonality

           MEAN SUCCESSIVE DIFF TEST   - perform a mean successive
                                         differences test for randomness

           BEST DISTRIBUTIONAL FIT Y   - search for best fitting distribution
                                        (univariate data, continuous
                                        distributions, no censoring)

           MCCOOL WEIBULL LOCATION TEST - test for samples of size 10 to 100
                                          to distinguish between a
                                          3-parameter and a 2-parameter
                                          Weibull distribution

    e) The REPLICATED and MULTIPLE options, support for matrix arguments
       and support for the TO syntax were added to additional analysis
       commands.  In addition, the output was reformatted for many of these
       commands.  Specifically,

          i) Support for MULTIPLE option:

             ABASIS, ANDERSON-DARLING K-SAMPLE, BARTLETT TEST, BBASIS,
             CAPABILITY ANALYSIS, CHI-SQUARE SD TEST, CUMULATIVE SUM,
             F LOCATION TEST, F TEST, FREQUENCY TEST, GOODNESS OF FIT,
             KOLM SMIR TWO SAMPLE TEST, KRUSKAL WALLIS, LEVENE TEST,
             LJUNG BOX TEST, MANN WHITNEY RANK SUM TEST, RUNS, SIGN TEST,
             SUMMARY, T TEST, TOLERANCE LIMITS, TWO SAMPLE CHI-SQUARE,
             VAN DER WAERDEN, WILCOXON SIGNED RANK TEST, WILK-SHAPIRO

             The interpretation of the MULTIPLE option depends on the
             data expected.

             a) When a single response variable is expected (e.g., the
                SUMMARY command), the MULTIPLE option means the test will
                be applied to each response variable independently.

             b) When two variables are expected where the first variable
                is the response variable and the second variable is a
                group-id variable (e.g., the KRUSKAL WALLIS TEST), the
                MULTIPLE option means that each variable is treated as
                a distinct group, no group-id variable is entered, and a
                single test is performed.

             c) When two response variables are expected (e.g., the
                F TEST), the MULTIPLE option will perform the test on all
                the pairwise combinations of response variables.  That is

                    F TEST Y1 TO Y4

                is equivalent to entering

                    F TEST Y1 Y2
                    F TEST Y1 Y3
                    F TEST Y1 Y4
                    F TEST Y2 Y3
                    F TEST Y2 Y4
                    F TEST Y3 Y4

         ii) Support for REPLICATION option:

             ABASIS, BBASIS, CAPABILITY ANALYSIS, CUMULATIVE SUM,
             FREQUENCY TEST, GOODNESS OF FIT, LJUNG-BOX, RUNS, SUMMARY,
             TOLERANCE LIMITS, WILK-SHAPIRO

        iii) Support for matrix arguments:

             All of the commands listed above for the MULTIPLE option
             now support matrix arguments.  The following additional
             commands also support matrix arguments

             BINOMIAL PROPORTION TEST,
             DIFFERENCE OF PROPORTION CONFIDENCE LIMITS,
             PROPORTION CONFIDENCE LIMITS

             Matrix arguments are not supported when the REPLICATION
             option is used.

         iv) Support for RTF formatted output has been extended to
             all cases where previously only HTML/LATEX formatted
             output was supported.  The HTML/LATEX/RTF support was
             extended to a number of additional commands.

          v) Many of the commands have reformatted the output for better
             clarity and readability.  These are not listed individually.

 6) The following miscellaneous commands were added.

    a) Added the command PWD to retrieve the current working
       directory.

    b) Added the command PSVIEW.  This command will preview the
       current plot (i.e., the dppl2f.dat file) with a Postscript
       viewer.  You can specify the program to use as the Postscript
       viewer with the command

           SET POSTSCRIPT VIEWER  /usr/bin/evince

       The default is Ghostview for both Windows and Linux/Unix.

    c) Added the following option to the CAPTURE command:

           CAPTURE SCRIPT <filename>

       This option saves the subsequent commands to a file
       without executing them.  The intended purpose of this
       is to allow scripts for external programs (e.g., Python,
       Perl, and so on) to be created within a Dataplot
       macro.  You can subsequently use the SYSTEM command to
       execute the script.

    d) Dataplot runs Ghostscript behind the scenes for several
       commands.  For Windows platforms, version 9.x of Ghostscript
       provides both 32-bit executables and 64-bit executables.  The
       following command was added to allow you to specify which
       version of Ghostscript is installed on your system

          SET GHOSTSCRIPT VERSION <32/64>

       This command is only applicable on Windows platforms.
       The default is "64".  This command has been added to the
       DPLOGF.TEX file that is installed by the Dataplot Windows
       installation.  If you have the 32-bit version of Ghostscript
       installed, it is recommended that you modify the DPLOGF.TEX
       file.

       This command is ignored for Unix/Linux and Mac OS X platforms.

 7) The following colors were added:

      R0 - R255  - turns on RED with an intensity level from 0
                   to 255
      Z0 - Z255  - turns on GREEN with an intensity level from 0
                   to 255
      B0 - B255  - turns on BLUE with an intensity level from 0
                   to 255

    Note the "Z" was used for GREEN because Gxxx is already used
    for gray scale colors.

    For devices that don't support full RGB colors, these will
    be mapped to RED, GREEN, and BLUE.

 8) A number of bugs have been fixed.

-----------------------------------------------------------------------
The following enhancements were made to DATAPLOT
August 2009 - November 2010.
-----------------------------------------------------------------------

 1) The following enhancements were made to the graphics commands.

    a) Several features are being developed for general implementation
       for the graphics commands.  These will be phased in over
       the next several releases.  Implementation of each of these
       features will be considered for each of the graphics commands
       on a case by case basis.

          i) REPLICATION - for this option, one or more group-id
             variables can be specified.  These group-id variables
             are cross tabulated and the plot is generated for
             each combination of the cross tabulated values.

             The LINE and CHARACTERS commands (and associated attribute
             setting commands) can be used to distinguish the
             various curves.

         ii) MULTIPLE - many Dataplot commands expect syntax like

                 BOX PLOT Y X

             where Y is the response variable and X is a group-id
             variable.

             In many cases, the groups may be in separate
             columns.  The MULTIPLE option will support the
             following syntax

                 MULTIPLE BOX PLOT Y1 Y2 Y3 Y4

             Although you can use the LET ... = STACK command to
             put the data in the Y X form, the MULTIPLE option
             makes that step unnecessary.

             For commands that except a single response (e.g.,
             BOX COX NORMALITY PLOT), the MULTIPLE option can be
             used to overlay several curves on the same plot.

             The MULTIPLE option cannot be used with the
             REPLICATION option.

        iii) SUBSET (highlighting) - for many plots, it may be
             useful to highlight certain points.  The SUBSET option
             typically specifies a group variable.  Based on this
             group variable, you can use the LINE and CHARACTER
             (and related) commands to highlight certain points.
             For example, the highlighted points might be drawn
             in a different color.

             Although this command is similar to REPLICATION, it
             is different.  For example, if you use REPLICATION to
             define two groups for a normal probability plot,
             two distinct probability plots are generated.  On the
             other hand, if you use SUBSET to define two groups for
             the normal probability plot, there is only one probability
             plot generated.  However, the two groups can be plotted
             with different attributes.

         iv) Matrix arguments and TO syntax - several Dataplot
             commands support a TO syntax.  For example,
             READ X1 TO X4 is equivalent to READ X1 X2 X3 X4.
             This syntax will be extended to more commands.

          v) Dataplot supports a matrix type.  Previously, matrix
             arguments were restricted to commands that specifically
             operated on matrices.  Commands that expect a
             univariate argument or that support the MULTIPLE 
             option are good candidates for adding support for
             matrix arguments.  Matrix arguments will not be
             supported for the REPLICATION option or for the case
             where multiple response variables are expected.
             Note that a matrix argument will be treated as a
             variable argument.  For example, NORMAL PROBABILITY PLOT M
             (where M is the name of a matrix) will generate a single 
             probability plot for all values in the matrix.

    b) Added the command

          TABULATION <stat> PLOT Y X1 ... X4  YLEVEL

       This plot is a bit of a mix between a fluctuation plot
       and a contour plot.  Enter HELP TABULATION PLOT for
       details.

       Some enhancements were made to the FLUCTUATION PLOT
       as well.  Enter HELP FLUCTUATION PLOT for details.

    c) Added a "BATCH MULTIPLE" option for the STRIP PLOT
       command.  Enter HELP STRIP PLOT for details.

    d) Made several changes to the HISTOGRAM command.

         i) By default, Dataplot sets the lower and upper class limits
            to xbar -/+ 6*s (with xbar and s denoting the sample mean
            and standard deviation), respectively.  This can
            occassionally result in a few outlying points being excluded
            from the histogram.  To adjust the lower and upper class
            limits so that these outlying points are included, enter the
            command

                 SET HISTOGRAM OUTLIERS ON

            To revert to the default, enter

                 SET HISTOGRAM OUTLIERS OFF

        ii) By default, the histogram draws all cells, even those with
            zero frequency.  To suppress these zero frequency cells,
            enter

                  SET HISTOGRAM EMPTY BINS OFF

             To restore the default, enter

                  SET HISTOGRAM EMPTY BINS ON

       iii) Previously, Dataplot only generated histograms for the case
            where the bin widths were equal.  This has been extended
            to the case with unequal bin widths.  The syntax is

                HISTOGRAM Y XLOW XHIGH

            with XLOW containing the values for the lower bin limit
            and XHIGH containing the values for the upper bin limit.

        iv) Added the following option

                SUBSET HISTOGRAM Y X

            In this case, X is a group-id variable.  This syntax
            can be used to highlight the contribution to the
            histogram for particular subsets of the data.

         v) Fixed a bug in the CUMULATIVE RELATIVE HISTOGRAM
            for the AREA case.  If SET RELATIVE AREA HISTOGRAM
            is set to AREA (the default), relative histograms
            are normalized so that the area is equal to 1 and
            if it set to PERCENT the sum of the bar heights is
            equal to 1.  The PERCENT case did not have a bug.

       For details, enter HELP HISTOGRAM.

    e) Made several enhancements to the FREQUENCY PLOT and
       KERNEL DENSITY commands.

          i) The SET HISTOGRAM OUTLIER option applies to the
             FREQUENCY PLOT.

         ii) As with the HISTOGRAM, non-equispaced bins are
             supported for the FREQUENCY PLOT:

                 FREQUENCY PLOT Y XLOW XHIGH

        iii) The REPLICATED and MULTIPLE options were added
             to these commands.  For the REPLICATED case, either
             one or two replication variables can be specified.
             Support was also added for matrix arguments and
             for the TO syntax.

    f) Made several enhancements to the BOX PLOT command.

       Support was added for the MULTIPLE and REPLICATED
       options.  Up to six replication variables can be
       specified.  The word REPLICATION is optional.

       For the REPLICATED case, you can control the spacing between
       groups.  Internally, Dataplot uses the CODE CROSS TABULATE
       command to generate a single combined group-id variable.  Enter
       HELP CODE CROSS TABULATE for details on how to control the
       spacing (the SET commands used by CODE CROSS TABULATE are
       supported for the BOX PLOT command).

       Support was added for matrix arguments for the MULTIPLE case or
       for the case where only a single argument is given.

    g) Made several enhancements to the

           BOX COX NORMALITY PLOT Y
           BOX COX HOMGENEITY PLOT Y
           BOX COX LINEARITY PLOT Y X

       commands.

       The REPLICATED option is supported for all 3 plots.  Either one or
       two replication variables can be supported.  The MULTIPLE option is
       supported for the BOX COX NORMALITY PLOT.

       The BOX COX NORMALITY plot supports matrix arguments for the MULTIPLE
       case or for the case where only a single argument is given.  The
       TO syntax is supported for all of these commands.

    h) For the PROBABILITY PLOT, added support for the MULTIPLE and
       REPLICATED (for up to 6 replication variables) options,
       support for matrix arguments, and support for the
       HIGHLIGHT option.

       In addition, you can enter the commands

           LET PPLOC   = <value>
           LET PPSCALE = <value>

       before entering the PROBABILITY PLOT command.  This adds location
       and scale parameters to the theoretical distribution.  This is
       intended for the case where the distribution parameters are
       estimated by a non-PPCC method (e.g., maximum likelihood) and
       you want to generate the probability plot using the estimated
       parameters.

    i) For the PPCC PLOT, added support for the MULTIPLE and
       REPLICATED (for up to 2 replication variables) options
       and support for matrix arguments.

    j) The BOOTSTRAP PLOT and JACKNIFE PLOT commands were updated
       to include reports in addition to the plots.

        i) If the BOOTSTRAP/JACKNIFE plot is applied to a statistic
           (e.g., BOOTSTRAP MEAN PLOT Y), the following tables are
           generated:

           1) An initial summary table.

           2) A table containing percent points for the computed
              statistic.

           3) A table containing percentile confidence limits for
              the statistic for various values of alpha.

       ii) If the BOOTSTRAP/JACKNIFE plot is applied to a
           a distributional fit (e.g., BOOTSTRAP WEIBULL PPCC PLOT Y
           or BOOTSTRAP WEIBULL MLE PLOT Y), the following tables are
           generated:

           1) An initial summary table.

           2) A table containing percentile confidence limits for
              each of the parameters of the distribution for various
              levels of alpha.

           3) If the SET MAXIMUM LIKELIHOOD PERCENTILS command was
              given, a table containing confidence limits for the
              specified percentiles will be generated.

       For both cases, the SET WRITE DECIMALS command can be
       used to specify the number of decimals to use in the
       tables and the CAPTURE HTML, CAPTURE LATEX, and
       CAPTURE RTF options are supported.

 2) Added or enhanced the following analysis comamnds:

    a) Similar to the graphics commands, the REPLICATED and MULTIPLE
       options and support for matrix arguments will be added to the
       analysis commands over the next several releases on a case by
       case basis.

    b) The output for the GRUBBS command was modified.

       Support was added for the MULTIPLE and REPLICATED options.
       Up to six replication variables can be specified.

       In addition, the following capability was added:

           GRUBB TEST Y LABID

       The LABID variable is used for identification purposes
       only in the output.

    c) In addition, the following new outlier tests were added:

           DIXON Y
           TIETJEN MOORE Y
           EXTREME STUDENTIZED DEVIATE Y

       The Dixon test is a small sample test for a single outlier.
       The Tietjen-Moore test is an generalization of the Grubbs
       test to the case of more than one outlier where the number
       of outliers must be specified exactly.  The extreme studentized
       deviate test is an generalization of the Grubbs test to the
       case of more than one outlier where only the upper bound on
       the number of outliers must be specified.

       These commands support the MULTIPLE and REPLICATED options
       in a similar manner as the GRUBBS command.

    d) For the following commands

           CONFIDENCE LIMITS
           BIWEIGHT CONFIDENCE LIMITS
           TRIMMED MEAN CONFIDENCE LIMITS
           MEDIAN CONFIDENCE LIMITS
           QUANTILE CONFIDENCE LIMITS
           DIFFERENCE OF MEANS CONFIDENCE LIMITS

       added support for the MULTIPLE and REPLICATION options.

       In addition, matrix arguments are now supported (except for
       the REPLICATION case).  The MULTIPLE option is not supported
       for the DIFFERENCE OF MEANS CONFIDENCE LIMITS case.

 3) Added the following miscellaneous commands:

    a) CPU TIME - this command prints the current CPU time
       used by the current Dataplot session.

    b) The CHARACTER command now accepts up to 16 characters for
       the plot symbol.  The previous limit was 4 characters.

       This capability is most useful for the case where the
       CHARACTER command is used to label specific points.  In
       particular, it can be useful for the CHARACTER AUTOMATIC
       command.

    c) The command

           CALL filename.dp

       can now also be run by entering

           filename.dp

       That is, the CALL is optional.

 4) Significant restructing is being performed for the
    goodness of fit and goodness of fit plots for probability
    distributions.  Much of this change is to reduce duplicate code,
    to make various goodness of fit commands more consistent,
    and to make some planned future updates easier to implement.

    Although much of this change is primarily internal and should
    be transparent to users, the following updates were made.

    a) The Anderson-Darling option was added as an alternative
       to the PPCC PLOT and KS PLOT:

           <dist> ANDERSON DARLING PLOT Y

       The Anderson-Darling is currently supported for ungrouped
       and uncensored data (i.e., the same as the KS test).

    b) The Anderson-Darling syntax was changed to

           <dist> ANDERSON DARLING GOODNESS OF FIT Y

    c) The output format for the Anderson-Darling, Kolmogorov-Smirnov,
       and chi-square goodness of fit was modified.

    d) The following command was added

           <dist> PPCC GOODNESS OF FIT Y

       This is currently supported for the raw data case (i.e.,
       ungrouped data) without censoring.  Currently, distributions
       with more than one shape parameter are not supported.

    e) The Anderson-Darling, Kolmogorov-Smirnov, and PPCC
       goodness of fit commands were updated to generate appropriate
       critical values dynamically via Monte Carlo simulation.  A
       few comments on this.

          i) There are 2 distinct cases.  In the first case, we
             assume the distribution parameters are known.  This
             is referred to as the "fully specified" case.  In this
             case, the simulations are performed using the specified
             parameters.

             In the second case, the distribution parameters are
             estimated from the data.  For this case, the 
             distribtuion parameters are estimated for each
             Monte Carlo sample using maximum likelihood (for the
             PPCC case, the PPCC plot is used instead of maximum
             likelihood).
  
             The following command is used to specify which method
             is used for the Monte Carlo simulations

                SET GOODNESS OF FIT <FULLY SPECIFIED/ESTIMATE>

         ii) Although the second case (i.e., estimate the parameters
             from the data) is the more realistic case, Dataplot
             does not support this for all distributions.  For the
             K-S and Anderson-Darling cases, a maximum likelihood
             estimation needs to be performed for each set of
             simulated values.  So if Dataplot supports maximum
             likelihood estimation for the specified distribution,
             then the "ESTIMATE" option is likely to be supported.
             The PPCC option is limited to distributions where
             there is at most one shape parameter.

        iii) Published tabes are available for a number of
             distributions for the Anderson-Darling and for
             the fully specified case for the Kolmogorov-Smirnov
             cases.  The advantage of using the published tables
             is speed since the simulation step does not need to
             be performed.  Simulation allows the Anderson-Darling
             and Kolmogorov-Smirnov critical values to be generated
             for cases where published tables are not available and
             also permits p-values to be returned.  Note that the
             critical values returned by Dataplot simulations may
             differ slightly from the published values due to a
             different random number generator being used.

             To specify whether "tabled" critical values or
             similated critical values will be used, enter

                 SET KOLMOGOROV SMIRNOV CRITICAL VALUE <TABLE/SIMU>
                 SET ANDERSON DARLING CRITICAL VALUE <TABLE/SIMU>

    f) The maximum likelihood estimation for the parameters
       of a distribution routines are being reviewed and updated.
       Some of the change is cosmetic (i.e., more consistent
       appearance), but in some cases the fitting algorithms are
       being improved.

       We are also reviewing the computational algorithms for some
       of the probability distributions.

    g) Made the following enhancements to the CONSENSUS MEANS command.

          i) Added Horn-Horn-Duncan and MINMAX estimates for the standard
             error to the DerSimonian Laird estimate.

         ii) Many of the methods require that the standard deviaitions
             be positive for all labs.  Zero standard deviations can
             result when a lab has a single observation or when all
             observations are equal.  Previously, Dataplot omitted
             all labs with zero standard deviations from the analysis
             (they were included in the initial summary table).

             However, some of the methods (specifically, the grand mean,
             mean of means, BOB, Bayesian Consensus Mean, and generalized
             confidence interval methods) can handle these methods.

        iii) The following consensus mean statistics can be computed

                  LET A = SUMMARY DERSIMONIAN LAIRD MEAN SD N
                  LET A = SUMMARY DERSIMONIAN LAIRD STANDARD ERROR MEAN SD N
                  LET A = SUMMARY DERSIMONIAN LAIRD HHD MEAN SD N
                  LET A = SUMMARY DERSIMONIAN LAIRD MINMAX MEAN SD N
                  LET A = SUMMARY MANDEL PAULE MEAN SD N
                  LET A = SUMMARY MANDEL PAULE STANDARD ERROR MEAN SD N
                  LET A = SUMMARY MODIFIED MANDEL PAULE MEAN SD N
                  LET A = SUMMARY MODIFIED MANDEL PAULE STANDARD ERROR MEAN SD N
                  LET A = SUMMARY VANGEL RUKHIN MEAN SD N
                  LET A = SUMMARY VANGEL RUKHIN STANDARD ERROR MEAN SD N
                  LET A = SUMMARY GENERALIZED CONFIDENCE INTERVAL MEAN SD N
                  LET A = SUMMARY GENERALIZED CONFIDENCE INTERVAL STANDARD ERROR
                                  MEAN SD N
                  LET A = SUMMARY BOB MEAN SD N
                  LET A = SUMMARY BOB STANDARD ERROR MEAN SD N
                  LET A = SUMMARY BCP MEAN SD N
                  LET A = SUMMARY BCP STANDARD ERROR MEAN SD N
                  LET A = SUMMARY MEAN OF MEANS MEAN SD N
                  LET A = SUMMARY MEAN OF MEANS STANDARD ERROR MEAN SD N
                  LET A = SUMMARY FAIRWEATHER MEAN SD N
                  LET A = SUMMARY FAIRWEATHER STANDARD ERROR MEAN SD N
                  LET A = SUMMARY SCHILLER-EBERHARDT MEAN SD N
                  LET A = SUMMARY SCHILLER-EBERHARDT STANDARD ERROR MEAN SD N
                  LET A = SUMMARY GRAYBILL DEAL MEAN SD N
                  LET A = SUMMARY GRAYBILL DEAL SINHA STANDARD ERROR MEAN SD N
                  LET A = SUMMARY GRAYBILL DEAL NAIVE STANDARD ERROR MEAN SD N
                  LET A = SUMMARY GRAYBILL DEAL ZHANG ONE STANDARD ERROR MEAN SD N
                  LET A = SUMMARY GRAYBILL DEAL ZHANG TWO STANDARD ERROR MEAN SD N
                  LET A = DERSIMONIAN LAIRDY X
                  LET A = DERSIMONIAN LAIRD STANDARD ERRORY X
                  LET A = DERSIMONIAN LAIRD HHDY X
                  LET A = DERSIMONIAN LAIRD MINMAXY X
                  LET A = MANDEL PAULEY X
                  LET A = MANDEL PAULE STANDARD ERRORY X
                  LET A = MODIFIED MANDEL PAULEY X
                  LET A = MODIFIED MANDEL PAULE STANDARD ERRORY X
                  LET A = VANGEL RUKHINY X
                  LET A = VANGEL RUKHIN STANDARD ERRORY X
                  LET A = GENERALIZED CONFIDENCE INTERVALY X
                  LET A = GENERALIZED CONFIDENCE INTERVAL STANDARD ERRORY X
                  LET A = BOBY X
                  LET A = BOB STANDARD ERRORY X
                  LET A = BCPY X
                  LET A = BCP STANDARD ERRORY X
                  LET A = MEAN OF MEANSY X
                  LET A = MEAN OF MEANS STANDARD ERRORY X
                  LET A = FAIRWEATHERY X
                  LET A = FAIRWEATHER STANDARD ERRORY X
                  LET A = SCHILLER-EBERHARDTY X
                  LET A = SCHILLER-EBERHARDT STANDARD ERRORY X
                  LET A = GRAYBILL DEALY X
                  LET A = GRAYBILL DEAL SINHA STANDARD ERRORY X
                  LET A = GRAYBILL DEAL NAIVE STANDARD ERRORY X
                  LET A = GRAYBILL DEAL ZHANG ONE STANDARD ERRORY X
                  LET A = GRAYBILL DEAL ZHANG TWO STANDARD ERRORY X

        The SUMMARY version of the these commands uses the summary
        statistics (means, standard deviations, sample size) while
        the other cases expect a response variable and a lab-id variable.

        These statistics can also be used with the various commands that
        support more than one response variable.  Enter HELP STATISTICS
        for details.

 5) The following statistics are now supported

        LET A = <H10/H12/H15/H17/H20> LOCATION Y
        LET A = <H10/H12/H15/H17/H20> SCALE Y
        LET A = DIFFERENCE OF <H10/H12/H15/H17/H20> LOCATION Y
        LET A = DIFFERENCE OF <H10/H12/H15/H17/H20> SCALE Y

        LET A = TIETJEN MOORE TEST Y
        LET A = TIETJEN MOORE MINIMUM TEST Y
        LET A = TIETJEN MOORE MAXIMUM TEST Y
        LET A = DIXON TEST Y
        LET A = DIXON MINIMUM TEST Y
        LET A = DIXON MAXIMUM TEST Y
        LET A = EXTREME STUDENTIZED DEVIATE TEST Y

        LET A = BINOMIAL RATIO NSUCC NTRIAL

        LET A = ROOT MEAN SQUARE ERROR Y
        LET A = DIFFERENCE OF ROOT MEAN SQUARE ERROR Y

 6) The following LET sub-commands were added:

        LET LOWLIM UPPLIM = BINOMIAL RATIO CONFIDENCE LIMITS 
                            P1 N1 P2 N2 ALPHA

        LET XCODE = CODE CROSS TABULATE X1 X2
        LET XCODE = CODE CROSS TABULATE X1 X2 X3
        LET XCODE = CODE CROSS TABULATE X1 X2 X3 X4
        LET XCODE = CODE CROSS TABULATE X1 X2 X3 X4 X5
        LET XCODE = CODE CROSS TABULATE X1 X2 X3 X4 X5 X6

        LET AO A0SD A1 A1SD = MATRIX <ROW/COLUMN> FIT M X

        LET Y = RANK INDEX X

        LET Y = COMBINE X1 ... XK

 7) The Windows version was upgraded to use version 11 of the
    Intel compiler (previously version 9 was used).  A number of
    Fortran 77 constructs are no longer supported by this version
    of the compiler.  A large number of coding changes were made
    to make the source code compatible with version 11.  These
    should be transparent (i.e., no change in how commands are
    used, although a number of potential bugs were corrected) to
    users.

 8) Following bug fixes:

    a) Fixed MOVE RELATIVE command.

    b) The standard deviation for the location parameter from
       linear fits (FIT Y X, QUADRATIC FIT Y X, etc.) was
       corrected.

    c) Corrected a bug when using the SET CONVERT CHARACTER command.

    d) Corrected the MEDIAN CONFIDENCE LIMITS for the Maritz-Jarrett
       method.

    e) A number of other miscellaneous bug fixes were made.

-----------------------------------------------------------------------
The following enhancements were made to DATAPLOT
March 2008 - July       2009.
-----------------------------------------------------------------------

 1) For Unix/Linux platforms and the gfortran compiler, added the
    command

       SET PROMPT ADVANCE <ON/OFF>

    This controls whether the Dataplot prompt appears as
    (for the OFF case)

       >
         enter new command here

    or (for the ON case)

       >enter new command here

    Although the ON case is preferred when running the command
    line version, it causes problems when running the GUI version
    with Tcl/Tk only (i.e., not using Expect).  For this reason,
    OFF is the default.  If you typically only run the command
    line version, you may want to add a SET PROMPT ADVANCE ON
    command to your dplogf.tex file (although you will need to
    comment this line out if you want to run the GUI).  Alternatively,
    you can enter the command manually when you initiate Dataplot.
    
 2) The following enhancements were made to the READ/WRITE commands.

    a) When reading numeric fields, Dataplot will check for
       the string

           NaN

       The NaN string is used to denote "not a number" on some
       systems.  If Dataplot encounters this string, it will
       insert the missing value.  This can be set with the
       command (the default is 0)

            SET READ MISSING VALUE  <value>

       The search for NaN is not case sensitive.

    b) Many programs will have a specific alpha string to denote
       a missing value.  You can set a character string that
       denotes a missing value with the command

            SET DATA MISSING VALUE  <value>

       Currently, a maximum of 4 characters is allowed for the
       missing value string.

       When reading numeric fields, Dataplot will check for the
       missing value string (if specified).  If Dataplot encounters
       this string, it will insert the missing value.  This can be set
       with the command (the default is 0)

            SET READ MISSING VALUE  <value>

       The search for the missing value string is not case sensitive.

    c) The maximum number of characters allowed for a single command
       line was increased to 255.  This applies to both reading
       commands from the terminal and reading commands from a file.

       In addition, you can now enter multiple continuation lines
       (previously, only a single continuation line was allowed).
       However, the combined command still has a maximum limit of
       255 characters.

    d) On Linux/Unix platforms, support has been added for the
       GNU readline library.  The readline library allows command
       line editing and history recall.  To use the readline
       capability, enter the command

          SET READ LINE ON

       Note that this capability only applies to commands that
       are entered from the terminal.

       The editing/history capabilities supported by readline are
       documented at the readline web site:

           http://tiswww.case.edu/php/chet/readline/rltop.html

       Dataplot requires version 6.x of the readline library.
       This is the current production version, although many
       systems may still be using version 5.2.

    e) Added the commands

         READ MATRIX TO VARIABLES FILE.DAT Z ROWID COLID

       In many cases

       Whether you read a matrix as a matrix or as 

    f) Added the commands

         READ STACKED VARIABLES FILE.DAT Z GROUPID <vari-list>

    g) Added the commands

         READ IMAGE TO VARIABLES FILE.DAT RED GREEN BLUE ROWID COLID
         READ IMAGE TO VARIABLES FILE.DAT GREY ROWID COLID

         READ IMAGE FILE.DAT GREY
         READ IMAGE RED FILE.DAT RED
         READ IMAGE GREEN FILE.DAT GREEN
         READ IMAGE BLUE FILE.DAT BLUE

       These commands allow you to read image data into Dataplot.
       The GD library is used to read the images in the following
       format:

          1) jpeg
          2) png
          3) gif

       Note: This update is not currently available for the
             Windows implementation.

    h) The WRITE RTF command will print variables in an RTF
       table (support was previously added for WRITE LATEX and
       WRITE HTML).  Note that HTML format is restricted to
       15 variables or less and LATEX and RTF format is
       restricted to 7 variables or less.

    i) Added the command

         TABLE WIDTH  <totwidth>   <nright>

       where <totwidth> and <nright> are variables that specify
       the total width of the field and the number of digits to
       the right of the decimal point.  Row one applies to
       variable one, row two applies to variable two, and so on.

       This is an alternative to the SET WRITE DECIMALS and
       SET WRITE FORMAT commands.  The SET WRITE DECIMALS command
       requires that all variables be printed with the same
       format.  Although the SET WRITE FORMAT allows more flexibility,
       it cannot be used for WRITE <RTF/LATEX/HMTL>.

       Up to 200 rows can be specified (if the number of variables
       being printed is greater than 200, it is recommended that
       you use the SET WRITE FORMAT command).

       A few comments on what can be specified for <ntot> and
       <nright>.  If NTOT and NRIGHT are the values for a given
       row, then the following apply:

       1) A value of -99 indicates that the default value
          should be used (this is 15 for NTOT and 7 for NWIDTH).

       2) If NRIGHT is a positive integer, then Fortran F format
          will be used (e.g.,  "3.26").

       3) If NRIGHT is 0, then an integer format will be used.

       4) If NRIGHT is -2, then a G15.7 format will be used.
          In this case, the Fortran compiler will decide between
          an F-based format or an E-based format depending on
          the particular number.  If NRIGHT is between -3 and
          -20, then a Fortran E-based format (Eyy.xx where the
          absolute value of NRIGHT specifies the "xx") will be
          used.

 3) Added the following graphics commands.

    a) Added the command

          DISCRETE CONTOUR PLOT Z ROW COL Z0

    b) Added the commands

          IMAGE PLOT GREY ROWID COLID
          IMAGE PLOT RED GREEN BLUE ROWID COLID

       This command allows you to render images.  The ability to
       support image plots is dependent on the capabilities of
       the specific graphics device and is currently supported
       on the following devices:

          1) Quickwin     - i.e., the command line version of
                            Dataplot under Microsoft Windows

          2) X11          - currently, only grey scale is supported.

          3) GD           - the GD device is used to generate jpeg,
                            PNG, and gif format files

          4) Postscript

    c) Added the command

          FLUCTUATION <stat> PLOT Y X1 X2
          FLUCTUATION <stat> PLOT Y X1 X2 X3
          FLUCTUATION <stat> PLOT Y X1 X2 X3 X4
          FLUCTUATION <stat> PLOT Y X1 X2 X3 X4 X5
          FLUCTUATION <stat> PLOT Y X1 X2 X3 X4 X5 X6

       to generate a fluctuation plot (this is a variant
       of the mosaic plot).  Enter HELP FLUCTUATION PLOT for
       details.

    d) Added the command

          STRIP PLOT Y
          STRIP PLOT Y2 X2
          BATCH STRIP PLOT Y TAG
          BATCH STRIP PLOT Y2 X2 TAG

       The strip plot is also known as a dot plot.  There
       are a number of variations of this plot.  Enter
       HELP STRIP PLOT for details.

    e) The <stat> PLOT command was updated to support multiple
       response variables.  For example,

           MEAN PLOT Y1 TO Y4 X

       That is, for each distinct value of X, there are now
       4 means plotted instead of just one.

       The following commands can be used to control the
       appearance of the plot:

           SET STATISTIC PLOT FORMAT   <DEX/OVERLAY>
           SET STATISTIC PLOT SUMMARY  <VARIABLE/GROUP>

    If the FORMAT option is set to OVERLAY and the SUMMARY option
    is set to VARIABLE, this is equivalent to the following:

        YLIMITS ...
        PRE-ERASE OFF
        ERASE
        MEAN PLOT Y1 X
        MEAN PLOT Y2 X
        MEAN PLOT Y3 X
        MEAN PLOT Y4 X
        PRE-ERASE ON

    That is, there will be a curve corresponding to each
    response variable and there will be a reference line
    corresponding to each variable.

    If the FORMAT option is set to DEX, then this plot uses a
    format similar to the DEX <stat> PLOT command.  That is, for
    each distinct value of X, there will be curve connecting the
    mean values for the 4 response variables.

    If the SUMMARY option is set to GROUP, there will be a single
    reference curve.  At each distinct value of X, a single overall
    mean is computed for all 4 of the response variables.

    In addition, the following option is added to this command:

        <stat> <zscore/uscore> PLOT

    If ZSCORE is given, then a z-score transformation (subtract the
    mean and then divide by the standard deviation) is computed
    on each response variable first.  If USCORE is given, then a
    u-score transformation (subtract the minimum and divide by the
    range) is computed on each column.  Note these z-score and
    u-score transformations apply to the entire response variable, not
    to each distinct group within the response variable.

 4) The following updates were made to the graphics output
    devices.

    a) For the SVG (Scalable Vector Graphics) device, graphics
       elements are now assigned "layers".  This may be useful
       if you import the SVG graphic into a graphics editing
       program (i.e., it may allow individual elements of the
       plot to be edited).

    b) The GD driver was enhanced to support hardware text (the
       previous implementation drew all characters using one
       of Dataplot's software fonts).

       There are two types of hardware characters supported:

          1) The GD library supports 5 built-in fonts: small,
             large, mediumbold, tiny, and huge.

             These are fixed size fonts.

          2) In addition, the GD library supports True Type Fonts
             (TTF).  This is the font type supported on the
             Microsoft Windows operating system.  These fonts
             are scalable.  Although these fonts were originally
             developed for Microsoft Windows, they can be used
             on Linux/Unix systems as well.

             Note that neither Dataplot nor the GD web site
             provides any of these fonts.  However, there are
             a number of web sites that provide these types of
             fonts (some are freely downloadable while others
             are proprietary).

    c) The Postscript driver was updated in the following ways.

       1) For presentation and publication graphs, it is desirable
          to use the Postscript typeset quality fonts.  However,
          the use of special characters (with the limited exception
          of the SP(), CR(), UC(), and LC() options) has required
          the use of one of the software Hershey fonts (e.g.,
          SIMPLEX or DUPLEX).  The Postscript device was upgraded
          to handle most of Dataplot's supported special characters.

          Specifically, the following are supported:

             i) subscripts and superscripts
            ii) Greek characters
           iii) A subset of the mathematical symbols and other
                special characters.  This is based on what is
                available in the Postscript symbol font.  Note
                that there is not a 1-to-1 correspondence between
                the characters available in the Postscript symbol
                font and the special characters supported by
                Dataplot.  The following is the list of Dataplot
                special characters that will be translated to
                equivalent characters in the Postscript symbol font:

                    INTE(), SUMM(), PROD(), INFI(), DOTP(),
                    DEL(), DIVI(), LT(), GT(), LTEQ(), GTEQ(),
                    NOT(), +-(), APPR(), TILD(), EQUI(), VARI(),
                    CARA(), TIME(), PART(), RADI(), SUBS(),
                    SUPE(), UNIO(), INTR(), ELEM(), THEX(),
                    THFO(), RAPO(), LBRA(), RBRA(), LCBR(),
                    RCBR(), LELB(), RELB(), RARR(), UARR(),
                    DARR(), VBAR(), HBAR(), DEGR() 

           The full set of special symbols supported by Dataplot
           is documented in chapter 13 of Volume I of the
           Reference Manual

           http://www.itl.nist.gov/div898/software/dataplot/refman1/
           ch13/homepage.pdf

    d) Added support for the Unux/Linux libplot library.  The libplot
       library is part of the "plotutils" package which includes the
       plot, tek2plot, pic2plot, plotfont, spline, and ode programs.

       This driver may not be available on some platforms.  If it
       is available, the following devices can be used:

          device <1/2/3>  LIBPLOT X           - X11
          device <1/2/3>  LIBPLOT PNM         - netPBM format (binary)
          device <1/2/3>  LIBPLOT PNM ASCII   - netPBM format (ascii)
          device <1/2/3>  LIBPLOT GIF         - gif
          device <1/2/3>  LIBPLOT AI          - adobe illustrator
          device <1/2/3>  LIBPLOT PS          - postscript
          device <1/2/3>  LIBPLOT FIG         - xfig
          device <1/2/3>  LIBPLOT PCL         - HP PCL (using HP-GL emulation)
          device <1/2/3>  LIBPLOT HPGL        - HP HP-GL
          device <1/2/3>  LIBPLOT TEKTRONIX   - Tektronix 4014
          device <1/2/3>  LIBPLOT REGIS       - Dec Regis
          device <1/2/3>  LIBPLOT META        - libplot metafile format
          device <1/2/3>  LIBPLOT SVG         - Scalable Vector Graphics
          device <1/2/3>  LIBPLOT PNG         - Portable Network Graphics
          device <1/2/3>  LIBPLOT CGM         - webcgm binary format
          device <1/2/3>  LIBPLOT CGM ASCII   - webcgm ascii format

        Note that many of these devices are redundant to devices already
        supported by Dataplot.  The new drivers are:

        a) The netPBM software supports conversion between a large number of
           bit-map formats.  The PNM format is an intermediate format (i.e.,
           a netPBM utility converts from the old bit-map file to PNM first
           and then converts the PNM file to the new format).  The netPBM
           software also contains a number of programs for manipulating
           images.  The PNM format is also supported by a number of
           image conversion programs.

        b) The xfig program is available on most Unix/Linux installations.
           It allows graphs to be edited.

        c) The ai format allows graphs to be generated in Adobe
           Illustrator's native format.

        d) The libplot meta format is used by the programs in the
           plotutils package.  It may also be supported by other
           Unix/Linux software.

        e) The webcgm format is an ANSI standard Computer Graphics Metafile
           (CGM) format.  Although CGM did not become a widely adopted
           standard, there are a number of word processing, graphics editing,
           and page publishing software programs that will import CGM files.
           Note that libplot supports a newer version of CGM that was
           developed to support inclusion on web pages.

        f) The native protocol for most HP printers is PCL.  Most desktop
           printers have adopted this format.  Note that the PCL driver in
           libplot is actually using the HP-GL emulation capability of
           PCL (HP-GL was originally developed for HP penplotters).

           In any event, this option may allow you to generate output
           directly for many desktop printers that do not support Postscript
           without going through an intermediate program such as Ghostview.

 5) The following updates were made to the analysis commands.

    a) The following updates were made to the CROSS TABULATE command:

       1) The number of cross-tabulation variables was increased to
          six (from two).  That is, you can cross-tabulate on a
          minimum of 2 variables and a maximum of six variables.

       2) The output can be generated in RTF format (support was
          previously added for Latex and HTML format).  RTF can
          be used to import the output into Microsoft Word.

          This enhancement was also made to the TABULATE command.

       3) You can use the SET WRITE DECIMALS command to specify
          the number of digits to the right of the decimal point
          in the output.

          This enhancement was also made to the TABULATE command.

       4) Since there is now a separate CHI-SQUARE INDEPENCE TEST
          command, the chi-square test option in the CROSS TABULATE
          command has been removed.

       5) For the LET <resp> = CROSS TABULATE <stat> ...
          command, added the following option:

              SET LET CROSS TABULATE <EXPAND/COLLAPSE>

          If EXPAND is specified, the number of rows in the
          output variable is equal to the number of rows in
          the input variables.  If COLLAPSE is specified, the
          number of rows in the output variable is equal to the
          number of distinct cross tabulation cells.

          If the COLLAPSE option is used, the following comamnds
          may be helpful

             LET X1D = CROSS TABULATE GROUP ONE   <var-list>
             LET X2D = CROSS TABULATE GROUP TWO   <var-list>
             LET X3D = CROSS TABULATE GROUP THREE <var-list>
             LET X4D = CROSS TABULATE GROUP FOUR  <var-list>

         For example, if you want to cross-tabulate the means
         for three classification variables, you could do
         something like

              LET YMEAN = CROSS TABULATE MEAN Y X1 X2 X3
              LET X1D   = CROSS TABULATE GROUP ONE X1 X2 X3
              LET X2D   = CROSS TABULATE GROUP TWO X1 X2 X3
              LET X3D   = CROSS TABULATE GROUP THREE X1 X2 X3
              PRINT YMEAN X1D X2D X3D

    b) Added the command

          BINOMIAL PROPORTION TEST P1 N1 P2 N2
          BINOMIAL PROPORTION TEST Y1 Y2

    c) Added the following commands to perform one-sample and
       two-sample proficiency analyses based on the ASTM
       E2489 - 06 standard:

          ONE SAMPLE PROFICIENCY TEST Y LABID
          TWO SAMPLE PROFICIENCY TEST Y LABID

    d) Added the following command to compute detection limits
       based on ASTM E2677

          LIMITS OF DETECTION Y X

 6) The following updates were made to the probability
    distributions.

    a) Added support for the following distributions

       1) 3-Parameter Logistic-Exponential

           LE3CDF(X,BETA)                - cdf function
           LE3CHAZ(X,BETA)               - cumulative hazard function
           LE3HAZ(X,BETA)                - hazard function
           LE3PDF(X,BETA)                - pdf function
           LE3PPF(P,BETA)                - ppf function
 
       2) Truncated Pareto

           TNPCDF(X,GAMMA,A,NU)          - cdf function
           TNPPDF(X,GAMMA,A,NU)          - pdf function
           TNPPPF(P,GAMMA,A,NU)          - ppf function
 
       3) Brittle Fracture

           BFRCDF(X,ALPHA,BETA,R)        - cdf function
           BFRPDF(X,ALPHA,BETA,R)        - pdf function
           BFRPPF(P,ALPHA,BETA,R)        - ppf function
 
       4) Pearson Type III

           PE3CDF(X,GAMMA)               - cdf function
           PE3PDF(X,GAMMA)               - pdf function
           PE3PPF(P,GAMMA)               - ppf function

       5) The Mielke's Beta-Kappa distributtion was
          renamed to

           MIECDF(X,K,THETA)            - cdf function
           MIEPDF(X,K,THETA)            - pdf function
           MIEPPF(P,K,THETA)            - ppf function

          Note also that BETA parameter was in fact a
          scale parameter and is now explicitly treated
          as such.

       6) Kappa

           KAPCDF(X,K,H)                 - cdf function
           KAPPDF(X,K,H)                 - pdf function
           KAPPPF(P,K,H)                 - ppf function

          Note that the Mielke's Beta-Kappa is a
          re-parameterized special case of the Kappa
          distribution.

    b) Added support for maximum likelihood estimation for
       the following distributions:

       reflected power
       Weibull (maximum case)
       Frechet (minimum case)
       generalized Pareto (minimum case)
       generalized extreme value (minimum case)
       Kappa
       Pearson type III

       Note that the Kappa  and Pearson type III actually
       implement L-moment estimates rather than maximum
       likelihood estimates.

 7) The following LET commands were added.

    a) Two-dimensional convex hulls can be computed
       with the command

           LET Y2 X2 = 2D CONVEX HULL Y X

    b) Two-dimensional minimum spanning trees can be
       computed with the commands

           LET Y2 X2 TAG = MINIMUM SPANNING TREE Y X
           LET Y2 X2     = MINIMUM SPANNING TREE D
     
       The first syntax is used when the input is a set of
       vertices.   The second syntax is used when the input
       is a distance matrix.

    c) Two-dimensional spanning forests can be
       computed with the commands

           LET Y2 X2 TAG = SPANNING FOREST EDGE1 EDGE2 Y X
           LET EDGE1 EDGE2 TAG NV = SPANNING FOREST EDGE1 EDGE2 NVERT
     
       The first syntax is most useful when you want to plot
       the spanning forest.


    d) The command

           LET Y2 X2 TAG = EDGES TO VERTICES EDGE1 EDGE2 Y X

       can be used to convert a list of edges in a graph
       to a list of vertices.  This is a convenience command
       to make plotting the graph easier.

    e) The commands

          LET X2 Y2       = SORT2 X Y
          LET Z1 Z2 Y2    = SORT3 X1 X2 Y
          LET Z1 Z2 Z3 Y2 = SORT4 X1 X2 X3 Y

       can be used to sort based on multiple fields.

    f) The commands

          LET Y = GATHER X INDEX
          LET Y = SCATTER X INDEX

       can be used to extract (or insert) data from one
       array into another based on a variable that contains
       row id's.

    g) Added the following matrix commands:

       The following is used to compute the permanent of a
       matrix.

          LET MOUT = MATRIX PERMANENT M

       The following is used to generate an adjacency matrix
       from a list of edges.

          LET ADJ = EDGES TO ADJACENCY MATRIX EDGE1 EDGE2 NVERT
          
       The following is used to permute the rows and columns
       of a matrix.

          LET MOUT = MATRIX RENUMBER M  VROW VCOL
          
       The following is used to compute the pseudo inverse
       of a matrix.

          LET MINV = PSEUDO INVERSE M

       The following is used to compute the coordinates for a
       biplot.

          LET Y X TAG = BIPLOT M

    h) Added the following commands:

          LET N = <value>
          LET Y = RANDOM SUBSET FOR I = 1 1 K
  
          LET N = <value>
          LET Y = RANDOM K-SET OF N-SET FOR I = 1 1 K
  
          LET N = <value>
          LET Y = RANDOM COMPOSITION FOR I = 1 1 K

          LET N = <value>
          LET K = <value>
          LET Y = RANDOM PARTITION

          LET N = <value>
          LET Y = RANDOM EQUIVALENCE RELATION

          LET N = <value>
          LET LAMBDA = DATA <values>
          LET Y = RANDOM YOUNG TABLEAUX LAMBDA

          LET Y = NEXT SUBSET 0
          LET Y = NEXT SUBSET N YPREV

          LET Y = NEXT K-SET OF N-SET 0
          LET Y = NEXT SUBSET N K YPREV

          LET Y = NEXT PERMUTATION 0
          LET Y = NEXT PERMUTATION N YPREV

          LET Y = NEXT COMPOSITION 0
          LET Y = NEXT COMPOSITION N K YPREV

          LET Y1 Y2 = NEXT EQUIVALENCE RELATION 0
          LET Y1 Y2 = NEXT EQUIVALENCE RELATION N YPREV YREPPREV

          LET Y = NEXT YOUNG TABLEAUX N LAMBDA
          LET Y = NEXT YOUNG TABLEAUX N LAMBDA Y
          LET VAL ROWID = CONVERT YOUNG TABLEAUX Y
          LET HOOK = YOUNG TABLEAUX HOOK LENGTH VAL ROWID

          LET Y2 X2 = PEAKS OF FREQUENCY TABLE Y

          LET AL AU = DIFFERENCE OF PROPORTION CONFIDENCE
                      LIMITS P1 N1 P2 N2 ALPHA
          LET PVAL  = DIFFERENCE OF PROPORTION HYPOTHESIS
                      TEST P1 N1 P2 N2 ALPHA
          LET PVAL  = DIFFERENCE OF PROPORTION LOWER TAIL
                      HYPOTHESIS TEST P1 N1 P2 N2 ALPHA
          LET PVAL  = DIFFERENCE OF PROPORTION UPPER TAIL
                      HYPOTHESIS TEST P1 N1 P2 N2 ALPHA

    i) Added the following commands for working with strings:

          LET SOUT = STRING MERGE          SORG  SADD  NSTART
          LET SOUT = STRING REPLACE        SORG  SADD  NSTART
          LET SOUT = STRING EDIT           SORG  SOLD  SNEW
          LET SOUT = STRING CONCATENATE    S1 S2 ...  SK
          LET SOUT = SUBSTRING             SORG  NSTART NSTOP
          LET SOUT = SUBSTRING             SORG  NSTART NSTOP
          LET SOUT = UPPER CASE            SORG
          LET SOUT = LOWER CASE            SORG
          LET IVAL = ICHAR                 SORG
          LET NLEN = STRING LENGTH         SORG
          LET NSTART NSTOP = STRING INDEX  SORG  SMATCH

    j) Added the following command for merging two sets of data

          LET ... = MERGE ...

       The merge is performed by matching columns in two sets of
       data and then carrying other variables of interest.  For
       details, enter HELP MERGE.

    k) Added the following commands for shifting the elements of
       a vector up (right) or down (left)

           LET Y2 = SHIFT Y NSHIFT
           LET Y2 = CIRCULAR SHIFT Y NSHIFT

    l) It is sometimes convenient to extract the index of the
       minimim, maximum, or extreme value (i.e., the largest
       absolute value) of a vector.  This can be done with the
       following commands

           LET A = INDEX MINIMUM Y
           LET A = INDEX MAXIMUM Y
           LET A = INDEX EXTREME Y

 8) The following miscellaneous updates were made.

    a) Added SAVE/RESTORE options to the FEEDBACK command.
       This is primarily useful for general purpose macros
       where you want to use the FEEDBACK OFF command and
       you want to restore the setting that was in place
       when the macro was called.

 9) Added the following command

       DIRECTION <HORIZONTAL/VERTICAL>

    For the TEXT command when a hardware font is being used,
    this specifies whether the text is drawn horizontally
    (the default) or vertically.

10) Fixed a number of bugs.

-----------------------------------------------------------------------
The following enhancements were made to DATAPLOT
July 2007 - February  2008.
-----------------------------------------------------------------------

 1) The following updates were made for probability
    distributions.

    a) Added the following new continuous distributions.

        1) Burr Type 2
 
           BU2CDF(X,R)                   - cdf function
           BU2PDF(X,R)                   - pdf function
           BU2PPF(P,R)                   - ppf function
 
        2) Burr Type 3
 
           BU3CDF(X,R,K)                 - cdf function
           BU3PDF(X,R,K)                 - pdf function
           BU3PPF(P,R,K)                 - ppf function
 
        3) Burr Type 4
 
           BU4CDF(X,R,C)                 - cdf function
           BU4PDF(X,R,C)                 - pdf function
           BU4PPF(P,R,C)                 - ppf function
 
        4) Burr Type 5
 
           BU5CDF(X,R,K)                 - cdf function
           BU5PDF(X,R,K)                 - pdf function
           BU5PPF(P,R,K)                 - ppf function
 
        5) Burr Type 6
 
           BU6CDF(X,R,K)                 - cdf function
           BU6PDF(X,R,K)                 - pdf function
           BU6PPF(P,R,K)                 - ppf function
 
        6) Burr Type 7
 
           BU7CDF(X,R)                   - cdf function
           BU7PDF(X,R)                   - pdf function
           BU7PPF(P,R)                   - ppf function
 
        7) Burr Type 8
 
           BU8CDF(X,R)                   - cdf function
           BU8PDF(X,R)                   - pdf function
           BU8PPF(P,R)                   - ppf function
 
        8) Burr Type 9
 
           BU9CDF(X,R,K)                 - cdf function
           BU9PDF(X,R,K)                 - pdf function
           BU9PPF(P,R,K)                 - ppf function
 
        9) Burr Type 10
 
           B10CDF(X,R)                   - cdf function
           B10PDF(X,R)                   - pdf function
           B10PPF(P,R)                   - ppf function
 
       10) Burr Type 11
 
           B11CDF(X,R)                   - cdf function
           B11PDF(X,R)                   - pdf function
           B11PPF(P,R)                   - ppf function
 
       11) Burr Type 12
 
           B12CDF(X,C,K)                 - cdf function
           B12PDF(X,C,K)                 - pdf function
           B12PPF(P,C,K)                 - ppf function
 
       12) DOUBLY PARETO UNIFORM
 
           DPUCDF(X,M,N,ALPHA,BETA)      - cdf function
           DPUPDF(X,M,N,ALPHA,BETA)      - pdf function
           DPUPPF(P,M,N,ALPHA,BETA)      - ppf function

       13) KUMARASWAMY
 
           KUMCDF(X,ALPHA,BETA)          - cdf function
           KUMPDF(X,ALPHA,BETA)          - pdf function
           KUMPPF(P,ALPHA,BETA)          - ppf function

       14) UNEVEN TWO-SIDED POWER
 
           UTSCDF(X,A,B,D,NU1,NU3,ALPHA) - cdf function
           UTSPDF(X,A,B,D,NU1,NU3,ALPHA) - pdf function
           UTSPPF(P,A,B,D,NU1,NU3,ALPHA) - ppf function

       15) SLOPE
 
           SLOCDF(X,ALPHA)               - cdf function
           SLOPDF(X,ALPHA)               - pdf function
           SLOPPF(P,ALPHA)               - ppf function

       16) TWO-SIDED SLOPE
 
           TSSCDF(X,ALPHA,THETA)         - cdf function
           TSSPDF(X,ALPHA,THETA)         - pdf function
           TSSPPF(P,ALPHA,THETA)         - ppf function

       17) OGIVE
 
           OGICDF(X,N)                   - cdf function
           OGIPDF(X,N)                   - pdf function
           OGIPPF(P,N)                   - ppf function

       18) TWO-SIDED OGIVE
 
           TSOCDF(X,N,THETA)             - cdf function
           TSOPDF(X,N,THETA)             - pdf function
           TSOPPF(P,N,THETA)             - ppf function

       19) REFLECTED POWER FUNCTION
 
           RPOCDF(X,C)                   - cdf function
           RPOCHAZ(X,C)                  - cumulative hazard function
           RPOHAZ(X,C)                   - hazard function
           RPOPDF(X,C)                   - pdf function
           RPOPPF(X,C)                   - ppf function

       20) POWER FUNCTION
 
           POWCHAZ(X,C)                  - cumulative hazard function
           POWHAZ(X,C)                   - hazard function

           The cdf, pdf, and ppf functions were already
           available.

       21) WAKEBY
 
           WAKPDF(X,BETA,GAMMA,DELTA)    - pdf function

           The cdf and ppf functions were added in a
           previous release.

       22) Muth
 
           MUTCDF(X,BETA)                - cdf function
           MUTPDF(X,BETA)                - pdf function
           MUTPPF(P,BETA)                - ppf function
 
       23) Logistic-Exponential
 
           LEXCDF(X,BETA)                - cdf function
           LEXCHAZ(X,BETA)               - cumulative hazard function
           LEXHAZ(X,BETA)                - hazard function
           LEXPDF(X,BETA)                - pdf function
           LEXPPF(P,BETA)                - ppf function
 
    b) The definitions for the exponential power, alpha, and
       Maxwell distributions were modified from

           PEXCDF(X,ALPHA,BETA,LOC,SCALE)
           PEXHAZ(X,ALPHA,BETA,LOC,SCALE)
           PEXCHAZ(X,ALPHA,BETA,LOC,SCALE)
           PEXPDF(X,ALPHA,BETA,LOC,SCALE)
           PEXPPF(P,ALPHA,BETA,LOC,SCALE)
           ALPCDF(X,ALPHA,BETA,LOC,SCALE)
           ALPHAZ(X,ALPHA,BETA,LOC,SCALE)
           ALPCHAZ(X,ALPHA,BETA,LOC,SCALE)
           ALPPDF(X,ALPHA,BETA,LOC,SCALE)
           ALPPPF(P,ALPHA,BETA,LOC,SCALE)
           MAXCDF(X,SIGMA,LOC,SCALE)
           MAXPDF(X,SIGMA,LOC,SCALE)
           MAXPPF(P,SIGMA,LOC,SCALE)

       to

           PEXCDF(X,BETA,LOC,SCALE)
           PEXHAZ(X,BETA,LOC,SCALE)
           PEXCHAZ(X,BETA,LOC,SCALE)
           PEXPDF(X,BETA,LOC,SCALE)
           PEXPPF(P,BETA,LOC,SCALE)
           ALPCDF(X,ALPHA,LOC,SCALE)
           ALPHAZ(X,ALPHA,LOC,SCALE)
           ALPCHAZ(X,ALPHA,LOC,SCALE)
           ALPPDF(X,ALPHA,LOC,SCALE)
           ALPPPF(P,ALPHA,LOC,SCALE)
           MAXCDF(X,LOC,SCALE)
           MAXPDF(X,LOC,SCALE)
           MAXPPF(X,LOC,SCALE)

       This reflects the fact that the ALPHA parameter for the
       exponential power distribution, the BETA parameter for the
       alpha distribution, and the SIGMA parameter for the Maxwell
       distribution are in fact scale parameters.  The random numbers,
       probability plots, ppcc/ks plots, and Kolmogorov
       Smirnov and chi-square gooodness of fit tests were
       updated to reflect this change as well.

    c) Added support for maximum likelihood estimation for
       the following distributions:

       Reflected generalized Topp and Leone
       Burr type 10
       Wakeby  (actually generates L-Moments estimates)
       exponential power

 2) Added the following statistics:

       LET A = LP LOCATION X
       LET A = LP VARIANCE X
       LET A = LP SD X

    These statistics are supported by the following commands:

        <stat> PLOT
        TABULATE <stat> 

        CROSS TABULATE <stat> 
        CROSS TABULATE <stat> PLOT
        LET Y = CROSS TABULATE <stat>

        LET Y = MATRIX <COLUMN/ROW/GRAND> <stat> M

        BOOTSTRAP <stat> PLOT
        JACKNIFE <stat> PLOT
        <stat> INFLUENCE CURVE
        <stat> BLOCK PLOT
        DEX <stat> PLOT

 3) Added the following for graphics output devices.

    a) Added the following device drivers

         AQUA    - Aquaterm for Mac OSX systems

                   Enter HELP AQUA for details.

    b) Added the following command
 
         SET POSTSCRIPT CONVERT CONVERT <PBM/PGM/PNM/TIFF/JPEG/PDF>

       This is a enhancement to the previously available
       command SET POSTSCRIPT CONVERT.  The SET POSTSCRIPT CONVERT
       command uses the Ghostscript command to automatically
       covert Dataplot Postscript output to one of the listed
       image formats.  One limitation was that the Ghostscript
       command did not provide a command line switch to
       generate a landscape orientation plot (which most
       Dataplot graphs need).  The "CONVERT CONVERT" option
       uses the "convert" program in Image Magic instead of
       Ghostscript.   This option does support landscape
       mode.

-----------------------------------------------------------------------
The following enhancements were made to DATAPLOT
March 2007 - June   2007.
-----------------------------------------------------------------------

 1) We have made the following updates for categorical data
    analysis.

    There are two basic types of data that the following
    commands address.

    a) We have two variables,each with n observations, where
       the first can have one of r mutually exclusive values
       and the second can have one of c mutually exclusive values.

       So each observation will fit into exactly one of the
       r levels of variable one and exactly one of the c levels
       of variable two.

       Your data can be either in raw form (two columns of data
       each with n rows) or summary form (an rxc table which
       will typically be read into Dataplot as a matrix).
       Each entry in the summary table is a count of how many
       times that particular combination occurred.

    b) If each variable can have exactly two outcomes (typically
       coded as 1/0), then we have the 2x2 special case.  There
       are a number of specialized methods for dealing with
       this type of data.

       For this type of data, the number of observations for
       the two variables need not be equal.

       Some examples of this type of data are:

        i) We have a diagnostic test to detect a disease.
           Variable one specifies whether the patient in
           fact has the disease (coded as 1) or not (coded
           as 0).  Variable two specifies whether the test
           detected the disease (coded as 1) or not (coded
           as 0).

       ii) We are testing instruments to determine whether or
           not they can detect a particular substance.  Variable
           one is the ground truth (coded as 1 when the substance
           is present and coded as 0 when it is not).  Variable
           two denotes whether the instrument detected the
           substance (1 for detected, 0 for not detected).

    The following capabilities have been added to Dataplot
    for analyzing these type of data.

    a) The following statistical tests were added:

          ODDS RATIO INDEPENDENCE TEST N11 N21 N12 N22
          ODDS RATIO INDEPENDENCE TEST Y1 Y2
          ODDS RATIO INDEPENDENCE TEST M

          CHI-SQUARE INDEPENDENCE TEST N11 N21 N12 N22
          CHI-SQUARE INDEPENDENCE TEST Y1 Y2
          CHI-SQUARE INDEPENDENCE TEST M

          FISHER EXACT TEST N11 N21 N12 N22
          FISHER EXACT TEST Y1 Y2
          FISHER EXACT TEST M

          MCNEMAR TEST N11 N21 N12 N22
          MCNEMAR TEST Y1 Y2
          MCNEMAR TEST M

          ODDS RATIO CHI-SQUARE TEST Y1 Y2
          ODDS RATIO CHI-SQUARE TEST Y1 Y2 X
          ODDS RATIO CHI-SQUARE TEST Y1 X1 Y2 X2

          MANTEL-HAENSZEL TEST Y1 Y2
          MANTEL-HAENSZEL TEST Y1 Y2 X
          MANTEL-HAENSZEL TEST Y1 X1 Y2 X2

    b) Added the following statistics:

          LET A = ODDS RATIO X1 X2
          LET A = ODDS RATIO STANDARD ERROR X1 X2
          LET A = LOG ODDS RATIO X1 X2
          LET A = LOG ODDS RATIO STANDARD ERROR X1 X2
          LET A = RELATIVE RISK X1 X2
          LET A = CRAMER CONTINGENCY COEFFICIENT X1 X2
          LET A = MATRIX GRAND CRAMER CONTINGENCY COEFFICIENT M
          LET A = PEARSON CONTINGENCY COEFFICIENT X1 X2
          LET A = MATRIX GRAND PEARSON CONTINGENCY COEFFICIENT M
          LET A = FALSE POSITIVE Y1 Y2
          LET A = FALSE NEGATIVE Y1 Y2
          LET A = TRUE POSITIVE Y1 Y2
          LET A = TRUE NEGATIVE Y1 Y2
          LET A = TEST SENSITIVITY Y1 Y2
          LET A = TEST SPECIFICITY Y1 Y2
          LET A = POSITIVE PREDICTIVE VALUE Y1 Y2
          LET A = NEGATIVE PREDICTIVE VALUE Y1 Y2

       These statistics are supported by the following commands:

          <stat> PLOT
          TABULATE <stat> 

          CROSS TABULATE <stat> 
          CROSS TABULATE <stat> PLOT

          BOOTSTRAP <stat> PLOT
          JACKNIFE <stat> PLOT

    c) Added the following graphics:

          ROC CURVE Y1 Y2 X   - generate a ROC curve

          ROSE PLOT Y         - generate a rose plot (also
          ROSE PLOT Y1 Y2       known as a four-fold plot)

          BINARY TABULATION PLOT Y1 Y2 X1 X2

          BINARY <type> PLOT Y1 Y2 X1

          where <type> is one of:

              CORRECT MATCH
              FALSE POSITIVE
              FALSE NEGATIVE
              TRUE POSITIVE
              TRUE NEGATIVE

          These "binary" plots are used to generate summary
          plots of "1/0" type data across groups.

          ASSOCIATION PLOT M        - generate an association plot
          ASSOCIATION PLOT Y1 Y2
          ASSOCIATION PLOT N11 N21 N12 N22

          SIEVE PLOT M              - generate a sieve plot
          SIEVE PLOT Y1 Y2
          SIEVE PLOT N11 N21 N12 N22

 2) We have made the following updates for probability
    distributions.

    a) Maximum likelihood estimates were added for the
       following distributions:

          Katz                 (generates moment estimates)
          slash
          triangular
          four parameter beta  (generates moment estimates)
          log beta
          beta normal

       The maximum likelihood for the two-sided power distribution
       was generalized to include the lower and upper limit
       parameters.

       The slash and triangular distributions have also been
       added to the BOOTSTRAP/JACKNIFE <dist> MLE PLOT command:

           BOOTSTRAP TRIANGULAR MLE PLOT Y
           JACKNIFE TRIANGULAR MLE PLOT Y
           BOOTSTRAP SLASH MLE PLOT Y
           JACKNIFE SLASH MLE PLOT Y

       The maximum likelihood estimation for the 
       two-sided power distribution was updated from the
       the standard case (lower and upper limits = 0 and 1)
       to the general case (lower and upper limits will be
       estimated from the data).  Also, the ML procedure for
       this distribution only applies if the N shape parameter
       is > 1.

    b) Added the following commands for binomial confidence
       intervals:

          LET A = EXACT BINOMIAL LOWER BOUND P N ALPHA
          LET A = EXACT BINOMIAL UPPER BOUND P N ALPHA
          LET ALOW AUPP = AGRESTI COULL LIMITS P N ALPHA

       The BINOMIAL MAXIMUM LIKELIHOOD command can generate
       these values for raw data.  The above LET commands are
       useful when you only have summary data (i.e., the p and n).

    c) Added the following plots:

          POISSON PLOT Y X
          GEOMETRIC PLOT Y X
          BINOMIAL PLOT Y X
          NEGATIVE BINOMIAL PLOT Y X
          LOGARITHMIC SERIES PLOT Y X

        These plots are alternatives to the PROBABILITY PLOT
        command.

          ORD PLOT Y

        This plot can help distinguish whether a Poisson,
        a negative binomial, or a logarithmic series
        distribution provides a more appropiate distributional
        model for a set of discrete data.

 3) Made the following updates to graphics commands.

    a) The HISTOGRAM command now accepts a matrix argument.

    b) Added the command

         BIVARIATE NORMAL TOLERANCE REGION PLOT Y1 Y2 X

 4) Added the following statistics:

       LET P1 = <value>
       LET P2 = <value>
       LET A = TRIMMED STANDARD DEVIATION Y

 5) Added the following command

       SET FATAL ERROR <IGNORE/TERMINATE/PROMPT>

    If an analysis or graphics command returns an error code,
    this command tells Dataplot how to respond:

       IGNORE - Dataplot will simply continue processing the
                next command.   This was the behavior before
                this command was added and is the default.

       TERMINATE - Dataplot will print a message and terminate
                   immediately.

       PROMPT - Dataplot will prompt whether you want to
                continue or terminate.

    This command was added primarily as a debugging option.
    If you are trying to debug a complex macro, it can be helpful
    to have Dataplot terminate (or prompt for termination)
    in order to locate where the initial error is occurring.
    Note that this command is not active if you are running
    the Graphical User Interface (GUI) version.

 6) Fixed a number of miscellaneous bugs.

-----------------------------------------------------------------------

The following enhancements were made to DATAPLOT
May 2006 - February  2007.
-----------------------------------------------------------------------

 1) The following updates were made for maximum likelihood estimates
    for distributions:

    a) The negative binomial was updated to distinguish between
       two cases: 1) the case where k is assumed known (p is
       estimated) and 2) the case where k is assumed unknown.
       For case 1), confidence limits for p were added.

    b) Maximum likelihood estimates were added for the
       following discrete distributions:

          zeta
          Borel-Tanner
          Lagrange-Poisson
          lost games
          beta-geometric
          Polya-Aeppli
          generalized logarithmic series
          geeta
          Consul
          quasi binomial type I
          generalized lost games
          generalized negative binomial

          topp and leone

    c) The binomial mle was updated in the following ways:

       1) For exact intervals, fixed a bug for extreme values
          of p and small samples.

       2) By default, Dataplot switches from the exact method
          to the normal approximation for sample sizes greater
          than 30 (Agresti-Coull intervals are always generated).
          You can specify the threshold with the command

              SET BINOMIAL NORMAL APPROXIMATION THRESHOLD <value>

       3) Some analysts prefer to use a continuity correction

             (p + 0.5)/(n + 1)

          You can specify whether to use the continuity
          correction by entering the command

              SET BINOMIAL CONTINUITY CORRECTION <ON/OFF>

          The default is OFF.

 2) The following distributional updates were made.

    a) The YULCDF was updated to use an explicit formula (as
       oppossed to direct summation).

    b) For the KS PLOT, the location and scale parameters are
       estimated via the probability plot.  For long-tailed
       distributions, more accurate estimates may be obtained
       by applying a biweight fit of the probability plot.
       To specify this option, enter the command

         SET PPCC PLOT LOCATION SCALE BIWEIGHT

       To restore the use of the regular least squares
       estimates of location and scale, enter

         SET PPCC PLOT LOCATION SCALE DEFAULT

    c) Added the following new continuous distributions.

        1) Asymmetric Log-Laplace
 
           ALDCDF(X,ALPHA,BETA)          - cdf function
           ALDPDF(X,ALPHA,BETA)          - pdf function
           ALDPPF(P,ALPHA,BETA)          - ppf function
 
        2) Log-Beta
 
           LBECDF(X,ALPHA,BETA,C,D)      - cdf function
           LBEPDF(X,ALPHA,BETA,C,D)      - pdf function
           LBEPPF(P,ALPHA,BETA,C,D)      - ppf function
 
        3) Topp and Leone
 
           TOPCDF(X,BETA)                - cdf function
           TOPPDF(X,BETA)                - pdf function
           TOPPPF(P,BETA)                - ppf function
 
        4) Generalized Topp and Leone
 
           GTLCDF(X,ALPHA,BETA)          - cdf function
           GTLPDF(X,ALPHA,BETA)          - pdf function
           GTLPPF(P,ALPHA,BETA)          - ppf function
 
        5) Reflected Generalized Topp and Leone
 
           RGTCDF(X,ALPHA,BETA)          - cdf function
           RGTPDF(X,ALPHA,BETA)          - pdf function
           RGTPPF(P,ALPHA,BETA)          - ppf function
 
        6) Wakeby:
 
           WAKCDF(X,BETA,GAMMA,DELTA)    - cdf function
           WAKPPF(P,BETA,GAMMA,DELTA)    - ppf function
 
    d) Added the following new discrete distributions.

        1) Beta-Geometric (Waring)
 
           BGECDF(X,ALPHA,BETA)         - cdf function
           BGEPDF(X,ALPHA,BETA)         - pdf function
           BGEPPF(X,ALPHA,BETA)         - ppf function
 
        2) Beta-Negative Binomial (generalized Waring)
 
           BNBCDF(X,ALPHA,BETA,k)       - cdf function
           BNBPDF(X,ALPHA,BETA,k)       - pdf function
           BNBPPF(X,ALPHA,BETA,k)       - ppf function
 
        3) Zeta
 
           ZETCDF(X,ALPHA)              - cdf function
           ZETPDF(X,ALPHA)              - pdf function
           ZETPPF(X,ALPHA)              - ppf function
 
        4) Zipf
 
           ZIPCDF(X,ALPHA,N)            - cdf function
           ZIPPDF(X,ALPHA,N)            - pdf function
           ZIPPPF(X,ALPHA,N)            - ppf function
 
        5) Borel-Tanner
 
           BTACDF(X,LAMBDA,N)           - cdf function
           BTAPDF(X,LAMBDA,N)           - pdf function
           BTAPPF(X,LAMBDA,N)           - ppf function
 
        6) Lagrange-Poisson
 
           LPOCDF(X,LAMBDA,THETA)       - cdf function
           LPOPDF(X,LAMBDA,THETA)       - pdf function
           LPOPPF(X,LAMBDA,THETA)       - ppf function
 
        7) Leads in Coin Tossing (Discrete Arcsine)
 
           LCTCDF(X,N)                  - cdf function
           LCTPDF(X,N)                  - pdf function
           LCTPPF(X,N)                  - ppf function

        8) Classical Matching
 
           MATCDF(X,K)                  - cdf function
           MATPDF(X,K)                  - pdf function
           MATPPF(X,K)                  - ppf function

        9) Polya-Aeppli

           PAPCDF(X,THETA,P)            - cdf function
           PAPPDF(X,THETA,P)            - pdf function
           PAPPPF(X,THETA,P)            - ppf function
 
       10) Generalized Logarithmic Series

           GLSCDF(X,THETA,BETA)         - cdf function
           GLSPDF(X,THETA,BETA)         - pdf function
           GLSPPF(X,THETA,BETA)         - ppf function
 
       11) Geeta

           GETCDF(X,THETA,BETA)         - cdf function
           GETPDF(X,THETA,BETA)         - pdf function
           GETPPF(X,THETA,BETA)         - ppf function
 
           This distribution can also be parameterized with
           MU and BETA.

       12) Quasi Binomial Type 1

           QBICDF(X,P,PHI)              - cdf function
           QBIPDF(X,P,PHI)              - pdf function
           QBIPPF(X,P,PHI)              - ppf function

       13) Generalized Negative Binomial

           GNBCDF(X,THETA,BETA,M)       - cdf function
           GNBPDF(X,THETA,BETA,M)       - pdf function
           GNBPPF(X,THETA,BETA,M)       - ppf function

       14) Truncated Generalized Negative Binomial

           GNTCDF(X,THETA,BETA,M,N)     - cdf function
           GNTPDF(X,THETA,BETA,M,N)     - pdf function
           GNTPPF(X,THETA,BETA,M,N)     - ppf function

       15) Discrete Weibull

           DIWCDF(X,Q,BETA)             - cdf function
           DIWPDF(X,Q,BETA)             - pdf function
           DIWPPF(X,Q,BETA)             - ppf function
           DIWHAZ(X,Q,BETA)             - hazard function
 
       16) Consul (a generalized geometric)

           CONCDF(X,THETA,M)            - cdf function
           CONPDF(X,THETA,M)            - pdf function
           CONPPF(X,THETA,M)            - ppf function
 
       17) Lost Games
 
           LOSCDF(X,P,R)                - cdf function
           LOSPDF(X,P,R)                - pdf function
           LOSPPF(X,P,R)                - ppf function
 
       18) Generalized Lost Games
 
           GLGCDF(X,P,J,A)              - cdf function
           GLGPDF(X,P,J,A)              - pdf function
           GLGPPF(X,P,J,A)              - ppf function
 
       19) Katz
 
           KATCDF(X,ALPHA,BETA)         - cdf function
           KATPDF(X,ALPHA,BETA)         - pdf function
           KATPPF(X,ALPHA,BETA)         - ppf function
 
    e) The Waring routines (WARCDF, WARPDF, WARPPF) routines
       were re-written to take advantage of their relationship
       to the beta-geometric (the Waring is simply a different
       parameterization of the beta-geometric).  This makes
       the Waring routines more computationally efficient and
       more accurate.

 3) Added the following LET sub-commands.

    a) Added the harmonic number and generalized harmonic
       number functions:

           LET A = HARMNUMB(N)
           LET A = HARMNUMB(N,M)

    b) For certain types of plots, it can be useful to add a
       small bit of random noise to a variable to avoid
       overplotting.  This is commonly referred to as jittering.
       To simplify this, the following command was added:

           LET DELTA <value>
           LET Y = JITTER X DELTA

       The value of DELTA is used to control the magnitude of
       the jittering.  That is, the value of x(i) will be
       changed to a value x(i) + noise where noise is in the
       range (-DELTA/2,DELTA/2).

 4) Made the following updates to the CONSENSUS MEANS command.

    a) If a within-lab standard deviation is zero (i.e., the lab
       has only a single unique measurement value), that lab
       will be omitted from the analysis (it will be included
       in the initial summary table).  Previously, Dataplot
       treated this as an error and would not run the
       consensus means analysis.

    b) Added the Fairweather method.  There are 3 separate
       methods for generating 95% confidence intervals for this
       method (the original method proposed by Fairweather,
       an improvement suggested by Cox, and a method developed
       by Ruhkin).  The output for this method is only printed
       if the minimum number of oberservations for a lab is
       greater than 5.

    c) Added the Bayesian Consensus Procedure (BCP) method of
       Hagwood and Guthrie.  This is a refinement of the BOB
       method.  For this method, the consensus mean and the
       standard deviation of the consensus mean are asymptotically
       equivalent to the posterior mean and standard deviation of
       a fully Bayesian method.

    d) Dataplot currently supports 12 methods.  Most users will
       only be interested in a subset of these methods.  You
       can now selectively turn individual methods on or off
       (all methods are on by default) with the commands:

          SET MANDEL PAULE <ON/OFF>
          SET MODIFIED MANDEL PAULE <ON/OFF>
          SET VANGEL RUHKIN <ON/OFF>
          SET BOB <ON/OFF>
          SET SCHILLER EBERHARDT <ON/OFF>
          SET MEAN OF MEANS <ON/OFF>
          SET GRAND MEAN <ON/OFF>
          SET GRAYBILL DEAL <ON/OFF>
          SET GENERALIZED CONFIDENCE INTERVAL <ON/OFF>
          SET DERSIMONIAN LAIRD <ON/OFF>
          SET FAIRWEATHER <ON/OFF>
          SET BAYESIAN CONSENSUS PROCEDURE <ON/OFF>

 5) The following updates and enhancements were made to
    the graphics commands.

    a) Added the command:

          SET 4-PLOT DISTRIBUTION <NORMAL/EXPONENTIAL>

       The 4-plot by default consists of a run sequence plot,
       a lag plot, a histogram, and a normal probability plot.
       The above command allows us to replace the normal
       probability plot with an exponential probability plot.
       This is useful when checking the assumptions for a
       Homogeneous Poisson Process (HPP) where we assume the
       interarrival times follow an exponential distribution.

    b) Added the command:

          REPAIR PLOT Y X CENSOR

       This is used to plot repair data where we may have
       multiple systems and each system may have a single
       censoring time (i.e., the time between the last repair
       and the end of the test).  Enter HELP REPAIR PLOT
       for details.

    c) Added the command:

          MEAN REPAIR FUNCTION PLOT Y X CENSOR

    d) Added the command

          TRILINEAR PLOT Y1 Y2 Y3

       This is used for plots where the rows of Y1, Y2, and
       Y3 are mixtures (i.e., they sum to either 1 (or 100
       if you are using percent units)).

 6) Updated the RELIABILITY TREND TEST in the following
    ways.

    a) Fixed a bug in the reverse arrangements test.

    b) Modified the output format for better clarity.

    c) Added support for multiple systems.  For multiple systems,
       the tests will be applied to each individual system and
       then composite tests will be performed.

    d) Added support for HTML, Latex, and RTF format.

 7) The following bug fixes were made:

    a) The 2 variable case for the chi-square goodness of fit
       test for discrete distributions had a bug.  This has
       been fixed.  For older versions, a work around is

           SET MINSIZE = 1
           LET Y3 XLOW XHIGH = COMBINE FREQUENCY TABLE Y2 X2
           POISSON CHI-SQUARE GOODNESS OF FIT Y3 XLOW XHIGH

    b) Some bugs with LET subcommands and SUBSETTING were
       corrected.

    c) A bug involving IF statements within nested loops was
       corrected.

    d) A few other miscellanous bug fixes were made.

-----------------------------------------------------------------------
The following enhancements were made to DATAPLOT
September 2005 - April 2006.
-----------------------------------------------------------------------

 1) For many one-factor plots, it is useful to sort the horizontal
    axis based on the value of some statistic (most commonly a
    location statistic such as the mean, median, minimum, or
    maximum).  The following commands was added to help generate
    these sorted plots:

       LET XSORT INDX = SORT BY <stat> X GROUPID

    For example, to generate a sorted mean plot for variables
    Y and X, you would do something like

      LET X2 INDX = SORT BY MEAN Y X
      X1TIC MARK LABEL FORMAT VARIABLE
      X1TIC MARK LABEL CONTENT INDX
      MEAN PLOT Y X2

    This can be used with the following types of plots

        i) <stat> PLOT Y X

           where <stat> is a desired statistic (e.g., MEAN or
           SD).

       ii) BOX PLOT Y X

      iii) PLOT Y X GROUP

    For details, enter HELP SORT BY STATISTIC.

    These plots often have alphabetic tick mark labels.  The
    following enhancements were made to simplify the use
    of alphabetic tick mark labels with sorted plots.

    a) The TIC MARK LABEL FORMAT and TIC MARK LABEL CONTENT
       commands were previously augmented to allow numeric
       variables, group label variables, or the row label
       variable as the contents for the tick mark labels.
       Specifically,

          LET LAB = DATA 50 40 30 20 10 0
          X1TIC MARK LABEL FORMAT VARIABLE
          X1TIC MARK LABEL CONTENT LAB

          LET IG = GROUP LABELS A B C D E
          X1TIC MARK LABEL FORMAT GROUP LABEL
          X1TIC MARK LABEL CONTENT IG

          X1TIC MARK LABEL FROMAT ROW LABELS

       This has been enhanced to allow an index variable to
       be specified on the above TIC MARK LABEL CONTENT
       commands (the index variable is typically generated by
       a SORT BY <stat> command).  The index variable specifies
       the order in which the tic mark labels will be generated.
    
       So the above examples can be augmented by

          LET X2 INDX = SORT BY MEAN Y X
          LET LAB = DATA 50 40 30 20 10 0
          X1TIC MARK LABEL FORMAT VARIABLE
          X1TIC MARK LABEL CONTENT LAB INDX

          LET X2 INDX = SORT BY MEAN Y X
          LET IG = GROUP LABELS A B C D E
          X1TIC MARK LABEL FORMAT GROUP LABEL
          X1TIC MARK LABEL CONTENT IG INDX

          LET X2 INDX = SORT BY MEAN Y X
          X1TIC MARK LABEL FROMAT ROW LABELS
          X1TIC MARK LABEL CONTENT INDX

    b) The LET ... = GROUP LABEL .... command was augmented in
       the following two ways.

        i) You can specify literal strings for group labels.
           For example,

              LET IG = GROUP LABEL  BATCHSP()1 BATCHSP()2 ...
                       BATCHSP()3 BATCHSP()4

           The strings are separated by spaces.  If you need to
           include a space in a particular string, use the
           SP() as in the above example.

       ii) Pre-defined strings can be used to define a group
           label variable.  For example,

              LET IG = GROUP LABEL ST1 TO ST10

           where ST1, ST2, ...., ST10 are previously defined
           strings.  The TO syntax is useful in this context
           when the number of strings is large.
        
       Dataplot's algorithm for parsing the GROUP LABEL command
       is:

         i) Dataplot first checks the character variables file
            (HELP SET CONVERT CHARACTER for details).  If the
            first name listed is found, Dataplot uses this
            character variable to define the group labels.
 
        ii) If a character variable is not found, Dataplot
            checks all the listed names to see if they are
            previously defined strings.  If they are, then
            Dataplot substitutes the values of these strings.
 
       iii) If one or more of the names is not a previously
            defined string, then Dataplot treats all of the
            names as literal text strings.

 2) You can now pass arguments to macros.

    To pass arguments to a macro, do something like

        CALL SAMPLE.DP  arg1  arg2 arg3

    Up tp 10 arguments may be passed (although limits on command
    line lengths still apply).  Arguments containing spaces or
    hyphens should be enclosed in quotes.  The character limit for
    a single argument is 40 characters.

    In the SAMPLE.DP macro, if a $1 is encountered, it will be
    replaced with "arg1", if a $2 is encountered, it will be
    replaced with "arg2" and so on.  A $0 will substitute the
    number of arguments given on the CALL command.

    This substitution will only occur if a command line is contained
    within a macro (i.e., if no macro is active, the "$" will not
    signal any substitution and it will remain in the command line
    as given).

    Dataplot currently only supports one level of argument
    substitition for macros.  That is, the values of the macro
    arguments (i.e., the $1, $2, etc.) will contain the values
    given by the most recent CALL command that specified at least
    one argument.  If you need to nest CALL commands with macro
    arguments, the recommended work around is to have the
    higher level macro extract any macro arguments passed to it
    into temporary variables or strings before calling any other
    macros.  For example, supposse SAMPLE.DP needs to call
    SAMPLE2.DP with arguments.  You could do something like
    the following in SAMPLE.DP:

        .  Start of SAMPLE.DP macro
        let string zzzzs1 = $1
        let string zzzzs2 = $2
        let string zzzzs3 = $3
            ...
        call sample2.dp  newarg1  newarg2

    The default character for argument substitution is the
    "$".  To use a different character, enter the command

       MACRO SUBSTITUTION CHARACTER <char>

 3) The following enhancements were made to the CAPTURE
    command (the CAPTURE command re-directs alphanumeric output
    to a file rather than displaying it on the screen).

    a) Sometimes it may be useful to have the output sent to
       both the screen and to a file.  You can do this by
       entering the command

          CAPTURE SCREEN ON

       To restore CAPTURE output only being sent to the
       CAPTURE file, enter the command

          CAPTURE SCREEN OFF

    b) Sometimes it may be useful to selectively send output to
       the CAPTURE file.  You can do this with the following
       commands:

          CAPTURE SUSPEND
          CAPTURE RESUME

      where SUSPEND specifies that output will be sent to the
      screen rather than the CAPTURE file (note that the CAPTURE
      file remains open) and RESUME will send the output to
      the currently open CAPTURE file.  You can enter as many
      CAPTURE SUSPEND/CAPTURE RESUME sequences as you like
      between a CAPTURE/END OF CAPTURE session.

      Note that OFF is a synonym for SUSPEND and ON is a
      synonym for RESUME.

 4) Made the following probability distribution updates:

    a) Added confidence intervals for the maximum likelihood
       estimates for the geometric distribution.

    b) Added confidence intervals for the maximum likelihood
       estimates for the Poisson distribution.

    c) Added support for the following new probability
       distributions:

       1) Added the type 2 generalized logistic distribution.
          Enter HELP GL2PDF for details.

       2) Added the type 3 generalized logistic distribution.
          Enter HELP GL3PDF for details.

       3) Added the type 4 generalized logistic distribution.
          Enter HELP GL4PDF for details.

       4) Added the Hosking parameterization of the generalized
          logistic distribution.  Enter HELP GL5PDF for details.

       5) Added the generalzied Tukey-Lambda distribution.  Enter
          HELP GLDPDF for details.

       6) Added the beta-normal distribution.  Enter HELP BNOPDF
          for details.

       7) Added the asymmetric log double exponential (Laplace)
          distribution.  Enter HELP ALDPDF for details.

 5) Added or modified the following analysis comamnds.

    a) The Durbin test for identifical effects in a two-way
       table for balanced incomplete block designs is supported
       with the command

          DURBIN TEST Y BLOCK TREATMENT

       Enter 

          HELP DURBIN TEST

       for details.

    b) The TOLERANCE LIMITS command generates both normal tolerance
       limits and non-parametric tolerance limits.  You can now
       specify only one of these with the commands

          NORMAL TOLERANCE LIMITS
          NONPARAMETRIC TOLERANCE LIMITS

    c) The GRUBS TEST for outlier detection was previously augmented
       to generate three distinct tests:

         i) a test for both the minimum and maximum points as
            outliers.
        ii) a test for the minimum points as an outliers.
       iii) a test for the maximum points as an outliers.

       This has now been modifed into three distinct commands:

          GRUBBS TEST Y
          GRUBBS MINIMUM TEST Y
          GRUBBS MAXIMUM TEST Y

       This was done so that the internally saved parameters
       (e.g., STATVAL, STATCDF, etc.) will now be correct for
       the appropriate test.
 
    d) The CONSENSUS MEANS command was modified in a number of
       ways.  Specifically,

       1) The output format was modified to make it more
          consistent and to provide better clarity.  In
          particular, a clearer distinction is made between
          standard uncertainty (the standard error of the
          consensus mean), expanded uncertainty (2*standard
          error) and expanded uncertainty based on a
          normal or t percent point value.

       2) Modified the summary tables.  There are now 4 summary
          tables generated:

             i) A summary table of the original data.
            ii) A summary table of the 95% confidence limits
                generated by each method
           iii) A summary table of the standard uncertainties
                generated by each method (i.e., the standard
                error of the consensus mean estimate)
            iv) A summary table of the expanded uncertainties
                generated by each method (i.e., the 2 times
                the standard error of the consensus mean estimate)

        3) Added the following new methods:

             i) The Graybill-Deal method now generates confidence
                limits using a method proposed by Andrew Rukhin.
                It also generates 4 distinct estimates of the
                variance of the consensus mean (the Sinha method,
                the naive method, and 2 methods proposed by
                Nien-Fan Zhang.  The commonly used naive method
                is know to seriously underestimate the variance
                for small sample sizes.

            ii) Added the generalized confidence interval method
                proposed by Hari Iyer and Jack Wang.

           iii) Added the DerSimonian-Laird method.

        4) Previous versions of Dataplot allowed you to create
           the CONSENSUS MEANS output in HTML format
           (CAPTURE HTML FILE.HTM) or Latex format
           (CAPTURE LATEX file.tex).  This was extended to
           include Rich Text Format (RTF).  The RTF option
           is used for creating output that can be read into
           Microsoft Word (RTF is a protocol Microsoft created
           for transporting word processing files between
           different word processing programs).  For example

               CAPTURE RTF FILE.RTF
               CONSENSUS MEAN Y X
               END OF CAPTURE

           You can then import FILE.RTF into Word.  Note that
           although RTF is suppossed to be a portable format,
           our experience is that non-Word word processors do a
           poor job of importing the Dataplot RTF files (tables
           tend to be problamatic for non-Word software and
           Dataplot is creating most of its RTF output as tables).

 6) The following updates were made to graphics output devices.

    a) The GD library, used to generate JPEG and PNG format
       graphs, was updated from version 1.84 to 2.033.  The
       primary consequence of this is that we can now generate
       GIF format files as well.  To generate GIF files, enter

          SET IPL1NA  PLOT.GIF
          DEVICE 2 GD GIF

    b) Dataplot can now generate graphs in Latex format.
       The primary motivation for using this format is
       to generate publication quaility graphs.  There are
       some unique features to this device driver that are
       described in detail in the HELP LATEX command.

 7) The following statistic command was added.

       LET A = RATIO Y1 Y2

    This statistic is the sum of Y1 divided by the sum of Y2.

    The following additional commands are supported:

       TABULATE RATIO Y1 Y2 X
       CROSS TABULATE RATIO Y1 Y2 X1 X2

       RATIO PLOT Y1 Y2 X
       RATIO CROSS TABULATE PLOT Y1 Y2 X1 X2

       BOOTSTRAP RATIO PLOT Y1 Y2
       JACKNIFE RATIO PLOT Y1 Y2

 8) The following special function library functions were added:

       I0INT   - integral of the modified Bessel function of the
                 first kind and order 0
       J0INT   - integral of the Bessel function of the first kind
                 and order 0
       K0INT   - integral of the modified Bessel function of the
                 third kind and order 0
       Y0INT   - integral of the Bessel function of the second kind
                 and order 0
       I0ML0   - difference of the modified Bessel function of the
                 first kind of order 0 and the modified Struve function
                 of order 0
       I1ML1   - difference of the modified Bessel function of the first
                 kind of order 1 and the modified Struve function of
                 order 1
       AIRINT  - integral of the Airy function Ai
       BIRINT  - integral of the Airy function Bi
       AIRYGI  - modified Airy function Gi
       AIRYHI  - modified Airy function Hi
       ATNINT  - integral of the inverse-tangent function

 9) Added the following LET subcommands:

    a) LET Y2 = REPLACE GROUPID GROUP2 Y1

       This command does the following:
   
       1) It matches the values in GROUP2 against GROUPID and
          returns the indices of the matching rows for the GROUPID
          array.
   
       2) The indices are used to access the corresponding value
          in the Y1 array.
   
       3) The corresponding row of Y2 is replaced with the Y1
          value.
   
       The abbreviated syntax
   
           LET Y2 = REPLACE GROUPID GROUP
   
       simply assigns a value of 1 in the corresponding row of Y2.
   
       Enter HELP REPLACE for details.

    b) LET Y2 X2 = MATRIX BIN M

       This command is used to generate a frequency table for
       the elements in a matrix.  This can be used to generate
       a histogram of the elements in a matrix.  For example,
 
          LET Y2 X2 = MATRIX BIN M
          HISTOGRAM Y2 X2

       Enter HELP MATRIX BIN for details.

    c) LET M = MATRIX TRUNCATION M IVALUE
       LET M = MATRIX LOWER TRUNCATION M IVALUE

       Set all values in the matrix M that are less than
       IVALUE to IVALUE.  This command can be used in conjunction
       with the MATRIX SUBTRACT command to remove background
       values from a matrix.  For example, if the background
       value is 5, do something like

           LET IBACK = 5
           LET IZERO = 0
           LET M = MATRIX SUBTRACT M IBACK
           LET M = MATRIX TRUNCATION M IZERO

       Likewise, you can use the following command to perform
       an upper truncation:

          LET M = MATRIX LOWER TRUNCATION M IVALUE

       That is, any values in M greater than IVALUE are set to
       IVALUE.

10) The SET HISTOGRAM CLASS WIDTH was previously implemented to
    specify different default class width algorithms for
    histograms.  This command was extended to apply to the
    following additional commands:

      LET Y2 X2 = BINNED Y
      LET Y2 X2 = MATRIX BIN Y
      NORMAL MIXTURE MAXIMUM LIKELIHOOD Y
      <dist> CHI-SQUARE GOODNESS OF FIT Y
      2 SAMPLE CHI-SQUARE GOODNESS OF FIT Y

11) Added the following command

        PROCESS ID

    This command will print the process id and save this
    process id in the internal parameter PID.

12) Made the following bug fixes.

    a) Previously, if all elements of a response variable were
       equal, the HISTOGRAM command would print an error message
       and not generate the histogram.  Dataplot will now
       print a warning message, but will generate a histogram
       with one non-zero class (it will generate one class above
       and one class below with zero count as well).

    b) In the TABULATE command, if all elements in the response
       variable are identifical, change from an error message to a
       warning message and perform the tabulation anyway.

    c) Corrected a bug in Friedman's test.  The previous version
       is correct if the original data is the rank within a block.
       The corrected version does not require that the data
       already be ranked.

    d) The WILK SHAPIRO command was not returning the p-value in
       the saved parameter PVALUE correctly.  This was corrected.

    e) For the command

          LET Z2 = BIVARIATE INTERPOLATION Z Y X Y2 X2

       the Y and X arguments were in the wrong order (i.e., the
       command was interperting Y X as X Y).  This was corrected.

    f) Fixed bugs in the 

          LET X = CHARACTER CODE IX1
          LET X = ALPHABETIC CHARACTER CODE IX1

       commands.

    g) The command 

           LET Y2 XLOW XUPP = COMBINE FREQUENCY TABLE Y X

       is used to combine low frequency bins.  The original
       implementation simply worked from left to right to
       combine the bins.  Since low frequency bins typically
       occur in the left and right tails, the algorithm was
       modified to move from the left tail to the center and
       then from the right tail to the center.

    h) Fixed a bug where the ORIENTATION command could cause
       Dataplot to hang on subsequent plots if no DEVICE 2
       command was defined and a software font was used to
       draw text.

    i) Dataplot creates and uses a number of temporary files
       in the current directory.

       If you have multiple sessions running from the current
       directory, this can create a problem for these temporary
       files.  In most cases, a conflict does not occur because
       Dataplot will open the file, read or write to the file,
       and then close the file immediately.  However, a few
       files, such as the plot files dppl1f.dat and dppl2f.dat,
       typically remain open.  The effect of different Dataplot
       sessions trying to access these files is system dependent.

       1. On Unix and Windows 98/NT4 platforms, the file will
          contain whatever was most recently written to it.

       2. On Windows 2000/XP platforms, the Dataplot session
          that opens the file first has a "lock" on the file.
          This causes any subsequent Dataplot session that tries
          to access the file to hang.

          This is particularly a problem with the GUI version
          on Windows 2000/XP.  Specifically, if the Dataplot GUI
          does not shut down cleanly, the underlying Dataplot
          executable does not get killed.  This then causes any
          future attempt to open the GUI to hang since the "dead"
          Dataplot executable has a lock on the file.  You have to
          use "Cntrl-Alt-Del" to bring up the Task Manager, select
          "Processes", and then manually kill any "DPLAHEY.EXE"
          processes in order to clear the dead process.

          In particualar, if you close the GUI by clicking the
          "x" in the upper right hand corner (rather than clicking
          the EXIT menu), this does not kill the underlying
          DPLAHEY.EXE process.

          As a partial solution to this problem, Dataplot should
          now trap this condition.  It will print a message
          indicating how to clear the "dead" DPLAHEY.EXE process.
          In addition, it will do one of two things in the current
          Dataplot process:

          a. It will attach the process id to the temporary file
             name and then re-open the file.

          b. It will simply ignore file (so if dppl2f.dat is locked,
             Dataplot will not write the current plot to dppl2f.dat
             in the current Dataplot session).

          You can specify which option Dataplot will use by entering
          one of the following commands in your startup file
          (c:\Program Files\NIST\DATAPLOT\DPLOGF.TEX):

              SET TEMPORARY FILE PID
              SET TEMPORARY FILE IGNORE

          The default is PID.

-----------------------------------------------------------------------
The following enhancements were made to DATAPLOT June - August     2005.
-----------------------------------------------------------------------

 1) The following matrix commands were added.

    a. The sum of all elements in a matrix can be computed with
       the following command

           LET A = MATRIX SUM M

    b. Previous versions of Dataplot allowed you to compute
       various column or row statistics
       (HELP MATRIX COLUMN STATISTIC or HELP MATRIX ROW STATISTIC
       for details).  This capability has been extended to the
       case of computing the statistics for the entire matrix
       with the command

          LET A = MATRIX GRAND <stat> M

       where <stat> denotes the desired the statistic (the list
       of supported statistics is the same as for the
       MATRIX COLUMN STATISTIC and MATRIX ROW STATISTIC commands.

    c. Previous versions of Dataplot allowed you to compute
       various column or row statistics
       (HELP MATRIX COLUMN STATISTIC or HELP MATRIX ROW STATISTIC
       for details).  This capability has been extended to the
       case where the matrix is divided into equal partitions
       with the command

          LET MOUT = MATRIX PARTITION <stat> M NROW NCOL

       with M, NROW, and NCOL denoting the input matrix, the number
       of rows in each sub-matrix, and the number of columns in
       each sub-matrix, respectively.  Note that this command
       returns a matrix (MOUT) of values.

       That is, the original matrix is divided into sub-matrices
       containing NROW rows and NCOL columns each.  The partition
       starts at row 1 and column 1. The number of rows in MOUT
       is determined by dividing the number of rows in M by NROW.
       Likewise, the number of columns is determined by dividing
       the number of columns in M by NCOL.  If this division
       does not result in an integer value (e.g., 23 columns
       in M and NCOL = 5 results in 3 columns left over), then the
       last column, or row, of MOUT will be based on whatever
       columns are left over.

       In addition, the MATRIX PARTITION command has been extended
       to accomodate unequal partitions where the partitions need
       not be contiguous.

       The syntax in this case is

          LET MOUT = MATRIX PARTITION <stat> M TAGROW TAGCOL

       with M denoting the input matrix.  In this case, TAGROW and
       TAGCOL are vectors with TAGROW having the same number of rows
       as M and TAGCOL having the same number of columns as M.
       The elements of TAGROW and TAGCOL identify which partition
       each element of M belongs to.  The output matrix will be
       dimensioned based on the number of distinct values in
       TAGROW and TAGCOL.

 2) The following commands were added to compute probability
    weighted moments and L-moments.

       LET P = PROBABILITY WEIGHTED MOMENTS Y
       LET L = L MOMENTS Y

 3) The following distributional updates were made.

    a. Made the following enhancements to the generalized Pareto
       maximum likelihood command.

       1. L-moment and elemental percentile estimates are now
          included.  The L-moment estimators are a refinement of
          probability weighted moments.  The elemental perecentile
          method is described in Castillo, Hadi, Balakrishnan, and
          Sarabia, "Extreme Value and Related Models with
          Applications in Engineering and Science", Wiley, 2005.
          One advantage of the elemental percentile approach is that
          it does not have the restricted domain for the shape
          parameter that the moment and maximum likelihood estimators
          have.

       2. The elemental percentile estimate is now used as the
          starting value for the maximum likelihood.  This seems
          to improve the convergence of the ML method.

       3. The methods used (moments, L-moments, elemental percentiles,
          and maximum likelihood) do not estimate a location
          parameter.

          By default, these methods will now use the minimum data
          value (minus an epsilon fudge factor) as the estimate of
          location.  The data will subtract this value before
          applying the estimation procedures.

          If you would like to provide your own location estimate,
          enter the command

              LET THRESHOL = <value>

          Any data values less than the value specified for
          THRESHOL will be omitted from the estimation.  Note that
          the generalized Pareto is often used in the context of
          modeling the distribution of "points above a threshold",
          so specifying a threshold greater than some of the data
          points is fairly common.

       4. The maximum likelihood estimates now include the normal
          approximation confidence intervals for the scale and
          shape parameters and, optionally, for select percentiles
          of the data.

          To specify percentile estimates, enter the command

               SET MAXIMUM LIKELIHOOD PERCENTILES  <varname>

          where <varname> specifies the name of a variable containing
          the desired percentiles.  You can specify DEFAULT to
          to use a default set of values.

          Be aware that for the generalized Pareto maximum
          likelihood estimation, a relatively large sample size
          may be required for the asymptotic normal approximations
          to become reasonably accurate.  Some studies have
          indicated sample sizes of at least 500 may be required.

    b. Added support for the maximum likelihood estimation for
       the inverted Weibull distribution:

           INVERTED WEIBULL MLE Y
           INVERTED WEIBULL MLE Y X

       The first syntax supports the full sample case.  It will
       return confidence intervals for the shape and scale
       parameters for various values of alpha (based on the
       normal approximations) and will return confidence intervals
       for selected percentiles if you have entered a
       SET MAXIMUM LIKELIHOOD PERCENTILES DEFAULT command.

       The second syntax supports the censored case.  This case
       currently only returns point estimates.

    c. The BINOMIAL MLE now returns improved confidence intervals.

    d. We have modified the output from a number of the maximum
       likelihood commands to make the output more consistent.

 3) Made a number of bug fixes.  In particular

    a. Fixed a bug where the following orm of the DERIVAIVE command
       wasn't being recognized:

           LET FUNCTION D = DERIVATIVE F WRT X

       This syntax should now work.

    b. Fixed the DIFFERENCE OF MEANS CONFIDENCE INTERVALS command
       (in adding support for the HTML/LATEX output, we had shut
       off the standard ASCII output).  Fixed the HTML outout
       for this command.

-----------------------------------------------------------------------
The following enhancements were made to DATAPLOT January - May     2005.
-----------------------------------------------------------------------

 1) Distributional Modeling Updates

    a. Dataplot provides extensive distributional modeling
       capabilities via probability plots and PPCC/KS plots.  One
       limitation of these methods is that they do not provide
       estimates for the uncertainty of the parameter estimates
       and for the distribution quantiles.

       The BOOTSTRAP ... PLOT command was enhanced to support
       distributional modeling for a number of distributions.
       This can be used to obtain confidence intervals for the
       distribution parameters, for selected percentiles of the
       distribution, and for the value of the PPCC (or K-S
       statistic).

       For details, enter

           HELP DISTRIBUTIONAL BOOTSTRAP

    b. For the case of one shape parameter, the PPCC plot was
       enhanced to support a group option (where group means
       multiple batches of data as oppossed to binned data).

       In this case, a separate curve is drawn for each batch
       of the data.  This can be used to check for a common
       shape parameter across multiple batches of data.  For
       details, enter

           HELP PPCC PLOT

    c. The PPCC PLOT and PROBABILITY PLOT commands support binned
       data.  Previously, the binning consisted of two variables:
       the first contained the bin frequencies and the second
       contaned the mid-point of the bins.  This form assumes
       the bins are of equal width.

       Some binned data may contain bins of unequal width.  The
       most common reason for the this is to combine bins in the
       tails which have low frequencies.

       The PPCC PLOT and PROBABILITY PLOT commands were updated
       to handle this case.  In this case, the syntax is

           <dist> PPCC PLOT Y XLOW XHIGH
           <dist> PROBABILITY PLOT Y XLOW XHIGH

       with Y, XLOW, and XHIGH denoting the frequency variable,
       the lower class boundary, and the upper class boundary,
       respectively.  For details, enter

           HELP PPCC PLOT
           HELP PROBABILITY PLOT

    d. The following enhancenets were made to the maximum
       likelihood estimation.

       1. Added confidence intervals for the location and scale
          parameters for the double exponential case
          (DOUBLE EXPONENTIAL MAXIMUM LIKELIHOOD Y).

       2. Added a weighted order statistics method to the Cauchy
          maximum likelihood estimation (CAUCHY MLE Y).  This method
          was added because it is the method recommended for the
          Cauchy Anderson-Darling test (see D'Agostino and Stephens,
          "Goodness-Of-Fit Techniques", Marcel Dekker, 1986, p. 164).

       3. Added support for the maximum case of the 2-parameter
          extreme value type 2 (Frechet) distribution.  This includes
          confidence intervals for the estimated parameters and
          for select percentiles (see
          SET MAXIMUM LIKELIHOOD PERCENTILES).

    e. The Anderson-Darling test now supports the extreme value
       type 2 (Frechet) for the maximum case and the Cauchy
       distribution.

    f. Added support for the minimum case for the generalized
       extreme value distribution.  Added the GEVHAZ and GEVCHAZ
       functions to compute the hazard and cumulative hazard
       functions for the generalized extreme value distribution.

    g. A number of distributions (Weibull, Gumbel, Frechet,
       and generalized extreme value) support both a minimum and
       a maximum case.  The command

           SET MINMAX <1/2>

       is used to specify which case (1 = minimum, 2 = maximum).
       If no MINMAX command is entered, previous versions used
       the value 1 as the default (this was chosen since the
       minimum case is what is typically used for the Weibull
       distribution).

       However, for the other distributions, the maximum case
       is generally the one most used.  For this reason, we
       added the value 0 to indicate the default where the default
       is now specific to each distribution.  For the Weibull, the
       default is the minimum and for the Gumbel, Frechet,  and
       generalized extreme value the default is the maximum.

 2) Interlaborartory Analysis Updates

    Dataplot added the following commands to perform an
    interlaboratory analysis as documented in 

      "Standard Practice for Conducting an Interlaboratory Study
      to Determine the Precision of a Test Method", ASTM
      International, 100 Barr Harbor Drive, PO BOX C700,
      West Conshohoceken, PA 19428-2959, USA.  This document is
      in support of ASTM Standard E 691 - 99.

    The specific commands added are:

       LET A = REPEATABILITY STANDARD DEVIATION Y LABID
       LET A = REPRODUCABILITY STANDARD DEVIATION Y LABID
       LET H = H CONSISTENCY STATISTIC Y LABID
       LET K = K CONSISTENCY STATISTIC Y LABID
       LET H TAG = H CONSISTENCY STATISTIC Y LABID MATID
       LET K TAG = K CONSISTENCY STATISTIC Y LABID MATID

       E691 INTERLAB  Y LABID MATID

    The E691 INTERLAB command generates four tables documentented
    in the above document.  The other comamnds are useful in
    generating the plots described in this standard.

    In addition, a number of built-in macros were added to
    generate the various graphs demonstrated in the standard.

    For more information, enter

       HELP E691 INTERLAB

 3) The following command can be useful in converting data in a
    two-way table to a format required by certain Dataplot
    commands

       LET Y MATID LABID = REPLICATED STACK X1 ... XK LAB

    The resulting output has the form

        X1(1)   1   LAB(1)
          .     .     .
        X1(n)   1   LAB(n)
        X2(1)   2   LAB(1)
          .     .     .
        X2(n)   2   LAB(n)
               ...
        Xk(1)   k   LAB(1)
          .     .     .
        Xk(n)   k   LAB(n)

    This is a variation of the STACK command.  The distinction is
    that the last variable entered is interpreted as a labid
    variable that is replicated for each of the response variables.
    For details, enter

        HELP REPLICATED STACK

 4) Extreme Value Analysis

    a. Enhancements were made to the CME and DEHAAN commands (these
       estimate the parameters for a generalized Pareto distribution).

    b. Added the following command

         PEAKS OVER THRESHOLD PLOT Y

       For details, enter PEAKS OVER THRESHOLD PLOT Y.

 5) Platform Specific Issues

    a) We have separated the Windows installation files into two
       distinct cases:
   
       a) Windows 2000/XP platforms
       b) Windows 95/98/NT4/ME platforms
   
       This was required for compiler compatibility reasons.  The
       Lahey LF90 and Compaq Visual Fortran compilers were starting
       to show some problems under Windows XP (specifically with
       Service Pack 2).
   
       For Windows 2000/XP, we have upgraded to the Intel 8.1
       Fortran compiler.  However, this compiler does not support
       Windows 98 and earlier platforms.  So the
       Windows 95/98/NT4/ME version is still built using the
       Lahey (for the GUI) and Compaq compilers.

    b) We have updated the Mac OSX installation.  There is now a
       single file that you download that includes the executable,
       the auxillary files, the source, the needed Tcl/Tk files,
       and the g77 compiler.  This simplifies the installation
       (e.g., you do not have to install Tcl/Tk yourself).

 6) We have started overhauling some of the menus for the graphical
    interface (GUI).  This will not be radically different, just an
    effort to provide better organization and clarity to the menus.

    This updating will occur over several releases.  The initial
    update has re-arranged the top level menus.  We have added
    a "Getting Started" menu to help new users.  The Reliability
    and Extreme Values menus have been reorganized.

 7) Dataplot uses the "." for the decimal point when reading data.
    Some countries use the "," for this purpose.

    We have added the command

       SET DECIMAL POINT <value>

    with <value> denoting the character to be used as the decimal
    point.

    Note that the use of this is currently fairly limited.  It is
    used in free-format reads only.  It is provided to allow
    international users the ability to read their data files
    without editing them.  Note that it does not apply if you
    use the SET READ FORMAT command to define a format for the
    data.  It is also not used for writing data nor for the
    output from Dataplot commands.

 8) Fixed a number of bugs.

    a. Fixed the COLUMN LIMITS where the specified limits are
       arrays (as oppossed to single scalar values) to work in
       the case where columns are of unequal length.

    b. Internally, Dataplot treats strings and functions
       interchangeably.  The one distinction is that strings
       preserve case.  However, when strings are operating as
       functions, we want them to be converted to upper case.

       Dataplot was updated so that when a string is used as a
       function, it is converted to upper case.  This also
       required some updates in the "^" and "&" string operators
       to handle case conversions appropriately.

    c. Fixed a bug in the Wilcox signed rank test when it was
       used for a 1-sample test.

    d. For generalized Pareto percent point function, the scale
       parameter was ignored.  This was corrected.

    e. Fixed a bug in the HFLPPF library function.

    f. The GRUBBS TEST checks for both the maximum and minimum
       values as outliers (relative to the normal distribution).
       This is actually two tests: one for the minimum value and
       one for the maximum value.  When testing for both, the
       value of alpha needs to be divided by 2.

       The fix was to have the Grubbs test generate output for
       3 tests:

         1) Test both the minimum and the maximum value (with the
            value of alpha adjusted appropriately).
         2) Test the minimum value only.
         3) Test the maximum value only.

       To suppress the one-sided tests, enter the command

           SET GRUBBS ONE SIDED OFF

    g. Fixed a bug in the discrete uniform random number generator.
       The algorithm was generating random numbers on the interval
       [1,N].  This was corrected to generate random numbers on the
       interval [0,N].

    h. If the PRINTING switch was set to OFF, the YATES command
       was not writing information to files "dpst1f.dat" and
       "dpst2f.dat".  This was corrected so that these files are
       printed regardless of the setting of the PRINTING switch.

-----------------------------------------------------------------------
The following enhancements were made to DATAPLOT June - December  2004.
-----------------------------------------------------------------------

 1) The following updates were made for probability distributions.

    A. The following enhancements were made to maximum likelihood
       estimation.

       1. The maximum likelihood output was rewritten for the
          normal, lognormal, exponential, Weibull, gamma, beta,
          Gumbel, and Pareto distributions.

          Support was added for the following:

          a. Improved confidence intervals for the distributional
             parameters.

          b. support for censored data was added for the normal,
             lognormal, exponential, Weibull, and gamma distributions.

          c. Confidence intervals for selected percentiles was added
             for the normal, lognormal, exponential, Weibull, gamma,
             beta, and Gumbel distributions.

       2. Added support for the Rayleigh, Maxwell, asymmetric
          Laplace, generalized Pareto, and normal mixture
          distributions:

             RAYLEIGH MAXIMUM LIKELIHOOD Y
             MAXWELL MAXIMUM LIKELIHOOD Y
             ASYMMETRIC LAPLACE MAXIMUM LIKELIHOOD Y
             GENERALIZED PARETO MAXIMUM LIKELIHOOD Y

             LET NCOMP = <value>
             NORMAL MIXTURE MAXIMUM LIKELIHOOD Y

          The NCOMP parameter is used to specify how many normal
          distributions to mix (it defaults to 2 if a value is not
          specified for NCOMP).

       The online help for the maximum likelihood was also rewritten.
       Enter

          HELP MAXIMUM LIKELIHOOD

       for details.

    B. Support was added for the following new distributions.

       Skew-Laplace (Skew Double Exponential) distribution:

       LET A = SDECDF(X,LAMBDA)    - cdf of skew-Laplace distribution
       LET A = SDEPDF(X,LAMBDA)    - pdf of skew-Laplace distribution
       LET A = SDEPPF(X,LAMBDA)    - ppf of skew-Laplace distribution

       Asymmetric Laplace (Asymmetric Double Exponential) distribution:

       LET A = ADECDF(X,LAMBDA)    - cdf of asymmetric Laplace
                                     distribution
       LET A = ADEPDF(X,LAMBDA)    - pdf of aysmmetric Laplace
                                     distribution
       LET A = ADEPPF(X,LAMBDA)    - ppf of asymmetric Laplace
                                     distribution

       Maxwell-Boltzman distribution:

       LET A = MAXCDF(X,SIGMA)     - cdf of Maxwell Boltzman
       LET A = MAXPDF(X,SIGMA)     - pdf of Maxwell Boltzman
       LET A = MAXPPF(X,SIGMA)     - ppf of Maxwell Boltzman

       Rayleigh distribution:

       LET A = RAYCDF(X)           - cdf of Maxwell Boltzman
       LET A = RAYPDF(X)           - pdf of Maxwell Boltzman
       LET A = RAYPPF(X)           - ppf of Maxwell Boltzman

       Generalized Inverse Gaussian distribution:

       LET A = GIGCDF(X,CHI,LAMBDA,THETA) - cdf of generalized inverse
                                            gaussian distribution
       LET A = GIGPDF(X,CHI,LAMBDA,THETA) - pdf of generalized inverse
                                            gaussian distribution
       LET A = GIGPPF(X,CHI,LAMBDA,THETA) - ppf of generalized inverse
                                            gaussian distribution

       Generalized Asymmetric Laplace distribution:

       LET A = GALCDF(X,KAPPA,TAU) - cdf of generalized asymmetric
                                     Laplace distribution
       LET A = GALPDF(X,KAPPA,TAU) - pdf of generalized asymmetric
                                     Laplace distribution
       LET A = GALPPF(X,KAPPA,TAU) - ppf of generalized asymmetric
                                     Laplace distribution

       Bessel I Function distribution:

       LET A = BEICDF(X,S1SQ,S2SQ,NU) - cdf of Bessel I function
                                        distribution
       LET A = BEIPDF(X,S1SQ,S2SQ,NU) - pdf of Bessel I function
                                        distribution
       LET A = BEIPPF(X,S1SQ,S2SQ,NU) - ppf of Bessel I function
                                        distribution

       McLeish (related to Bessel K function) distribution:

       LET A = MCLCDF(X,ALPHA) - cdf of McLeish distribution
       LET A = MCLPDF(X,ALPHA) - pdf of McLeish distribution
       LET A = MCLPPF(X,ALPHA) - ppf of McLeish distribution

       Generalized McLeish (related to Bessel K function) distribution:

       LET A = GMCCDF(X,ALPHA,A) - cdf of McLeish distribution
       LET A = GMCPDF(X,ALPHA,A) - pdf of McLeish distribution
       LET A = GMCPPF(X,ALPHA,A) - ppf of McLeish distribution

    C. The following random number generators, plots, and commands
       were added:

         LET LAMBDA = <value>
         LET Y = SKEW LAPLACE RANDOM NUMBERS FOR I = 1 1 N
         SKEW LAPLACE PROBABILITY PLOT Y
         SKEW LAPLACE KOLMOGOROV SMIRNOV GOODNESS OF FIT Y
         SKEW LAPLACE CHI-SQUARE GOODNESS OF FIT Y
         SKEW LAPLACE PPCC PLOT Y
         SKEW LAPLACE KS PLOT Y

         LET LAMBDA = <value>
         LET Y = ASYMMETRIC LAPLACE RANDOM NUMBERS FOR I = 1 1 N
         ASYMMETRIC LAPLACE PROBABILITY PLOT Y
         ASYMMETRIC LAPLACE KOLMOGOROV SMIRNOV GOODNESS OF FIT Y
         ASYMMETRIC LAPLACE CHI-SQUARE GOODNESS OF FIT Y
         ASYMMETRIC LAPLACE PPCC PLOT Y
         ASYMMETRIC LAPLACE KS PLOT Y

         LET Y = MAXWELL RANDOM NUMBERS FOR I = 1 1 N
         MAXWELL PROBABILITY PLOT Y
         MAXWELL KOLMOGOROV SMIRNOV GOODNESS OF FIT Y
         MAXWELL CHI-SQUARE GOODNESS OF FIT Y

         LET Y = RAYLEIGH RANDOM NUMBERS FOR I = 1 1 N
         RAYLEIGH PROBABILITY PLOT Y
         RAYLEIGH KOLMOGOROV SMIRNOV GOODNESS OF FIT Y
         RAYLEIGH CHI-SQUARE GOODNESS OF FIT Y

         LET CHI = <value>
         LET LAMBDA = <value>
         LET THETA = <value>
         LET Y = GENERALIZED INVERSE GAUSSIAN RANDOM NUMBERS ...
                 FOR I = 1 1 N
         GENERALIZED INVERSE GAUSSIAN PROBABILITY PLOT Y
         GENERALIZED INVERSE GAUSSIAN KOLMOGOROV SMIRNOV ...
                 GOODNESS OF FIT Y
         GENERALIZED INVERSE GAUSSIAN CHI-SQUARE ...
                 GOODNESS OF FIT Y

         LET KAPPA = <value>
         LET TAU = <value>
         LET Y = GENERALIZED ASYMMETRIC LAPLACE RANDOM NUMBERS ...
                 FOR I = 1 1 N
         GENERALIZED ASYMMETRIC LAPLACE PROBABILITY PLOT Y
         GENERALIZED ASYMMETRIC LAPLACE KOLMOGOROV SMIRNOV ...
                 GOODNESS OF FIT Y
         GENERALIZED ASYMMETRIC LAPLACE CHI-SQUARE ...
                 GOODNESS OF FIT Y

         LET S1SQ = <value>
         LET S2SQ = <value>
         LET NU = <value>
         LET Y = BESSEL I FUNCTION RANDOM NUMBERS FOR I = 1 1 N
         BESSEL I FUNCTION PROBABILITY PLOT Y
         BESSEL I FUNCTION KOLMOGOROV SMIRNOV GOODNESS OF FIT Y
         BESSEL I FUNCTION CHI-SQUARE GOODNESS OF FIT Y

         LET ALPHA = <value>
         LET Y = MCLEISH RANDOM NUMBERS FOR I = 1 1 N
         MCLEISH PROBABILITY PLOT Y
         MCLEISH KOLMOGOROV SMIRNOV GOODNESS OF FIT Y
         MCLEISH CHI-SQUARE GOODNESS OF FIT Y
         MCLEISH PPCC PLOT Y
         MCLEISH KS PLOT Y

         LET ALPHA = <value>
         LET A = <value>
         LET Y = GENERALIZED MCLEISH RANDOM NUMBERS FOR I = 1 1 N
         GENERALIZED MCLEISH PROBABILITY PLOT Y
         GENERALIZED MCLEISH KOLMOGOROV SMIRNOV GOODNESS OF FIT Y
         GENERALIZED MCLEISH CHI-SQUARE GOODNESS OF FIT Y
         GENERALIZED MCLEISH PPCC PLOT Y
         GENERALIZED MCLEISH KS PLOT Y

    D. Dataplot uses the following defintion for the generalized
       Pareto probability density function:

          f(x,gamma) = (1+gamma*x)**(-(1/gamma)-1)

       However, many sources (e.g., Johnson, Kotz, and Balakrishnan)
       define the generalized Pareto as:

          f(x,gamma) = (1-gamma*x)**((1/gamma)-1)

       That is, the sign of gamma is reversed.  The following
       command was added:

          SET GENERALIZED PARETO DEFINITION <JOHNSON AND KOTZ/SIMIU>

       was added.  A value of JOHNSON or KOTZ for this command
       will use the second definition given.  Any other value
       will use the first (default) definition.

    E. For the Pareto and Pareto type 2 distributions, what is
       typically referred to as the location parameter (the A
       parameter) is not a location parameter in the technical
       sense that the relation

           f(x;gamma,loc) = f((x-loc);gamma,0)

       does not hold (it is a location parameter in the sense
       that it defines a lower bound for the Pareto, but not the
       Pareto type 2, distribution).

       For this reason, we modified the Dataplot definition to
       treat A as a second shape parameter.  For example, the
       Pareto PDF function is

           PARPDF(x,gamma,a,loc,scale)

       The A, LOC, and SCALE parameters are optional (A will
       default to 1 if not given).

    F. The following enhancements were made to the probability
       plot and ppcc/ks plots.
 
       Note that both the probability plot and the ppcc plot
       ultimately depend on computing the percent point function
       for the specified distribution.  If the percent point function
       is fast to compute (e.g., if it exists as a simple, closed
       formula), then these plots can be generated rapidly even if the
       number of data points is large.  On the other hand, some percent
       point functions can require a good deal of computation.  For
       example, some distributions compute the cumulative distribution
       function via numerical integration and then compute the percent
       point function by inverting the cumulative distribution
       function.  In these cases, the ppcc/ks plots can take too long
       to generate to be practical (this tends to be less of an issue
       with probability plots).
 
       1. The following commands can be used to control how many
          points are used to generate probability and ppcc/ks
          plots, respectively:
  
              SET PROBABILITY PLOT DATA POINTS <value>
              SET PPCC PLOT DATA POINTS <value>
 
          The algorithm is to compute <value> equally spaced
          percentiles of the full data set and then use these
          percentiles in generating the probability and
          ppcc/ks plot.
 
          Using this command involves a trade-off between speed
          and accuracy.  For distributions with simple, closed
          formualas or fast approximations for the percent point
          function, there is little reason not to use the full data
          set.  However, for many distributions, the ppcc plot or
          ks plot can become impractical as the number of data points
          increases.
 
          The minimum number of points is 20.  The number of
          points is typically set between 50 and 100.  You may
          want to use less than 50 points for a few distributions
          with particularly expensive percent point functions.
          For distributions with only moderately expensive percent
          point functions, you may want to go as high as 100 or
          200.
 
       2. For the ppcc (or ks) plot, each point on the plot
          represents one underlying probability plot (which in
          return requires n, where n is the sample size, computations
          of the percent point function.  For distributions with
          one shape parameter, Dataplot typically uses 50 points
          (i.e., there are 50 underlying probability plots
          computed).  For two shape parameters, Dataplot typically
          uses between 20 and 50 values for each shape parameter.
          It decreases the number of values used when the percent
          point function is expensive to compute.
 
          The following command allows you to explicitly specify
          how many probability plots are generated by the ppcc plot:
 
             SET PPCC PLOT AXIS POINTS <value1>  <value2>
 
          with <value1> and <value2> denoting the number of values
          to use for the first and second shape parameters,
          respectively.  Specifying <value2> is optional.
 
          Set these values to 0 in order to revert to the Dataplot
          default.
 
          There are actually two reasons for using this command.
          If the percent point function is fast to compute (e.g.,
          the Weibull distribution), you may want to increase the
          number of points in order to generate a finer grid.  On
          the other hand, if the percent point function is
          expensive to compute, you may want to decrease the
          number of points to speed up the generation of the plot.
          
       3. If the ppcc (or ks) plot has two shape parameters, then
          the default graphical format is to plot the ppcc (or
          ks) value on the y-axis.  Each curve on the plot
          represents one value of one shape parameter while the
          value of the x-axis coordinate represents the value of
          the other shape parameter.  To reverse the roles of the
          shape parameters, enter the command
 
              SET PPCC PLOT AXIS ORDER REVERSE
 
          To restore the default, enter
 
              SET PPCC PLOT AXIS ORDER DEFAULT
 
       4. The PPCC PLOT will write the following to the file
          dpst2f.dat (in the current directory):

            PPCC   LOCATION    SCALE       SHAPE1      SHAPE2
            VALUE  PARAMETER   PARAMETER   PARAMETER   PARAMETER

          This can be useful for plotting how the estimate of location
          and scale change as the shape parameter changes.  In some
          cases, a less optimal value of the shape parameters may
          be preferred if it generates more realistic estimates for
          location and scale.

       5. The PROBABILITY PLOT and PPCC PLOT were updated to support
          multiply censored data.

          The syntax is

              <dist> CENSORED PROBABILITY PLOT Y X
              <dist> CENSORED PPCC PLOT Y X

          The X variable identifies which points represent failure
          and which represent censoring times.  Specifically,
          X = 1 implies a failure time and X = 0 represents a
          censoring time.  The word CENSORED is required to
          distinguish this syntax from the syntax for binned
          data.  Censored probability plots and censored ppcc
          plots do not apply to binned data.

          Dataplot supports two algorithms for determining plot
          coordinates for a censored probability plot.

           i. The uniform order statistic medians are generated
              based on the full sample size.  However, only
              values that represent a failure time are actually
              plotted.

          ii. Instead of uniform order statistic medians, the
              plotting positions for the failure times are
              computed using the Kaplan-Meier product limit
              estimate:

                 U(i) = ((n+0.7)/(n+0.4))*
                        PRODUCT[q=1 to i][(n-q+0.7)/(n-q+1.7)]

              with n denoting the full sample size and q denoting
              failure times only.  The theoretical quantile is then
              the percent point function of U(i).

          The censored ppcc plot is then based on the correlation
          coefficient of the censored probability plot.

          To specify which censoring algorithm to use, enter the
          commands

             SET CENSORED PROBABILITY PLOT
                 <uniform order statistc medians/kaplan-meier>

             SET CENSORED PPCC PLOT
                 <uniform order statistc medians/kaplan-meier>

          The default is to use the uniform order statistic medians
          algorithm.

    G. The following enhancements were made to the
       Kolmogorov-Smirnov goodness of fit command and the KS PLOT.
       plot and ppcc/ks plots.
 
       1. The KS PLOT for the binned case (<dist> KS PLOT Y X) now
          automatically plots the chi-square goodness of fit
          statistic rather than the Kolmogorov-Smirnov goodness of
          fit statistic.  This is done since the chi-square goodness
          of fit is expliticly based on binned data.  Note that
          bins with a size less than 5 are automatically combined
          so that the minimum bin size is at least 5.

       2. The KS PLOT will write the following to the file
          dpst2f.dat (in the current directory):

            PPCC   LOCATION    SCALE       SHAPE1      SHAPE2
            VALUE  PARAMETER   PARAMETER   PARAMETER   PARAMETER

          This can be useful for plotting how the estimate of location
          and scale change as the shape parameter changes.  In some
          cases, a less optimal value of the shape parameters may
          be preferred if it generates more realistic estimates for
          location and scale.

 2) The following graphics commands were added.

    a. Univariate average shifted histograms can be generated with
       the command:

           ASH HISTOGRAM Y

 3) The following analysis commands were added.

    a. Cochran's test can be performed with the command

           COCHRAN TEST Y X

       where Y is a response variable and X is a group identifier
       variable.  Cochran's test is an alternative to the
       Kruskal-Wallis test when the response variable is dichotomous
       (i.e., only 2 possible values).

    b. The Kruskal-Wallis test was enhanced to write the pairwise
       multiple comparisons to the file dpst1f.dat.

    c. Van Der Waerden's test can be performed with the command

          VAN DER WAERDEN TEST Y X

       where Y is a response variable and X is a group identifier
       variable.  Van Der Waerden's test is an alternative to
       KRUSKAL WALLIS that is based on normal scores of the ranks.

 4) The following statistics and LET subcommands were added.

    a. Kendell's tau can be computed with the command

          LET A = KENDELL TAU Y1 Y2

    b. For the chi-square goodness of fit, it is generally advisable
       to combine bins with small counts (typically, 5 is recommended
       as a minimum bin size).  To convert equal width bins to
       variable width bins with a minimum bin count, enter the
       commands

          LET MINSIZE = <value>
          LET Y2 XLOW XUPPER  = Y X

    c. The commands

         LET Y2 X2 = ASH BINNED Y
         LET Y2 X2 = COUNTS ASH BINNED Y

       generate frequency tables based on the average shifted
       histogram (see ASH HISTOGRAM above).  The first syntax returns
       the relative frequency while the second syntax returns a
       count.

 5) The following enhancements were made to the READ command.

     a. In previous versions of Dataplot, if your data set contained
        rows with an unequal number of columns, Dataplot would only
        read the number of variables corresponding to the row
        with the minimum number of columns.

        If you would like Dataplot to pad missing columns with a
        missing value, enter the command

           SET READ PAD MISSING COLUMNS ON

        For example, if you enter the command

           READ FILE.DAT X1 X2 X3 X4 X5

        then rows with less than five columns will set the missing
        rows to a missing value.  To set the numeric value that
        represents a missing value, enter

           SET READ MISSING VALUE <value>

        where <value> denotes the desired numeric value.

        To reset the default behavior, enter the command

           SET READ PAD MISSING COLUMNS OFF

        In some cases, missing columns would be indicative of an
        error in the data file.

     b. The SUBSET/EXCEPT/FOR clause on a READ command was ambiguous.
        The ambiguity aries from the fact that it is not clear whether
        the SUBSET/EXCEPT/CLAUSE command refers to the lines in the
        data file being read or to the output variables that are
        created by the READ command.  We address this with the
        following command:

           SET READ SUBSET  <PACK/DISPERSE>   <PACK/DISPERSE>

        In this command, PACK means the SUBSET/EXCEPT/FOR clause
        does not apply while DISPERSE means that it does.  The
        first setting applies to the input file while the second
        setting applies to the created data variables.

        This is demonstrated with the following example (note that
        P-D means the data file is set to PACK and the output
        variable is set to DISPERSE).  The first column is the
        data in the file while the remaining columns show what
        the resulting data variable should look like.

                  READ FILE.DAT  X  FOR I = 1  2  10

           X      P-D       P-P          D-P      D-D
          ===========================================
           1       1         1            1        1
           2       0         2            3        0
           3       2         3            5        3
           4       0         4            7        0
           5       3         5            9        5
           6       0         6            -        0
           7       4         7            -        7
           8       0         8            -        0
           9       5         9            -        9
          10       -        10            -        -

     The default setting is PACK-DISPERSE (this is the default
     because this is the behavior of previous versions of Dataplot).

 6) Miscellaneous Updates

    a. Added the command

         SET POSTSCRIPT DEFAULT COLOR <ON/OFF>

       Postscript devices can be either black and white or color.
       Dataplot assumes black and white by default.  After the
       DEVICE <2/3> POSTSCRIPT command, you can enter

          DEVICE <2/3> COLOR ON

       Although this works fine for DEVICE 2, it presents
       complications for DEVICE 3 (this is the device used by the
       PP command to print the current graph to a Postscript
       printer).  Dataplot opens/closes this device as needed
       without the user entering any commands.  It can be
       difficult to determine when to insert a DEVICE 3 COLOR ON
       command.
 
       If you enter
 
          SET POSTSCRIPT DEFAULT COLOR ON
 
       then Dataplot will assume Postscript devices are color
       (this applies to both DEVICE 2 and DEVICE 3, although it
       is primarily motivated for DEVICE 3 output). 

    b. The default algorithm for class width in Dataplot is to
       use 0.3*s where s is the sample standard deviation.

       A number of different algorithms have been proposed to
       obtain "optimal" class widths.  The command

          SET HISTOGRAM CLASS WIDTH <DEFAULT/SD/IQ/NORMAL/
                                    NORMAL CORRECTED>

       can be used to specify the default class width that Dataplot
       will use for the HISTOGRAM and ASH HISTOGRAM commands.
       Additional choices may be added in future releases.

       The current choices are:

           DEFAULT           - use 0.3*s
           SD                - use 0.3*s
           NORMAL            - use 2.5*s/n**(1/3)
           NORMAL CORRECTED  - start with 2.5*s/n**(1/3).  If the
                               skewness is between 0 and 3, multiply
                               this by the correction factor:
                               1/(1 - 0.006*skew + 0.27*skew**2 -
                               0.0069*skew**3).
                               If the kurtosis - 3 is between 0 and 6,
                               multiply by the correction factor:
                               1 - 0.2*(1 - EXP(-0.7*(kurt - 3)))
           IQ                - use 2.603*IQ/N**(1/3) where IQ is the
                               interquartile range

       The NORMAL width is an optimal choice (in the sense of
       minimizing the integrated mean square error of the histogram)
       if the data is in fact normal.  The NORMAL CORRECTED provides
       correction factors for moderate skewness and kurtosis.  The
       IQ replaces s with a robust estimate of scale (the
       interquartile range) and should provide a reasonable bin width
       for a wide range of underlying distributions.

       Since the "optimal" choice of bin width is dependent on
       the underlying distribution of the data, it is difficult
       to provide a default bin width that will work well in all
       cases (we are typically using the histogram to help determine
       what that underlying distribution actually is).

       An explicit CLASS WIDTH command will override the default
       class width algorithm.

    c. For the chi-square goodness of fit test, it is usually
       recommended that classes with less than 5 observations be
       combined in order to obtain a reasonably accurate
       approximation.  Given data that is binned into equal size
       bins, you can automatically combine bins with small
       frequencies with the commands

           LET MINSIZE = <value>
           LET Y3 XLOW XHIGH = COMBINE FREQUENCY TABLE Y2 X2

       The variables XLOW and XHIGH will contain the lower and upper
       boundary values for the classes (since bins will no longer be
       of equal length), respectively.  The value for MINSIZE defines
       the minimum frequency for a class (it defaults to 5).

       You can then generate a chi-square goodness of fit test
       with the command

            <dist> CHISQUARE GOODNESS OF FIT Y3 XLOW XHIGH

       A typical sequence of commands for generating a chi-square
       goodness of fit test for a discrete distribution, starting
       from raw data, is

            LET AMIN = MINIMUM Y
            LET AMAX = MAXIMUM Y
            CLASS LOWER AMIN
            CLASS UPPER AMAX
            CLASS WIDTH 1
            LET Y2 X2 = BINNED Y
            LET MINSIZE = 5
            LET Y3 XLOW XHIGH = COMBINE FREQUENCY TABLE Y2 X2
            <dist> CHISQUARE GOODNESS OF FIT Y3 XLOW XHIGH

    d. The CORRELATION MATRIX and COVARIANCE MATRIX compute the
       correlation and covariance matrices, respectively, of the
       columns of a matrix.  If you would like these to be
       generated from the rows of the matrix, you can enter the
       commands

           SET CORRELATION MATRIX DIRECTION ROW
           SET COVARIANCE  MATRIX DIRECTION ROW

       To reset to the columns, enter

           SET CORRELATION MATRIX DIRECTION COLUMN
           SET COVARIANCE  MATRIX DIRECTION COLUMN

 7) Bug Fixes:

    a. There was a bug reading numbers of the form

         -.23

       In this case, the minus sign was being lost.  You can
       work around this by entering the number as

         -0.23

      This bug is fixed in the current version.

      NOTE: This bug was introduced in the 1/2004 version.

   b. There was a bug reading rows containing a single character.
      This has been fixed.  If you encounter this bug, you can
      work around it by inserting a leading space in the data
      file.

      NOTE: This bug was introduced in the 1/2004 version.

   c. The SET commands that accepted file names as arguments did
      not support quoting.  Enclosing the file name in quotes is
      required when the file names contains spaces or hyphens.
      This has been corrected.

   d. There was a bug in the SUMMARY command where in some cases
      it did not extract the correct data.  This has been fixed.

   e. There was a bug in the KAPLAN MEIER PLOT command that caused
      the censoring variable to not be recognized.  This has been
      corrected.

   f. A bug was corrected for nested IF blocks.

   g. The following loop syntax (i.e., the start value is greater than
      the stop value for positive increments or the start value is
      less than the stop value for negative increments) was executing
      the loop twice.

         LOOP FOR K = 6  1  5
             PRINT K
         END OF LOOP

      This has been corrected so that the loop does not execute
      (the value of K will be undefined after processing the loop).

----------------------------------------------------------------------
The following enhancements were made to DATAPLOT February - May  2004.
----------------------------------------------------------------------

 1) The following updates were made for probability distributions.

    a. Support was added for the following new distributions.

       Log-skew-normal distribution:

       LET A = LSNCDF(X,LAMBDA,SD) - cdf of log-skew-normal
                                     distribution
       LET A = LSNPDF(X,LAMBDA,SD) - pdf of log-skew-normal
                                     distribution
       LET A = LSNPPF(P,LAMBDA,SD) - ppf of log-skew-normal
                                     distribution

       Log-skew-t distribution:

       LET A = LSTCDF(X,NU,LAMBDA,SD) - cdf of log-skew-normal
                                        distribution
       LET A = LSTPDF(X,NU,LAMBDA,SD) - pdf of log-skew-normal
                                        distribution
       LET A = LSTPPF(P,NU,LAMBDA,SD) - ppf of log-skew-normal
                                        distribution

       G-and-H distribution:

       LET A = GHCDF(X,G,H)           - cdf of g-and-h distribution
       LET A = GHPDF(X,G,H)           - pdf of g-and-h distribution

       Note that the ppf function was added in a previous update.

       Hermite distribution:

       LET A = HERCDF(X,A,B)          - cdf of Hermite distribution
       LET A = HERPDF(X,A,B)          - pdf of Hermite distribution
       LET A = HERPPF(P,A,B)          - ppf of Hermite distribution

       Yule distribution:

       LET A = YULCDF(X,P)            - cdf of Yule distribution
       LET A = YULPDF(X,P)            - pdf of Yule distribution
       LET A = YULPPF(P,P)            - ppf of Yule distribution

    b. The following pdf functions were added (these distributions
       previously supported the cdf and ppf functions).

       LET A = NCTPDF(X,NU,LAMBDA)      - pdf of non-central t
       LET A = DNTPDF(X,NU,L1,L2)       - pdf of doubly non-central t
       LET A = NCCPDF(X,NU,LAMBDA)      - pdf of non-central chi-square
       LET A = NCFPDF(X,NU1,NU2,L1)     - pdf of non-central F
       LET A = DNFPDF(X,NU1,NU2,L1,L2)  - pdf of doubly non-central F
       LET A = NCBPDF(X,A,B,LAMBDA)     - pdf of non-central Beta

       These pdf functions are computed by taking the numerical
       derivative of the corresponding cdf function.  You may
       at times get warning messages that the derivative has not
       converged with sufficient accuracy (this occurs most frequently
       with the non-central Beta distribution).

    c. The following enhancements were made to maximum likelihood
       estimation.

       1. The binomial case now generates lower and upper confidence
          limits based on the Agresti and Coull approximation.

       2. The lognormal case now generates confidence limits for
          the shape and scale parameters.

       3. Support was added for the following distributions:

            LOGARITHIC SERIES MAXIMUM LIKELIHOOD Y
            GEOMETRIC MAXIMUM LIKELIHOOD Y
            BETA BINOMIAL MAXIMUM LIKELIHOOD Y
            NEGATIVE BINOMIAL MAXIMUM LIKELIHOOD Y
            HYPERGEOMETRIC MAXIMUM LIKELIHOOD Y
            HERMITE MAXIMUM LIKELIHOOD Y
            YULE MAXIMUM LIKELIHOOD Y

            FATIGUE LIFE MAXIMUM LIKELIHOOD Y
            GEOMETRIC EXTREME EXPONENTIAL MAXIMUM LIKELIHOOD Y
            FOLDED NORMAL MAXIMUM LIKELIHOOD Y
            CAUCHY MAXIMUM LIKELIHOOD Y

       4. For the Johnson SU/SB distribution, a percentile
          estimator is now available (a method of moments
          estimator was previously available):

             JOHNSON PERCENTILE Y

          Note that this estimator will automatically determine
          whether a SB or SU estimator is appropiate.  Also, you
          can define a constant Z used by this estimator by
          entering the command (before the JOHNSON PERCENTILE
          command):

             LET Z = <value>

          This value is typically set between 0.5 and 1 with a
          default value of 0.54.  As the sample size gets larger,
          then values of Z closer to 1 are appropriate (e.g.,
          for a sample of size 1,000, a value of 0.8 works well).

       5. Support for Latex and HTML output was added to most
          supported distributions.

    d. The following random number generators were added:

         LET NU = <value>
         LET LAMBDA = <value>
         LET Y = NONCENTRAL T RANDOM NUMBERS FOR I = 1 1 N

         LET NU = <value>
         LET LAMBDA1 = <value>
         LET LAMBDA2 = <value>
         LET Y = DOUBLY NONCENTRAL T RANDOM NUMBERS FOR I = 1 1 N

         LET NU = <value>
         LET LAMBDA = <value>
         LET Y = NONCENTRAL BETA RANDOM NUMBERS FOR I = 1 1 N

         LET GAMMA = <value>
         LET Y = GENERALIZED LOGISTIC RANDOM NUMBERS FOR I = 1 1 N

         LET GAMMA = <value>
         LET Y = GENERALIZED HALF-LOGISTIC RANDOM NUMBERS FOR I = 1 1 N

         LET ALPHA = <value>
         LET BETA = <value>
         LET Y = HERMITE RANDOM NUMBERS FOR I = 1 1 N

         LET P = <value>
         LET Y = YULE RANDOM NUMBERS FOR I = 1 1 N

         LET A = <value>
         LET C = <value>
         LET Y = WARING RANDOM NUMBERS FOR I = 1 1 N

         LET A = <value>
         LET B = <value>
         LET C = <value>
         LET Y = GENERALIZED WARING RANDOM NUMBERS FOR I = 1 1 N

       The t, F, and chi-square random number generators were
       updated to accept non-integer values for the degrees of
       freedom parameters.

    e. The following additions were made to the probability plot,
       Kolmogorov-Smirnov goodness of fit, chi-sqaure goodness of
       fit, and ppcc plot commands:

          LET LAMBDA = <value>
          LET SD = <value>
          LOG SKEW NORMAL PROBABILITY PLOT Y
          LOG SKEW NORMAL KOLMOGOROV-SMIRNOV GOODNESS OF FIT Y
          LOG SKEW NORMAL CHI-SQUARE GOODNESS OF FIT Y

          LOG SKEW NORMAL PPCC PLOT Y

          LET LAMBDA = <value>
          LET SD = <value>
          LET NU = <value>
          LOG SKEW T PROBABILITY PLOT Y
          LOG SKEW T KOLMOGOROV-SMIRNOV GOODNESS OF FIT Y
          LOG SKEW T CHI-SQUARE GOODNESS OF FIT Y

          LET G = <value>
          LET H = <value>
          G AND H PROBABILITY PLOT Y
          G AND H KOLMOGOROV-SMIRNOV GOODNESS OF FIT Y
          G AND H CHI-SQUARE GOODNESS OF FIT Y

          G AND H PPCC PLOT Y

          LET ALPHA = <value>
          LET BETA = <value>
          HERMITE PROBABILITY PLOT Y
          HERMITE CHI-SQUARE GOODNESS OF FIT Y

          HERMITE PPCC PLOT Y

          LET P = <value>
          YULE PROBABILITY PLOT Y
          YULE CHI-SQUARE GOODNESS OF FIT Y

          YULE PPCC PLOT Y

    f. The Anderson Darling test was updated to support the
       generalized Pareto distribution:

         ANDERSON-DARLING GENERALIZED PARETO TEST Y

       The maximum likelihood estimation for the generalized
       Pareto is still undergoing algorithmic development, so
       you should specify the shape and scale parameter for
       the generalized Pareto (before invoking the Anderson-Darling
       test) as follows:

          LET GAMMA = <value of shape parameter>
          LET A = <value of scale parameter>

    g. An optional definition was added for the geometric
       distribution.

       The default defintion for the geometric distribution is the
       number of failures before the first success is obtained in
       a sequence of Bernoulli trials.  The alternate definition
       is the number of trials up to and including the first
       success in a series of Bernoulli trials.  This definition
       simply shifts the geometric distribution to start at X = 1
       rather than X = 0.

       To specify the alternate definition, enter the command

          SET GEOMETRIC DEFINITION DLMF

       To restore the default definition, enter the command

          SET GEOMETRIC DEFINITION JOHNSON AND KOTZ

    h. The negative binomial was updated to support non-integer
       arguments for the number of failures shape parameter
       (i.e., k).

    i. A number of bug fixes and algorithmic improvements were made
       for the ppcc plots with two shape parameters and the random
       number generation for a few distributions.

 2. The following enhancements were made to the PPCC PLOT and
    PROBABILITY PLOT commands.

    a. For some long tailed distributions, there can be large
       variability in the tails.  This can distort the estimates
       of location, PPA0,  and scale, PPA1, of the line fitted
       to the probability plot.  To address this, Dataplot now
       also returns PPA0BW and PPA1BW.  These are the estimates
       obtained by performing two iterations of biweight
       weighting of the residuals.

       In most cases, the use of PPA0 and PPA1 is preferred.
       However, if the probability plot indicates the prescence
       of extreme outliers in the tails, PPA0BW and PPA1BW may
       provide better estimates for the location and scale
       parameters.

    b. The following command was added as a variant of the
       ppcc plot:

          <dist> KS PLOT Y

       where <dist> is any of the distributions supported by
       the PPCC PLOT command.

       This plot uses a similar concept to the ppcc plot.
       However, it uses the value of the Kolmogorov-Smirnov
       goodness of fit statistic rather than the correlation
       coefficient of the probability plot as the measure 
       of distributional fit.  In this, the goal is to minimize
       the Kolmogorov-Smirnov goodness of fit statistic.

       Although we are still developing experience with this
       plot, a few prelimary recommendations are:

       1. For most continuous distributions with one shape
          parameter, the PPCC PLOT and KS PLOT generate similar
          estimates for the shape parameter.

       2. The KS PLOT seems to perform better for at least some
          distributions with two shape parameters.

       3. The KS PLOT generates a smoother plot for discrete
          distributions.

       For additional information, enter

          HELP KS PLOT

    c. For the PPCC PLOT and KS PLOT, the following command
       allows you to specify the desired format for the
       plot when there are two shape parameters:

          SET PPCC FORMAT  <TRACE/3D>

       For the default setting, TRACE, these plots are generated
       as a multi-trace 2D plot.  That is, the Y axis will
       represent the correlation (or value of the
       Kolmogorov-Smirnov statistic), the X axis will represent
       the value of the second shape parameter, and each trace
       will represent one of the values for the first shape
       parameter.

       If this value is set to 3D, the plot is represented as
       a 3D surface plot.

 3. Sometimes data may only be available in the form of a frequency
    table.  However, some Dataplot commands may expect the data
    in a "raw" format.  The following command was added to convert
    frequency data to raw data:

       LET Y = FREQUENCY TO RAW X FREQ

    For example,

        X   FREQ
        --------
        0   3
        1   2
        2   4

    would be converted to

        0
        0
        0
        1
        1
        2
        2
        2
        2

-------------------------------------------------------------------------
The following enhancements were made to DATAPLOT June 2003-January  2004.
-------------------------------------------------------------------------

 1) The following enhancements were made to the Dataplot I/O
    capabilities.

    a) Previously, the Dataplot READ command was updated to
       handle the syntax

          READ  FILE.DAT

       In this case, Dataplot simply assigns the names X1, X2,
       and so on to the variables.  Many packages accept data
       files where the first line contains the variable names.
       To support this in Dataplot, do the following:

          SET READ VARIABLE LABEL ON
          READ FILE.DAT

       In this case, Dataplot will interpret the first line
       read as the variable names in the file.

    b) Dataplot has previously not supported reading character
       variables in data files (with the one execption of READ ROW
       LABELS).  If encountered, Dataplot would generate an error
       message and not read the data file correctly.  To address
       this, we have added the command

           SET CONVERT CHARACTER <ON/IGNORE/ERROR>

       Setting this to ERROR will continue the current Dataplot
       action of reporting an error.  This is recommended for the
       case when a file is suppossed to contain only numeric data
       and the presence of character data is in fact indicative
       of an error in the data file.  Setting this to IGNORE will
       instruct Dataplot to simply ignore any fields containing
       character data.  Setting this to ON will read character fields
       and write them to the file "dpzchf.dat".

       There are some restrictions on when Dataplot will try to
       read character data:

          1) This only applies to the variable read case.  That
             is, READ PARAMETER and READ MATRIX will ignore
             character fields or treat them as an error.

          2) Dataplot will only try to read character data from
             a file.  When reading from the keyboard (i.e., when
             READ is specified with no file name), character data
             will be ignored when a SET CONVERT CHARACTER ON is
             specified.

          3) This capability is not supported for the SERIAL READ
             case.

          4) The SET READ FORMAT command does not accept the
             "A" format specification for reading character
             fields.

       Some of these restrictions may be addressed in subsequent
       releases of Dataplot.

       Enter HELP CONVERT CHARACTER for details.

    c) The COLUMN LIMITS command has been updated to accept
       variable arguments.  For example,

            COLUMN LIMITS  LOWER  UPPER

       with LOWER and UPPER denoting variables (as oppossed to
       parameters) each with N elements.  Dataplot will parse
       the data file assuming that field one of the data is in
       columns LOWER(1) to UPPER(1), field two of the data is
       in LOWER(2) to UPPER(2) and so on.  Note that only one
       numeric or character variable will be read in each field.

       Many programs, Excel for example, will write data to ASCII
       files with the data values either left or right justified
       to a given column.  If the ASCII file is written so that
       the decimal point is in a fixed column, then using the
       SET READ FORMAT is typically recommended rather than
       the COLUMN LIMITS with variable arguments.
       
       If the data file contains columns of equal length, then
       using this form of the COLIMNM LIMITS command is not
       necessary.  However, there are two cases where it is useful:

          1) If you only want to read selected fields in the data
             file, then this form of the COLUMN LIMITS command
             easily allows you to do this.

          2) If the data columns are of unequal length, as ASCII
             files created from Excel often are, then this form
             of the COLUMN LIMITS allows these data files to be
             read correctly. If a given field is empty, Dataplot
             interprets it as a missing value.

             By default, Dataplot will set the missing value to 0. 
             If you would like to specify a value other than zero,
             then enter the command

                SET READ MISSING VALUE  <value>

             where <value> is the desired value.

        Enter HELP COLUMN LIMITS for details.

    d) If Excel writes a comma delimited ASCII file (.CSV), then
       missing values are denoted with ",,".  In order to interpert
       these files correctly, you can enter the command

          SET READ DELIMITER  <value>

       where <value> specifies the desired delimiter.  The default
       delimiter is a comma.

       If Dataplot encounters the delimiter before any valid data
       has been found, it interprets this as a missing value. 
       Missing values are set to 0 unless a SET READ MISSING VALUE
       command has been entered (see above).

    We have added a section in the online help files that provides
    general guidance on reading ASCII data files in Dataplot.
    This consolidates information documented under a number of
    different commands.  For details, enter

         HELP ASCII FILES

 2) The SET CONVERT CHARACTER ON command allows you to read
    character variables.  We have added the following commands
    that operate on these character variables.

    a) Many character variables are in fact group-id variables.
       In order to allow you to use these group-id variables
       in a numeric context, the following two commands were added:

           LET Y = CHARACTER CODE IX
           LET Y = ALPHABETIC CHARACTER CODE IX

       with IX denoting the name of a character variable that
       has been read into Dataplot and Y denoting the name of a
       numeric variable that will be created by this command.

       Both of these commands identify the unique rows in the
       character variable (Dataplot checks for exact matches, it
       does not try to guess if a typo has occurred, etc.).  If
       there are K unique rows, Dataplot will generate coded values
       as the integer values from 1 to K.  The distinction is that
       CHARACTER CODE will perform the coding in the order that the
       unique rows are encoutered in the file while ALPHABETIC
       CHARACTER CODE will sort the unique character rows and
       code based on the alphabetic order.

    b) Character variables are frequently used as group-id
       variables (e.g., Male and Female to identify sex).  The
       following command creates a group-id variable from a
       character variable:

          LET IG = GROUP LABELS MONTH

       with MONTH denoting the name of a character variable.
       The name IG will be used to denote a group-id variable.
       The number of rows in IG will be equal to the number of
       unique rows in MONTH.  Up to 5 group-id variables can be
       created and the maximum number of rows for a group-id
       variable is the maximum number of rows for a numeric
       variable divided by 100.

    c) You can create a row label variable with the READ ROW LABEL
       command.  Alternatively, you now enter the command

           LET ROWLABEL = MONTH

       with MONTH denoting the name of a character variable.
       Note that the variable name on the left hand side of the
       "=" must be ROWLABEL for this command to work.

    d) The TIC MARK LABEL FORMAT and TIC LABEL CONTENT commands
       have been updated to suppor the following:

           TIC MARK LABEL FORMAT GROUP LABEL
           TIC MARK LABEL CONTENT IG

           TIC MARK LABEL FORMAT ROW LABEL

           TIC MARK LABEL FORMAT VARIABLE
           TIC MARK LABEL CONTENT YVAR

      Setting the tic mark label format to GROUP LABEL instructs
      Dataplot to use a group label variable for the contents
      of the tic mark label.  The TIC MARK LABEL CONTENT command
      is then used to specify the name of the group label variable
      to use.

      Setting the tic mark label format to VARIABLE is similar to
      the GROUP LABEL case.  However,  in this case a numeric
      variable is specified rather than a group label variable.
      This allows you to place your own numeric tic mark labels.
      For example, you can use this to generate a "reverse" axis.

      Setting the tic mark label format to ROW LABEL allows you
      to use the row labels as the content for the tic mark labels.
      For example, this can be useful for labeling a bar chart.

 3) Support for the following univariate distributions was added:

       LET A = TRACDF(X,A,B,C,D)   - cdf of trapezoid distribution
       LET A = TRAPDF(X,A,B,C,D)   - pdf of trapezoid distribution
       LET A = TRAPPF(P,A,B,C,D)   - ppf of trapezoid distribution

       LET A = GTRCDF(X,A,B,C,D,NU1,NU3,ALPHA) - cdf of generalized
                                                 trapezoid distribution
       LET A = GTRPDF(X,A,B,C,D,NU1,NU3,ALPHA) - pdf of generalized
                                                 trapezoid distribution
       LET A = GTRPPF(P,A,B,C,D,NU1,NU3,ALPHA) - ppf of generalized
                                                 trapezoid distribution

       LET A = FTCDF(X,NU)         - cdf of folded t distribution
       LET A = FTPDF(X,NU)         - pdf of folded t distribution
       LET A = FTPPF(P,NU)         - ppf of folded t distribution

       LET A = SNCDF(X,ALPHA)      - cdf of skew normal distribution
       LET A = SNPDF(X,ALPHA)      - pdf of skew normal distribution
       LET A = SNPPF(P,ALPHA)      - ppf of skew normal distribution

       LET A = STCDF(X,NU,ALPHA)   - cdf of skew t distribution
       LET A = STPDF(X,NU,ALPHA)   - pdf of skew t distribution
       LET A = STPPF(X,NU,ALPHA)   - ppf of skew t distribution

       LET A = SLACDF(X)           - cdf of slash distribution
       LET A = SLAPPF(P)           - ppf of slash distribution

       LET A = IBCDF(X,ALPHA,BETA) - cdf of inverted beta distribution
       LET A = IBPPF(P,ALPHA,BETA) - ppf of inverted beta distribution

       LET A = GHCDF(X,G,H)        - cdf of g-and-h distribution
       LET A = GHPPF(P,G,H)        - ppf of g-and-h distribution

       LET A = MAKCDF(X,XI,L,T)    - cdf of Gompertz-Makeham distribution
       LET A = MAKPDF(X,XI,L,T)    - pdf of Gompertz-Makeham distribution
       LET A = MAKPPF(P,XI,L,T)     - ppf of Gompertz-Makeham distribution

       LET A = GHPPF(P,G,H)        - ppf of g-and-h distribution

       LET A = ZIPPDF(X,ALPHA)     - pdf of Zipf distribution

     Note that the IBPDF and SLAPDF functions were implemented
     previously.  The GHPDF function is still under development.

     You can generate random numbers for these distributions
     with the commands

           LET A = <value>
           LET B = <value>
           LET C = <value>
           LET D = <value>
           LET Y = TRAPEZOID RANDOM NUMBERS FOR I = 1 1 N

           LET A = <value>
           LET B = <value>
           LET C = <value>
           LET D = <value>
           LET NU1 = <value>
           LET NU3 = <value>
           LET ALPHA = <value>
           LET Y = GENERALIZED TRAPEZOID RANDOM NUMBERS FOR I = 1 1 N

           LET NU = <value>
           LET Y = FOLDED T RANDOM NUMBERS FOR I = 1 1 N

           LET ALPHA = <value>
           LET Y = SKEWED NORMAL RANDOM NUMBERS FOR I = 1 1 N

           LET NU = <value>
           LET ALPHA = <value>
           LET Y = SKEWED T RANDOM NUMBERS FOR I = 1 1 N

           LET G = <value>
           LET H = <value>
           LET Y = G AND H RANDOM NUMBERS FOR I = 1 1 N

           LET XI = <value>
           LET LAMBDA = <value>
           LET THETA = <value>
           LET Y = GOMPERTZ-MAKEHAM RANDOM NUMBERS FOR I = 1 1 N

           LET ALPHA = <value>
           LET Y = ZIPF RANDOM NUMBERS FOR I = 1 1 N

     Random numbers for the slash and inverted beta distributions
     were added previously.

     You can generate the following probability plots and goodness
     of fit tests

           LET A = <value>
           LET B = <value>
           LET C = <value>
           LET D = <value>
           TRAPEZOID PROBABILITY PLOT Y
           TRAPEZOID KOLMOGOROV-SMIRNOV GOODNESS OF FIT TEST Y
           TRAPEZOID CHI-SQUARE GOODNESS OF FIT TEST Y

           LET A = <value>
           LET B = <value>
           LET C = <value>
           LET D = <value>
           LET NU1 = <value>
           LET NU3 = <value>
           LET ALPHA = <value>
           GENERALIZED TRAPEZOID PROBABILITY PLOT Y
           GENERALIZED TRAPEZOID KOLMOGOROV-SMIRNOV GOODNESS OF FIT TEST Y
           GENERALIZED TRAPEZOID CHI-SQUARE GOODNESS OF FIT TEST Y

           LET NU = <value>
           FOLDED T PROBABILITY PLOT Y
           FOLDED T KOLMOGOROV-SMIRNOV GOODNESS OF FIT TEST Y
           FOLDED T CHI-SQUARE GOODNESS OF FIT TEST Y

           FOLDED T PPCC PLOT Y

           LET NU = <value>
           LET LAMBDA = <value>
           SKEW T PROBABILITY PLOT Y
           SKEW T KOLMOGOROV-SMIRNOV GOODNESS OF FIT TEST Y
           SKEW T CHI-SQUARE GOODNESS OF FIT TEST Y

           SKEW T PPCC PLOT Y

           LET LAMBDA = <value>
           SKEW NORMAL PROBABILITY PLOT Y
           SKEW NORMAL KOLMOGOROV-SMIRNOV GOODNESS OF FIT TEST Y
           SKEW NORMAL CHI-SQUARE GOODNESS OF FIT TEST Y

           SKEW NORMAL PPCC PLOT Y

           LET G = <value>
           LET H = <value>
           G AND H PROBABILITY PLOT Y
           G AND H KOLMOGOROV-SMIRNOV GOODNESS OF FIT TEST Y
           G AND H CHI-SQUARE GOODNESS OF FIT TEST Y

           G AND H PPCC PLOT Y

           LET XI = <value>
           LET LAMBDA = <value>
           LET THETA = <value>
           GOMPERTZ-MAKEHAM PROBABILITY PLOT Y
           GOMPERTZ-MAKEHAM KOLMOGOROV-SMIRNOV GOODNESS OF FIT TEST Y
           GOMPERTZ-MAKEHAM CHI-SQUARE GOODNESS OF FIT TEST Y

    c) Added the following commands

          JOHNSON SU MOMENTS Y
          JOHNSON SB MOMENTS Y

       to compute method of moment estimates for the Johnson SU
       and Johnson SB distributions.

    d) The GUMBEL MAXIMUM LIKELIHOOD command was extended to support
       both the minimum and maximum cases (the previous version was
       restricted to the maximum case).  Before the GUMBEL MAXIMUM
       LIKELIHOOD command, enter the command

          SET MINMAX 1

       to specify the minimum case and 

          SET MINMAX 2

       to specify the maximum case.

    e) Enter the following command to generate Dirichelet random numbers:

          LET M = DIRICHLET RANDOM NUMBERS ALPHA N

       with ALPHA denoting a vector containing the shape parameters of
       the Dirichlet distribution and N denoting a scalar that specifies
       the number of rows to generate.  M will be a matrix with N rows
       and k columns (where k is the number of elements in the ALPHA
       vector).

       You can also compute the Dirichlet probability density or the
       log of the Dirichlet probability density with the commands

          LET M = DIRICHLET PDF X ALPHA
          LET M = DIRICHLET LOG PDF X ALPHA

    f) Enter the following command to generate correlated uniform
       random numbers:

          LET U = MULTIVARIATE UNIFORM RANDOM NUMBERS SIGMA N

       with SIGMA denoting the variance-covariance matrix of
       a multivariate normal distribution and N denoting the number
       of rows to generate.

    g) The Anderson-Darling goodnes of fit test was enhanced to
       include the following distributions:

          ANDERSON-DARLING LOGISTIC TEST Y
          ANDERSON-DARLING DOUBLE EXPONENTIAL TEST Y
          ANDERSON-DARLING UNIFORM TEST Y

       The uniform case is for the uniform distribution on the
       (0,1) interval.  This can also be used for fully specified
       distributions (i.e., the shape, location, and scale
       parameters are not estimated from the data).  Simply
       calculate the appropriate CDF function with the specified
       shape, location, and scale parameters (this converts the
       data to the (0,1) interval) and apply the test for a
       uniform distribution.
 
    h) The following maximum likelihood estimation commands were
       added:

          LOGISTIC MAXIMUM LIKELIHOOD Y
          UNIFORM MAXIMUM LIKELIHOOD Y
          BETA MAXIMUM LIKELIHOOD Y

       The BETA and UNIFORM cases generate both method of moments and
       maximum likelihood estimates.

       The beta case estimates the lower and upper limits of the
       data from the minimum and maximam data values, respectively,
       and then computes the maximum likelihood estimates for the
       alpha and beta shape parameters.

    i) Support was added for the following random number
       generators:

       1) FIBONACCI CONGRUENTIAL - a mixture of the Fibonnaci generator
                                   with a congruential generator

       2) MERSENNE TWISTER - Fortran 90 implementation of the
                             Mersenned twister generator (may not be
                             valid on platforms that are compiled
                             with Fortran 77 compilers)

       Enter HELP RANDOM NUMBER GENERATOR for details.

    j) Fixed the inverse gaussian and reciprocal inverse gaussian
       probability functions.  The MU parameter was treated as a
       location parameter in original implementation.  However, it
       is really a shape parameter.  So IGPDF and RIGPDF can now be
       called via

           IGPDF(X,GAMMA,MU,LOC,SCALE)
           RIGPDF(X,GAMMA,MU,LOC,SCALE)

       The MU parameter is treated as an optional parameter (LOC and
       SCALE are also optional).  MU is set to 1 if it is omitted.

       The MU parameter can also be specified for random numbers
       and probability plots.  If the MU parameter is not set, it
       will automatically be set to 1 (no error message is printed).
       The PPCC plot for these two distributions is now generated for
       both the gamma and mu parameters (i.e., a 3D plot is generated).
       If you want the PPCC plot assuming MU =1 for the inverse
       gaussian case, you can use the WALD PPCC PLOT command (the
       Wald distribution is a special case of the inverse gaussian
       where MU is set to 1).

 4) Added the following analysis commands:

    a) Support for linear and quadratic calibration is available via
       the following commands:

          LINEAR CALIBRATION Y X Y0
          QUADRATIC CALIBRATION Y X Y0

       The LINEAR CALIBRATION command performs a linear calibration
       analysis using eight different methods.  The QUADRATIC
       CALIBRATION command performs a quadratic calibration analysis
       using three different methods.

       Enter HELP CALIBRATION for details.

    b) The Friedman test for two-way analysis of variance on ranks
       is supported with the command

          FRIEDMAN TEST Y BLOCK TREATMENT

       Enter 

          HELP FRIEDMAN TEST

       for details.

    c) The frequency and cumulative sum tests for randomness are
       supported with the commands

          FREQUENCY TEST Y

          LET M = <value for block size>
          FREQUENCY WITHIN A BLOCK TEST Y

          CUMULATIVE SUM TEST Y

       These tests are used for sequences of 0's and 1's (Dataplot
       just checks for two distinct values, the higher value is
       set to 1 and the lower value is set to 0).

       To test a uniform random number generator, do something like
       the following:

          LET N = 1
          LET P = 0.5
          LET Y = BINOMIAL RANDOM NUMBERS FOR I = 1 1 10000
          FREQUENCY TEST Y

       For details, enter

          HELP FREQUENCY TEST
          HELP CUMULATIVE SUM TEST

 5) The following enhancements were made to the BOOTSTRAP PLOT command

    a) Extended the grouped case to handle two groups (previously
       one group was supported).

    b) For the grouped (either one or two groups), the following
       information is written to file:

       DPST1F.DAT - the full set of bootstrap estimates for the
                    statistic (group-id in column 1, bootstrap
                    statistic in column 2)

       DPST2F.DAT - writes the group-id and the corresponding mean,
                    standard deviation, and the 0.025, 0.975, 0.05,
                    0.95, 0.0005, and 0.995 quantiles

    c) Added the following form of the command

          BCA BOOTSTRAP <stat> PLOT Y

       This generates BCa bootstrap confidence intervals as defined
       by Efron.  At the expense of additional computation, it
       generates bootstrap confidence intervals that are second order
       accurate (the percentile bootstrap confidence intervals are
       first order accurate).

       Enter HELP BOOTSTRAP PLOT for further information.

 6) The CAPTURE HTML (for generating Dataplot output in HTML format)
    capability has been extended to additional analysis commands.

    In addition, Dataplot output can now be generated in Latex format
    with the command

        CAPTURE LATEX  file.tex

    with "file.tex" denoting the name where the Latex output is
    generated.  An END OF CAPTURE terminates the generation of
    Latex output.

    The CAPTURE HTML and CAPTURE LATEX commands now generate formatted
    output for the following commands:

        SUMMARY
        TABULATE
        CROSS TABULATE

        CONSENSUS MEAN
        CONSENSUS MEAN PLOT
        LINEAR CALIBRATION
        QUADRATIC CALIBRATION

        YATES ANALYSIS
        FIT
        ANOVA
        FRIEDMAN TEST

        WILK SHAPIRO
        ANDERSON DARLING
        KOLMOGOROV-SMIRNOV GOODNESS OF FIT
        CHI-SQUARE GOODNESS OF FIT
        EXPONENTIAL MAXIMUM LIKELIHOOD
        GUMBEL MAXIMUM LIKELIHOOD
        WEIBULL MAXIMUM LIKELIHOOD
        LOGISTIC MAXIMUM LIKELIHOOD
        PARETO MAXIMUM LIKELIHOOD
        UNIFORM MAXIMUM LIKELIHOOD
        BETA MAXIMUM LIKELIHOOD

        CONFIDENCE LIMITS
        DIFFERENCE OF MEANS CONFIDENCE LIMITS
        BIWEIGHT LOCATION CONFIDENCE LIMITS
        TRIMMED MEAN CONFIDENCE LIMITS
        MEDIAN/QUANTILE CONFIDENCE LIMITS
        T TEST
        F TEST
        CHI-SQUARE TEST
        GRUBB TEST
        LEVENE TEST

        FREQUENCY TEST
        FREQUENCY WITHIN A BLOCK TEST
        CUSUM TEST

    In addition, WRITE HTML and WRITE LATEX commands have been added
    to allow the generation of one-way tables.

    We plan to implement this capability for most of the analysis
    commands over the course of the next year or so.  In addition,
    we are investigating a similar capability for Rich Text
    Format (RTF), which would allow importation into Word and
    other word processing programs.

    Output from unsupported commands is enclosed in "<PRE>" and
    "</PRE>" tags for HTML and within the "begin{\verbatin}"
    environment for Latex.  Enter 

        HELP HTML
        HELP LATEX

    for details.

 7) Dataplot has previously supported a LET ... = DERIVATIVE ...
    command that generates analytic derivatives.  However, this was
    supported for a rather limited set of functions (enter
    HELP DERIVATIVE for details).  We have added the commands

       LET A = NUMERICAL DERIVATIVE F WRT X FOR X = X0
       LET Y = NUMERICAL DERIVATIVE F WRT X

    to compute derivatives numerically.  The distinction in the
    above syntax is that the first command computes a single
    derivative while the second syntax computes the derivative
    for a vector of values (define X to contain the points at
    which you want the derivative computed).  For details, enter

        HELP NUMERICAL DERIVATIVE f

 8) Fixed following bugs:

    a) Fixed the READ and WRITE commands to handle hyphens inside
       of quoted file names correctly (only applies if
       SET FILE NAME QUOTE ON entered).

    b) The substitution character, "^", was modified to treat
       anything other than a letter, a number, or an underscore
       as terminator for the Dataplot name.  Note that although you
       can use some special characters in Dataplot names, this
       is strongly discouraged.

    c) Fixed a bug where the file name restriction of 80 characters
       was actually a restriction on the entire command line.  This
       has been fixed so that file name may be up to 80 characters
       and the full command line may be more than 80 characters.

    d) Fixed a bug with the CAPTURE FLUSH command.

    e) If an improper format is given on the SET WRITE FORMAT,
       Dataplot will now return an error message rather than
       crashing.

    f) Fixed a bug in the generation of non-central chi-square,
       non-central F, and doubly non-central F random numbers.

-----------------------------------------------------------------------
The following enhancements were made to DATAPLOT April-May        2003.
-----------------------------------------------------------------------

 1) Added the following plot commands

       PARALLEL COORDINATES PLOT Y1 ... YK

    The parallel coordinates plot is a technique for plotting
    multivariate data.  Enter HELP PARALLEL COORDINATES PLOT
    for details.

 2) Added support for the following statistics:

        LET A = SN SCALE Y1
        LET A = QN SCALE Y1
        LET A = DIFFERENCE OF SN Y1 Y2
        LET A = DIFFERENCE OF QN Y1 Y2

        LET P1 = 10
        LET P2 = 10

    Enter HELP for the given statistic for details (e.g.,
    HELP DIFFERENCE OF SN).

    In addition, these statistics are supported for the following
    plots and commands

       <STAT> STATISTIC PLOT Y1 Y2 X
       CROSS TABULATE <STAT> STATISTIC PLOT Y1 Y2 X1 X2
       BOOTSTRAP <STAT> PLOT Y1 Y2 X1 X2
       JACKNIFE <STAT> PLOT Y1 Y2 X1 X2

       TABULATE <STAT> Y1 Y2 X
       CROSS TABULATE <STAT> Y1 Y2 X1 X2
       LET Z = CROSS TABULATE <STAT> Y1 Y2 X1 X2

    The DIFFERENCE OF COUNTS statistic is not supported for these
    plots and commands (since it will simply be zero for all
    cases).
 
    The SN SCALE and QN SCALE statistics are also supported for
    the following additional commands

       DEX <STAT> PLOT Y X1 ... XK
       <STAT> BLOCK PLOT Y X1 ... XK
       <STAT> INFLUENCE CURVE Y
       <STAT> INTERACTION PLOT Y X1 X2
       LET Y = MATRIX COLUMN <STAT> M 
       LET Y = MATRIX ROW <STAT> M 

 3) The following probability distribution commands were added:

    a) The following commands for multivariate random numbers
       were added:

          LET W = WISHART RANDOM NUMBERS MU SIGMA N
          LET U = INDEPENDENT UNIFORM RANDOM NUMBERS LOWL UPPL NP
          LET M = MULTIVARIATE T RANDOM NUMBERS MU SIGMA NU N
          LET M = MULTINOMIAL RANDOM NUMBERS P N NEVENTS

       For details, enter

          HELP WISHART RANDOM NUMBERS
          HELP INDEPENDENT UNIFORM RANDOM NUMBERS
          HELP MULTIVARIATE T RANDOM NUMBERS
          HELP MULTINOMIAL RANDOM NUMBERS

     b) The following multivariate cumulative distribution and
        probability density/mass function commands were added:

           LET M = MULTIVARIATE NORMAL CDF SIGMA UPPL
           LET M = MULTIVARIATE NORMAL CDF SIGMA LOWL UPPL
           LET M = MULTIVARIATE T CDF SIGMA UPPL
           LET M = MULTIVARIATE T CDF SIGMA LOWL UPPL
           LET M = MULTINOMIAL PDF X P

        These compute the cdf for multivariate normal and
        multivariate t distributions and the pdf for the multinomial
        distribution.  For details, enter

           HELP MULTIVARIATE NORMAL CDF
           HELP MULTIVARIATE T CDF
           HELP MULTINOMIAL PDF
           
     c) Support for the following univariate distributions was
        added:

           LET A = LANCDF(X)       - cdf of Landau distribution
           LET A = LANPDF(X)       - pdf of Landau distribution
           LET A = LANPPF(P)       - ppf of Landau distribution
           LET A = LANDIF(X)       - derivative of Landau pdf
           LET A = LANXM1(X)       - first moment function of
                                     Landau distribution
           LET A = LANXM2(X)       - second moment function of
                                     Landau distribution

           LET A = ERRCDF(X,ALPHA) - cdf of error distribution
           LET A = ERRPDF(X,ALPHA) - pdf of error distribution
           LET A = ERRPPF(X,ALPHA) - ppf of error distribution

           LET A = SLAPDF(X)       - pdf of slash distribution
           LET A = IBPDF(X,ALPHA)  - pdf of inverted beta distribution

        The cdf and ppf functions for the slash and inverted
        beta distributions are still being developed.

        You can generate random numbers for these distributions
        with the commands

           LET Y = LANDAU RANDOM NUMBERS FOR I = 1 1 N
           LET Y = SLASH RANDOM NUMBERS FOR I = 1 1 N

           LET ALPHA = <value>
           LET Y = ERROR RANDOM NUMBERS FOR I = 1 1 N

           LET ALPHA = <value>
           LET Y = INVERTED BETA RANDOM NUMBERS FOR I = 1 1 N

        The error distribution is also referred to as the
        Subbotin, exponential power, or general error distribution.
        There are several different parameterizations of this
        distribution.  Dataplot uses the parameterization of
        Tadikamalla in "Random Sampling From the Exponential
        Power Distribution", Journal of the American Statistical
        Association, September, 1980.  Enter HELP ERRPDF for
        details.

     d) Support was added for the following random number
        generators:

 
        1) GENZ - Alan Genz generator

        2) LUXURY - based on the Marsagalia and Zaman
           borrow-and-carry generator.  Uses a code written
           by F. James and incorporating improvements by
           M. Luscher.

        Enter HELP RANDOM NUMBER GENERATOR for details.

 4) Added the following command:

       LET Y2 X2 = STACK Y1 Y2 ... YK

    This command appends the variables Y1, Y2, ..., YK into
    the single variable Y2.  In addition, X2 contains a
    group identifier variable (values corresponding to Y1 are
    set to 1, values corresponding to Y2 are set to 2, and so on).

    Many Dataplot commands (e.g., BOX PLOT, MEAN PLOT, ANOVA)
    require data be in the two-variable format (i.e., a response
    variable and a group identifier variable).  However, many
    data files will simply have each response variable in a
    separate column.  The STACK command provides a convenient
    way to generate the data in the form needed by many Dataplot
    commands.


-----------------------------------------------------------------------
The following enhancements were made to DATAPLOT January-March    2003.
-----------------------------------------------------------------------

 Note that this is the version of Dataplot included on the
 "official" e-Handbook/Dataplot CD released in 5/2003.

 1) The Windows 95/98/ME/NT/2000/XP installation now uses
    InstallShield.  This should simplify the installation of
    Dataplot on Windows platforms.

 2) A few tweaks were made to the Postscript device.

    a) Previously, Dataplot started a new page when the device
       was intialized.  It also started a new page when the first
       plot was generated.  This was to ensure that a fresh
       page was started if you were generating diagrammatic
       graphics before the first plot.  However, it caused
       a blank page to be printed for most applications.
       Dataplot now automatically keeps track so that the first
       plot will not generate the unneeded page erase.

    b) Previously, the LANDSCAPE WORDPERFECT orientation (this
       results in a landscape orientation on a portrait page)
       was supported for encapsulated Postscript, but not for
       regular Postscript.  This orientation is now supported
       for regular Postscript.

    c) Dataplot allows you to switch between the various
       orientations (LANDSCAPE, PORTRAIT, LANDSCAPE WORDPERFECT,
       SQUARE) when using Postscript.  For this reason, it sets
       the bounding box for an 11x11 inch page.

       The following command

           SET POSTSCRIPT BOUNDING BOX <FIXED/FLOAT>

       can be used to modify this behavior.  If the value is
       FLOAT (the default), the bounding box is set for an
       11x11 inch page.  If the value is set to FIXED, the
       bounding box will be set according to whatever the current
       orientation is when the device is initialized.  However,
       you should not change the orientation if FIXED is used.

       If you are simply using the Postscript output for printing,
       then you do not need to worry about this command.  However,
       it may occasionally be useful if are importing the Postscript
       output into an external program.

 3) Postscript was added to the list of devices supported by
    the CAPTURE HTML command (see 3) for the August-December 2002
    updates).

    If a DEVICE 2 CLOSE command is encountered when CAPTURE HTML
    is on and the device is set to postscript, Dataplot will first
    use Ghostscript to convert the Postscript output to JPEG.
    The JPEG file will have the same file name as the original
    postscript file, but its extension will be changed to "jpg"
    (e.g., the default name "dppl1f.dat" results in a JPEG file
    called "dppl1f.jpg").  Dataplot will add an "<img" tag in
    the generated HTML file for this JPEG file.

    A couple of points to note.

    a. This assumes that Ghostscript is installed on your
       system.  For Unix platforms, Ghostscript is launched with
       a "gs" command.  On Windows platforms, Ghostscript is
       launched with "C:\GS\GS704\GS\BIN\GSWIN32C.EXE".

       If you need to change the path for the Ghostscript command,
       enter the following command:

           SET GHOSTSCRIPT PATH  <path-name>

       For example, on my Windows system, I use

           SET GHOSTSCRIPT PATH F:\GS\GS704\GS\BIN\

       We suggest that you add this command to your Dataplot
       startup file "dplogf.tex".

    b. We suggest using either the ORIENTATION PORTRAIT or the
       ORIENTATION LANDSCAPE WORDPERFECT command to set the
       orientation.  Plots with a landscape orientation are
       rotated in the Dataplot Postscript output (in order to
       make full use of the page).  Currently, Ghostscript does
       not support a command line switch to rotate the graph.
       This means that landscape plots will be rotated vertically
       on the web page (you can use an external program, GIMP for
       example, to rotate the JPEG files if you like).

 4) Dataplot uses a vector graphics model.  However, when you want
    to incorporate Dataplot graphics into other applications, it
    is often preferrable to work with bitmapped graphics.

    Dataplot now supports the command:

        SET POSTSCRIPT CONVERT <device>

    where <device> is one of the following:

        JPEG  - for jpeg
        PDF   - for Portable Document Format (PDF)
        TIFF  - for Tiff
        PBM   - PBM Portable Bit Map Format (supports black and white)
        PGM   - PBM Portable Grey Map Format (supports grey scale)
        PPM   - Portable Pixmap Format (supports color)
        PNM   - PBM Portable Anymap Format (operates on PBM, PGM, or
                PPM formats)

    If <device> is set to one of the choices above, a DEVICE 2 CLOSE
    command is encountered, and the device is set to postscript, Dataplot
    first uses Ghostscript to convert the Postscript output to the
    requested format.  The converted file will have the same file name
    as the original postscript file, but its extension will be changed to
    "jpg", "pdf", "tif", "pbm", "pgm", "ppm", or "pnm" depending on
    the value of <device>.  For example, if <device> is "PDF", the default
    name "dppl1f.dat" results in a PDF file called "dppl1f.pdf").

    As noted above in 3), this option assumes Ghostscript is installed
    on your local system.  You can use the SET GHOSTSCRIPT PATH
    described above to set the path for Ghostscript.

    Also, as noted in 3), we suggest using either the ORIENTATION PORTRAIT
    or the ORIENTATION LANDSCAPE WORDPERFECT command to set the
    orientation.

    A few additional points:

    a. The original postscript file is not deleted.  An additional
       plot file, with a different extension, is created.

    b. The bit map formats are generally most useful when there is
       one image per file.  You can do something like the following:

          SET POSTSCRIPT CONVERT JPEG
          SET IPL1NA plot1.ps
          DEVICE 2 POSTSCRIPT
             ...  generate plot 1 ...
          DEVICE 2 CLOSE
          SET IPL1NA plot2.ps
          DEVICE 2 POSTSCRIPT
             ...  generate plot 2 ...
          DEVICE 2 CLOSE

       This will result in the files plot1.ps, plot1.jpg, plot2.ps, and
       plot2.jpg.

       The PDF files may be an exception to this.  Depending on how
       you want to use the generated plots, you can either
       create all the plots in a single PDF file or put each plot
       in a separate PDF file (using the above logic).

    c. If the CAPTURE HTML switch is on, PDF files are incorporated
       into the generated HTML file.  For PDF files, no file
       conversion is performed.  Instead, a link to the PDF file is
       added to the HTML page.

       The advantage of the PDF format over JPEG is that it is typically
       of higher quality than the JPEG file.  The disadvantage is that
       you have to link to another page to view it.

 5) The CAPTURE HTML command can be used to save Dataplot numeric
    and graphics output in an HTML page.  By default, Dataplot
    generates fairly minimal "header" and "footer" HTML code
    (basically, it sets a white background and not much else).

    If your basic purpose is to simply create a web viewable page,
    then this is sufficient.  However, many sites have specific style
    guidelines for web pages.  These can typically be incorporated into
    the "header" and "footer" of the HTML page.

    In order to provide additional flexibility to the appearance
    of the web pages created using CAPTURE HTML, Dataplot now
    supports the following two commands:

       SET HTML HEADER FILE <header-file-name>
       SET HTML FOOTER FILE <footer-file-name>

    If these commands are given, Dataplot will add the contents of
    <header-file-name> to the beginning and the contents of
    <footer-file-name> to the end of the generated HTML file.

    The Dataplot HELP directory contains the files
    "sed_header.htm" and "sed_footer.htm".  These can be used as
    examples for developing your own templates (these implement
    some NIST specific information, so they are not intended to be
    used directly by non-NIST users).

    Note that Dataplot does no error checking on these files.  We
    recommend that you view a page containing the intended header
    and footer to detect problems with your HTML code.

    Dataplot will only read 240 characters per line in these file.

 6) One current limitation in Dataplot has been that reading data
    from ASCII files was limited to a maximum of 132 columns.  The
    only way arround this was to use the SET READ FORMAT.  However,
    this did not work if the data did not have a consistent format.

    The default limit was raised to 255 columns.  To read even
    longer data lines, use the command MAXIMUM RECORD LENGTH.
    Enter HELP MAXIMUM RECORD LENGTH for details.

 7) The following commands were added:

       TRIMMED MEAN CONFIDENCE LIMITS Y
       MEDIAN CONFIDENCE LIMITS Y

    These provide confidence intervals for robust estimates of
    location.  Enter

       HELP TRIMMED MEAN CONFIDENCE LIMITS
       HELP MEDIAN CONFIDENCE LIMITS

    for details.

 8) The following plot commands were added:

       VIOLIN PLOT Y X
       SHIFT PLOT Y X

    The VIOLIN PLOT is a mix of a a box plot and a kernel density
    plot.  The shift plot is a variation of quantile-quantile or
    Tukey mean-difference plots.

    Enter HELP VIOLIN PLOT and HELP SHIFT PLOT for details.

 9) The Hotelling control chart capability was upgraded in the following
    way:

    a) A distinction is now made between phase I and phase II plots.
       The previous implementation was effectively a phase I plot.

    b) Support was added for the individual observations case.

    Enter

        HELP HOTELLING CONTROL CHART

    for details.

10) The Ljung-Box test for randomness was added.  This test is based
    on the autocorrelation plot and is commonly used in the context
    of ARIMA modeling.  Enter

         HELP LJUNG BOX TEST

    for details.

11) Added support for the following statistics:

        LET A = DIFFERENCE OF MEANS Y1 Y2
        LET A = DIFFERENCE OF MIDMEANS Y1 Y2
        LET A = DIFFERENCE OF MEIDANS Y1 Y2
        LET A = DIFFERENCE OF TRIMMED MEANS Y1 Y2
        LET A = DIFFERENCE OF WINSORIZED MEANS Y1 Y2
        LET A = DIFFERENCE OF GEOMETRIC MEANS Y1 Y2
        LET A = DIFFERENCE OF HARMONIC MEANS Y1 Y2
        LET A = DIFFERENCE OF HODGES-LEHMAN Y1 Y2
        LET A = DIFFERENCE OF BIWEIGHT LOCATIONS Y1 Y2

        LET A = DIFFERENCE OF STANDARD DEVIATIONS Y1 Y2
        LET A = DIFFERENCE OF VARIANCES Y1 Y2
        LET A = DIFFERENCE OF AAD Y1 Y2
        LET A = DIFFERENCE OF MAD Y1 Y2
        LET A = DIFFERENCE OF SN Y1 Y2
        LET A = DIFFERENCE OF QN Y1 Y2
        LET A = DIFFERENCE OF INTERQUARTILE RANGE Y1 Y2
        LET A = DIFFERENCE OF WINSORIZED SD Y1 Y2
        LET A = DIFFERENCE OF WINSORIZED VARIANCE Y1 Y2
        LET A = DIFFERENCE OF BIWEIGHT MIDVARIANCE Y1 Y2
        LET A = DIFFERENCE OF BIWEIGHT SCALE Y1 Y2
        LET A = DIFFERENCE OF PERCENTAGE BEND MIDVARIANCE Y1 Y2
        LET A = DIFFERENCE OF GEOMETRIC SD Y1 Y2
        LET A = DIFFERENCE OF RANGE Y1 Y2
        LET A = DIFFERENCE OF MIDRANGE Y1 Y2
        LET A = DIFFERENCE OF SKEWNESS Y1 Y2
        LET A = DIFFERENCE OF KURTOSIS Y1 Y2
        LET A = DIFFERENCE OF RELATIVE SD Y1 Y2
        LET A = DIFFERENCE OF COEFFICIENT OF VARIATION Y1 Y2
        LET A = DIFFERENCE OF SD OF MEAN Y1 Y2
        LET A = DIFFERENCE OF RELATIVE VARIANCE Y1 Y2
        LET A = DIFFERENCE OF VARIANCE OF MEAN Y1 Y2

        LET A = DIFFERENCE OF QUANTILE Y1 Y2
        LET A = DIFFERENCE OF MINIMUM Y1 Y2
        LET A = DIFFERENCE OF MAXIMUM Y1 Y2
        LET A = DIFFERENCE OF EXTREME Y1 Y2
        LET A = DIFFERENCE OF MAXIMUM Y1 Y2
        LET A = DIFFERENCE OF MAXIMUM Y1 Y2

        LET A = DIFFERENCE OF SUM Y1 Y2
        LET A = DIFFERENCE OF COUNTS Y1 Y2

    Enter HELP for the given statistic for details (e.g.,
    HELP DIFFERENCE OF MEANS).

    In addition, these statistics are supported for the following
    plots and commands

       <STAT> STATISTIC PLOT Y1 Y2 X
       CROSS TABULATE <STAT> STATISTIC PLOT Y1 Y2 X1 X2
       BOOTSTRAP <STAT> PLOT Y1 Y2 X1 X2
       JACKNIFE <STAT> PLOT Y1 Y2 X1 X2

       TABULATE <STAT> Y1 Y2 X
       CROSS TABULATE <STAT> Y1 Y2 X1 X2
       LET Z = CROSS TABULATE <STAT> Y1 Y2 X1 X2

    The DIFFERENCE OF COUNTS statistic is not supported for these
    plots and commands (since it will simply be zero for all
    cases).

12) The follwing miscellaneous changes were made:

    a) A correction was made in the computation of the Herrell-Davis
       quantile estimate.  Enter HELP QUANTILE for details.

    b) The SEARCH command now returns the line number that the
       first match is found on in the internal parameter
       LINENUMB.  This can occassionaly be useful when writing
       macros.

    c) If no variable name is given on the READ command, Dataplot
       will now try to automatically determine the variables.

       There are two cases:

        i) If the command SKIP AUTOMATIC was previously entered,
           Dataplot will skip all lines until a line starting
           with "----" is encountered.  It will then backup one
           line and read the variable list from that line.

           This case is primarily used when reading data files
           that come with the Dataplot distribution (i.e., the
           files in the Dataplot "DATA" sub-directory).  Most,
           though not all, of these files follow that convention.

       ii) If a SKIP AUTOMATIC command has not been entered,
           Dataplot will read the first line of the file and
           determine the number of columns of data.  It will then
           automatically name the variables X1 X2 ... XK (where
           K is the number of variables).

           Note that any SKIP, COLUMN LIMITS, or ROW LIMITS
           commands will be honored when reading the first
           line to determine the number of variables.
           
       This capability only applies when reading variables (i.e.,
       it is not supported for the READ PARAMETER, READ STRING,
       or READ MATRIX cases).  Also, it only applies when reading
       from a file, not when reading from the terminal.

    d) Some bugs were fixed.

----------------------------------------------------------------------

The following enhancements were made to DATAPLOT August-December 2002.
----------------------------------------------------------------------

 1) Added the following command:

       AUTO TEXT <ON/OFF>

    Entering AUTO TEXT ON will prepend a TEXT to all subsequent
    lines until an AUTO TEXT OFF command is encoutered.  This
    command is used in generating word slides.  Enter

        HELP AUTO TEXT

    for details.

 2) The list of supported statistics has been expanded for the
    following commands:

      BLOCK PLOT
      DEX PLOT
      TABULATE
      CROSS TABULATE
      MATRIX ROW STATISTIC
      MATRIX COLUMN STATISTIC
      CROSS TABULATE (LET)

    Enter the corresponding HELP command for a complete list
    of supported statistics.

 3) The CAPTURE command added the following option:

       CAPTURE HTML <file-name>

    This writes the output from the CAPTURE command in HTML
    format.  Note that most commands simply use a
    <PRE> ... </PRE> syntax.  Currently, the exceptions are the
    TABULATE, CROSS TABULATE, SUMMARY, CONSENSUS MEAN, and
    CONSENSUS MEAN PLOT commands.  These commands write the output
    using HTML table syntax.

    This can be used in conjunction with the WEB command.  For
    example,

       SKIP 25
       READ RIPKEN.DAT Y X1 X2
       ECHO ON
       CAPTURE HTML C:\TABLE.HTM
       TABULATE MEAN Y X1
       CROSS TABULATE MEAN Y X1 X2
       END OF CAPTURE
       WEB file://C:\TABLE.HTM

    In addition, if DEVICE 2 is set to PNG, JPEG, or SVG, Dataplot
    will incorporate the graphics into the web page using the
    IMG tag (the SVG device uses the EMBED tag).  For example,

       device 1 x11
       .
       skip 25
       read berger1.dat y x
       .
       line blank solid
       character x blank
       echo on
       capture html fit.htm
       set ipl1na data.png
       device 2 gd png
       title original data
       plot y x
       device 2 close
       fit y x
       set ipl1na pred.png
       device 2 gd png
       title predicted line
       plot y pred vs x
       device 2 close
       end of capture
       .
       web file:///home/heckert/dataplot/solaris/fit.htm

 4) The maximum number of lines in a loop was raised from 500 to
    1,000.

 5) For the command line version of Dataplot under Windows (i.e.,
    the version built with the Microsoft/Visual Fortran compiler),
    we added the following enahncements.

    a) By default, the text window and graphics window overlap.
       Some users prefer a "tiled" mode where there is no overlap
       of these windows.  You can specify the option "-tile" on
       the command line to request this mode.  For example, to
       specify true color, with a "large" window, and tiled mode,
       execute Dataplot with the following command:

          C:\DATAPLOT\DATAPLOT.EXE  -true  -large  -tile

       (this is normally set by right clicking on the Dataplot
       shortcut to bring up the properties menu).

    b) The title for the text window now says "Dataplot Text
       Window".

    c) For the command line version, the menus accessible from
       the frame window are simply the default menus put up
       by the Visual Fortran compiler.  We have now modified
       the menus under the "Help" menu to access the Dataplot
       help rather than help for the Visual Fortran compiler.

    d) The following command was added:

          SET QWIN SYSTEM  <SYSTEMQQ/WINEXEC>

       This commands allows two options for how Dataplot
       executes operating system commands (e.g., the SYSTEM
       command, the WEB HELP, and several others).

       By default, the SYSTEMQQ library routine is used.  The
       primary drawback of this option is that control does
       not return to Dataplot until the operating system
       command completes execution.

       Alternatively, the WINEXEC command can be used.  The
       advantage of this command is that control passes back
       to Dataplot after the command is issued (i.e., it doesn't
       wait for the command to complete).  However, the drawback
       is that DOS type commands do not work.

       The basic recommendation is that if you want to execute
       a Windows application (e.g., the browser, notepad, Word,
       etc.), then set this option to WINEXEC.  However, to issue
       a DOS type command via the SYSTEM command, you should set
       this option to SYSTEMQQ (the default).

       In particular, I recommend setting the WINEXEC comamnd
       before using the WEB HELP (or any other command starting
       with WEB).  This allows you to enter additional Dataplot
       commands without exiting the browser.

 6) Under Windows, printing Dataplot graphs has been a bit of an
    issue.  Dataplot supports Postscript printers and printers
    that provide HPGL emulation (i.e., most HP LaserJet printers)
    directly.  However, there is no built-in support for the
    various ink jet and desk jet printers.  The recommended
    solution has been to install the freely downloadable
    Ghostview/Ghostscript software and use this to print Dataplot
    generated Postscript files on non-Postscript printers.  The
    limitations of this approach were that the PP command (which
    prints the most recent graph from within a Dataplot session)
    was not available and you had to run a separate program to print
    your graphs.

    Ghostview also provides the command "GSPRINT" that will
    run ghostscript automatically to print a Postscript file
    on a generic Windows printer.  We have added the following
    command to Dataplot:

       SET GHOSTSCRIPT PRINTER <ON/OFF>

    If you want the PP command to print to either a Postscript
    printer or a printer with HP-GL emulation, then it is
    recommended that you leave this switch set to OFF.  For all
    other printers, we recommend you set this switch to ON.

    Enter the command

        HELP GHOSTSCRIPT PRINTER

    for details.

 7) Following bug fixes were made.

    a) A bug was fixed with the TO syntax with the SERIAL READ
       command (e.g., SERIAL READ FILE.DAT X1 TO X5).

----------------------------------------------------------------------
The following enhancements were made to DATAPLOT April-July      2002.
----------------------------------------------------------------------

 1) Added support for the following probability distribution
    functions.

    a) Two-Sided Power

       TSPCDF(X,THETA,N)
       TSPPDF(X,THETA,N)
       TSPPPF(X,THETA,N)

       LET THETA = <value>
       LET N = <value>
       LET Y = TWO-SIDED POWER RANDOM NUMBERS FOR I = 1 1 100

       LET THETA = <value>
       LET N = <value>
       TWO-SIDED POWER PROBABILITY PLOT Y

       TWO-SIDED POWER PPCC PLOT Y

       LET THETA = <value>
       LET N = <value>
       CHI-SQUARE TWO-SIDED POWER GOODNESS OF FIT TEST Y

       LET THETA = <value>
       LET N = <value>
       KOLMOGOROV-SMIRNOV TWO-SIDED POWER GOODNESS OF FIT TEST Y

       LET A = <lower limit>
       LET B = <upper limit>
       TWO-SIDED POWER MAXIMUM LIKELIHOOD Y

       Note: The MLE estimator assumes that the value of the lower
       and upper limits (default to 0 and 1) are known and fixed.
       It returns estimates for THETA and N.

    b) Bi-Weibull

       BWECDF(X,SCALE1,GAMMA1,LOC2,SCALE2,GAMMA2)
       BWEPDF(X,SCALE1,GAMMA1,LOC2,SCALE2,GAMMA2)
       BWEPPF(P,SCALE1,GAMMA1,LOC2,SCALE2,GAMMA2)
       BWEHAZ(X,SCALE1,GAMMA1,LOC2,SCALE2,GAMMA2)
       BWECHAZ(X,SCALE1,GAMMA1,LOC2,SCALE2,GAMMA2)

       LET SCALE1 = <value>
       LET GAMMA1 = <value>
       LET LOC2 = <value>
       LET SCALE2 = <value>
       LET GAMMA2 = <value>
       LET Y = BIWEIBULL RANDOM NUMBERS FOR I = 1 1 100

       LET SCALE1 = <value>
       LET GAMMA1 = <value>
       LET LOC2 = <value>
       LET SCALE2 = <value>
       LET GAMMA2 = <value>
       BIWEIBULL PROBABILITY PLOT Y

       LET SCALE1 = <value>
       LET GAMMA1 = <value>
       LET LOC2 = <value>
       LET SCALE2 = <value>
       LET GAMMA2 = <value>
       CHI-SQUARE BIWEIBULL GOODNESS OF FIT TEST Y

       LET SCALE1 = <value>
       LET GAMMA1 = <value>
       LET LOC2 = <value>
       LET SCALE2 = <value>
       LET GAMMA2 = <value>
       KOMOGOROV-SMIRNOV BIWEIBULL GOODNESS OF FIT TEST Y

    c) Multivariate normal distribution

       LET MU = DATA <list of p means>
       READ MATRIX SIGMA
          <pxp set of values>
       END OF DATA
       LET N = <value>
       LET M = MULTIVARIATE NORMAL RANDOM NUMBERS MU SIGMA N

       Note that M will be an NxP matrix.  N is the number of rows
       generated for each component and their are P components to
       the multivariate normal.  SIGMA is the pxp variance-covariance
       matrix of the multivariate normal.  SIGMA will be checked to
       ensure that it is a positive definite matrix.  MU is a vector
       specifying the means of the p components.
   
       This command utilizes a code written by Charlie Reeves when
       he was a member of the NIST Statistical Engineering Division.

    d) Logarithmic series distribution

       Added randon number generation for this distribution.  For
       example, 

            LET THETA = 0.7
            LET Y = LOGARITHMIC SERIES RANDOM NUMBERS FOR I = 1 1 500

       The cdf, pdf, and ppf functions are already available for
       this distribution.

 2) Made the following updates to the FIT command:

    a) Added the command:

         SET FIT ADDITIVE CONSTANT <ON/OFF>

       If OFF, then Dataplot does not include a constant term
       in a multi-linear fit (i.e., FIT Y X1 X2 ...).  The
       default is to include the additive constant.

    b) If Dataplot detects a singularity in a multi-linear fit,
       it now prints an error message.  Previously, it simply
       set all the parameter estimates to 0 and terminated the
       fit.  In addition, Dataplot explictly checks for two
       types of singularities: a column that contains all the same
       values (this essentially adds an addtional constant term) and
       for two columns being equal.

    c) Added the command:

          LET M = CREATE MATRIX X1 ... XK

       where X1 ... XK designates a list of previously defined
       variables.

       This command has a similar function as the MATRIX DEFINITION
       command.  However, the MATRIX DEFINITION command
       creates matrices from variables that are contiguous
       (the order of variables is determined by the order
       in which they were created in Dataplot).  The
       CREATE MATRIX command does not have this restriction.
       The variables need not be contiguous.
       

       This command is useful for creating a design matrix
       in regression problems that can be used as input for
       some of the new commands that follow.

    d) Added the command:

          LET C = CATCHER MATRIX X

       This computes the catcher matrix, X*(X'X)**(-1).  This
       matrix is used in the computation of certain regression
       diagnostics (e.g., Variance Inflation Factors, Partial
       Regression Plots).  This command greatly simplifies the
       writing of macros to generate these regression diagnostics
       (and allows larger design matrices to be used).  Enter
       HELP CATCHER MATRIX for details.

    e) Added the command:

          LET XTXINV = XTXINV MATRIX X

       This computes the matrix (X'X)**(-1).  This
       matrix is used in the computation of certain regression
       diagnostics (e.g., DFBETA statistic) and in computing
       certain confidence and prediction intervals for multi-linear
       fits.  This command simplifies the writing of macros to
       generate these regression diagnostics and intervals
       (and allows larger design matrices to be used).  Enter
       HELP XTXINV MATRIX for details.

    f) Added the command:

          LET C = CONDITION INDICES X

       where X is the design matrix for a multi-linear fit
       (note that you need to create the indpendent variables,
       including a column containing all 1's, as a matrix).

       The condition indices provide a measure of colinearity
       in the design matrix.  Enter HELP CONDITION INDICES for
       details.

    g) Added the command:

          LET VIF = VARIANCE INFLATION FACTORS X

       where X is the design matrix for a multi-linear fit
       (note that you need to create the indpendent variables,
       including a column containing all 1's, as a matrix).

       The variance inflation factors provide a measure of
       colinearity in the design matrix.  Enter
       HELP VARIANCE INFLATION FACTORS for details.

    h) Added the following plot commands:

         PARTIAL REGRESSION PLOT Y X1 ... XK  XI
         PARTIAL RESIDUAL PLOT Y X1 ... XK  XI
         PARTIAL LEVERAGE PLOT Y X1 ... XK  XI
         CCPR PLOT Y X1 ... XK  XI

         MATRIX PARTIAL REGRESSION PLOT Y X1 ... XK
         MATRIX PARTIAL RESIDUAL PLOT Y X1 ... XK
         MATRIX PARTIAL LEVERAGE PLOT Y X1 ... XK
         MATRIX CCPR PLOT Y X1 ... XK  XI

       These generate partial regression plots, partial residual
       plots, partial leverage plots, and component and
       component-plus-residual (CCPR) plots for a multi-linear fit.
       These plots are typically used to assess the effect of
       a variable on the fit given the effect of other variables
       already included in the fit.

       There are 2 forms for the command.

       In the first form, a single plot is generated.  In this case,
       the last variable listed is the "primary" variable.  That is,
       this is the variable we are considering adding/deleting from
       the fit.  Note that this variable should already be listed.
       That is, a fit of Y versus X1 to XK is performed (including XI),
       then the plot assesses the effect of XI on the fit.

       In the second form, a multiplot is generated where each
       of the indpendent variables is used as the primary variable.

       Enter

          HELP PARTIAL REGRESSION PLOT
          HELP PARTIAL RESIDUAL PLOT
          HELP PARTIAL LEVERAGE PLOT
          HELP CCPR PLOT

       for details.

    i) For multi-linear fits, the output for DPST2F.DAT was
       enhanced to include Bonferroni and Hotelling joint
       confidence limits, respectively, for the predicted values.  
       By default, a 95% interval is generated.  To use a different
       alpha value, enter the following command before the fit:

            LET ALPHA = 0.90

       In addition, the output for DPST1F.DAT now includes
       the t critical value and lower and upper joint Bonferroni
       confidence limits for the parameters.  The format 5E15.7
       is used in writing these values.

       In addition, for multi-linear fits, the regression ANOVA
       table is written to the file DPST5F.DAT.  In addition, the
       values for R**2, adjusted R**2, and the Press P statistic are
       also printed to this file.  Theses three statistics are
       saved as the internal parameters RSQUARE, ADJRSQUA, and PRESSP,
       respectively.

    j) One weakness in the Dataplot multi-linear fit routine
       has been the lack of any "forward selection/backward
       selection/best subsets" capabilities.

       The command

          BEST CP Y X1 ... XK

      was added to identify the best candidate models using
      the Mallow's CP criterion.  Enter HELP BEST CP for details.

    k) Added the command:

          BOOTSTRAP FIT Y X1 .... XK

       This performs a bootstrap linear/multilinear fit.  Bootstrap
       linear fits are an alternative to weighting and transformation
       when the assumptions for multilinear fitting are not
       satisfied (that is, the errors from the fit are independent and
       have a common distribution, typically assumed to be normal, with
       common location and scale).  Enter HELP BOOTSTRAP FIT for
       details.

 3) Added support for alternative random number generators.  Note
    that the default generator (i.e., the one that has been in
    Dataplot for many years) is based on Fibonacci sequence as
    defined by Marsagalia.  Note that this is equivalent to the
    generator UNI of Jim Blue, David Kahaner, and George Marsagalia
    that is in the CMLIB library.

    Support is now provided for a linear congruential generator
    written by Fullerton (CMLIB routine RUNIF) and a multiplicative
    congruential generator (ACM algorithm 599).  In addition,
    2 generators based on the generalized feedback shift
    register (GFSR) methods are supported.  The first is based on the
    original algorithm of Lewis and Payne (Journal of the ACM,
    Volume 20, pp. 456-468).  The second is an alternative
    implementation given by Fushimi and Tezuka (Journal of the
    ACM, Volume 26, pp. 516-523).  Both are based on codes
    given by Monohan (2000) in "Numerical Methods of Statistics".
    Support is also provided for the Applied Statistics algorithm
    183.  AS183 is based on the fractional part of the sum of 3
    multiplicative congruential generators.  It requires 3 integers
    be specified initially.  Dataplot uses the multiplicative
    congruenetial generator (which does depend on the SEED command)
    to randomly generate these 3 integers.

    These 6 generators are used to generate uniform random numbers.
    Random numbers for other distributions are then derived from
    these uniform random numbers.

    To specify the uniform random number generator, use the command

        SET RANDOM NUMBER GENERATOR FIBONACCI
        SET RANDOM NUMBER GENERATOR LINEAR CONGRUENTIAL
        SET RANDOM NUMBER GENERATOR MULTIPLICATIVE CONGRUENTIAL
        SET RANDOM NUMBER GENERATOR GFSR
        SET RANDOM NUMBER GENERATOR FUSHIMI
        SET RANDOM NUMBER GENERATOR AS183

    Note that you can use the SEED command to change the random numbers
    generated as well.  The SEED does not apply to the 2 GFSR
    generators (these each have their own initialization routines).

 4) Added support for the following special functions.

    a) Fermi-Dirac function

       FERMDIRA(X,ORDER)

       where ORDER is the order of the function.  ORDER can be
       -0.5, 0.5, 1.5, or 2.5 (Dataplot uses an epsilon of 0.1,
       any order not within epsilon of one of the above values
       results in an error.  Enter HELP FERMDIRA for details.

 5) Added support for the following statistics:

       LET A = WINSORIZED VARIANCE Y
       LET A = WINSORIZED SD Y
       LET A = WINSORIZED COVARIANCE Y X
       LET A = WINSORIZED CORRELATION Y X
       LET A = BIWEIGHT MIDVARIANCE Y X
       LET A = BIWEIGHT MIDCOVARIANCE Y X
       LET A = BIWEIGHT MIDCORRELATION Y X
       LET A = PERCENTAGE BEND MIDVARIANCE Y
       LET A = PERCENTAGE BEND CORRELATION Y1 Y2
       LET A = HODGES LEHMAN Y
       LET A = TRIMMED MEAN STANDARD ERROR
       LET A = <XQ> QUANTILE Y
       LET A = <XQ> QUANTILE STANDARD ERROR Y

    Enter

        HELP WINSORIZED VARIANCE
        HELP WINSORIZED SD
        HELP WINSORIZED COVARIANCE
        HELP WINSORIZED CORRELATION
        HELP BIWEIGHT MIDVARIANCE
        HELP BIWEIGHT MIDCOVARIANCE
        HELP BIWEIGHT MIDCORRELATION
        HELP PERCENTAGE BEND MIDVARIANCE
        HELP PERCENTAGE BEND CORRELATION
        HELP HODGES LEHMAN
        HELP TRIMMED MEAN STANDARD ERROR
        HELP QUANTILE
        HELP QUANTILE STANDARD ERROR

    for details.

 6) Added the following plot:

      <stat> INFLUENCE CURVE Y XSEQ

    where <stat> is one of the built-in supported statistics,
    Y is a response variable, and XSEQ is a sequence of x values.

    The plot is generated by looping through the values in XSEQ.
    For a given value of XSEQ, the value of <stat> is computed for
    that value of XSEQ along with the values in Y.  The vertical
    axis of the plot contains the computed statistic while the
    horizontal axis contains the value of XSEQ.

    This plot is of interest in the field of robust statistics.
    For details, enter HELP INFLUENCE CURVE.

 7) For the ANOVA command, the residual standard deviations for
    various models are written to the file DPST3F.DAT (these are
    the same values that appear in the fitted output).  This
    allows these values to be read back in as a variable, which
    is occassionally useful in writing macros that involve an
    ANOVA step.

 8) The PROBE command now recognizes the following:

        PROBE IDMAN(1)
        PROBE IDMAN(2)
        PROBE IDMAN(3)

    This identifies the current manufacturer for devices 1, 2, and
    3 respectively.  In addition, the value of PROBEVAL is set
    if the returned manufacturer is one of the following:

          X11           = 1
          QWIN          = 2
          REGI          = 3
          TEKT          = 4
          OPGL          = 5
          QUAR or MACI  = 6
          POST or PS    = 7
          HP or HPGL    = 8
          GENE          = 9
          GD            = 10
          QUIC          = 11
          CALC          = 12
          ZETA          = 13
          GKS           = 14
          LAHE          = 15
          PRIN          = 16
          LATE          = 17
          SVG           = 18
          DISC          = 19

    In addition, the device model can be extracted via the commands

        PROBE IDMOD(1)
        PROBE IDMOD(2)
        PROBE IDMOD(3)

        PROBE IDMO2(1)
        PROBE IDMO2(2)
        PROBE IDMO2(3)

        PROBE IDMO3(1)
        PROBE IDMO3(2)
        PROBE IDMO3(3)

    The following PROBE commands were added to return the
    operating system and compiler, respectively.

        PROBE IOPSY1
        PROBE ICOMPI

    For IOPSY1, the value of PROBEVAL is also set:

        UNIX   = 1    (Unix)
        PC-D   = 2    (Windows)
        VMS    = 3    (VAX/VMS)
        other  = 0

    For ICOMPI, the value of PROBEVAL is also set:

        f77    = 1    (the Unix Fortran compiler)
        MS-F   = 2    (the Microsoft, now Compaq, Fortran compiler)
        LAHE   = 3    (the Lahey Fortran compiler)
        other  = 0

    In general, if the PROBE command returns a string value of ON,
    OPENED, or YES, it sets the value of the PROBEVAL parameter to 1.
    Similarly, if the PROBE command returns a string value of OFF,
    CLOSED, or NO, it sets the value of the PROBEVAL parameter to 0.

    The above uses of PROBE are primarily of value in writing
    general purpose macros.  In particular, macros that are intended
    to be used by others.

 9) The following command was added:

        CAPTURE FLUSH

    The purpose of this command is to allow Dataplot text output
    to be written to the graphics output file.  This can be useful
    when you are writing a macro and you want the analytic output
    (for example, the output from a fit) to be included with the
    graphics output.  The following shows a sample of how this
    command is used:

       device 1 x11
       device 2 postscript
       .
       title automatic
       skip 25
       read gear.dat y x
       .
       mean plot y x
       .
       move 5 95
       margin 5
       capture junk.dat
       tabulate mean y x
       capture flush
       end of capture
       .
       device 2 close
       system lpr dppl1f.dat

    The initial CAPTURE command directs text output to the
    file "junk.dat".  When the CAPTURE FLUSH command is
    encountered, the capture file is closed, an ERASE command
    is generated for the  graphics devices, the contents of
    the capture file are printed on the graphics devices using
    the TEXT command (i.e., each line of the file generates a
    distinct TEXT command), and then the capture file is re-opened
    (it will start at the beginning).

    Since the lines are generated with the TEXT command, the
    appearance of the text can be controlled with the various
    TEXT attribute commands.  Also, it is recommended that
    CRLF be set to ON (the default), a MOVE command be given to
    set the position for the first line of the text, and a MARGIN
    command be entered to set the beginning x-coordinate for the
    line.

    Some output may be too long to display on one page.  You
    can control the number of lines printed per page with the
    following command:

        SET CAPTURE LINES <value1>  ... <value5>

    Up to 5 values may be entered.  The first value is for the
    first page of output, the second value is for the second
    page of output, and so on.  If more than 5 values are
    entered, then the page limits start over (i.e., page 6 uses
    the value for page 1, page 7 uses the value for page 2, and
    so on).  The default is 25 lines for all pages.

    If the MULTIPLOT switch is ON, the initial page erase is
    suppressed.  The following example shows how this feature
    can be used:

       .
       device 1 x11
       device 2 ps
       device 1 font simplex
       .
       title automatic
       skip 25
       read gear.dat y x
       .
       multiplot 2 2
       multiplot corner coordinates 0 0 100 100
       multiplot scale factor 2
       .
       mean plot y x
       sd plot y x
       .
       move 5 98
       margin 5
       plot
       capture junk.dat
       tabulate mean y x
       capture flush
       end of capture
       move 5 98
       plot
       capture junk.dat
       tabulate sd y x
       capture flush
       end of capture
       .
       end of multiplot
       .

    Note that the null PLOT command is used to move to the
    next plot area without actually generating a plot.

    This example draws a mean and standard deviation plot
    on the first row and then suplements that with the numeric
    values generated using the TABULATE command on the second
    row.

    The following two commands are also available.

        SET CAPTURE NUMBER  <ON/OFF>
        SET CAPTURE BOX   <ON/OFF>

    If SET CAPTURE NUMBER ON is entered, the output lines are
    numbered.  This is primarily a convenience function to help
    determine what values to enter for the SET CAPTURE LINES command
    in order to generate breaks at the appropriate spots.

    If SET CAPTURE BOX ON is entered, a box will be drawn for each
    page of the output.  Use the BOX 1 CORNER COORDINATES command,
    before the CAPTURE FLUSH, to specify the cooridinates of the
    box.  Use the various BOX attribute commands to set the
    properties of the box.

10) The following enhancements were made to the IF command:

    a) You can now test for strings with the IF command.  That is,

         LET STRING S = TEST
         IF S = TEST
            PRINT S
         ENDS OF IF
   
         LET STRING S = TEST
         IF S <> "NOT TEST"
            PRINT S
         ENDS OF IF
   
       Note that "=" and "<>" are the only comparisons allowed (i.e.,
       no "<" or ">").
   
       The argument on the left of the "=" must be the name of a
       previously defined string.  The argument to the right of the
       "=" is a literal string.  The string can be enclosed in
       dooble quotes, ", if it contains spaces.  If there are no
       double quotes, the string is assumed to end once the first
       space is encountered.
   
    b) Support was added for a ELSE and ELSE IF clauses.  For 
       example,

          IF A = 2
             PRINT "A = 2"
          ELSE
             PRINT "A NOT EQUAL 2"
          END OF IF

       or

          IF A = 2
             PRINT "A = 2"
          ELSE IF A = 1
             PRINT "A = 1"
          ELSE
             PRINT "A NOT EQUAL 2 AND A NOT EQUAL 1"
          END OF IF

    c) A bug was fixed for the IF ... NOT EXIST and IF ... EXIST
       cases.  Also, these now test whether the name exists as a
       parameter, string, variable, or matrix (previously, it only
       checked if it was a parameter).

11) One problem with reading files in Dataplot has been the
    inability to handle directory and file names with embedded
    spaces.  The command

      SET FILE NAME QUOTE <ON/OFF>

    was added to address this problem.  If ON is specified,
    then the file name may be enclosed in double quotes (").
    All text, including spaces, until the matching ending double
    quote is found are considered a part of the file name (no
    provision is made for file names containing a double quote
    character).  If OFF is specified, this feature is disabled.

    The default is OFF to accomodate quoted strings on the WRITE
    that might contain a "." (which is what Dataplot uses to
    identify a file name).  For example,

       WRITE "Example of writing a string."

    The following will work as intended:

       SET FILE NAME QUOTE ON
       WRITE "C:\ My Data\STRING.OUT"  "String to STRING.OUT"

12) Modified the output for the SIGN TEST, SIGNED RANK TEST, and
    the RANK SUM test to have better clarity.
    
13) Added the following to the BOOTSTRAP PLOT command:

     BOOTSTRAP CORRELATION PLOT Y X
     BOOTSTRAP RANK COVARIANCE PLOT Y X
     BOOTSTRAP RANK CORRELATION PLOT Y X
     BOOTSTRAP COVARIANCE PLOT Y X

     BOOTSTRAP LINEAR CALIBRATION PLOT Y X
     BOOTSTRAP QUADRATIC CALIBRATION PLOT Y X

14) Fixed several bugs.

----------------------------------------------------------------------
The following enhancements were made to DATAPLOT November-March  2002.
----------------------------------------------------------------------

 1) Added the following probability distributions.

    a) Geometric Extreme Exponential

       GEECDF(X,GAMMA)
       GEEPDF(X,GAMMA)
       GEEPPF(X,GAMMA)
       GEEHAZ(X,GAMMA)
       GEECHAZ(X,GAMMA)

       LET GAMMA = <value>
       LET Y = GEOMETRIC EXTREME EXPONENTIAL RANDOM NUMBERS FOR I = 1 1 100

       LET GAMMA = <value>
       GEOMETRIC EXTREME EXPONENTIAL PROBABILITY PLOT Y

       GEOMETRIC EXTREME EXPONENTIAL PPCC PLOT Y

       LET GAMMA = <value>
       CHI-SQUARE GEOMETRIC EXTREME EXPONENTIAL GOODNESS OF FIT TEST Y

       LET GAMMA = <value>
       KOLMOGOROV-SMIRNOV GEOMETRIC EXTREME EXPONENTIAL GOODNESS OF FIT TEST Y

    b) Johnson SB

       JSBCDF(X,ALPHA1,ALPHA2)
       JSBPDF(X,ALPHA1,ALPHA2)
       JSBPPF(X,ALPHA1,ALPHA2)

       LET ALPHA1 = <value>
       LET ALPHA2 = <value>
       LET Y = JOHNSON SB RANDOM NUMBERS FOR I = 1 1 100

       LET ALPHA1 = <value>
       LET ALPHA2 = <value>
       JOHNSON SB PROBABILITY PLOT Y

       JOHNSON SB PPCC PLOT Y

       LET ALPHA1 = <value>
       LET ALPHA2 = <value>
       CHI-SQUARE JOHNSON SB GOODNESS OF FIT TEST Y

       LET ALPHA1 = <value>
       LET ALPHA2 = <value>
       KOLMOGOROV-SMIRNOV JOHNSON SB GOODNESS OF FIT TEST Y

    c) Johnson SU

       JSUCDF(X,ALPHA1,ALPHA2)
       JSUPDF(X,ALPHA1,ALPHA2)
       JSUPPF(X,ALPHA1,ALPHA2)

       LET ALPHA1 = <value>
       LET ALPHA2 = <value>
       LET Y = JOHNSON SU RANDOM NUMBERS FOR I = 1 1 100

       LET ALPHA1 = <value>
       LET ALPHA2 = <value>
       JOHNSON SU PROBABILITY PLOT Y

       JOHNSON SU PPCC PLOT Y

       LET ALPHA1 = <value>
       LET ALPHA2 = <value>
       CHI-SQUARE JOHNSON SU GOODNESS OF FIT TEST Y

       LET ALPHA1 = <value>
       LET ALPHA2 = <value>
       KOLMOGOROV-SMIRNOV JOHNSON SU GOODNESS OF FIT TEST Y

    d) Generalized Tukey-Lambda
       Note: still being tested/developed.  In particular,
       negative values of shape parameter are not working.

       GLDCDF(X,LAMBDA3,LAMBDA4)
       GLDPDF(X,LAMBDA3,LAMBDA4)
       GLDPPF(X,LAMBDA3,LAMBDA4)

       LET LAMBDA3 = <value>
       LET LAMBDA4 = <value>
       LET Y = GENERALIZED TUKEY LAMBDA RANDOM NUMBERS FOR I = 1 1 100

       LET LAMBDA3 = <value>
       LET LAMBDA4 = <value>
       GENERALIZED TUKEY LAMBDA PROBABILITY PLOT Y

       GENERALIZED TUKEY LAMBDA PPCC PLOT Y

       LET LAMBDA3 = <value>
       LET LAMBDA4 = <value>
       CHI-SQUARE GENERALIZED TUKEY LAMBDA GOODNESS OF FIT TEST Y

       LET LAMBDA3 = <value>
       LET LAMBDA4 = <value>
       KOLMOGOROV-SMIRNOV GENERALIZED TUKEY LAMBDA GOODNESS OF FIT TEST Y

 2) Added support for the following new statistics.

    a)  LET A = BIWEIGHT LOCATION Y

    b)  LET A = BIWEIGHT SCALE Y

    For more information, enter the following commands:

        HELP BIWEIGHT LOCATION
        HELP BIWEIGHT SCALE

 3) Added support for a biweight based confidence interval:

       BIWEIGHT CONFIDENCE INTERVAL Y

    For more information, enter the following command:

        HELP BIWEIGHT CONFIDENCE INTERVAL

 4) Added the following command:

       SET BOX PLOT WIDTH  <VARIABLE/FIXED>

    This specifies whether box plots are drawn with fixed width
    or variable width boxes.  In variable width box plots, the
    width of the box is proportional to the maximum group sample
    size.  That is, the largest width is used for the box plot
    with the largest sample size.  The remaining box plots
    compute a scale factor that is the sample size of the given
    box plot relative to the maximum sample size.

    The default is variable width.  This is recommended in most cases
    as it conveys additional information regarding the relative
    sample sizes.  However, there are cases where it is desirable
    to turn this feature off (e.g., when multiple BOX PLOT commands
    are used to overlay box plots on the same page.

 5) Added the following commands:

      SET 4PLOT MULTIPLOT  <ON/OFF>
      SET 6PLOT MULTIPLOT  <ON/OFF>

    Setting these switches ON specifies that the multiplot corner
    coordinates will be used to size the 4-PLOT and 6-PLOT,
    respectively.  The default is OFF (i.e., the plot sizes are
    hard-coded to a default value).  If set to ON, then you
    can use the MULTIPLOT CORNER COORDINATES to size the
    graphs.

 6) ROBUSTNESS PLOT was added as a synonym for BLOCK PLOT.

 7) Support was added for the Scalable Vector Graphics (SVG)
    graphics output.  SVG is an XML based vector graphics format
    that is expected to become increasingly popular for web based
    applications.  SVG format files can also be imported into
    several popular graphics editing programs.  For more information,
    enter

        HELP SVG

 8) The VERSION command was re-activated.

 9) Fixed several bugs.

----------------------------------------------------------------------
The following enhancements were made to DATAPLOT May-October     2001.
----------------------------------------------------------------------

 1) Added support for kernel density plots.  Enter

      HELP KERNEL DENSITY PLOT

    for details.

 2) Added the following command:

      CONSENSUS MEAN PLOT

    This plot summarizes the results of a consensus means analysis.
    Enter

      HELP CONSENSUS MEANS PLOT

    for details.

 3) Added the following probability distributions.

    a) Inverted Weibull

       IWECDF(X,GAMMA)
       IWEPDF(X,GAMMA)
       IWEPPF(X,GAMMA)
       IWEHAZ(X,GAMMA)
       IWECHAZ(X,GAMMA)

       LET GAMMA = <value>
       LET Y = INVERTED WEIBULL RANDOM NUMBERS FOR I = 1 1 100

       LET GAMMA = <value>
       INVERTED WEIBULL PROBABILITY PLOT Y

       INVERTED WEIBULL PPCC PLOT Y

       LET GAMMA = <value>
       CHI-SQUARE INVERTED WEIBULL GOODNESS OF FIT TEST Y

       LET GAMMA = <value>
       KOLMOGOROV-SMIRNOV INVERTED WEIBULL GOODNESS OF FIT TEST Y

    b) Log Double Exponential

       LDECDF(X,ALPHA)
       LDEPDF(X,ALPHA)
       LDEPPF(X,ALPHA)

       LET ALPHA = <value>
       LET Y = LOG DOUBLE EXPONENTIAL RANDOM NUMBERS FOR I = 1 1 100

       LET ALPHA = <value>
       LOG DOUBLE EXPONENTIAL PROBABILITY PLOT Y

       LOG DOUBLE EXPONENTIAL PPCC PLOT Y

       LET ALPHA = <value>
       CHI-SQUARE lOG DOUBLE EXPONENTIAL GOODNESS OF FIT TEST Y

       LET ALPHA = <value>
       KOLMOGOROV-SMIRNOV LOG DOUBLE EXPONENTIAL GOODNESS OF FIT TEST Y

 4) Added support for random number for the following distributions:

       LET Y = COSINE RANDOM NUMBERS FOR I = 1 1 100
       LET Y = ANGLIT RANDOM NUMBERS FOR I = 1 1 100
       LET Y = HYPERBOLIC SECANT RANDOM NUMBERS FOR I = 1 1 100
       LET Y = ARCSIN RANDOM NUMBERS FOR I = 1 1 100
       LET Y = HALF-LOGISTIC RANDOM NUMBERS FOR I = 1 1 100

       LET GAMMA = <value>
       LET Y = DOUBLE WEIBULL RANDOM NUMBERS FOR I = 1 1 100

       LET GAMMA = <value>
       LET Y = DOUBLE GAMMA RANDOM NUMBERS FOR I = 1 1 100

       LET GAMMA = <value>
       LET Y = INVERTED GAMMA RANDOM NUMBERS FOR I = 1 1 100

       LET GAMMA = <value>
       LET Y = LOG GAMMA RANDOM NUMBERS FOR I = 1 1 100

       LET GAMMA = <value>
       LET Y = GENERALIZED EXTREME VALUE RANDOM NUMBERS FOR I = 1 1 100

       LET DELTA = <value>
       LET Y = LOG LOGISTIC RANDOM NUMBERS FOR I = 1 1 100

       LET BETA = <value>
       LET Y = BRADFORD RANDOM NUMBERS FOR I = 1 1 100

       LET B = <value>
       LET Y = RECIPROCAL RANDOM NUMBERS FOR I = 1 1 100

       LET C = <value>
       LET B = <value>
       LET Y = GOMPERTZ RANDOM NUMBERS FOR I = 1 1 100

       LET P = <value>
       LET Y = POWER NORMAL RANDOM NUMBERS FOR I = 1 1 100

       LET P = <value>
       LET SD = <value>
       LET Y = POWER LOGNORMAL RANDOM NUMBERS FOR I = 1 1 100

       LET ALPHA = <value>
       LET BETA = <value>
       LET Y = POWER EXPONENTIAL RANDOM NUMBERS FOR I = 1 1 100

       LET ALPHA = <value>
       LET BETA = <value>
       LET Y = ALPHA RANDOM NUMBERS FOR I = 1 1 100

       LET GAMMA = <value>
       LET THETA = <value>
       LET Y = EXPONENTIATED WEIBULL RANDOM NUMBERS FOR I = 1 1 100

 5) Extended the ppcc plot to handle distributions with 2
    shape parameters.  Specifically,

      BETA PPCC PLOT
      GOMPERTZ PPCC PLOT
      ALPHA PPCC PLOT
      EXPONENTIAL POWER PPCC PLOT
      EXPONENTIATED WEIBULL PPCC PLOT

    This generates a 3-d plot of ppcc value over the range of
    values taken by the 2 shape parameters.

    Support for several additional 2-shape parameter distributions
    is still being tested.

    Enter HELP PPCC PLOT for details.

 6) Made some updates to the STANDARDIZE command.

    a) LET Y2 = USCORE Y X1 X2

       This syntax generates a u-score (i.e., subtract the minimum
       and divide by the range).  This effectively translates
       the variable to a uniform (0,1) scale (much as the z-score
       translates to a standard normal scale).

    b) LET Y2 = SCALE STANDARDIZE Y X1 X2

       This divides by the scale statistic, but does not subtract
       the location statistic first.

    c) Support was added for additional location and scale
       statistics.

    Enter HELP STANDARDIZE for details.

 7) Added the command

      LET Y2 = CROSS TABULATE <stat> Y X1 X2

    where <stat> is one of approximately 25 statistics.

    This command is related to, but different than, the
    analysis command CROSS TABULATE.  This command stores
    the value of the cross tabulated statistic in
    each row of Y2 (where Y2 is the same length as the original
    array Y).  The purpose of this form of the cross tabulation
    is to allow the cross tabulated values to be used in
    subsequent computations (e.g., to compute statistics not
    supported directly by Dataplot).

    For more information, enter the following command:

        HELP CROSS TABULATE (LET)

    In this case, you need to specify the "(LET)" in order to
    avoid ambiguity with other CROSS TABULATE  commands.

 8) Added support for the following new statistics.

    a)  LET A = INTERQUARTILE RANGE Y

    For more information, enter the following commands:

        HELP INTERQUARTILE RANGE

 9) Added the following commands:

      LET A = COMMON DIGITS Y
      LET A = NUMBER OF COMMON DIGITS Y

    These commands return the common digits, and the number of
    common digits, of a vector of numbers.  For example, given
    the numbers 3.214, 3.216, 3.217, and 3.219, the common digits
    are 3.21 and the number of common digits is 2.  The common digits
    are tested to the the RIGHT of the decimal point only (although
    Dataplot does include the portion to the left of the decimal
    point when returning the value of the common digits).  If the
    numbers do no match in their integer portion, Dataplot does
    not return any common digits.  This is a convenience command
    that was added to simplify some macros we were writing.

10) Added the following command:

      LET Y = MATCH X VAL
      LET Z2 = MATCH X VAL Z

    This command matches each value in VAL against X.  For the
    first syntax, it returns the index of the X array where the
    match was found.  A match is that value that is closest in
    absolute value (i.e., an exact match is not required, so
    a match will always be returned).  For the second syntax,
    the index is used to extract the value in Z corresponding to
    the matched index.  This second syntax in fact implements the
    most common use of this command (i.e., the index is usually
    not of interest in itself, rather it is used to extract
    appropriate values from another variable).

11) A few bug fixes were made.  In partiuclar,

    a) The ANDERSON DARLING WEIBULL TEST was modified slightly.
       You no longer get an error message if the GAMMA parameter
       is not specified.  This GAMMA was not actually being used.
       The command now does the following:

        i) If no GAMMA (shape parameter) or BETA (scale parameter)
           has been predefined, maximum likelihood estimates are
           computed automatically.

       ii) If GAMMA and BETA are pre-defined, then the test is
           based on these values.  This allows you to test the goodness
           of fit for parameter values obtained by methods other than
           maximum likelihood.

     b) Made a few fixes in the SINGLE SAMPLE ACCEPTANCE PLAN
        command.  Specifically, it now requires P1 < P2.  In addition,
        a maximum number of iterations has been added to detect
        convergence problems (although this usually caused by P1 > P2).
        Also modified the documentation for this command to provide
        more realistic examples.

     c) Fixed some bugs in the GD device driver (JPEG and PNG
        support).

     d) The COLUMN LIMITS command now works with READ STRING
        (when the string is read from a file).

     e) The output for a number of confirmatory tests was modified
        for clarity.  Note that the underlying computations were
        not modified, just the presentation of the output.

----------------------------------------------------------------------
The following enhancements were made to DATAPLOT February-April  2001.
----------------------------------------------------------------------

 1) The online help files have been substantially updated.
    Specifically, the additions over the last three years are
    now (mostly) incorporated into the help files and the
    web documentation.

 2) Added support for generating JPEG and PNG image formats.
    Enter HELP GD for details.  These device drivers are dependent
    on several external libraries, so support may not be
    available on all platforms.

 3) Added the following command:

      CHARACTER AUTOMATIC SIGN <varname>

    This is similar to the CHARACTER AUTOMATIC command.  However,
    it makes the character "+", "-", or "0" depending on the
    sign of the value in <varname>.  This is sometimes useful
    when writing macros for design of experiment applications.

 4) PROBE is used to determine the current value of internal
    Dataplot variables.  Added the following values that can
    now be accessed with PROBE.

       FX1MIN
       FX1MAX
       FY1MIN
       FY1MAX
       GX1MIN
       GX1MAX
       GY1MIN
       GY1MAX
       DX1MIN
       DX1MAX
       DY1MIN
       DY1MAX

    The FX1MIN, FX1MAX, FY1MIN, FY1MAX define the current
    axis limits, DX1MIN, DX1MAX, DY1MIN, DY1MAX define the
    current data limits, and GX1MIN, GX1MAX, GY1MIN, GY1MAX
    are the current "fixed" limits (i.e., limits set by the
    LIMITS command).

    The most common use is to PROBE the values for FX1MIN,
    FX1MAX, FY1MIN, and FY1MAX to determine the current
    axis limits.  This can sometimes be useful when writing
    complex macros.  For example,

       PLOT SIN(X) FOR X = 0 0.1 6
       PROBE FX1MIN
       LET XAXISMIN = PROBEVAL
       PROBE FX1MAX
       LET XAXISMAX = PROBEVAL
       PROBE FY1MIN
       LET YAXISMIN = PROBEVAL
       PROBE FY1MAX
       LET YAXISMAX = PROBEVAL

 5) Added the following command:

      LET Y2 = STANDARDIZE Y
      LET Y2 = STANDARDIZE Y X1
      LET Y2 = STANDARDIZE Y X1 X2

    This command standardizes a variable, Y, based on either
    no groups, one group, or two groups.  You can standardize
    for both mean and standard deviation or just by the mean.
    By standardize, we mean subtract the mean and divide by the
    standard deviation.  Alternative measures for location and
    scale are allowed.  For details, enter

       HELP STANDARDIZE

 6) By default, the size of characters in subscripts or superscripts
    are set to 1/2 the current character size.  You can set the
    scale factor using the following commands:

      SET SUPERSCRIPT VERTICAL SCALE <value>
      SET SUPERSCRIPT HORIZONTAL SCALE <value>

    These set the height and width of the character respectively.

 7) The CAPABILITY command was significantly enhanced.  Enter
    
        HELP CAPABILITY

    for details.

 8) Support was added for orthogonal distance regression.  Enter

        HELP ORTHOGONAL DISTANCE FIT

    for details.
    
 9) Support was added for consensus means using Mandel-Paule,
    modified Mandel-Paule, Vangel-Ruhkin (maximum likelihood),
    Schiller-Eberhardt, and bounds on bias (BOB) methods.  Enter

        HELP CONSENSUS MEANS

    for details.
    
10) Some bugs were fixed.

    In particular, diagrammatic graphics drawn in data units rather
    than screen units (e.g., DRAWDATA, MOVEDATA) were not drawn
    correctly for log scales.  This has been fixed.  An error
    message is printed if a WEIBULL or NORMAL axis scale is detected.

----------------------------------------------------------------------
The following enhancements were made to DATAPLOT January         2000.
----------------------------------------------------------------------

 1) Added the following commands.

    a) LEGEND <numb> UNITS <DATA/SCREEN>

       This command allows legend coordinates to be interpreted
       in either the screen 0 to 100 units (SCREEN, the default) or in
       units of the plot (DATA).

    b) ...LABEL OFFSET <value>
       ...LABEL JUSTIFICATION <value>

       These commands allow you to set the horizontal offset
       (in Dataplot 0 to 100 screen units, the LABEL DISPLACEMENT
       allows you to set the vertical offset) and justification of
       the axis labels.  These commands were motivated by some
       of the new multiplots discussed below.  However, they
       can be used at any time (although usage should be rare).

    c) You can use CR() in text strings to start a new line.
       Up to 10 lines may be entered, although more than 3 lines
       is rare.  Each of the lines use the same plot attributes
       (e.g., all left justified or all center justified).
       This applies to both hardware and software fonts and
       is used for all types of text.  The most common usages
       are to create multiline titles and legends and to use
       multiple lines with alphabetic tic mark labels.

    d) By default, the Dataplot HISTOGRAM and FREQUENCY POLYGONS
       range from -6 to +6 standard deviations from the mean.
       Although in most cases, this is more than adequate,
       Dataplot did not warn you if points were found outside
       this range.  Dataplot now flags the number of points
       outside this range (separate messages for points below
       and points above).  No message is printed if all points
       are within the range.  The CLASS LOWER and CLASS UPPER
       commands can be used if you need to widen the range.

    e) Dataplot now supports row labels and variable labels.

       Row labels are strings of up to 32 characters that
       are used to identify a row of the data.  To define
       the row label, do something like the following:

          SKIP 25
          COLUMN LIMITS 1 19
          READ ROW LABELS AUTO79.DAT
          COLUMN LIMITS 20 132
          READ AUTO79.DAT Y1 TO Y12

       The COLUMN LIMITS are almost always used when reading the
       row labels.  Typically, you read a file once for the
       numeric data and then a second time for the row labels.

       Currently, the use of row labels is only supported
       with the CHARACTER command (see below).  However, we
       anticipate additional usage of this feature in future
       updates.

       A long label (up to 52 characters) can be associated with a
       variable name (which is currently limited to 8 characters).
       Variable labels are specified with (note that the variable
       name must already be defined).

           VARIABLE LABEL <var name>  <var label>

       The label may contain spaces.  Variable labels are currently
       supported in three ways:

        i) Some of the new multi-plotting commands (discussed
           below) automatically make use of variable labels.
       ii) You can use the "^" to substitute a variable label
           for a variable name in text strings.  For example,

              LET Y = NORMAL RAND NUMBERS FOR I = 1 1 100
              VARIABLE LABEL Y NORMAL RANDOM NUMBERS
              Y1LABEL ^Y
              PLOT Y

           Previously, Dataplot only supported substitutions
           for parameters and strings.  Now, if a variable name
           is found, it checks to see if a label has been defined.
           If yes, the label is substituted for the variable name.
           If not, the variable label is left as is (with the
           "^" removed).
       
      iii) The X1LABEL AUTOMATIC and Y1LABEL AUTOMATIC commands
           will now substitute the varialbe label for the variable
           name on the x and y axes respectively.

    f) The following special options were added for the
       CHARACTER command:
    
          ROWID     - uses the row number as the plot character
          ROWLABEL  - uses the row label as the plot character
          XVALUE    - uses the x-coordinate of the point as the
                      plot character
          YVALUE    - uses the y-coordinate of the point as the
                      plot character
          XYVALUE   - uses (x-coor,y-coor) as the plot character
          TVALUE    - uses the tag value as the plot character
                      (Dataplot assigns a curve-id, the tag,
                      to each point)
          ZVALUE    - this is a special form that is specific to
                      certain commands.  For a few commands (currently
                      the DEX CONTOUR PLOT and the CROSS TABULATE 
                      PLOT, but we expect a few
                      additional plots to support this form in future
                      releases), Dataplot writes a numeric value into
                      an internal array.  The value in this array is
                      used as the plot symbol.  Using this with
                      unsupported plot types may have unpredictable
                      results (it will depend on what is stored in
                      the internal array).  This option is typically
                      set automatically by Dataplot in the
                      background, so currently users should not
                      set this directly.

       The ROWID and ROWLABEL are typically only used for the
       PLOT command (i.e., not for HISTOGRAM, etc.).  This option
       keeps track of any subsetting (i.e., SUBSET/FOR/EXCEPT
       clauses on the plot command) when identifying the point.
       However, the results may be unpredictable for graphics other
       than the PLOT command.

       The most common use of this command is to identify specific
       points on the plot (typically with the ROWLABEL option).
       A typical sequence would be

         CHARACTER X
         PLOT Y X
         PRE-ERASE OFF
         LIMITS FREEZE
         CHARACTER ROWLABEL
         PLOT Y X SUBSET Y > 90
       
    g) The STATISTIC PLOT command now supports the
       CORRELATION, RANK CORRELATION, COVARIANCE, and RANK
       COVARIANCE cases.

    h) The command

          SET PARAMETER EXPANSION <NUMERIC/EXPONENTIAL>

       was added.  This command applies when substituting the
       value of a parameter using "^".  Normally, this was
       intended for putting numeric values in text lagels.  In this
       case, it is desirable to limit the number of digits.  However,
       when used with the FIT command (parameters you want to remain
       constant rather than be fitted are often entered this way),
       you may need to specify high precision.  If NUMERIC (the
       default) is specified, the current algorithm for parameter
       substitution is used.  If EXPONENTIAL is specified, the 
       parameter is entered using scientific notation.  For example,

          (0.123456789012*10**(2))

    i) The command

          SET SORT DIRECTION <ASCENDING/DESCENDING>

       was added.  This command specifies whether the sorts
       performed by SORT and SORTC are ascending or descending
       sorts (the default is ascending).

 2) The following new plots were added.

    a) INTERACTION PLOT Y X1 ... XK
       <stat> INTERACTION PLOT Y X1 ... XK

       These plot Y versus X1*X2* ... *XK and are primarily intended
       for DEX applications.  Specifically, it serves as the
       building block for the DEX INTERACTION PLOT discussed below.
       It is actually the DEX INTERACTION PLOT that is typically
       generated by the user.  This command supports the same
       set of statistics as the STATISTIC PLOT command.
      
       The case of most interest for the DEX plots is 2 X variables,
       but these plots will in fact handle an arbitrary number
       up to 25.


    b) CROSS TABULATE <stat> PLOT Y X1 X2
       CROSS TABULATE <stat> PLOT Y1 Y2 X1 X2
       CROSS TABULATE PLOT X1 X2
       CROSS TABULATE PLOT <stat> X1 X2

       This command performs a cross-tabulation on X1 and X2.
       It computes the statistic given by <stat> for the response
       values (Y) in each cell of the cross tabulation.  The list
       of supported statistics is the same as for the
       STATISTIC PLOT command.  Most of the supported statistics
       expect a single response variable.  A few expect two
       (e.g., LINEAR CORRELATION).  The COUNT (or NUMBER) expect
       no response variables.

       The output of this command plots the computed statistic
       on the Y axis.  The X axis coordinate is determined from
       the two group variables in the following way:

        i) The levels of the first group variable (X1 in the above
           examples) are plotted at 1, 2, 3, etc.

       ii) For each level of the group 1 variable, the levels of
           the group 2 variable are scaled +/- 0.2 around the
           level of the group 1 variable.

       For example, if X1 has 2 levels (at 1 and 2) and X2 has
       3 levels (1, 2, and 3), then the following x-coordinates
       are used:

             X1         X2        X-COOR
             ============================
              1          1        0.8
              1          2        1.0
              1          3        1.2
              2          1        1.8
              2          2        2.0
              2          3        2.2

       The syntax CROSS TABULATE X1 X2 is a special case.  It plots
       the value of X1 on the X axis and the value of X2 on the
       Y axis.  The plot character is then set to the count
       for that cell (this is done automatically and you do not need
       to set the plot character).  This form of the plot has
       application in the design of experiments.

       Note that this command is an extension of the STATISTIC PLOT
       command.  However, instead of one group variable, there
       are two group variables.


       The command 

           SET CROSS TABULATE PLOT DIMENSION <1/2>

       can be used to specify an alternative format for this
       plot.  If "1", then the format of the plot is described
       as above.  If "2", then the format is similar to the
       CROSS TABULATE X1 X2 format.  That is,

           SET CROSS TABULATE PLOT DIMENSION 2
           CROSS TABULATE MEAN PLOT Y X1 X2

       will print the value of the mean of Y at the value of X1 on
       the X axis and the value of X2 on the Y axis.   Essentially,
       this is the tabled values in graphic format.  You can
       use this format to generate plots where you want to print
       a numeric value at (X,Y), that is some value other than
       X or Y.  You can define a response variable Z with the
       desired values to print and then use the CROSS TABULATE
       MEAN PLOT (if there is only one value, the mean is equal
       to that value).

    c) DEX CONTOUR PLOT Y X1 X2 YCONT

       This plots a dex contour plot for the case when X1 and X2
       have 2 levels (represented by the values -1 and 1). In
       addition, one or more center points (X1 and X2 both 0)
       may be present.  Any points where X1 and X2 are not equal
       to -1, 1, or 0 are ignored.  The array YCONT contains the
       contour levels.

       The appearance of the plot is controlled by the settings
       of the LINE and CHARACTER command.  Specifically,

          trace 1 = label for center point and the points
                    at (-1,-1), (-1,1), (1,1), (1,-1).  The
                    character setting should be ZVAL and line
                    should be blank.
          trace 2 = center point.  If no center point was specified,
                    this point is not generated (and the CHAR and LINE
                    settings need to be adjusted accordingly).
          trace 3 = line connecting (-1,-1), (1,-1), (1,1), (-1,1)
          trace 4+= the contour lines start with trace 4.  There is
                    one trace for each value of YCONT.
     
       This command implements the algorithm previously available
       in the built-in DEXCONT.DP macro as a Dataplot command.

       As an example of this command, you can enter

          SKIP 25
          READ BOXYIELD.DAT Y X1 X2
          LET YCONT = SEQUENCE 50 2 70
          CHARACTER ZVAL CIRCLE CIRCLE
          CHARACTER FILL OFF ON ON
          LINE BLANK BLANK BLANK
          DEX CONTOUR PLOT Y X1 X2

    d) YATES CUBE PLOT Y X1 X2 X3

       This plots a Yates cube plot for the case when X1, X2, and
       X3 are factor variables with exactly two levels.  It plots
       the value of the response variable, Y, at each vertex.
       This plot is used in 2**(3) factorial and fractional
       factorial designs.

 3) Dataplot now supports sub-regions on plots.  Sub-regions are
    motivated by the desire to denote "engineering limits"
    on a plot.  That is, a rectangle, denoting an acceptance
    region in both the X and Y directions, is drawn on the
    plot and then the plots are overlaid on top of this.

    Although the subregion capability was motivated for the
    purpose of denoting engineering limits, they can in fact
    be used for whatever purpose you want.

    The SUBREGION commands are:

        SUBREGION <ON/OFF> <ON/OFF> <ON/OFF> ....
        SUBREGION XLIMITS <lower value> <upper value>
        SUBREGION <id> XLIMITS <lower value> <upper value>
        SUBREGION YLIMITS <lower value> <upper value>
        SUBREGION <id> YLIMITS <lower value> <upper value>

    Up to 10 subregions may be defined.  In most applications,
    only a single subregion is plotted.  The SUBREGION <ON/OFF>
    switch determines whether or not the given subregion is
    plotted.  The SUBREGION XLIMITS/YLIMITS commands specify
    the lower and upper bounds of the rectangle.  If no
    <id> is specified, the limits are set for the first subregion.
    If <id> is specified, it should be between 1 and 10.

    You do not need to adjust the settings for the CHARACTER, LINE,
    BAR, and SPIKE when using subregions.  Dataplot automatically
    shifts these in the background.  The attributes of the SUBREGION
    are defined by:

        REGION FILL <ON/OFF>
        REGION COLOR <COLOR>
        REGION BORDER LINE <linetype>
        REGION BORDER COLOR <color>

    The REGION FILL and REGION COLOR determine the attributes of
    the interior of the rectangle.  The two most common choices
    are to leave it blank or to fill it with some type of light gray
    scale color.  The attributes of the box border are set with
    the REGION BORDER LINE and REGION BORDER COLOR commands.  The
    standard line types (BLANK,SOLID, DASH, DOTTED, etc.) are
    supported.  Although only one setting was given above, if you
    have defined multiple subregions, then you should define
    multiple settings in the above command.

    A typical sequence of commands would be

        SUBREGION ON
        SUBREGION XLIMITS 0.35 0.42
        SUBREGION YLIMITS 2000 3000
        REGION FILL ON
        REGION BORDER LINE DASH
        REGION COLOR G90
        PLOT ....
        SUBREGION OFF

    Some points to note about subregions are:

    a) The subregions are plotted before any of the plot
       curves.  The significance of this is that a solid filled
       subregion will be drawn and then the regular plot points
       are drawn on top.  The effect of this can be hardware
       dependent.  On X11 and Postscript devices, a solid character
       can be seen on top of a light gray scale box (if the gray
       scale gets too dark, the plot points are no longer
       distinguishable).  However, on some hardware devices, you may
       not be able to see points plotted on top of a solid fill
       region.  In this case, plot the border of the subregion and
       leave the interior blank.

       It is this order of plotting that distinguishes the
       subregion from simply using a BOX <id> command to plot
       rectangular regions on the screen.

    b) Although most commonly used with the PLOT command, subregions
       can in fact be used with any Dataplot graphics command.

    c) Currently, only rectangular subregions are supported.
       We expect that to be generalized to polygonal regions
       in the future.

 4) Dataplot now saves the following internal parameters after
    all plots (not just those generated with PLOT):

    PLOTCORR - correlation of the X and Y coordinates on the plot
    PLOTCOR1 - correlation of the X and Y coordinates on the plot
               with a tag value of 1.  This can be useful for
               plots that generate reference lines (which you
               do not want included in the correlation computation
    PLOTYMAX - maximum Y coordinate
    YMAXINDE - index of the maximum Y coordinate
    PLOTYMIN - minimum Y coordinate
    YMININDE - index of the minimum Y coordinate
    PLOTXMAX - maximum X coordinate
    XMAXINDE - index of the maximum X coordinate
    PLOTXMIN - minimum X coordinate
    XMININDE - index of the minimum X coordinate
    NACCEPT  - number of plot points inside the first subregion
               (0 if no subregions defined)
    NREJECT  - number of plot points outside the first subregion
               (0 if no subregions defined)
    NTOTAL   - number of plot points (NACCEPT + NREJECT)
               (0 if no subregions defined)

 5) The following multiplots were added:

    SCATTER PLOT MATRIX  Y1 Y2 ... YK
    FACTOR PLOT Y1 X1 ... YK
    CONDITIONAL PLOT Y X TAG

    a) SCATTER PLOT MATRIX Y1 ... YK

       This generates all the pairwise scatter plots of Y1 ... YK
       on a single page.

    b) FACTOR PLOT Y X1 ... XK
 
       This generates the plots Y VS X1, Y VS X2, .... , Y VS XK
       on a single page.

    c) CONDITIONAL PLOT Y X TAG

       This generates PLOT Y VERSUS X for each unique value in
       TAG on a single page.

    There are a lot of variations possible with these types of
    plots.  For example, the basic concept is not limited to
    scatter plots.  For example, you can generate all the pairwise
    bihistograms instead of the pairwise scatter plots.  There are
    many options in terms of labeling, what plot goes on the
    diagonal, and so on.

    There are various SET commands that control the appearance
    and nature of these plots.  Enter

        HELP SCATTER PLOT MATRIX
        HELP CONDITIONAL PLOT
        HELP FACTOR PLOT

    for a complete description of what is available.

    Two variations of the SCATTER PLOT MATRIX are important enough
    to be given special names:

         DEX INTERACTION PLOT
         YOUDEN MATRIX PLOT

    These are described under HELP SCATTER PLOT MATRIX.

 6) Fixed the following bugs.

    a) The MULTIPLOT SCALE FACTOR did not work correctly with
       the software fonts.

    b) Entering "character blank", i.e., the blank is in lower case,
       plotted BLAN as the plot character when DEVICE 1 FONT SIMPLEX
       was used.

    c) Using SP() with a software font did not work.

    d) The BOX SHADOW OFF command was fixed to set the shadow
       height and width to zero rather than to the default.



----------------------------------------------------------------------
The following enhancements were made to DATAPLOT January - July   1999.
----------------------------------------------------------------------

 1) Modified the IF command so that if there is an error (e.g., one
    of the parameters is not defined), the IF status is set to
    FALSE rather than being undefined.

 2) Added the following time series commands.

    a) Added the command 

         LET PERIOD = <value>
         LET START = <value>
         SEASONAL SUBSERIES PLOT Y

       A seasonal subseries plot is used to determine if there
       is significant seasonality in a time series.  Instead of
       a straight time order plot, it splits the plot into
       the corresponding seasons (or periods).  For example, for
       monthly data, all the January values are plotted, then all
       the February values, and so on.  Reference lines are drawn
       at the seasonal means.

    b) Added the command

         LET PERIOD = <value>
         LET STLWIDTH = <value>
         LET STLSDEG = <0/1>
         LET STLTDEG = <0/1>
         LET STLROBST = <0/1>
         SEASONAL LOWESS Y     (or SEASONAL LOESS Y)
         READ DPST1F.DAT SEAS TREND

       The SEASONAL LOWESS command decomposes a time series into
       trend, seasonal, and residual components using techniques
       based on locally weighted least squares.  That is,
      
          X(t) = TREND(t) + SEAS(t) + RES(t)

       The seasonal and trend components are written to the file
       DPST1F.DAT (dpst1f.dat on Unix systems) and can be read
       back into Dataplot for further plotting and analysis.  The
       internal variable RES contains the residual component and
       the internal variable PRED contains the trend plus the
       seasonality component.

       The SEASONAL LOWESS command accepts a number of options
       which can be defined by the LET commands above.  The most
       important is the PERIOD parameter which identifies the number
       of seasons (e.g., 12 for monthly data).  The STLWIDTH
       parameter identifies the number of data points to use
       in the LOWESS steps and defaults to N/10.  It is similar
       to specifying the LOWESS FRACTION for standard LOWESS
       smoothing.  The more points used, the more smoothing that
       occurs.  The STLSDEG and STLTDEG parameters identify the
       polynomial degree used in the lowess for the seasonal and
       trend components respectively.  By default, the seasonal
       lowess performs some robustness iterations.  Enter
       LET STLROBST = 1 to suppress this.

       This technique is described in

           Cleveland, Cleveland, McRae, and Terpenning, "STL: A
           Seasonal-Trend Decomposition Procedure Based on Loess",
           Statistics Research Report, AT&T Bell Laboratories.

    c) Added an ARIMA modeling capability.  The command is:

         ARMA Y AR DIFF MA SAR SDIFF SMA SPERIOD

       where

         Y       = the response variable
         AR      = the order of auto-regressive terms
         DIFF    = number of differences to apply.  DIFF is typically
                   0, 1, or 2.  Differencing is one technique for
                   removing trend.
         MA      = order ot the moving average terms
         SAR     = order of seasonal auto-regressive terms.
         SDIFF   = number of seasonal differences to apply.  It is
                   typically 0, 1, or 2.
         SMA     = order of seasonal moving average terms.
         SPERIOD = period for seasonal terms.  It defaults to 12
                   (if a seasonal component is included).

       If there is no seasonal component, the last 4 terms may be
       omitted.

       To minimize the amount of screen output, but to also to
       keep the maximum amount of information, Dataplot writes
       most of the output to files.  Speficially,

          dpst1f.dat - the parameters and the standard deviations
                       of the parameters from the ARMA fit.  The
                       order is:
                       1) Autoregressive terms
                       2) Seasonal autoregressive terms
                       3) Mean term
                       4) Moving average terms
                       5) Seasonal moving average terms
          dpst2f.dat - this file contains:
                       1) Row number
                       2) Original series (i.e., Y)
                       3) Predicted values
                       4) Standard deviation of predicted values
                       5) Residuals
                       6) Standardized residuals
          dpst3f.dat - Intermediate outut from iterations before
                       convergence.  This is generally useful if
                       the ARMA fit does not converge.
          dpst4f.dat - The parameter variance-covariance matrix.
          dpst5f.dat - The forecast values for (N/10)+1 observations
                       ahead.  Specifically,
                       1) The forecasted values
                       2) The standard deviation of the forecasted
                          values.
                       3) The lower 95% confidence band for the
                          forecast.
                       4) The upper 95% confidence band for the
                          forecast.

       Dataplot allows you to define the starting values by
       defining the variable ARPAR.  The order of the parameters
       is as given for the file dpst1f.dat above.  By default,
       all parameters are set to 1 except for the mean term which
       is set to 0.

       In addition, you can define the variable ARFIXED to fix
       certain parameters to their start values.  That is, you
       define ARPAR to specify the start values.  If the
       corresponding element of ARFIXED is zero, the parameter is
       estimated as usual.  If ARFIXED is one, then the parameter
       is fixed at the start value.  The most common use of this
       is to set certain parameters to zero.  For example, if
       you fit an AR(2) model and you want the AR(1) term to be
       zero, you could enter the following:

           LET ARPAR = DATA 0 1
           LET ARFIXED = DATA 1 0
       
       Dataplot uses the STARPAC library (developed by 
       Janet Rogers and Peter Tyrone of NIST) to compute the
       ARIMA estimates.

       ARIMA modeling is covered in many time series texts.  It is
       beyond the scope of this news file to discuss ARIMA modeling.
       However, to use ARIMA models, it is generally recommended
       that the series be at least 50 observations long.  In addition,
       if the series is dominated by the trend and seasonal factors,
       an explicit trend, seasonal, and random component decomposition
       method, such as the seasonal lowess described above, is
       generally preferred to an explicit ARIMA model.

 3) Added support for location and scale parameters for an additional
    15 distributuins.  Entering the command

         LIST DISTRIBU.

    will list the distributions table.  This table shows which
    distributions support location and scale parameters.

 4) Added the following statistics:

    Added the CNPK capability index statistics:

       LET LSL = <value>
       LET USL = <value>

       LET A = CNPK Y

    This statistic is now also supported for the following plots:

       LET LSL = <value>
       LET USL = <value>

       CNPK PLOT Y X
       DEX CNPK PLOT Y

    The LSL and USL specify the lower specification and upper
    specificiation engineering limits.  The CNPK is a variant of the
    CPK capability indices used for non-normal data and is defined as:

      CNPK = MIN(A,B)

    where

        A = (USL-MEDIAN)/(P(.995)-MEDIAN)
        B = (MEDIAN-LSL)/(MEDIAN-P(0.005))

    P(0.995) and P(0.0050 are the 99.5 and 0.5 percentiles of the
    data respectively.

    Added the geometric mean and standard deviation and the
    harmonic mean statistics.

       LET A = GEOMETRIC MEAN Y
       LET A = GEOMETRIC STANDARD DEVIATION Y
       LET A = HARMONIC MEAN Y

    This statistic is now also supported for the following plots:

       GEOMETRIC MEAN PLOT Y X
       GEOMETRIC STANDARD DEVIATION PLOT Y X
       HARMONIC MEAN PLOT Y X

       BOOTSTRAP GEOMETRIC MEAN PLOT Y X
       BOOTSTRAP GEOMETRIC STANDARD DEVIATION PLOT Y X
       BOOTSTRAP HARMONIC MEAN PLOT Y X
       JACKNIFE GEOMETRIC MEAN PLOT Y X
       JACKNIFE GEOMETRIC STANDARD DEVIATION PLOT Y X
       JACKNIFE HARMONIC MEAN PLOT Y X

    The geometric mean is defined as:

       XGM = (PRODUCT(Xi))**(1/N)

    The geometric standard deviation (SD means standard deviation of)
    is defined as:

       XSD = EXP(SD(LOG(Xi)))

    The harmonic mean is defined as:

       XHM = N/SUM(1/Xi)

 5) Added the Wilks-Shapiro test for normality.  The following
    commands are equivalent.

       WILKS SHAPIRO NORMALITY TEST Y
       WILKS SHAPIRO TEST Y
       WILKS SHAPIRO Y

    There must be at least 3 values in Y.  The computed significance
    level is not neccessarily valid for N >= 5,000.  This command
    uses algorithm R94 from the Applied Statistics Journal.

 6) Added the studentized range CDF and PPF functions.

       LET A = SRACDF(X,V,R)
       LET A = SRAPPF(P,V,R)

    where V is the degrees of freedom and R is the number of
    samples.  X must be positive, V must be >= 1, and 
    R must be >= 2.  For most applications, R = V + 1.  The PPF
    function is only supported for values in the range 0.90 to 0.99.

    The studentized range is defined as:

       Q = Range/(Standard deviation)

    The studentized range is used in constructing confidence intervals
    and significance levels for tests for multiple comparison in
    analysis of variance problems.

 7) Updated the Weibull maximum likelihood estimates to suport
    censored data (both type 1 and type 2 and multiple).  It also now
    generates confidence intervals for the estimate (for various
    significance levels).  The command

       SET CENSORING TYPE <NONE/1/2/MULTIPLE>

    defines the censoring type.  The EXPONENTIAL MLE output was
    modified to be more readable and consistent with the Weibull
    output.

 8) Added the following quality control commands.

    a) Added the following command to generate binomial based single
       sample acceptance plans:

           SINGLE SAMPLE ACCEPTANCE PLOT P1 P2 ALPHA BETA

      where

           P1 = Acceptable Quality Level
           P2 = Lot Tolerence Percent Defective
           ALPHA = Probability of a Type I error
           BETA =  Probability of a Type II error

    b) Added a command to generate the average run length for the
       cumulative sum (cusum) control chart.  The average run length
       is the average number of observations that are entered
       before the system is declared out of control.

           LET S0 = <value>
           LET K = <value>
           LET H = <value>

       These commands set parameters required by the cusum ARL
       calculation.  Specifically,

           S0     = start-up value for the cumulative sum.  This is
                    usually zero.  However, it can be set to a
                    positive initial value for a fast initial
                    response (FIR) cusum chart.
           H      = defines the value which signals that the cusum
                    is "out of control".  A value of 5 is a common
                    choice.
           K      = the value of k is set to one half of the smallest
                    shift in location (in standard deviation units)
                    that you want to detect.  A common choice is a
                    1-sigma shift, that is k = 0.5.

           LET Y = ONE-SIDED CUSUM ARL DELTA
           LET Y = CUSUM ARL DELTA

       where DELTA defines the difference between the target value
       of the process and the true value of the process.  This is
       a variable that is usually defined to be a sequence of values.
       For example, 

           LET DELTA = SEQUENCE 0 .01 0.5

       That is, this command returns the average run length for
       a series of values that define the difference between the
       target value and the true value of the process. 

       A typical sequence of commands would be

           LET K = 0.5
           LET H = 5
           LET S0 = 0
           LET DELTA = SEQUENCE 0 .01 1.0
           LET Y = CUSUM ARL DELTA
           PLOT Y DELTA

       This command was implemented using Applied Statistics
       algorithm 258.  If unreasonable values are specified for the
       parameters, this algorithm can generate unreasonable results.

 9) Added the following commands:

         ANOP LIMITS <low> <high>
         PROPORTION CONFIDENCE LIMITS Y
         DIFFERENCE OF PROPORTION CONFIDENCE LIMITS Y1 Y2

    to generate a confidence interval for proportions and the
    difference of two proportions respectively.  The ANOP
    LIMITS command is used to define the lower and upper bounds
    that define a success.  The confidence intervals are based 
    on the direct binomial computations, not the normal
    approximation, so it is not limited by small N.
   
10) Added the command

       WEB HANDBOOK <keyword>

    This command access the NIST/SEMATECH Engineering Statistics
    Handbook.  A beta version of the Handbook will be released
    May, 1999 (http://www.itl.nist.gov/div898/handbook/).

    The <keyword> is matched against a file of keywords to
    go to the appropriated location in the handbook.  This
    command is used primarily by the Dataplot GUI, but it can
    also be entered by the end-user.  If you want to see a list
    of the supported keywords, enter

        LIST HANDBK.TEX

    The handbook provides tutorial information on many common
    engineering statistical capabilities.  This complements the
    WEB HELP command, which accesses the on-line Dataplot Reference
    Manual.  The on-line Reference Manual is primarily concerned
    with how you implement a statistical technique while the
    Handbook provides more of a statistical tutorial.

    If your site has downloaded the Handbook, enter a command
    like the following:

        SET HANDBOOK URL http://ketone.cam.nist.gov/cf/handbook/

    to define the home directory for the handbook.

    The web commands SET BROWSER and SET NETSCAPE OLD apply to
    the WEB HANDBOOK as well.  SET BROWSER defines the browser
    and SET NETSCAPE OLD allows you to use a currently open
    browser for the WEB HANDBOOK command.  These commands are
    discussed in more detail later in this news file.

11) Added the following non-parameteric tests.

    a) The following are non-parametric alternatives to the
       2-sample t test (i.e., test the hypothesis U1 = U2 where U1
       and U2 are the population means for 2 samples).

          SIGN TEST Y1 Y2
          SIGN TEST Y1 Y2 D0
          SIGN TEST Y1 MU

          SIGNED RANK TEST Y1 Y2
          SIGNED RANK TEST Y1 Y2 D0
          SIGNED RANK TEST Y1 MU

          RANK SUM TEST Y1 Y2
          RANK SUM TEST Y1 Y2 D0

       where Y1 and Y2 are the response variables and D0 and MU
       are parameters.  Specify D0 to test U1 - U2 = D0.  The
       2-sample test can also be used for the 1-sample test
       U1 = MU.

       The SIGN TEST and SIGNED RANK TEST commands only apply to
       paired samples.  The RANK SUM TEST command does not require
       equal sample sizes.

    b) The following performs the Kruskal-Wallis non-parametric
       1-sample ANOVA.

           KRUSKAL WALLIS Y X

       where Y is the response variable and X is the factor
       variable.

12) Added the following plot commands:

    a) TUKEY MEAN-DIFFERENCE PLOT Y1 Y2

       A Tukey mean-difference plot is an enhancement of the
       quantile-quantile (q-q) plot.  It converts the interpretation
       of the q-q plot from the differences around a diagonal line
       to the differences around a horizontal line.  If T(i) and
       D(i) are the vertical and horizontal coordinates for the q-q
       plot, the Tukey mean-difference plot is (T(i) - D(i)) versus
       (T(i) + D(i))/2.  A horizontal reference line is drawn at
       zero.

    b) SPREAD LOCATION PLOT Y TAG

       The spread-location (s-l) plot is a robust alternative to
       the homoscedasticity plot.

       Given a response variable Y and a group-id variable X, 
       the homoscedasticity plot is the group standard deviations
       versus the group means.  This is a graphical measure of
       constant spread across groups.

       The s-l plot has the square roots of the absolute value of
       the Y(i) minus their group medians on the vertical axis and
       the group medians on the horizontal axis.  A reference line
       connects the group medians.

       When setting the LINE and CHARACTER commands, the reference
       line is the first trace and the data starts with trace 2
       (each group is identified as a unique trace).  That is, to
       draw the data points as circles and the reference line as a
       solid line, do something like the following

           CHARACTER CIRCLE ALL
           CHARACTER BLANK
           LINE BLANK ALL
           LINE SOLID
           SPREAD LOCATION PLOT Y X

    c) RF SPREAD PLOT

       The residuals-fitted (r-f) spread plot is a graphical measure
       of the goodness of fit.  That is, this command is preceeded
       by some type of fit.  It plots percent point (or quantile)
       plots of the fitted values minus their mean and the residuals
       arranged side by side with a common vertical scale.

       The vertical spread of the residuals compared to the vertical
       spread of the fitted values gives an indication of how much
       of the variation is explained by the fit.

13) Added the following special functions:

    a) LET A = ABRAM0(X,ORD)
 
       This computes the Abramowitz function for order ORD.
       currently, ORD can be an integer from 0 to 100.

    b) LET A = CLAUSN(X)
 
       This computes the Clausen integral.

    c) LET A = DEBEYE(X,ORD)
 
       This computes the Debeye function of order ORD.  ORD
       can be 1, 2, 3, or 4.

    d) LET A = EXP3(X)
 
       This computes the cubic exponential integral.

    e) LET A = GOODST(X)
 
       This computes the Goodwin and Stanton integral.

    f) LET A = LOBACH(X)
 
       This computes the Lobachevski's integral.

    g) LET A = SYNCH1(X)
       LET A = SYNCH2(X)
 
       This computes the synchrotron radiation functions.

    h) LET A = STROM(X)
 
       This computes Stromgren's integral.

    i) LET A = TRAN(X,ORD)
 
       This computes the transport integrals of order ORD.
       ORD can be 2, 3, 4, 5, 6, 6, 8, or 9.

    These special functions are computed using ACM algorithm 757.
    Formulas for these functions are given in:

        Allan MacLead, "ACM Transactions of Mathematical Software",
        Vol. 22, No. 3, September 1996, pp. 288-301.

13) Fixed a bug in the CD command for Unix platforms.  The CD command
    allows you to set the default directory.

    A few other miscellaneous bugs have also been fixed.

----------------------------------------------------------------------
The following enhancements were made to DATAPLOT September - Dec 1998.
----------------------------------------------------------------------

 1) Added the following MATRIX commands:

       LET MEAN = MATRIX GROUP MEANS M TAG
       LET SD   = MATRIX GROUP SD    M TAG
       LET SPOOL = POOLED VARIANCE-COVARIANCE MATRIX M TAG

    The MATRIX GROUP MEANS and MATRIX GROUP SD commands compute
    the group means and standard deviations of a matrix.
    The POOLED VARIANCE-COVARIANCE MATRIX computes a pooled
    variance-covariance matrix.

    These commands all operate on a matrix (M) and a group
    id variable (TAG).  The TAG variable has the same number of
    rows as the matrix M.  The values of TAG are typically integers
    and they identify the group to which the corresponding row
    of the matrix belongs.

    The MATRIX GROUP MEANS/SD commands return a matrix with the
    same number of columns as the original matrix M and with
    the number or rows equal the number of groups identified
    by the TAG variable.  That is, MEANS(2,3) is the mean of
    of the third variable of the second group.

    The pooled variance-covariance matrix:

        SPOOL = (1/SUM(N(i)-1)) * SUM((1/N(i)-1)*C(i)))

    where N(i) is the number of elements in group i and C(i)
    is the variance-covariance matrix of the rows belonging to
    group i.  An earlier implementation of this command
    works with 2 matrices (and no group id variable).  This
    version of the command still works.  That is, if the second
    argument to POOLED VARIANCE-COVARIANCE MATRIX command is
    a matrix, it is assumed that there are 2 groups and the
    data for each group is stored in a separate matrix.  If the
    second argument is a variable, it is assumed that it is a
    group id variable and the data for all matrices are stored
    in a single matrix.  For the 2 group case, either syntax
    will work.  For more than 2 groups, only the new syntax
    will work.

 2) The following control chart enhancements were added:

    a) HOTELLING CONTROL CHART Y1 Y2 ... YK GROUP

       This commands implements a Hotelling multivariate
       control chart.   Given p response variables, the Hotelling
       control chart computes and plots the following for each group:

         T-square = n*(xbar - u0)'SINV(xbar - u0)

       N is the size of the group, xbar is a vector of the p
       sample means for the subgroup, and u0 is a vector of the
       p sample means for the entire data set.  That is a 1-sample
       Hotelling test is computed to test whether the means for
       a given group are equal to the overall sample means.

       An upper control limit (there is no lower control limit)
       is drawn at the appropriate F statistic for the Hotelling
       test.  The value of alpha for the F test is chosen so
       that alpha/(2*p) = 0.00135.  This corresponds to the
       3-sigma value for a univariate chart.  You can specify
       your own control limit, set by whatever criterion that
       you deem appropriate,  by entering the command:

           LET USL = <value>

       You can control the appearance of this chart by setting
       the lines and character switches.  The traces are:

          Trace 1 = T-square values
          Trace 2 = Zero reference line
          Trace 3 = Dataplot calculated control limit
          Trace 4 = User specified upper control limit

       For example, to draw the T-square values as a solid line
       and an X, no zero reference line, the Dataplot control
       limit as a dotted line, and no user specified control
       limit, enter the commands:

           LINE SOLID BLANK DOTTED BLANK
           CHARACTER X BLANK BLANK BLANK

    b) CUSUM CONTROL CHART Y X

       This command implements a mean cumulative sum control
       chart.

       There are numerous variations on how cusum control
       charts are implemented.  Dataplot follows the methods
       discussed by Thomas Ryan in "Statistical Methods for
       Quality Improvement".  Dataplot does the following:

         i) Positive and negative sums are computed as follows:

               SUMH = MAX[0,(z(i) - k) + SUMH(i-1)] 
               SUML = MAX[0,(-z(i) - k) + SUML(i-1)] 

            SUMH and SUML have initial values of 0.  Z(i) is
            the z-score of the ith group (that is, the sub-group
            mean minus the overall mean divided by the
            standard deviation of xbar. 

            Dataplot plots the negative of SUML.  This is to
            avoid overlap for the plottting of SUMH and SUML.
            SUMH is plotted on the positive scale vertically and
            SUML is plotted on the negative scale vertically.

            The value of k is set to one half of the smallest
            shift in location (in standard deviation units)
            that you want to detect.  Dataplot by default selects
            a 1-sigma shift, that is k = 0.5.  To overide this,
            enter the command

                 LET K = <value>

       ii)  By defauult, Dataplot sets the control limit at
            a value of 5.  That is, if the one of the sums exceeds
            5, the process is deemed out of control.  To override
            the default value, enter the command

                LET H = <value>

            The value for H is typically between 4 and 5.

 3) The following command was added:

    TOLERANCE LIMITS Y

    This computes univariate two-sided tolerance limits for the normal
    case and for the distribution free case.

    Tolerance limits are a generalization of confidence limits
    for the mean.  However, instead of a confidence limit for a
    single value, it provides confidence limits for the interval
    that contains a given percentage of the data (this is called
    the coverage).  That is, for 90% coverage, we are finding
    a confidence interval that contains 90% of the data.

 4) Bug fixes:

    a) The PP command was fixed for the LAHEY and Microsoft PC
       versions of Dataplot.

    b) Fixed the RESET VARIABLES command so that it would not
       delete parameters, functions, and strings.  Note that
       RESET DATA still deletes them.

 5) Added the percentile statistic:

       LET A = <value> PERCENTILE Y

    where <value> is a number between 0 and 100.

    This statistic is now also supported for the following plots:

       LET P100 = <value>

       PERCENTILE PLOT Y X
       BOOTSTRAP PERCENTIL PLOT Y
       JACKNIFE PERCENTILE PLOT Y
       PERCENTILE BLOCK PLOT Y
       DEX PERCENTILE PLOT Y

    The LET P100 = <value> command defines the percentile you
    want to compute for all of these plots.

    Fixed a small bug in the ...DECILE command.

 6) Added the CPM and CC capability index statistics:

       LET LSL = <value>
       LET USL = <value>
       LET TARGET = <value>

       LET A = CPM Y
       LET A = CC Y

    This statistic is now also supported for the following plots:

       LET LSL = <value>
       LET USL = <value>
       LET TARGET = <value>

       CPM PLOT Y X
       DEX CPM PLOT Y
       CC PLOT Y X
       DEX CC PLOT Y

    The LSL, USL, and TARGET specify the lower specification,
    upper specificiation, and target engineering limits.  The
    CPM is a variant of the CP and CPK capability indices and
    is defined as:

      CPM = (USL-LSL)/(6*SQRT(S**2+(XBAR-TARGET)**2))

    where XBAR and S are the sample mean and standard deviation.
    For this index, the larger the better.

    The CC statistic is defined as:

      CC = MAX((TARGET-XBAR)/(TARGET-LSL),(XBAR-TARGET)/USL)

    For this index, the smaller the better.

 7) Added the following commands:

       <dist> CHI-SQUARE GOODNESS OF FIT TEST Y
       <dist> CHI-SQUARE GOODNESS OF FIT TEST Y X
       <dist> CHI-SQUARE GOODNESS OF FIT TEST Y X1 X2

       <dist> KOLMOGOROV-SMIRNOV GOODNESS OF FIT TEST Y

    These commands test whether or not a data set
    comes from a specified distribution.  All distributions for
    which Dataplot can generate a cdf function are supported (there
    are 70+ such distributions in Dataplot).  The names are identical
    to the names used for the PROBABILITY PLOT command.

    A couple of notes on these commands:

    a) The KOLMOGOROV-SMIRNOV test is not supported for discrete
       distributions.

    b) The CHI-SQUARE test works with either binned or unbinned
       data.

       Dataplot supports 2 types of pre-binned data.  If your data
       has equal sized bins, then the X variable contains the
       mid-point of each bin.  If your bins may be of different
       sizes, then the X1 variable is the lower limit of each
       class and X2 is the upper limit of each class.  Unequal
       bins usually result from combining classes with low expected
       frequency.  

       It uses the same rules for binning as it does for the
       HISTOGRAM command.  That is, the class width is 0.3*S where S
       is the standard deviation of Y.  The upper and lower limits are
       the mean plus or minus 6 times the standard deviation.
       The BINNED command generates counts while the RELATIVE BINNED
       generates relative frequency.
   
       As with the histogram, you can override these defaults with the
       following commands:
   
          CLASS WIDTH <value>
          CLASS LOWER <value>
          CLASS UPPER <value>

    c) You need to specify shape parameters for distributions that
       require it.  For example,

         LET GAMMA = 2
         GAMMA CHI-SQUARE GOODNESS OF FIT Y

       The parameter names are equivalent to the names used for
       the PROBABILITY PLOT command.

       Location and shape parameters can be specified genrically
       for the CHI-SQUARE and KOLMOGOROV-SMIRNOV tests respectively
       by entering:

         LET CHSLOC = <value>
         LET CHSSCALE = <value>

         LET KSLOC = <value>
         LET KSSCALE = <value>

       These are optional.

 8) Added the following commands:

       2-SAMPLE CHI-SQUARE TEST Y1 Y2
       2-SAMPLE KOLMOGOROV-SMIRNOV TEST Y1 Y2

    These 2 commands test whether 2 data samples come from a
    common (unspecified) distribution.  Y1 and Y2 do not need
    to be the same size.

 9) Updated the TABULATE and CROSS-TABULATE commands.  The computed
    group id's and the value of the statistic are written to
    the file DPST1F.DAT (or dpst1f.dat on Unix).  This simplifies
    using the results in further analysis.  For example, to
    compute the group means and store them in a variable, do
    something like the following:

       TABULATE MEANS Y X
       SKIP 1
       READ DPST1F.DAT GROUPID YMEANS
       SKIP 0

    The CROSS-TABULATE is similar, except there are 2 group-id
    variables instead of 1.
    
10) Added the following command:

       LET Y2 X = BINNED Y  (or LET Y2 X = FREQUENCY TABLE Y)
       LET Y2 X = RELATIVE BINNED Y
                  (or LET Y2 X = RELATIVE FREQUENCY TABLE Y)

    Here, Y2 will contain the counts (or frequencies) and X will
    contain the bin mid-points.

    This command bins your data.  It uses the same rules as the
    histogram.  That is, the class width is 0.3*S where S is the
    standard deviation of Y.  The upper and lower limits are
    the mean plus or minus 6 times the standard deviation.
    The BINNED command generates counts while the RELATIVE BINNED
    generates relative frequency.

    As with the histogram, you can override these defaults with the
    following commands:

       CLASS WIDTH <value>
       CLASS LOWER <value>
       CLASS UPPER <value>

    The command SET RELATIVE HISTOGRAM <AREA/PERCENT> specifies
    whether or not relative binning is computed so that the area
    sums to 1 or so that the frequencies sum to 1.  The first option,
    which is the default, is useful when using the
    relative binning as an estimate of a probability distribution.
    The second option is useful when you want to see what percentage
    of the data falls in a given class.

----------------------------------------------------------------------
The following enhancements were made to DATAPLOT June - August   1998.
----------------------------------------------------------------------

 1) Added the following command:

      EMPIRICAL CDF PLOT Y

    This generates an empirical CDF plot.

 2) Made the following enhancements to the QWIN (the Microsoft
    95/NT version) device driver:

    Added support for "true color".  Previously, if the user
    had true color set for the display, the screen colors were
    all black (i.e., you couldn't see the output).  
    Note that true color is something you set from the 
    Windows 95/NT control panel, not something that Dataplot
    can set automatically.  That is, you set true color or
    standard VGA mode from the control panel and then you
    enter the appropriate Dataplot commands to support that
    mode.

    a) If you have your display set to true color, enter the
       following commands in the C:\DPLOGF.TEX file:

           SET QWIN COLOR DIRECT
           DEVICE 1 QWIN

       Note that the order is significant here.  The color model
       is set when the QWIN device is initialized, so the 
       SET QWIN COLOR command must come before the DEVICE 1 QWIN
       command.  Also, it is recommended that you put these commands
       in the DPLOGF.TEX file so that you do not get the initial
       blank screen where you cannot see the text that you type.
       The command SET QWIN COLRO VGA resets the default.

    b) For true color, the QWIN device driver supports the full
       complement of colors recognized by Dataplot (HELP COLORS
       for a description of the Dataplot color model).  The default
       VGA mode only supports 16 colors.

    c) The foreground and background colors for the text window
       can now be set for both standard VGA and true color modes.
       The following 2 commands, if used, should be entered
       after the SET QWIN COLOR <DIRECT/VGA> command and before
       the DEVICE 1 QWIN command:

          SET QWIN TEXT BACKGROUND COLOR <index>
          SET QWIN TEXT FOREGROUND COLOR <index>

       where <index> is an integer identifying the desired
       color (HELP COLOR gives the index to color mapping in
       Dataplot).  For VGA mode, <index> is restricted to 0 to
       15.  For DIRECT mode, <index> is restricted to 0 to 88.
       The default for both VGA and DIRECT mode is a white
       foreground on a black background.  The colors for the
       graphics window are set by the normal Dataplot COLOR
       commands (e.g., BACKGROUND COLOR BLUE, LINE COLOR RED).

 3) Added the following new matrix commands:

    The following 2 commands are used to obtain row or column
    statistics for a matrix.

       LET Y = MATRIX ROW <STAT> M
       LET Y = MATRIX COLUMN <STAT> M

    where <STAT> is one of: MEAN, MIDMEAN, TRIMMED MEAN,
    WINSORIZED MEAN, MEDIAN, SUM, PRODUCT, SD (or STANDARD DEVIATION),
    SD OF MEAN, VARIANCE, VARIANCE OF MEAN, RELATIVE VARIANCE,
    RELATIVE STANDARD DEVIATION, COEFFICIENT OF VARIATION, 
    AVERAGE ABSOLUTE DEVIAITION, MEDIAN ABSOLUTE DEVIATION, RANGE,
    MIDRANGE, MAXIMUM, MINIMUM, EXTREME, LOWER HINGE, UPPER HINGE,
    LOWER QUARTILE, UPPER QUARTILE, SKEWNESS, KURTOSIS, 
    AUTOCOVARIANCE, AUTOCORRELATION.

    The following command computes an overall mean for the matrix:

        LET A = MATRIX MEAN M

    The following command calculates the quadratic form of a
    vector and a matrix.  The quadratic form is: x'Mx where x
    is a vector and M is a matrix.  Quadratic forms are used
    frequently in  multivariate statistical calculations.

        LET A = QUADRATIC FORM M X

    The following command is a commonly used quadratic form:

        LET Y = DISTANCE FROM MEAN M

    This command generates:

         Di = (Xi - XMEAN)'SINV(Xi-XMEAN)

    where Xi is the ith row, XMEAN is a vector of the column
    means, and SINV is the inverse of the variance-covariance
    matrix.   That is, Di is the distance of the ith row of the
    matrix from the mean.  Note that in the Dataplot command, you
    specify the original matrix, not the variance-covariance matrix.

    The following command cacluate X*X' for the vector X.  The
    result is a pxp matrix where p is the number of rows of X.
    This computation is used in some multivariate analyses.

        LET M = VECTOR TIMES TRANSPOSE X

    The following command is used to create linear combinations:

        LET Y2 = LINEAR COMBINATION M C

    If the matrix M has p columns and n rows, C should be a vector
    with p rows.  This commands calculates:

        y2 = c(1)*M1 + c(2)*M2 + c(3)*M3 + ... + c(p)*Mp

    where M1, M2, ... are the columns of the matrix.  The result
    is a vector with n rows.

    The following commands are used to calculate various distance
    matrices:

        LET D = EUCLIDEAN ROW DISTANCE M
        LET D = EUCLIDEAN COLUMN DISTANCE M

        LET D = MAHALANOBIS ROW DISTANCE M
        LET D = MAHALANOBIS COLUMN DISTANCE M

        LET D = MINKOWSKY ROW DISTANCE M
        LET D = MINKOWSKY COLUMN DISTANCE M

        LET D = CHEBYCHEV ROW DISTANCE M
        LET D = CHEBYCHEV COLUMN DISTANCE M

        LET D = BLOCK ROW DISTANCE M
        LET D = BLOCK COLUMN DISTANCE M

    It is often desirable to scale the original matrix before
    calculating a distance matrix.  The following commands can
    be used to scale the original matrix:

         SET MATRIX SCALE <NONE/MEAN/SD/RANGE/ZSCORE>
         LET MSCAL = MATRIX ROW SCALE M
         LET MSCAL = MATRIX COLUMN SCALE M

    The SET MATRIX SCALE command is used to define the type of
    scaling to perform.  You can scale either across rows or down
    columns.
   
    The following command computes the pooled sample
    variance-covariance matrix for two matrices:

         LET MOUT = POOLED VARIANCE-COVARIANCE MATRIX MA MB
    
    Note that MA and MB should have the same number of columns.
    However, the number of rows can vary.

    The following computes a 1-sample Hotelling T-square test:

         LET A = 1-SAMPLE HOTELLING T-SQUARE M Y

    The 1-sample Hotelling t-square tests the following hypothesis:

         H0: U=U0

    Here, U0 is a vector of population means.  That is, the
    hypothesied means for each column of the matrix.  In the
    above syntax, M is a matrix containing the original data
    and Y is a vector containing the hypothesized means.  The
    returned parameter A contains the value of the Hotelling
    T-square test statistic.  The critical values corresponding
    to alpha = .90, .95, .99, and .995 are saved in the internal
    parameters B90, B95, B99, and B995.

    The following computes a 2-sample Hotelling T-square test:

         LET A = 2-SAMPLE HOTELLING T-SQUARE MA MB

    The 2-sample Hotelling t-square tests the following hypothesis:

         H0: U1=U2

    Here, U1 is a vector of population means for sample 1 and
    U2 is a vector of population means for sample 2.  In the
    above syntax, MA is a matrix containing the original data
    for sample 1 and MB is a matrix containing the original data
    for sample 2.  MA and MB must have the same number of columns.
    However, they can have a different number of rows.  The
    returned parameter A contains the value of the Hotelling
    T-square test statistic.  The critical values corresponding
    to alpha = .90, .95, .99, and .995 are saved in the internal
    parameters B90, B95, B99, and B995.

    The following 2 commands add or delete rows of a matrix:

        LET M = MATRIX ADD ROW M Y
        LET M = MATRIX DELETE ROW M ROWID

    Here, M is a matrix, Y is a variable with the number of rows
    equal to the number of columns in M, and ROWID is a scalar
    identifying the row to delete.

 4) Fixed a bug in the character fill for the QWIN device
    driver (DEVICE 1 QWIN for Windows 95/NT).  Removed the line
    CHARACTER FILL COLOR from the sample DPLOGF.TEX file (this
    caused problems for Postscript output).
  
 5) Added support for SP() in the LET STRING command.  SP() will
    be converted to a single space.  Previously, LET STRING packed
    out any spaces in the string.

 6) Added the command:

        LET Y2 = EXPONENTIAL SMOOTHING Y ALPHA

    This performs an exponential smoothing of Y.  The formual is:

        Y2(1) = Y(1)
        Y2(I) = ALPHA*Y(I) + (1-ALPHA)*Y(I-1),  I > 1

    ALPHA is the smoothing parameter and should be greater than
    0 and less than 1.

 7) The PROBE command is used to return the values of certain
    internal parameters and strings.  This command was updated
    so that the returned value is automatically saved.  If the
    returned value is an integer or real number, then the value
    is stored in the internal parameter PROBEVAL.  If the
    returned value is a string, then the value is stored in the
    internal string PROBESTR.  PROBESTR and PROBEVAL can then be
    used in the same way as other parameters and strings.

    This feature is typically used in macros.  For example, you
    might want to use the machine maximum value as a "missing
    value" indicator.  A host independent way of using this value
    would now be:

       PROBE CPUMAX
       LET MACHMAX = PROBVAL

    You could then use the parameter MACHMAX wherever you wanted
    to define a missing value.

 8) Multiplots create new 0 to 100 coordinate units for each
    subplot and character sizes are scaled according to this
    new subplot area.  Although this is generally desirable,
    sometimes the resulting character sizes are too small or
    distorted if the rows to columns ratio is too far from 1.

    As a convenience, the following command was added to allows
    all character sizes to be scaled when multiplotting is
    in effect:

        MULTIPLOT SCALE FACTOR 3
        MULTIPLOT SCALE FACTOR 1 2

    In the first syntax, both the height and width sizes are
    scaled (by 3 in this example) by the same factor.  In the
    second syntax, the height and width are scaled separately
    (the height by 1 and the width by 2 in this example).
    The word FACTOR is optional in the command.

    The scale factor is multiplied by the requested size.  For
    example, if the title size is 2 and the scale factor is 3,
    then the effective size will be 6.  The scale factor is
    ignored if multi-plotting is not in effect.

    This command allows character sizes to be easily adjutsted
    for multiplots without having to enter a number of separate
    size commands before the multiplot (and then after the 
    multiplot to return to normal values).

----------------------------------------------------------------------
The following enhancements were made to DATAPLOT January - May   1998.
----------------------------------------------------------------------

 1) Reliability/Extreme Value Updates

    a) Added the following commands for finding maximum likelihood
       estimates for distribution parameters.

          WEIBULL MAXIMUM LIKELIHOOD Y
          EXPONENTIAL MAXIMUM LIKELIHOOD Y
          DOUBLE EXPONENTIAL MAXIMUM LIKELIHOOD Y
          NORMAL MAXIMUM LIKELIHOOD Y
          LOGNORMAL MAXIMUM LIKELIHOOD Y
          PARETO MAXIMUM LIKELIHOOD Y
          GAMMA MAXIMUM LIKELIHOOD Y
          INVERSE GAUSSIAN MAXIMUM LIKELIHOOD Y
          GUMBEL MAXIMUM LIKELIHOOD Y (or EV1)
          POWER MAXIMUM LIKELIHOOD Y
          BINOMIAL MAXIMUM LIKELIHOOD Y
          POISSON MAXIMUM LIKELIHOOD Y

       At this time, only the parameter estimates are computed,
       that is no standard errors or confidence intervals for the
       estimates are computed.

   
       There are various synonyms for these commands.  For example,
   
             WEIBULL MAXIMUM LIKELIHOOD ESTIMATE Y
             WEIBULL MAXIMUM LIKELIHOOD Y
             WEIBULL MLE ESTIMATE Y
             WEIBULL MLE Y
   
       are all equivalent.  Similar synonyms apply to the other
       commands.

       The exponential case is an exception in that it does
       print confidence intervals.  It also supports type 1 and
       type 2 censored data.  For example, the full sample case
       is:

           SET CENSORING TYPE NONE    (this is the default)
           EXPONENTIAL MLE Y

       Type 1 censoring is censoring at a fixed time t0.  This
       is handled via:

           SET CENSORING TYPE 1
           LET TEND = <censor time>
           EXPONENTIAL MLE Y

       If you have data values that are censored before time t0, then
       create a TAG variable with 1 for failure times and 0 for
       censoring times.  You would the  enter:

           EXPONENTIAL MLE Y TAG

       Type 2 censoring is censoring after R failures have been
       observed.  This case is handled via:

           SET CENSORING TYPE 2
           EXPONENTIAL MLE Y TAG

       where TAG is variable with 1 for failure times and 0 for
       censoring times.

       Related to this are the commands
   
             DEHAAN Y
             CME Y
   
       These generate parameter estimates for the generalized
       Pareto distribution for extreme value applications.
   
    b) Added the following commands:

       1)   LET Y = CUMULATIVE HAZARD X TAG
            LET Y = HAZARD X TAG
   
          where X is a list of failure times and TAG is an array
          that identifies the value as a failure time (TAG = 1) or
          a censoring time (TAG = 0).
   
       2)   LET Y = INTERARRIVAL TIMES X
   
          where X is a list of failure times.  This is similar to
          the SEQUENTIAL DIFFERENCE command in that it calculates
          X(I)-X(I-1).  However, it sorts the data first and the
          first interarrival time is set equal to X(1).
   
       3)   LET Y = CUMULATIVE AVERAGE X
            LET Y = CUMULATIVE MEAN X
   
          As the name implies, this computes the cumulative mean of
          a variable.  One use of this is to compute cumulative mean
          time between failures for reliability data.
   
       4)   LET Y = REVERSE X
            LET Y = FLIP X
   
          This reverses the order of a variable (i.e., Y(1)=X(N),
          Y(2)=X(N-1), and so on).  For example, if you want to
          sort from high to low instead of low to high, you can enter
      
             LET Y = SORT X
             LET Y = REVERSE Y

       5)    LET ALPHA = <value>
             LET BETA = <value>
             LET Y = POWER LAW RANDOM NUMBERS FOR I = 1 1 N

          This generates N failure times from a non-homogeneous
          Poisson process following the power law.  That is,

             M(t) = alpha*t**beta     alpha, beta > 0

          where M(t) is the expected number of failures at time
          t.  The random failure times are generated from the
          formula for the interarrival times (i.e., the CDF for
          the waiting time for the next failure given a failure at
          time T):

             F (t) = 1 - EXP(-ALPHA*[(T+t)**BETA-T**BETA]
              T

    c) The following 2 plots were added:

          KAPLAN MEIER PLOT Y TAG
          MODIFIED KAPLAN MEIER PLOT Y TAG
   
       Here, Y is a list of failure times and TAG identifies censored
       data.  A value of 1 for TAG means that the corresponding Y
       value is a failure time and a value of 0 means that the 
       corresponding Y value was censored.  The TAG variable is
       optional (if omitted, no censoring is performed).
   
       Kaplan-Meier estimates are discussed in most texts in survival
       or reliability analysis.  The modified Kaplan-Meier is a
       slightly adjusted form of the estimate.
   
       The X axis of the plot is failure time and the Y axis is
       an estimate of survival (or reliability).  Some analysts
       prefer that the Y axis be CDF estimate (i.e., 1 - Survival).
       Enter the command
   
         SET KAPLAN MEIER CDF
   
       to specify this (and SET KAPLAN MEIER RELIABILITY to reset it).
   
       If you want the numeric Kaplan Meier estimates, do
   
          KAPLAN MEIER PLOT Y TAG
          LET RELI = YPLOT
          LET FAILTIME = XPLOT
   
       The variables RELI and FAILTIME can be used in subsequent
       commands to do further analysis.

    d) The following plots were added:

          EXPONENTIAL HAZARD PLOT Y TAG
          NORMAL HAZARD PLOT Y TAG
          LOGNORMAL HAZARD PLOT Y TAG
          WEIBULL HAZARD PLOT Y TAG

       Hazard plots are similar to probability plots.  However,
       they can be used with censored data and are commonly used
       in reliability studies.

    e) Added the following command:

           DUANE PLOT Y
   
       Given a set of failure times T, the Duane plot is 
       Ti/i (where i is the index from 1 to N) versus Ti on
       a log-log scale.  You do not need to specify XLOGON or YLOG ON
       as Dataplot does this automatically.  Dataplot also resets
       the original values for these switches after the Duane plot
       is completed.
   
       A line is fit to the plotted data.  Various parameters from
       the fit are saved as internal parameters (enter
       STATUS PARAMETERS after the DUANE PLOT to see what they are).
       A typical use would be:
   
           READ FAILURE.DAT Y
           Y1LABEL CUMULATIVE MEAN TIME BETWEEN FAILURE
           X1LABEL FAILURE TIME
           CHARACTER X BLANK
           LINE BLANK SOLID
           DUANE PLOT Y
           JUSTIFCATION CENTER
           MOVE 50 7
           TEXT SLOPE OF FITTED LINE = ^BETA 
           MOVE 50 4
           TEXT INTERCEPT OF FITTED LINE = ^ALPHA 

    f) The following command was added:

           RELIABILITY TRENDS TEST Y
   
       This command is used in reliability applications to determine
       if repair times show a significant trend.  It computes the
       following 3 tests:
   
       a) Reverse Arrangement Test
       b) Military Handbook Test
       c) Laplace Test
   
       The last 2 tests require the censoring time.  This is entered
       (before the RELIABILITY TRENDS TEST) as:
   
           LET TEND = <value>
   
       The value of TEND should be greater than the maximum value
       of the response variable.

    Some of the Probability and Recipe updates discussed below are
    also relevant to reliability applications.

 2) Probability Updates

    a) Added optional location and scale parameters for many of the
       probability functions.

       Specifically, the following functions now support both location
       and scale parameters:
   
          CAUCDF, CAUPDF, CAUPPF, CAUSF
          DEXCDF, DEXPDF, DEXPPF, DEXSF
          DGACDF, DGAPDF, DGAPPF
          DWECDF, DWEPDF, DWEPPF
          EV1CDF, EV1PDF, EV1PPF
          EV2CDF, EV2PDF, EV2PPF
          EWECDF, EWEPDF, EWEPPF
          EXPCDF, EXPPDF, EXPPPF
          FLCDF, FLPDF, FLPPF
          GAMCDF, GAMPDF, GAMPPF
          GEVCDF, GEVPDF, GEVPPF
          GGDCDF, GGDPDF, GGDPPF
          GLOCDF, GLOPDF, GLOPPF
          HFCCDF, HFCPDF, HFCPPF
          HFNCDF, HFNPDF, HFNPPF
          IGCDF, IGPDF, IGPPF
          LGACDF, LGAPDF, LGAPPF
          LGNCDF, LGNPDF, LGNPPF
          LLGCDF, LLGPDF, LLGPPF
          LOGCDF, LOGPDF, LOGPPF
          NORCDF, NORPDF, NORPPF
          RIGCDF, RIGPDF, RIGPPF
          WEICDF, WEIPDF, WEIPPF
   
       NOTE: The help files and Reference Manual refer to the
             location parameter for the 2-parameter inverse gaussian
             (IG), reciprocal inverse gaussian (RIG), Wald (WAL), and
             fatigue life (FL) distributions.  This is actually the
             scale parameter for these distributions.
   
       The following added a location parameter only:
   
          HFLCDF, HFLPDF, HFLPPF
          PA2CDF, PA2PDF, PA2PPF
          PARCDF, PARPDF, PARPPF
          PEXCDF, PEXPDF, PEXPPF
          PLNCDF, PLNPDF, PLNPPF
          PNRCDF, PNRPDF, PNRPPF
          VONCDF, VONPDF, VONPPF
          WALCDF, WALPDF, WALPPF
          WCACDF, WCAPDF, WCAPPF
   
       The following added a scale parameter only:
   
          GEPCDF, GEPPDF, GEPPPF
          POWCDF, POWPDF, POWPPF
   
       The following added a lower and upper limit (which is then
       converted by Dataplot into location and scale parameters).
   
          UNICDF, UNIPDF, UNIPPF, UNISF
          BETCDF, BETPDF, BETPPF, BETSF

    b) Added the following hazard and cumulative hazard functions:

       NOTE: In the following, LOC and SCALE specify location and
             scale parameters respectively and are optional.  For the
             uniform, the lower and upper limits are specified (and
             are converted by Dataplot to location and scale
             parameters) and are also optional.  All other parameters
             are the standard shape parameters for the distribution.
        
          UNIHAZ(X,LOWER,UPPER)    - uniform hazard function
          UNICHAZ(X,LOWER,UPPER)   - uniform cumulative hazard function
   
          NORHAZ(X,LOC,SCALE)       - normal hazard function
          NORCHAZ(X,LOC,SCALE)      - normal cumulative hazard function
          LGNHAZ(X,SD,LOC,SCALE)    - normal hazard function
          LGNCHAZ(X,SD,LOC,SCALE)   - normal cumulative hazard function
          PNRHAZ(X,SD,P,LOC)        - power normal hazard function
          PNRCHAZ(X,SD,P,LOC)       - power normal cumulative hazard
                                      function
          PLNHAZ(X,SD,P,LOC)        - power log-normal hazard function
          PLNCHAZ(X,SD,P,LOC)       - power log-normal cumulative
                                      hazard function
   
          EXPHAZ(X,LOC,SCALE)        - exponential hazard function
          EXPCHAZ(X,LOC,SCALE)       - exponential cumulative hazard
                                       function
          WEIHAZ(X,GAMMA,LOC,SCALE)  - Weibull hazard function
          WEICHAZ(X,GAMMA,LOC,SCALE) - Weibull cumulative hazard
                                       function
          EWEHAZ(X,GAMMA,THETA,LOC,SCALE)  - exponentiated Weibull
                                             hazard function
          EWECHAZ(X,GAMMA,THETA,LOC,SCALE) - exponentiated Weibull
                                             cumulative hazard function
          GAMHAZ(X,GAMMA,LOC,SCALE)  - gamma hazard function
          GAMCHAZ(X,GAMMA,LOC,SCALE) - gamma cumulative hazard function
          IGAHAZ(X,GAMMA,LOC,SCALE)  - inverted gamma hazard function
          IGACHAZ(X,GAMMA,LOC,SCALE) - inverted gamma cumulative hazard
                                       function
          GGDHAZ(X,GAMMA,K,LOC,SCALE)  - generalized gamma hazard
                                         function
          GGDCHAZ(X,GAMMA,K,LOC,SCALE) - generalized gamma cumulative
                                         hazard function
   
          EV1HAZ(X,GAMMA,LOC,SCALE)  - Gumbel hazard function
          EV1CHAZ(X,GAMMA,LOC,SCALE) - Gumbel cumulative hazard
                                       function
          EV2HAZ(X,GAMMA,LOC,SCALE)  - Frechet hazard function
          EV2CHAZ(X,GAMMA,LOC,SCALE) - Frechet cumulative hazard
                                       function
          GEPHAZ(X,GAMMA,SCALE)      - generalized Pareto hazard
                                       function
          GEPCHAZ(X,GAMMA,SCALE)     - generalized Pareto cumulative
                                       hazard function
   
          IGHAZ(X,GAMMA,LOC,SCALE)   - inverse gaussian hazard function
          IGCHAZ(X,GAMMA,LOC,SCALE)  - inverse gaussian cumulative
                                       hazard function
          WALHAZ(X,GAMMA,LOC)        - Wald hazard function
          WALCHAZ(X,GAMMA,LOC)       - Wald cumulative hazard function
          RIGHAZ(X,GAMMA,LOC,SCALE)  - reciprocal inverse gaussian
                                       hazard function
          RIGCHAZ(X,GAMMA,LOC,SCALE) - reciprocal inverse gaussian
                                       cumulative hazard function
          FLHAZ(X,GAMMA,LOC,SCALE)   - fatigue life hazard function
          FLCHAZ(X,GAMMA,LOC,SCALE)  - fatigue life cumulative hazard
                                       function
   
          PARHAZ(X,GAMMA,LOC)        - Pareto hazard function
          PARCHAZ(X,GAMMA,LOC)       - Pareto cumulative hazard
                                       function
          ALPHAZ(X,ALPHA,BETA)       - alpha hazard function
          ALPCHAZ(X,ALPHA,BETA)      - alpha cumulative hazard function
          PEXHAZ(X,ALPHA,BETA)       - exponetial power hazard function
          PEXCHAZ(X,ALPHA,BETA)      - exponential power cumulative
                                       hazard function
   
       NOTE: The hazard function is defined as:
   
                h(x) = pdf(x)/(1-cdf(x))
   
             and the cumulative hazard function is defined as:
   
                H(x) = -log(1-cdf(x))
   
             where pdf and cdf are the probability density and
             cumulative distribution functions respectively.  These
             functions can be used to generate hazard and cumulative
             hazard functions for distributions that Dataplot does
             not support directly.

    c) Added the mixture of 2 normal probability functions.
       Specifically,

         NORMXCDF(X,U1,SD1,U2,SD2,PMIX)
         NORMXPDF(X,U1,SD1,U2,SD2,PMIX)
         NORMXPPF(P,U1,SD1,U2,SD2,PMIX)
   
       where U1 and SD1 are the mean and standard deviation of the
       first normal distribution, U2 and SD2 are the mean and standard
       deviation of the second normal distribution, and PMIX is
       the mixing proportion (between 0 and 1).
   
       You can generate a probability plot as follows:
   
         LET U1 = <value>
         LET SD1 = <value>
         LET U2 = <value>
         LET SD2 = <value>
         LET P = <value>
         NORMAL MIXTURE PROBABILITY PLOT Y
   
       You can generate random numbers as follows:
   
         LET U1 = <value>
         LET SD1 = <value>
         LET U2 = <value>
         LET SD2 = <value>
         LET P = <value>
         LET Y = NORMAL MIXTURE RANDOM NUMBERS FOR I = 1 1 1000

    d) Added the inverted gamma probability functions:

         IGACDF(X,GAMMA,LOC,SCALE)
         IGAPDF(X,GAMMA,LOC,SCALE)
         IGAPPF(P,GAMMA,LOC,SCALE)
   
       This is not really a new function.  It is simply the
       generalized gamma function with the second shape parameter
       set to -1.  We added it as a separate set of functions since
       it is a common distribution in certain applications.
   
       Also added:
   
          LET GAMMA = <value>
          INVERSE GAMMA PROBABILITY PLOT
   
          INVERSE GAMMA PPCC PLOT

    e) Added following discrete PPCC PLOT commands:
   
           BINOMIAL PPCC PLOT
           NEGATIVE BINOMIAL PPCC PLOT
           LOGARIOTHMIC SERIES PPCC PLOT
   
       For the binonial and negative binomial, N must be specified
       (and then P is computed).
   
    f) Fixed the PROBABILITY PLOT X Y and PPCC PLOT X Y commands
       to handle zero count bins correctly.

 3) Recipe Updates

    a) Added support for multi-factor recipe fits.  For example,
       a common model is:

          Y = A0 + A1*X1 + A2*X1**2 + A3*X2 + A4*X2**2 + A5*X1*X2

       In Dataplot, the recipe analysis could be done as follows:

         READ FILE.DAT Y X1 X2 BATCH
         READ FILE2.DAT XP1 XP2
         LET X1S = X1*X1
         LET X2S = X2*X2
         LET X1X2 = X1*X2
         LET XP1S = XP1*XP1
         LET XP2S = XP2*XP2
         LET XP1P2 = XP1*XP2
         .
         RECIPE FIT FACTORS 5
         RECIPE FIT Y X1 X1S X2 X2S X1X2 BATCH XP1 XP1S XP2 XP2S XP1P2
         PRINT TOL

       XP1 and XP2 are the points at which you want the tolerance
       values computed.  If they are omitted, then the tolerance
       values are computed at the unique points in the design
       matrix (i.e., all the unique combinations of X1 and X2).
       The BATCH variable is a batch identifier and is optional.
       X1 and X2 must have the same number of points and XP1 and
       XP2 should have the same number of points.  However, X1 and
       XP1 do not need to have the same number of points (and they
       usually will not).  The primary output from the RECIPE command
       is the tolerance values (by default, saved in TOL).  Commands
       for setting the probability confidence and content are
       the same as for the 1-factor recipe fit.

    b) Recipe is generally used in the context of setting tolerance
       limits as defined in MIL-17 Handbook.  A number of other
       statistical techniques are defined in this handbook.
       Dataplot had previously added support for the Grubbs test,
       Levene's test for shifts in scale, and the F test for shifts
       in location.  The following additional tests defined in the
       handbook are now supported as well:
       
          ANDERSON-DARLING <DIST> TEST Y
          where DIST is: NORMAL, LOGNORMAL, WEIBULL, EXTREME VALUE

          ANDERSON-DARLING K-SAMPLE TEST Y X

          WEIBULL MAXIMUM LIKELIHOOD Y
          B BASIS <DIST> TOLERANCE LIMIT Y 
          A BASIS <DIST> TOLERANCE LIMIT Y 
          where DIST is: NORMAL, LOGNORMAL, WEIBULL, NON-PARAMETRIC
 
       The Anderson-Darling 1-sample test is used to determine if a
       data set can be assumed to come from a certain distribution.
       The EXTREME VALUE distribution is the type 1 extreme value
       distribution.  The k-sample Anderson-Darling test is used
       to test if groups of data are the same (in the sense of
       coming from the same distribution with common location and
       scale).  It is typically used to determine if data coming
       different batches can be treated as if they came from the
       same batch.  The WEIBULL MAXIMUM LIKELIHOOD command is used
       to generate maximum likelihood estimates of the 2-parameter
       Weibull distribution (the shape and scale parameters).
       The B BASIS and A BASIS commands are used to generate 
       b basis and a basis tolerance limits for a variable
       for a few common distributions.

       See the MIL-17 Handbook for more information on these
       techniques.

 4) Matrix Updates

    Modified matrix commands to make more efficient use of
    storage.  Upped default maximum number of rows from 1,500 to
    3,000.

    Added a DIMENSION MATRIX COLUMNS <val> and DIMENSION MATRIX ROWS
    <val> command.  This is used to dimension temporary matrices
    in the matrix routines.  Note that unlike the DIMENSION command
    for variables, this command does not erase any previously
    created data.  It is only used to dimension temporary matrices
    in the matrix code, not to store the original data.
    Each temporary matrix has a maximum of 920,000/3 elements.
    However, you cannot dimension the number of rows in a matrix
    to be greater than the number of rows in a variable.


 5) Miscellaneous Updates

    a) Added the commands:
   
          LINE <SAVE/RESTORE>
          CHARACTER <SAVE/RESTORE>
   
       These were motivated by the graphical user interface, but they
       can be used directly by the user as well.
   
    b) Added the commands:
   
          SET PRINTER <id>
          PROBE PRINTER <id>
   
       These allow the user to specify the printer name for the
       PP command.  It is currently supported for the Unix and
       Windows 95/NT versions.  It would be straightforward to support
       on other systems as well.
   
    c) The ANOVA code was significantly rewritten.
   
       1) The maximum number of factors was increased from 5 to 10.
       2) The output was modified.  Specifically, an ANOVA table was
          added other output was re-arranged.
       3) Some information is now written out to files DPST1F.DAT
          and DPST2F.DAT.  This is usefule if you need to use some
          of the ANOVA quantities in further analysis.
       4) A check is now made to see if you have a balanced design
          (i.e., all cells have an equal number of observations).
          A warning message will be printed if an unbalanced case is
          detected.  Note that the Dataplot calculations are based on
          the assumption of balanced data.  However, it will still
          run the ANOVA for the unbalanced case (the output will
          not be accurate in this case).
   
    d) Added CODED as synonym for CODE (LET Y = CODE X or
       LET Y = CODED X).
   
    e) Modified data reads so that non-printing characters are
       converted to spaces.
   
    f) The BOOTSTRAP PLOT command was augmented so that the following
       parameters are now automatically saved:
   
          BMEAN  - mean of the plotted bootstrap values
          BSD    - standard deviation of the plotted bootstrap values
          B001   - the 0.1% percentile of the plotted bootstrap values
          B005   - the 0.5% percentile of the plotted bootstrap values
          B01    - the 1.0% percentile of the plotted bootstrap values
          B025   - the 2.5% percentile of the plotted bootstrap values
          B05    - the 5.0% percentile of the plotted bootstrap values
          B10    - the 10% percentile of the plotted bootstrap values
          B20    - the 20% percentile of the plotted bootstrap values
          B80    - the 80% percentile of the plotted bootstrap values
          B90    - the 90% percentile of the plotted bootstrap values
          B95    - the 95% percentile of the plotted bootstrap values
          B975   - the 97.5% percentile of the plotted bootstrap values
          B99    - the 99% percentile of the plotted bootstrap values
          B995   - the 99.5% percentile of the plotted bootstrap values
          B999   - the 99.9% percentile of the plotted bootstrap values
   
      These values are typically used in setting confidence levels.
   
      Also, the BOOTSTRAP COEFFICENT OF VARIATION PLOT and
      BOOTSTRAP RELATIVE VARIANCE PLOT commands were added.
   
    g) Some code not used by the user was added for the graphical
       front-end.

    h) Raised the maximum number of lines in a loop from 200 to 500.

    i) Fixed some minor bugs.

----------------------------------------------------------------------
The following enhancements were made to DATAPLOT October - December 1997.
----------------------------------------------------------------------

 1) The WRITE command was updated to allow

        WRITE VARIABLES ALL (or WRITE ALL VARIABLES)

    This was added to support some updates to the frontend, but it
    can be used in the command line as well.  Currently, a maximum
    of 25 variables will be printed.

 2) An update was made to allow exponential notation in commands
    where a number or parameter is expected.  For example,

       LET Y = DATA 1.2E-7  2.0E3  4.26E+4

    The above example shows the 3 forms of the E notation that
    are currently recognized.  Note that using "D" instead of
    "E" is not currently supported.

    Parsing of expressions (e.g., transformations under LET,
    definition of functions, FIT expressions) is not yet supported.
    That is,

       LET Y(1) = 1.2E-3

    does NOT work as of yet.  The parsing of expresions under
    LET is handled in a different part of the code.  Support
    may be added at a later time.

 3) The command SKIP AUTOMATOC or SKIP ---- can be used to
    skip all lines in a data file until the first line
    containing a  "----" string is found.  It does not have to
    start in column 1.  This was added primarily to
    to support the data files provided with Dataplot.  However,
    you can use this with your own data files as well.

    If no line with "----" is found, Dataplot rewinds the file
    and tries to read data starting with the first line of the
    file.

    This option only applies if the read is performed on a file.
    If the read is from the terminal, SKIP AUTOMATIC is
    equivalent to a SKIP 0.

 4) The following 2 commands were added:

       AUTOCOMOVEMENT PLOT Y
       CROSS COMOVEMENT PLOT Y1 Y2

    These are similar to the AUTOCORRELATION PLOT and the
    CROSS CORRELATION PLOT commands.  However, they are based
    on the COMOVEMENT statistic rather than the correlation
    statistic.   At this time, no reference lines indicating
    statistical significance are drawn.

 5) The following special function was added:

       LET A = PSIFN(X,K)  - scaled k-th derivative of the PSI (or
                             DIGAMMA) function

    Note that this computes a SCALED version of the function,
    specifically

       ((-1)**(K+1)/GAMMA(K+1))*PSI(X,K)

    where GAMMA is the gamma function and PSI(X,K) is the unscaled
    function.  Also, it is the k-th derivative of PSI, not of
    the log gamma function.  That is, K=1 computes the 
    trigamma function, not the digamma function.

 6) The DELETE command was modified so that blanked out values
    are reset to zero instead of machine negative infinity.

 7) Added IF EXIST command.  An IF NOT EXIST command was added
    several years ago.  This commands works as follows:

       IF A EXIST
         PRINT A
       END OF IF

    where A is a parameter.  A will be printed if it already
    exists.

 8) Added the command REPLOT to regenerate the most recently
    created plot.  Although this was motivated by enhancements
    to the graphical user interface, it can be useful in command
    line mode as well.

----------------------------------------------------------------------
The following enhancements were made to DATAPLOT September       1997.
----------------------------------------------------------------------

 1) Added a SLEEP <n> command to pause for <n> seconds.  This is useful for
    macros so plots can be displayed for a given period of time without
    requiring user intervention to continue (as needed by the PAUSE command).
    This command is platform dependent and is currently implemented for Unix
    and Windows 95/NT versions.

    Added a CD command to change the current directory.  This command is
    platform dependent and has currently been implemented for the
    Windows 95/NT version.  This command is particularly useful for the
    Windows 95/NT version since when Dataplot is executed from a screen
    icon, the default directory is the the directory where the Dataplot
    executable resides.  The SYSTEM command cannot be used to change the
    current directory since a "SYSTEM CD <directory>" does not persist
    after the SYSTEM command completes execution.

 2) Added Mark Vangel's RECIPE code.  RECIPE stands for "REgression Confidence
    Intervals on PErcentiles".  It is used to calculate basis values for
    regression models with or without a random "batch effect".

    A full discussion of RECIPE is beyond the scope of this brief news item.
    Complete technical documentation for RECIPE is available at the following
    Web site:

       http://www.itl.nist.gov/div898/software/recipe/

    This discusses RECIPE in general, not the Dataplot implementation.

    The basic RECIPE commands are:

      RECIPE FIT Y X BATCH XPRED       - linear regression, polynomial models
      RECIPE ANOVA Y  X1 ... XK  BATCH - ANOVA, multilinear models

    The primary output from the RECIPE command is a set of tolerance values.
    These are saved in the internal Dataplot variable TOL by default.  This
    variable can be plotted and manipulated like any other Dataplot variable.

    The RECIPE documentation (on the above web site) also discusses a program
    called SIMCOV.  SIMCOV is used to determine whether or not Saitterthwaite
    approximation is adequate in determing the tolerance values.  SIMCOV 
    uses simulation to determine this.  The following commands implement
    the SIMCOV program in Dataplot.

      RECIPE SIMCOV FIT Y X BATCH XPRED    - linear regression, polynomial models
      RECIPE SIMCOV ANOVA Y  X1 ... XK  BATCH - ANOVA, multilinear models

    The following commands set switches for the RECIPE and SIMCOV analyses.

      RECIPE FIT DEGREE <N>            - polynomial degree for RECIPE FIT
      RECIPE FACTORS <N>               - number of factors for RECIPE ANOVA
      RECIPE OUTPUT <VAR>              - name of variable to contain computed 
                                         tolerance values
      RECIPE SATTERTHWAITE <YES/NO>    - specifies whether or not Satterthwaite
                                         approximation is used
      RECIPE PROBABILITY CONTENT <VAL> - value for probability content
      RECIPE CONFIDENCE <VAL>          - value for probability content
      RECIPE CORRELATION <N>           - the number of correlation values at
                                         which to compute SIMCOV probabilities
      RECIPE SIMCOV REPLICATES <N>     - the number of replications for SIMCOV
      RECIPE SIMPVT REPLICATES <N>     - the number of replications for SIMPVT
                                         (applies when Satterthwaite
                                         approximation not used)

    In addition, the following commands were added to support RECIPE
    analyses (these techniques recommended by the MIL-HDBK-17E):

      GRUBB TEST Y        - performs the Grubb test for outliers
      LEVENE TEST Y X     - performs the Levene test for homogenuous variances
                            (similar, but more robust for non-normal distributions,
                            to Bartlett's test)
      F LOCATION TEST Y X - performs an F test for homogenuous locations

    These capabilities were originally implemented as the macros GRUBB.DP, LEVENE.DP,
    and FTESTLOC.DP which have been added to the Dataplot macro directory.

    In addition, four data sets (VANGEL31.DAT, VANGEL32.DAT, VANGEL33.DAT, and
    VANGEL34.DAT) that can be analyzed with RECIPE were added to the Dataplot
    data sets directory.  Corresponding macros (VANGEL31.DP, VANGEL32.DP, VANGEL33.DP,
    and VANGEL34.DP) were added to the Dataplot programs directory.

 3) The following control charts were added:

      EWMA  CONTROL CHART Y      - exponentially weighted moving average control chart
      EWMA  CONTROL CHART Y X    - exponentially weighted moving average control chart
      MOVING AVERAGE CONTROL CHART Y   - moving average control chart
      MOVING AVERAGE CONTROL CHART Y X - moving average control chart
      MOVING RANGE CONTROL CHART Y     - moving range control chart
      MOVING RANGE CONTROL CHART Y X   - moving range control chart
      MOVING SD CONTROL CHART Y        - moving standard deviation control chart
      MOVING SD CONTROL CHART Y X      - moving standard deviation control chart

    These work in a similar fashion to previously available control charts.

    An important feature of all control charts was omitted from previous
    documentation (this feature has actually been available for quite some time).
    Dataplot allows you to specify the target and lower and upper
    control limits by entering the commands:

       LET A = TARGET = <value>       - the target value
       LET A = USL <value>            - the upper control limit
       LET A = LSL <value>            - the lower control limit

    The data is drawn as trace 1, the target value and limits derived from the
    data are drawn as traces 2, 3, and 4, and the user specified target and
    control limits (if given) are drawn as traces 5, 6, and 7.  You can control
    which of these values are actually plotted by setting the LINE and CHARACTER
    commands appropriately.

 4) The REPEAT GRAPH, SAVE GRAPH, and LIST GRAPH commands that were previously
    added for X11 installations have been extended to support the Microsoft
    Windows 95/NT implementation.  The commands work on Windows 95/NT as they
    do for Unix.  The primary difference is that the plots are saved in
    Windows bitmap format.  The Windows 95/NT still needs a little tidying up
    (the default positioning isn't ideal yet), but it is functional.

 5) The following special functions were added:

     LET A = CGAMMA(XR,XC)     - real component of complex gamma
     LET A = CGAMMAI(XR,XC)    - complex component of complex gamma

     LET A = CLNGAM(XR,XC)     - real component of complex log gamma
     LET A = CLNGAMI(XR,XC)    - complex component of complex log gamma

     LET A = CBETA(AR,AC,BR,BC)  - real component of complex beta
     LET A = CBETAI(AR,AC,BR,BC) - complex component of complex beta

     LET A = CLNBETA(AR,AC,BR,BC)  - real component of complex beta
     LET A = CLNBETAI(AR,AC,BR,BC) - complex component of complex beta

     LET A = CPSI(XR,XC)       - real component of complex psi
     LET A = CPSII(XR,XC)      - complex component of complex psi

     LET A = CHM(X,A,B)        - confluent hypergeometric M function
     LET A = HYPERGEO(X,A,B,C) - hypergeometric function (for restricted values of X,
                                 convergent case x < 1)

     LET A = PBDV(X,A)         - parabolic cylinder function (Dv)
     LET A = PBDV1(X,A)        - derivative of parabolic cylinder
                                 function (Dv)
     LET A = PBVV(X,A)         - parabolic cylinder function (Vv)
     LET A = PBVV1(X,A)        - derivative of parabolic cylinder
                                 function (Vv)
     LET A = PBWA(X,A)         - parabolic cylinder function (Wa) (only for X < 5)
     LET A = PBWA1(X,A)        - derivative of parabolic cylinder
                                 function (Wa) (only for X < 5)

     LET A = BER(XR)           - Real component of Kelvin Ber function
     LET A = BERI(XR)          - Complex component of Kelvin Ber function
     LET A = BER1(XR)          - Real component of derivative of Kelvin Ber
                                 function
     LET A = BERI1(XR)         - Complex component of derivative of Kelvin Ber
                                 function
     LET A = KER(XR)           - Real component of Kelvin Ker function
     LET A = KERI(XR)          - Complex component of Kelvin Ker function
     LET A = KER1(XR)          - Real component of derivative of Kelvin Ker
                                 function
     LET A = KERI1(XR)         - Complex component of derivative of Kelvin Ker
                                 function

     LET A = ZETA(S)           - Riemann zeta function - 1 (s > 1)
     LET A = ETA(S)            - eta function - 1 (s >= 1)
     LET A = CATLAN(S)         - Catlan Beta function - 1 (s >= 1)

     LET A = BINOMIAL(N,M)     - Binomial coefficent of N and M
     LET A = BINOM(N,M)        - Binomial coefficent of N and M

     LET A = EN(N)             - Euler number of order N
     LET A = EN(X,N)           - Euler polynomial of order N
     LET A = BN(N)             - Bernoulli number of order N
     LET A = BN(X,N)           - Bernoulli polynomial of order N

     LET A = BERNOULLI NUMBERS FOR I = 1 1 N - Bernoulli numbers
     LET A = EULER NUMBERS FOR I = 1 1 N     - Euler numbers


----------------------------------------------------------------------
The following enhancements were made to DATAPLOT July            1997.
----------------------------------------------------------------------

 1. Added support for printing tic mark labels in exponential
    format for linear scales.  Enter the command

       ...TIC MARK LABEL FORMAT EXPONENTIAL

    The default is to write the number with an E15.7 format.
    To control the number of decimal points, enter the command

       ...TIC MARK LABEL DECIMAL <n>

    where <n> is a positive integer.  For example, if
    <n> is 4, the number is printed with an E12.4 format.

 2) For the diagrammatic graphics commands that draw a figure
    (AND, AMPLIFIER, ARC, ARROW, BOX, CAPACITOR, CIRCLE, DIAMOND,
    CUBE, ELLIPSE, GROUND, HEXAGON, INDUCTOR, LATTICE, NOR, OR,
    OVAL, PYRAMID, POINT, RESISTOR, SEMI-CIRCLE, TRIANGLE)
    were updated to include a "DATA" option (similar to the
    DRAWDATA and MOVEDATA commands).  This "DATA" option draws the
    plot in units of the most recent plot rather than 0 to 100
    screen units.  For example, ELLIPSE DATA <list of points>
    draws the ellipse in units of the most recent plot.

    Similar to the DATA option, there is a RELATIVE option in the
    above commands.  Although this capability has actually been
    available in Dataplot for quite some time, it was left out
    of the documentation for the diagrammatic graphics commands.
    Relative drawing means that the first point is drawn in
    absolute units and all subsequent points are relative to the
    prior point.  For example DRAW RELATIVE 10 10 2 3
    would draw a line from (10,10) to (12,13).

    The word "DATA" should come before the word "RELATIVE"
    in these commands.  There are actually 4 forms to these
    commands.  For example,

       ELLIPSE X1 Y1 X2 Y2 X3 Y3
       ELLIPSE DATA X1 Y1 X2 Y2 X3 Y3
       ELLIPSE RELATIVE X1 Y1 X2 Y2 X3 Y3
       ELLIPSE DATA RELATIVE X1 Y1 X2 Y2 X3 Y3

    The first form draws in absolute screen 0 to 100 units,
    the second form draws in absolute units of the most recent plot,
    the third form draws in relative screen 0 to 100 units, and
    the fourth form draws in relative units of the most recent plot.

 3) POLYGON was added to the list of diagrammatic commands.  This
    command takes the following form:

      POLYGON X Y       <SUBSET/EXCEPT/FOR qualification>
      POLYGON DATA X Y  <SUBSET/EXCEPT/FOR qualification>
      POLYGON RELATIVE X Y       <SUBSET/EXCEPT/FOR qualification>
      POLYGON RELATIVE DATA X Y  <SUBSET/EXCEPT/FOR qualification>

    The first form plots the polygon in 0 to 100 screen units while
    the second form plots the data in units of the most recent plot.
    The third and fourth forms are similar, but they use relative 
    coordinates (the first coordiante pair is in absolute units,
    the remaining are coordinates relative to the previous point).
    Note that X and Y are arrays, not lists of points as used by
    the other diagrammatic graphics commands.  Since these are
    arrays, the SUBSET, EXCEPT, and FOR qualifications can be
    applied to the list of points, although this is not common
    in the context of this command.  

    Setting the last point to the first point (i.e., closing the
    polygon) is not required since Dataplot does this automatically.
    As with the other diagrammatic graphics commands, the attributes
    of the border of the polygon are set via the first setting
    of the LINE commands (e.g., LINE DASH, LINE COLOR BLUE, LINE
    THICKNESS 0.3).  The attributes of the interioir of the polygon
    are set with the various REGION attribute commands (e.g.,
    REGION FILL ON, REGION FILL COLOR BLUE). 

----------------------------------------------------------------------
The following enhancements were made to DATAPLOT January-April   1997.
----------------------------------------------------------------------

 1. A check is now performed to determine if DPPL2F.DAT is opened
    successfully upon starting Dataplot.  If not, an error message is
    printed and Dataplot is terminated.  The typical cause for this
    is trying to run Dataplot in a read only directory.  This change
    provides a more graceful exit.

 2. The Dataplot Reference Manual is now available on-line.  The
    Dataplot home page can be accessed from a Web browser using
    the URL:

       http://www.itl.nist.gov/div898/software/dataplot/homepage.html

    The Reference Manual is under the "documentation" table entry.
    The following should be noted:

    a) In order for these commands to work, you need to have
       a web browser available on your system.

       The Dataplot web pages display correctly with the Netscape,
       Internet Explorer, and HotJava 1.1  browsers.  They do not
       display correctly with the HotJava 1.0, Mosaic, or character
       oriented browsers.  We do not have access to other browsers,
       so we can make no specific comment on them.

    b) The Reference Manual is in PDF format (Portable Document
       Format), so it requires a PDF viewer.  Typically, this is the
       Adobe Acrobat Reader.  This reader is supported on most common
       platforms and can be downloaded for free.  The PC installation 
       typically takes about 10-15 minites to download and install.
       For best performance, it is strongly recommended that the
       Adobe Acrobat reader be installed as a plug-in (this is
       done automatically for Netscape on the PC) rather than
       as a helper application.  The documentation web page contains
       a link to the Adobe Acrobat web site for downloading the
       reader.

    In addition, several commands are now available for accessing
    the Web, and the Dataplot Web pages and Reference Manual in
    particular, from within Dataplot.

    The first command is:

        WEB 
        WEB NIST/SIMA/HPPC/SED/DATAPLOT
        WEB <url address>

    By default, this command activates Netscape with the specified
    URL.  If no URL is given, the NIST home page is used.  Several
    keywords are recognized.  For example, SED activates the
    NIST Statistical Engineering Division home page.

    The second command is:

        WEB HELP <string>

    This command is similar to the standard Dataplot HELP command.
    However, it accesses the on-line Reference Manual rather than
    the ASCII text help files.  <string> will usually be a Dataplot
    command (e.g., WEB HELP FIT, WEB HELP PLOT).  However, many
    special keywords are also recognized.  For example, WEB HELP or
    WEB HELP DATAPLOT access the Dataplot home page.  Enter the
    command:

        LIST REFMAN.TEX

    to see a list of recognized keywords (the upper case entries in
    columns 1-40 identify the keywords while columns 40+ identify the
    associated URL).

    The WEB and WEB HELP commands are supported for Unix platforms
    and for the Windows 95/NT version.

    A few SET commands were added to support the WEB and WEB HELP
    commands.

    a) By default, Dataplot tries to use the Netscape browser.  On
       Unix, it tries to do this by entering the command "netscape".
       On Windows 95/NT, it enters

          "C:\Program Files\NETSCAPE\NAVIGATOR\PROGRAM\netscape.exe"

       If you wish to use a different browser, or if Netscape is
       installed in a different location, you can enter the
       following command:

          SET BROWSER <file name>

       where <file name> is the string that activates your preferred
       browser.  In particular, if you prefer to use the Internet
       Explorer under Windows 95/NT, you can enter:

           SET BROWSER "C:\Program Files\Plus!\Microsoft Internet\iexplore.exe"

       The enclosing quotes are required because the file name contains
       spaces.  Again, check to see if this is the proper path on
       your system.

       Alternatively, you can enter the Unix command

           setenv BROWSER <file name>

       or the Windows 95/NT command

          SET BROWSER=<file name>

       to set the browser.  These are typically placed in your
       start-up files (.login or .cshrc for Unix, AUTOEXEC.BAT for
       Windows 95/NT).  You can shorten the browser name if you add
       the correct directory to your path.

    b) For the WEB command, the default URL is the NIST home page.
       You can change the default with the following Dataplot command:

           SET URL <default URL>

       For the WEB HELP command, the default URL is the Dataplot
       home page on the public NIST web server.  This can be
       changed (for example, if you have installed the Dataplot
       web pages and Reference Manual on a local site) by entering
       the command:

          SET DATAPLOT URL <location of Dataplot web pages>

       Alternatively, you can enter the Unix commands

          setenv URL <location of default URL>
          setenv DPURL <location of Dataplot web pages>

       or the Windows 95/NT commands

          SET URL=<location of default URL>
          SET DPURL=<location of Dataplot web pages>

    For Unix platforms, the following command was added to tell
    Dataplot to use a currently open NETSCAPE window (this command
    is not needed for the PC):

       SET NETSCAPE <OLD/NEW>

    These commands have been tested with NETSCAPE on Unix and
    with Netscape and the Internet Explorer on the PC.

    One important difference between the Unix and PC versions of
    these commands should be noted.  Under Unix, once the WEB command
    is initiated, control returns to Dataplot after the browser is
    started.  You can independently navigate in the the browser and
    enter additional Dataplot commands.  However, on the PC, control
    does not return to Dataplot until you exit the browser.

 3. The following commands were added to allow previously viewed
    graphs to be saved for later recall.  The primary purpose is
    to allow comparisons of a previous graph to a current graph.
    These commands are currently only supported for the X11 graphics
    device (available on most Unix implementations).

       SAVE PLOT <file>    (or SAVE GRAPH, SP, SG)
       SAVE PLOT <file> AUTOMATIC
       SAVE PLOT AUTOMATIC

       REPEAT PLOT <file>  (or REPEAT GRAPH, RP, RG, VIEW PLOT,
                           VIEW GRAPH, VG, VP)
       REPEAT PLOT <+n>
       REPEAT PLOT <-n>

       LIST PLOT (or LIST GRAPH, LP, LG)

       CYCLE PLOT (or CYCLE GRAPH, CG, CP)

       PIXMAP TITLE <title>

    As a technical note, the plots are saved in X11 "bitmap" format.
    This is distinct from the X11 image format that is used by
    xwd to save a screen image.  This choice was made for performance
    reasons (xlib provides direct routines for reading and writing
    bitmaps, but not for reading and writing images).  The primary
    limitations are:

      i) Color is not supported for X11 bitmaps.  Elements drawn
         in color will not be saved in the bitmap.

     ii) You cannot use the X11 tools xwd and xwud to view the
         saved plots independently of Dataplot.  However, they
         can be viewed by any software the reads X11 bitmaps.

    The saved plots are essentially screen dumps.  There is
    currently no "linking" in the sense that if a given variable
    is changed the saved plots are automatically updated.

    The SAVE GRAPH command saves the current plot in the user
    specified file.  If no file name is specified, then the file
    name "pixmap.<n>", where <n> is a counter, is used.

    The keyword AUTOMATIC tells Dataplot to automatically save all
    subsequent plots.  With the AUTOMATIC option, Dataplot does not
    save the current graph until the next plot is generated.  This is
    done in order to correctly handle multi-plots and diagrammatic
    graphics.  That is, the current graph is saved whenever a screen
    erase is performed.  If a filename is provided, this will be used
    as the base (the ".<n>" is added).  For example,
    SAVE PLOT HISTOGRAMS AUTOMATIC saves subsequent plots in
    the files HISTOGRAMS.1, HISTOGRAMS.2, and so on.  Enter SAVE GRAPH
    AUTOMATIC OFF to terminate the automatic saving of the plots.

    The REPEAT PLOT command reads a saved plot and draws it in a
    window that is distinct from the normal Dataplot X11 graphics
    window.  If no file is specified, or if <n> is 0 for REPEAT
    PLOT, the most current saved plot is drawn.  A <+n> takes the
    Nth plot from the current list.  A <-n> takes the "current - n"th
    plot from the current plot list.  The DEVICE 1 X11 command
    must be entered before the REPEAT PLOT command can be used.
    The REPEAT PLOT command can redraw plots that were created in
    a previous Dataplot session.  In fact, it will successfully
    redraw any file that is in the X11 bitmap format (but not in
    xwd format).

    The LIST PLOT command lists the currently saved plots (by
    sequence number, file name, and title).  It only lists plots
    saved in the current session.  However, this includes graphs
    created in a previous Dataplot session that have been redrawn
    with the REPEAT GRAPH command.  Dataplot does not maintain a
    database of previously saved plots.

    The CYCLE PLOT command allows you to cycle through the pixmaps
    in the current list by clicking mouse buttons.  Clicking the
    left mouse button moves down in the current list, clicking the
    right mouse button moves up in the current list, and clicking
    the middle mouse button returns control to Dataplot.  At least
    one REPEAT PLOT command should be entered before using this
    command.

    The PIXMAP TITLE command allows you to specify the title for
    a saved plot.  This title is simply for convenience in listing
    the saved plots.  It is not saved as part of the file and the
    title only applies to the current Dataplot session.  The default
    title is the file name.

    The pixmap title applies to the current plot when the SAVE GRAPH
    command is entered.  It does not matter whether the PLOT or
    PIXMAP TITLE command is entered first.

    Be aware that for SAVE GRAPH AUTOMATIC the saving for a given
    plot is not executed until the next screen erase (typically the
    next plot) is encountered to allow for multi-plotting and the
    addition of diagrammatic graphics to a plot.  The order of
    the commands would typically be something like:

       SAVE GRAPH AUTOMATIC
       4-PLOT Y
       PIXMAP TITLE 4-PLOT
       PLOT Y
       PIXMAP TITLE PLOT Y
       HISTOGRAM Y
       PIXMAP TITLE HISTOGRAM

    The main point here is that the PIXMAP TITLE comes AFTER the
    plot command.

    Unlike the regular TITLE command, the PIXMAP TITLE command does
    not persist.  That is, it applies only to the next saved plot and
    then reverts to the default of using the file name.

 4. Added following special functions:

    a) LAMBDA(X,V)  - Lambda function (V can be integer or real)
    b) LAMBDAP(X,V) - derivative of Lambda function (V can be integer
                      or real)
    c) H0(X)        - Struve function order 0
    d) H1(X)        - Struve function order 1
    e) HV(X,V)      - Struve function order V
    f) L0(X)        - modified Struve function order 0
    g) L1(X)        - modified Struve function order 1
    h) LV(X,V)      - modified Struve function order V

    i) Added LOGBETA as synonym for LNBETA and LNGAMMA as synonym for
       LOGGAMMA.

 5. The following bug fixes were made:

    a) Fixed bug where TEXT command automatically generated a software
       font (introduced by the DEVICE FONT command).

    b) Fixed bug in the ANOVA command.

    c) Fixed bug with ERASE command on Windows NT version.

    d) Fixed bug in HELP with conflict between STATUS and
       STATISTIC PLOT.

    e) Fixed bug if software font used and CHARACTER BLANK was
       entered in lower case.

    f) Fixed bug where CREATE <file> went into an infinite loop if
       a CALL command was encountered.  The CALL command will now
       be saved correctly in the CREATE file.  Note that the commands
       in the CALL file are not saved in the CREATE file (they are
       already saved as part of the CALL macro file).


----------------------------------------------------------------------
The following enhancements were made to DATAPLOT October-November 1996.
----------------------------------------------------------------------

 1. A native mode Windows 95/NT version is now available.  This
    version was created using the Microsoft Windows 95 compiler.
    The initial release supports the command line version only.
    We will attempt over the next several months to port the
    Tcl/Tk based graphical user interface to the Windows 95/NT
    environment.

    To generate graphics to the screen for this version, enter
    the following command:

        DEVICE 1 QWIN

    Enter the command HELP QWIN for details of using this device.

 2. For encapsulated Postscript files, DATAPLOT based the bounding
    box parameters assuming an 11 x 11 inch page.  This was done to
    accomodate both landscape and portrait orientation plots. 

    Unfortunately, this did not generate satisfactory results when
    importing DATAPLOT graphics into WordPerfect and other text
    processing software.  The user had to do a fair amount of manual
    rotation and scaling of plots.

    DATAPLOT now adjusts the bounding box depending on the orientation.
    It uses 11 x 8.5 inch for landscape orientation and 8.5 x 11 inch
    for portrait.  However, most text processors ignore the rotation
    and translation that the landscape plots request.  To compensate
    for this, the following command was added:

         ORIENTATION LANDSCAPE WORDPERFECT

    This essentially generates a landscape orientation on a portrait
    page.  That is, the bounding box specifies an 8.5 x 6.5 inch page.
    This generates execellent results with Word Perfect (users should
    normally never need to adjust the bounding box parameters or
    perform manual rotation and translation in Word Perfect).

    This option is only recognized for encapsulated Postscript.
    Regular Postscript should still use ORIENTATION LANDSCAPE.

 3. Fixed a few bugs:

    a. Macros now accept more than 1,000 lines.
    b. Unix executables were not finding certain auxillary files
       if the file names were entered in lower case.
    c. NORMAL PLOT fixed.

 4. The output for the YATES command was modified to be more readable
    and informative.

----------------------------------------------------------------------
The following enhancement was made to DATAPLOT July              1996.
----------------------------------------------------------------------

 1. The previous fix (checking the HOME environment variable for the
    user's root directory) was refined a bit.  If HOME is defined,
    it looks for dplogf.tex in that directory.  If dplogf.tex is
    not found, instead of printing an error message, it then strips
    off the path name and looks for it in the current directory and
    then in the DATAPLOT directory (typically /usr/local/lib/dataplot).
    Note that if an error message is printed saying that this file is
    not found, DATAPLOT will still run.  This file simply lets you
    enter some DATAPLOT commands when starting DATAPLOT (i.e., for
    setting your preferred defaults).  There should not be any
    negative side effects if this file is not executed.

 2. Unix versions will check for the environment variable
    DATAPLOT_WEB.  If this variable is defined, DATAPLOT assumes it
    is being run from the web (e.g., from Mosaic or Netscape).
    Currently, the only effect is that certain files that DATAPLOT
    typically creates in the current directory, such as dppl1f.dat
    and dpconf.tex, are opened in the /tmp directory.  This may or
    may not be expanded upon as we gain more experience running
    DATAPLOT from web servers.

 3. We built a "double precision" version for the Sun.  That is,
    the -p8 option was used so that single precision numbers are
    64-bit rather than 32-bit.  The only complication was in how the
    X11 routines were called (these are compiled with 32-bit real
    numbers).  Changes were made to the X11 driver to allow a
    "compile flag" to be set based on which case (i.e., 32 or 64-bit)
    is desired.  This means that DATAPLOT can be easily built on any
    Unix system that supports the "-p8" option (or a compiler switch
    that provides a similar capability).

 4. A version of DATAPLOT was built using the LAHEY compiler
    (previously, the OTG compiler was used).  This version allows
    DATAPLOT to be run on PC's without special AUTOEXEC.BAT and
    CONFIG.SYS files (and therefore no rebooting to run DATAPLOT).

    A device driver that uses the LAHEY graphics library is also
    available.  Enter

        DEVICE 1 LAHEY
        DEVICE 1 FONT SIMPLEX    (this described below)

 5. The following command was added:

        DEVICE <1/2/3> FONT <font name>

    This allows the screen device to use a different font than the
    printed output.  This was specifically motivated for the LAHEY
    device driver.  This driver does a very poor job with hardware
    characters.  Using a software font avoids this problem, but
    often hardware characters are desired for the printed Postscript
    output (to take advantage of the typset quality fonts available
    with Postscript).  Using the DEVICE 1 FONT SIMPLEX allows us
    to get decent characters on the screen and still retain the
    ability to use the Postscript fonts.  Although this command
    was motivated by the LAHEY device, it is also useful for other
    screen devices (e.g., X11 hardware fonts are a fixed size, so
    only 1 character size is available at a time, Tektronix devices
    are limited to 4 discrete sizes, etc.).

 6. Previously, log scales required at least 1 full cycle (e.g.,
    10 to 100).  It is now possible to get around this limitation.
    For example, to have a log scale go from 85 to 125, do the
    following:

        YLOG ON
        YLIMITS 100 100
        YTIC OFFSET 15 25
        PLOT Y

    The key is that the lower and upper bound on the LIMITS command
    must be the same and at least one of the TIC OFFSETS must be
    greater than zero.  Major TICS will be generated at this bound
    and also at the frame limits.  Minor tics will be plotted
    where appropriate.  Also, the TIC OFFSET is always interpreted
    in data units for this case (i.e., can't specify the offset
    in DATAPLOT 0 to 100 coordinates as you normally can).

 7. Several bugs were fixed.

----------------------------------------------------------------------
The following enhancement was made to DATAPLOT June              1996.
----------------------------------------------------------------------

 For Unix systems, check for the HOME environment variable.  This
 normally specifies the user's home directory.  If present, DATAPLOT
 looks for the user's start-up file (dplogf.tex) in the user's home
 directory rather than the current directory.  This means you no
 longer have to include the start-up file in each directory from
 which you run DATAPLOT.  If HOME is not found, look for dplogf.tex
 in the current directory .  Note that if HOME is found and dplogf.tex
 is not found in the home directory, DATAPLOT will NOT look for
 it in the current directory.

----------------------------------------------------------------------
The following enhancements were made to DATAPLOT MAY,            1996.
----------------------------------------------------------------------

 1) Fixed a bug where the X11 driver bombed if being run remotely
    and the SET X11 PIXMAP ON command was used.

 2) Fixed a bug where the 3D-PLOT was bombing when a large number of
    points were plotted.

----------------------------------------------------------------------
The following enhancements were made to DATAPLOT FEBRUARY-APRIL, 1996.
----------------------------------------------------------------------

 1) The following probability functions were added:

      LET A = BBNCDF(X,ALPHA,BETA,N) - beta-binomial cumulative
                                       distribution function
      LET A = BBNPDF(X,ALPHA,BETA,N) - beta-binomial probability
                                       density function
      LET A = BBNPPF(P,ALPHA,BETA,N) - beta-binomial percent point
                                       function
      LET A = BRACDF(X,BETA)   - Bradford cumulative distribution
                                       function
      LET A = BRAPDF(X,BETA)   - Bradford probability density function
      LET A = BRAPPF(P,BETA)   - Bradford percent point function
      LET A = DGACDF(X,GAMMA)  - double gamma cumulative distribution
                                 function
      LET A = DGAPDF(X,GAMMA)  - double gamma probability density
                                 function
      LET A = DGAPPF(P,GAMMA)  - double gamma percent point function
      LET A = FCACDF(X,U,SD)   - folded Cauchy cumulative distribution
                                 function
      LET A = FCAPDF(X,U,SD)   - folded Cauchy probability density
                                 function
      LET A = FCAPPF(P,U,SD)   - folded Cauchy percent point function
      LET A = GEXCDF(X,LAM1,LAM2,S) - generalized exponential
                                      cumulative distribution function
      LET A = GEXPDF(X,LAM1,LAM2,S) - generalized exponential
                                      probability density function
      LET A = GEXPPF(P,LAM1,LAM2,S) - generalized exponential
                                      percent point function
      LET A = GLOCDF(X,ALPHA)  - generalized logistic cumulative
                                 distribution function
      LET A = GLOPDF(X,ALPHA)  - generalized logistic probability
                                 density function
      LET A = GLOPPF(P,ALPHA)  - generalized logistic percent point
                                 function
      LET A = KAPCDF(X,AK,B,T) - Mielke's beta-kappa cumulative
                                 distribution function
      LET A = KAPPDF(X,AK,B,T) - Mielke's beta-kappa probability
                                 density function
      LET A = KAPPPF(P,AK,B,T) - Mielke's beta-kappa percent point
                                 function
      LET A = NCCPDF(X,V,DELTA) - non-central chi-square probability
                                  density function
      LET A = PEXCDF(X,ALPHA,BETA) - exponential power cumulative
                                     distribution function
      LET A = PEXPDF(X,ALPHA,BETA) - exponential power probability
                                     density function
      LET A = PEXPPF(P,ALPHA,BETA) - exponential power percent point
                                     function

    The following probability plots were added:

      LET ALPHA = <value>
      LET BETA = <value>
      LET N = <value>
      BETA BINOMIAL PROBABILITY PLOT Y

      LET BETA = <value>
      BRADFORD PROBABILITY PLOT Y

      LET GAMMA = <value>
      DOUBLE GAMMA PROBABILITY PLOT Y

      LET M = <value>
      LET SD = <value>
      FOLDED CAUCHY PROBABILITY PLOT Y

      LET LAMBDA1 = <value>
      LET LAMBDA2 = <value>
      LET S = <value>
      GENERALIZED EXPONENTIAL PROBABILITY PLOT Y

      LET ALPHA = <value>
      GENERALIZED LOGISTIC PROBABILITY PLOT Y

      LET BETA = <value>
      LET THETA = <value>
      LET K = <value>
      MIELKE BETA-KAPPA PROBABILITY PLOT Y

      LET ALPHA = <value>
      LET BETA = <value>
      EXPONENTIAL POWER PROBABILITY PLOT Y

    The following probability plot correlation coefficient plots
    were added:

      BRADFORD PPCC PLOT Y
      DOUBLE GAMMA PPCC PLOT Y
      GENERALIZED LOGISTIC PPCC PLOT Y

 2) The WRITE command was updated to handle a maximum of 25 variables
    (up from 10).

    Support was added for writing Fortran unformatted data files.
    This was done primarily for sites that have created "mega" size
    versions of DATAPLOT where the time entailed in reading and writing
    large data files becomes important.  For standard size DATAPLOT
    (typically a maximum of 10,000 rows with 10 columns for 100,000
    data points total), the use of the SET READ FORMAT and SET WRITE
    FORMAT commands provides adequate performance.  However, the
    unformatted read and write capability is available regardless of
    the workspace size.  The advantage of unformatted read and writes
    is that the data files are much smaller (typically by a factor of
    10 or more) and reading and writing the data significantly faster.
    The disadvantage is that unformatted files are binary, and thus
    cannot be modified or viewed with a standard text editor.  Also,
    Fortran unformatted files are NOT transportable across different
    computer systems.  Also, unformatted Fortran files are NOT
    equivalent to C language byte stream files (these types of files
    are not currently supported in DATAPLOT).

    An unformatted write is accomplished by entering the command:

       SET WRITE FORMAT UNFORMATTED

    and then entering a standard WRITE command.  For example,

       WRITE LARGE.DAT X1 X2 X3

    There are 2 ways to create the unformatted file in Fortran.  For
    example, suppose X and Y are to be written to an unformatted
    file.  The WRITE can be generated by:

    a)    WRITE(IUNIT) (X(I),Y(I),I=1,N)
    b)    WRITE(IUNIT) X,Y

    The distinction is that (a) stores the data as X(1), Y(1),
    X(2), Y(2), ..., X(N), Y(N) while (b) stores all of X then
    all of Y.  There is no inherent advantage in either method in
    terms of performance or file size.  The SET WRITE FORMAT
    UNFORMATTED command only supports (a).

    Unformatted writing is supported only for variables or matrices
    (i.e., not for parameters or strings).

    Be aware that Fortran unformatted files are NOT transportable
    across systems.  This is due to the fact that the file contains
    various header bytes (the Fortran standard leaves implementation
    of this up to vendor) that are not standard.  Also, the storage
    of real numbers can vary between platforms.  This means that
    the SET WRITE FORMAT UNFORMATTED command can NOT be used to write
    raw binary files (as might be produced by a C program) and it
    cannot, in general, be used to write unformatted Fortran files
    that can be read on systems other than the one you are running
    DATAPLOT on.

 3) The command SET RELATIVE HISTOGRAM <AREA/PERCENT> was added to
    specify whether or not relative histograms (and relative
    bi-histograms) are drawn so that the area under the histogram
    sums to 1 or so that the heights of the histograms sum to 1.
    The first option, which is the default, is useful when using the
    relative histogram as an estimate of a probability distribution.
    The second option is useful when you want to see what percentage
    of the data falls in a given class.

 4) For Unix versions, the location of the DATAPLOT auxillary files
    can be specified with the following Unix command:

       setenv DATAPLOT_FILES <directory name>

    This can be useful if you do not have super user permission to
    copy the files into the /usr/local/lib/dataplot directory and
    you do not have a cooperative system adminstrator.

 5) The LET STRING command was modified so that the case of the
    text in the string is preserved as entered.  Note that the
    LET FUNCTION command still converts text to upper case.

    The READ STRING command was modified so that it ignores the
    SET READ FORMAT command.

 6) Numerous minor bugs were fixed.

-----------------------------------------------------------------
The following enhancements were made to DATAPLOT AUGUST-OCTOBER, 1995.
-----------------------------------------------------------------

1) The Numerical Recipes routine for calculating complex roots
   was replaced with a CMLIB routine.  There is no change in the
   command syntax.

2) The Numerical Recipes routine for calculating the fast Fourier
   transform was replaced with CMLIB routines.  A couple of changes
   were made as follows:

   a) the CMLIB routine does not require zero padding so that
      the length of the variable is a power of two.  Previously,
      DATAPLOT did this automatically.  It no longer does.  However,
      the CMLIB algorithm loses efficiency if the length is not a
      factor of small primes.  In this case, you may wish to zero
      pad the variable yourself before calling the FFT command.

   b) The SET FOURIER EXPONENT <+/-> command was corrected to work
      as intended (the default implemented the + case, which was really
      the only option that worked).  In addition, this command was
      extended to apply to the FOURIER and INVERSE FOURIER command
      as well as the FFT and INVERSE FFT commands.  Enter
      HELP FOURIER EXPONENT for more information on this command.

   c) Most FFT routines return the data in the following order:

         F(1)              = zero frequency
         F(2) ... F(N/2)   = smallest positive frequency to largest
                             positive frequency
         F(N/2+1)          = aliased point that contains the largest
                             positive and the largest negative frequency
         F(N/2+2) ... F(N) = negative frequencies from largest
                             magnitude to smallest magnitude

      By default, DATAPLOT returns the data in the following order:

         F(1)              = aliased point that contains the largest
                             positive and the largest negative frequency
         F(2) ... F(N/2)   = Largest positive frequency to smallest
                             positive frequency
         F(N/2+1)          = zero frequency
         F(N/2+2) ... F(N) = negative frequencies from smallest
                             magnitude to largest magnitude

      The command SET FOURIER ORDER <STANDARD/DATAPLOT> was 
      implemented to allow you to specify which order to use.
      The option STANDARD returns the first order while the option
      DATAPLOT returns the second order.

 3) Support was added for hypergeometric, non-central chi-square,
    singly and doubly non-central F, half-cauchy and folded normal
    random numbers, 

    The following probability functions were added:

      LET A = ANGCDF(X)  - anglit cumulative distribution function
      LET A = ANGPDF(X)  - anglit density function
      LET A = ANGPPF(X)  - anglit percent point function
      LET A = ARSCDF(X)  - arcsin cumulative distribution function
      LET A = ARSPDF(X)  - arcsin density function
      LET A = ARSPPF(X)  - arcsin percent point function
      LET A = DWECDF(X,G) - double Weibull cumulative distribution
                            function
      LET A = DWEPDF(X,G) - double Weibull density function
      LET A = DWEPPF(X,G) - double Weibull percent point function
      LET A = EWECDF(X,G) - exponentiated Weibull cumulative
                            distribution function
      LET A = EWEPDF(X,G) - exponentiated Weibull density function
      LET A = EWEPPF(X,G) - exponentiated Weibull percent point function
      LET A = FNRCDF(X,U,SD) - folded normal cumulative distribution
                               function
      LET A = FNRPDF(X,U,SD) - folded normal probability density
                               function
      LET A = FNRPPF(X,U,SD) - folded normal percent point function
      LET A = GEVCDF(X,G) - generalized extreme value cumulative
                            distribution function
      LET A = GEVPDF(X,G) - generalized extreme value density function
      LET A = GEVPPF(X,G) - generalized extreme value percent point
                            function
      LET A = GOMCDF(X,C,B) - Gompertz cumulative distribution function
      LET A = GOMPDF(X,C,B) - Gompertz probability density function
      LET A = GOMPPF(X,C,B) - Gompertz percent point function
      LET A = HFCCDF(X)   - half-Cauchy cumulative distribution function
      LET A = HFCPDF(X)   - half-Cauchy density function
      LET A = HFCPPF(X)   - half-Cauchy percent point function
      LET A = HFLCDF(X,G) - generalized half-logistic cumulative
                            distribution function
      LET A = HFLPDF(X,G) - generalized half-logistic density function
      LET A = HFLPPF(X,G) - generalized half-logistic percent point
                            function
      LET A = HSECDF(X)   - hyperbolic secant cumulative distribution
                            function
      LET A = HSEPDF(X)   - hyperbolic secant density function
      LET A = HSEPPF(X)   - hyperbolic secant percent point function
      LET A = LGACDF(X,G) - log-gamma cumulative distribution function
      LET A = LGAPDF(X,G) - log-gamma density function
      LET A = LGAPPF(X,G) - log-gamma percent point function
      LET A = PA2CDF(X,G) - Pareto type 2 cumulative distribution
                            function
      LET A = PA2PDF(X,G) - Pareto type 2 density function
      LET A = PA2PPF(X,G) - Pareto type 2 percent point function
      LET A = TNRCDF(X,A,B,U,SD) - truncated normal cumulative
                                   distribution function
      LET A = TNRPDF(X,A,B,U,SD) - truncated normal probability density
                                   function
      LET A = TNRPPF(X,A,B,U,SD) - truncated normal percent point
                                   function
      LET A = TNECDF(X,X0,U,SD) - truncated exponential cumulative
                                  distribution function
      LET A = TNEPDF(X,X0,U,SD) - truncated exponential probability
                                  density function
      LET A = TNEPPF(X,X0,U,SD) - truncated exponential percent point
                                  function
      LET A = WCACDF(X,G) - wrapped-up Cauchy cumulative distribution
                            function
      LET A = WCAPDF(X,G) - wrapped-up Cauchy density function
      LET A = WCAPPF(X,G) - wrapped-up Cauchy percent point function

    The following probability plots were added:

      ANGLIT PROBABILITY PLOT Y
      ARCSIN PROBABILITY PLOT Y
      HYPERBOLIC SECANT PROBABILITY PLOT Y
      HALF CAUCHY PROBABILITY PLOT Y

      LET M = <value>
      LET SD = <value>
      FOLDED NORMAL PROBABILITY PLOT Y

      LET A = <value>
      LET B = <value>
      LET M = <value>  (optional, defaults to 0)
      LET SD = <value> (optional, defaults to 1)
      TRUNCATED NORMAL PROBABILITY PLOT Y

      LET X0 = <value>
      LET M = <value>  (optional, defaults to 0)
      LET SD = <value> (optional, defaults to 1)
      TRUNCATED EXPONENTIAL PROBABILITY PLOT Y

      LET GAMMA = <value>
      DOUBLE WEIBULL PROBABILITY PLOT Y
      LOG GAMMA PROBABILITY PLOT Y
      GENERALIZED EXTREME VALUE PROBABILITY PLOT Y  (or GEV PROB PLOT)
      PARETO SECOND KIND PROBABILITY PLOT Y   (or PARETO TYPE 2)
      HALF LOGISTIC PROBABILITY PLOT Y  (GAMMA optional for this case)
      
      LET GAMMA = <value>
      LET THETA = <value>
      EXPONENTIATED WEIBULL PROBABILITY PLOT Y

      LET C = <value>
      LET B = <value>
      GOMPERTZ PROBABILITY PLOT Y

      LET C = <value>
      WRAPPED CAUCHY PROBABILITY PLOT Y

    The following probability plot correlation coefficient plots were
    added:

      LOG GAMMA PPCC PLOT Y
      DOUBLE WEIBULL PPCC PLOT Y
      GENERALIZED EXTREME VALUE PPCC PLOT Y  (or GEV PPCC PLOT)
      PARTEO SECOND KIND PPCC PLOT Y   (or PARETO TYPPE 2 PPCC PLOT)
      WRAPPED CAUCHY PPCC PLOT Y
      HALF LOGISTIC PPCC PLOT Y

 4) The following character option was added:

       CHARACTER PIXEL

    This option plots a single "pixel" on a given device.  In addition,
    when this option is given, the CHARACTER SIZE is interpreted as
    an integer expansion factor.  For example, CHARACTER SIZE 10 will
    plot a 10x10 pixel block.

    This option has been implemented for the Tektronix, X11,
    Postscript, HP-GL, Regis, HP-2622, and Sun devices.  Other devices
    will print a message saying this option is unavailable (although
    additional devices will be added later).

    Although this capability was added with some possible future
    enhancements in mind, it can be useful in some plots such as
    fractal plots.

-----------------------------------------------------------------
The following enhancements were made to DATAPLOT JULY, 1995.   
-----------------------------------------------------------------

Support was added for various types of orthogonal polynomials.
The following commands were added.

   LET A = LEGENDRE(X,N)       Compute the Legendre polynomial of
                               order n
   LET A = LEGENDRE(X,N,M)     Compute the associated Legendre
                               polynomial of order n and degree m
   LET A = NRMLEG(X,N)         Compute the normalized Legendre
                               polynomial of order n
   LET A = NRMLEG(X,N,M)       Compute the associated normalized
                               Legendre polynomial of order n and
                               degree m
   LET A = LEGP(X,N)           Compute the Legendre function of the
                               first kind of order n
   LET A = LEGP(X,N,M)         Compute the associated Legendre function
                               of the first kind of order n and degree m
   LET A = LEGQ(X,N)           Compute the Legendre function of the
                               second kind of order n
   LET A = LEGQ(X,N,M)         Compute the associated Legendre function
                               of the second kind of order n and
                               degree m
   LET A = SPHRHRMR(X,P,N,M)   Compute the real component of the
                               spherical harmonic function
   LET A = SPHRHRMC(X,P,N,M)   Compute the complex component of the
                               spherical harmonic function
   LET A = LAGUERRE(X,N)       Compoute the Laguerre polynomial of
                               order n
   LET A = LAGUERRL(X,N,A)     Compute the generalized Laguerre
                               polynomial of order n
   LET A = NRMLAG(X,N)         Compute the normalized Laguerre
                               polynomial of order n
   LET A = CHEBT(X,N)          Compute the Chebyshev T (first kind)
                               polynomial of order n
   LET A = CHEBU(X,N)          Compute the Chebyshev U (second kind)
                               polynomial of order n
   LET A = JACOBIP(X,N,A,B)    Compute the Jacobi polynomial of order n
   LET A = ULTRASPH(X,N,A)     Compute the Ultraspherical (or
                               Gegenbauer) polynomial of order n
   LET A = HERMITE(X,N)        Compute the Hermite polynomial of order n
   LET A = LNHERMIT(X,N)       Compute the log of the absolute value of
                               the Hermite polynomial of order n
   LET A = HERMSGN(X,N)        Compute the sign of the Hermite
                               polynomial (1 for positive, -1 for
                               negative, 0 for zero)

In addition, an alpha version of a graphical user interface is
available on some Unix systems.  You can check with your local site
installer to see if it is available on your system.  If it is
available, it is typically executed by entering the command:

    xdp

At NIST, the frontend has been installed on the CAML Sun's and
SGI's as well as the Convex.  There are no plans to install it
on the Cray.  For non-NIST sites, the following non-DATAPLOT programs
must be installed:

   1) Tcl/TK   - Tool Commmand Language
   2) Expect   - a program for controlling the dialog among 
                 interactive programs.

These are both popular public domain Unix utilities that can be
installed on most common Unix platforms.

-----------------------------------------------------------------
The following enhancements were made to DATAPLOT APRIL, 1995.   
-----------------------------------------------------------------

 1) Support was added for reading Fortran unformatted data files.
    This was done primarily for sites that have created "mega" size
    versions of DATAPLOT where the time entailed in reading large
    data files becomes important.  For standard size DATAPLOT
    (typically a maximum of 10,000 rows with 10 columns for 100,000
    data points total), the use of the SET READ FORMAT command
    provides adequate performance.  However, the unformatted read
    capability is available regardless of the workspace size.  The
    advantage of unformatted reads is that the data files are much
    smaller (typically by a factor of 10 or more) and reading the
    data significantly faster.  The disadvantage is that unformatted
    files are binary, and thus cannot be modified or viewed with a
    standard text editor.  Also, Fortran unformatted files are NOT
    transportable across different computer systems.

    An unformatted read is accomplished by entering the command:

       SET READ FORMAT UNFORMATTED

    and then entering a standard READ command.  For example,

       READ LARGE.DAT X1 X2 X3

    There are 2 ways to create the unformatted file in Fortran.  For
    example, suppose X and Y are to be written to an unformatted
    file.  The WRITE can be generated by:

    a)    WRITE(IUNIT) (X(I),Y(I),I=1,N)
    b)    WRITE(IUNIT) X,Y

    The distinction is that (a) stores the data as X(1), Y(1),
    X(2), Y(2), ..., X(N), Y(N) while (b) stores all of X then
    all of Y.  There is no inherent advantage in either method in
    terms of performance or file size.  The SET READ FORMAT
    UNFORMATTED command assumes (a).  To specify (b), enter the
    command:

          SET READ FORMAT COLUMNWISE (or UNFORMATTEDCOLUMNWISE)

    Unformatted reading is supported only for variables or matrices
    (i.e., not for parameters or strings).  Also, it only applies
    when reading from a file.  The limits for the maximum number of
    rows and columns for a matrix still apply (500 rows and 100
    columns on most systems).  When reading a matrix, the number of
    columns must be specified via the SET UNFORMATTED COLUMNS
    command.  For example,

          SET READ FORMAT UNFORMATTED
          SET UNFORMATTED COLUMNS 25
          READ MATRIX.DAT M

    The maximum size of the file that DATAPLOT can read is equal to
    the workspace size on your implementation (100,000 or 200,000
    points on most installations).  For larger files, it will read
    up to this number of data values.

    The data is assumed to be a rectangular grid of data written in
    a single chunk.  Only single precision real numbers are
    supported.  By default, the entire file (up to the maximum number
    of points) is read.  DATAPLOT does provide 2 commands to allow
    some control of what portion of the file is read:

          SET UNFORMATTED OFFSET <value>
          SET UNFORMATTED RECORDS <value>

    The OFFSET specifies the number of data values at the begining of
    the file to skip.  This is useful for skipping header lines
    (similar to a SKIP command for reading ASCII files) and other
    miscellaneous values.  The RECORDS value is useful for reading
    part of a larger file.

    Be aware that Fortran unformatted files are NOT transportable
    across systems.  This is due to the fact that the file contains
    various header bytes (the Fortran standard leaves implementation
    of this up to vendor) that are not standard.  Also, the storage
    of real numbers can vary between platforms.  This means that
    the SET READ FORMAT UNFORMATTED command can NOT be used to read
    raw binary files (as might be produced by a C program) and it
    cannot, in general, be used to read unformatted Fortran files
    created on systems other than the one you are running DATAPLOT on.

 2) The following mathematical library functions were added:

       LET A = HEAVE(X,C)  - Heavside function (=1 if X>=C, 0
                             otherwise, C is 0 if no second argument)
       LET A = CEIL(X)     - ceiling function (integer value of x
                             rounded to positive infinity
       LET A = FLOOR(X)    - floor function (integer value rounded o
                             negative infinity)
       LET A = STEP(X)     - step function (synonym for FLOOR(X))
       LET A = GCD(X1,X2)  - greatest common divisor of X1 and X2

 3) The following command was added:
   
       LET A = MAD Y          - medain absolute deviation

    MEDIAN ABSOLUTE DEVIATION is a synonym for MAD.  Given a variable
    X with median value MED, the MAD is defined as the median of
    the absolute value of (X-MED).

    The BOOTSTRAP PLOT, JACKNIFE PLOT, STATISTIC PLOT, BLOCK PLOT, and
    DEX PLOT commands were modified to support the MAD and AAD
    statistics.

 4) The PHD command was renamed DEX PHD.  In addition, some I/O was
    fixed in these routines.

 5) Some bugs were fixed in the EDIT command.  A few other
    miscellaneous bugs were fixed.

 7) The following functions were added to the probability library.

      LET A = ALPCDF(X,ALPHA,BETA)  - alpha cumulative distribution
                                      function
      LET A = ALPPDF(X,ALPHA,BETA)  - alpha density function
      LET A = ALPPPF(X,ALPHA,BETA)  - alpha percent point function

      LET A = CHCDF(X,NU)           - chi cumulative distribution
                                      function
      LET A = CHPDF(X,NU)           - chi density function
      LET A = CHPPF(X,NU)           - chi percent point function

      LET A = COSCDF(X)             - cosine cumulative distribution
                                      function
      LET A = COSPDF(X)             - cosine density function
      LET A = COSPPF(X)             - cosine percent point function

      LET A = DLGCDF(X,THETA)       - logarithmic series cumulative
                                      distribution function
      LET A = DLGPDF(X,THETA)       - logarithmic series density
                                      function
      LET A = DLGPPF(X,THETA)       - logarithmic series percent point
                                      function

      LET A = GGDCDF(X,ALPHA,C)     - generalized gamma cumulative
                                      distribution function
      LET A = GGDPDF(X,ALPHA,C)     - generalized gamma density function
      LET A = GGDPPF(X,ALPHA,C)     - generalized gamma percent point
                                      function

      LET A = LLGCDF(X,DELTA)       - log-logistic cumulative
                                      distribution function
      LET A = LLGPDF(X,DELTA)       - log-logistic density function
      LET A = LLGPPF(X,DELTA)       - log-logistic percent point
                                      function

      LET A = PLNCDF(X,P,SD)        - power lognormal cumulative
                                      distribution function
      LET A = PLNPDF(X,P,SD)        - power lognormal density function
      LET A = PLNPPF(X,P,SD)        - power lognormal percent point
                                      function

      LET A = PNRCDF(X,P,SD)        - power normal cumulative
                                      distribution function
      LET A = PNRPDF(X,P,SD)        - power normal density function
      LET A = PNRPPF(X,P,SD)        - power normal percent point function

      LET A = POWCDF(X,C)           - power function cumulative
                                      distribution function
      LET A = POWPDF(X,C)           - power function density function
      LET A = POWPPF(X,C)           - power function percent point
                                      function

      LET A = WARCDF(X,C,A)         - Waring cumulative distribution
                                      function
      LET A = WARPDF(X,C,A)         - Waring density function
      LET A = WARPPF(P,C,A)         - Waring percent point function

      LET A = NCTPDF(X,NU,DELTA)    - non-central t density function
                                      (density and percent point
                                      functions were added previously)
      LET A = TNRPDF(X,A,B)         - truncated normal density function
      LET A = FNRPDF(X,U,SD)        - folded normal density function

    The Yule distribution is a special case of the Waring
    distribution.  Set A to 1 or simply omit the A parameter.

    The generalized gamma distribution can handle negative values
    for the C parameter (although not zero).  Specifically, a value
    of C = -1 is the inverted gamma distribution.

    In addition, the log-normal cdf, pdf, and ppf functions were
    upgraded to handle the standard deviation shape parameter (LGNCDF,
    LGNPDF, LGNPPF).  This parameter defaults to 1 if not specified.

    In addition the following probability plots were added.

      COSINE PROBABILITY PLOT Y

      LET ALPAHA = <value>
      LET BETA = <value>
      ALPHA PROBABILITY PLOT Y

      LET P = <value>
      LET SD = <value>  (this parameter optional, defaults to 1)
      POWER NORMAL PROBABILITY PLOT Y

      LET P = <value>
      LET SD = <value>   (this parameter optional, defaults to 1)
      POWER LOGNORMAL PROBABILITY PLOT Y

      LET SD = <value>
      LOGNORMAL PROBABILITY PLOT Y

      LET C = <value>
      POWER FUNCTION PROBABILITY PLOT Y

      LET NU = <value>
      CHI PROBABILITY PLOT Y

      LET THETA = <value>
      LOGARITMIC SERIES PROBABILITY PLOT Y

      LET DELTA = <value>
      LOG LOGISTIC PROBABILITY PLOT Y

      LET GAMMA = <value>
      LET C = <value>
      GENERALIZED GAMMA PROBABILITY PLOT Y

      LET A = <value>  (can omit for the Yule distribution)
      LET C = <value>
      GENERALIZED GAMMA PROBABILITY PLOT Y

    In addition the following PPCC plots were added.

      LET SD = <value>  (this parameter optional, defaults to 1)
      POWER NORMAL PPCC PLOT Y

      LET SD = <value>   (this parameter optional, defaults to 1)
      POWER LOGNORMAL PPCC PLOT Y

      LET SD = <value>
      LOGNORMAL PPCC PLOT Y

      CHI PPCC PLOT Y

      VON MISES PPC PLOT Y

      POWER FUNCTION PPCC PLOT Y

      LOG LOGISTIC PPCC PLOT Y

    In addition the following random number generator was added.

      LET C = <value>
      LET Y = POWER FUNCTION RANDOM NUMBERS FOR I = 1 1 N

-----------------------------------------------------------------
The following enhancements were made to DATAPLOT NOVEMBER, 1994.
-----------------------------------------------------------------

 1) The following mathematical library functions were added:

       LET A = FRESNS(X)     - Fresnel sine integral
       LET A = FRESNC(X)     - Fresnel cosine integral
       LET A = FRESNF(X)     - Fresnel auxillary function f integral
       LET A = FRESNG(X)     - Fresnel auxillary function g integral
       LET A = SN(X,M)       - Jacobian elliptic sn function
       LET A = CN(X,M)       - Jacobian elliptic cn function
       LET A = DN(X,M)       - Jacobian elliptic dn function
       LET A = PEQ(XR,XI)    - the real component of the Weirstrass
                               elliptic function (equianharmomic case)
       LET A = PEQI(XR,XI)   - the complex component of the Weirstrass
                               elliptic function (equianharmomic case)
       LET A = PEQ1(XR,XI)   - the real component of the first 
                               derivative of the Weirstrass elliptic
                               function (equianharmomic case)
       LET A = PEQ1I(XR,XI)  - the complex component of the first 
                               derivative of the Weirstrass elliptic
                               function (equianharmomic case)
       LET A = PLEM(XR,XI)   - the real component of the Weirstrass
                               elliptic function (cwlemniscatic case)
       LET A = PLEMI(XR,XI)  - the complex component of the Weirstrass
                               elliptic function (lemniscatic case)
       LET A = PLEM1(XR,XI)  - the real component of the first 
                               derivative of the Weirstrass elliptic
                               function (lemniscatic case)
       LET A = PLEM1I(XR,XI) - the complex component of the first 
                               derivative of the Weirstrass elliptic
                               function (lemniscatic case)

-----------------------------------------------------------------
The following enhancements were made to DATAPLOT OCTOBER, 1994.
-----------------------------------------------------------------

 1) The following mathematical library functions were added:

       LET A = BETA(ALPHA,BETA)    - complete Beta function
       LET A = LNBETA(ALPHA,BETA)  - log of complete Beta function
       LET A = BETAI(X,ALPHA,BETA) - incomplete Beta function

       LET A = GAMMAI(X,GAMMA)     - incomplete Gamma function
       LET A = GAMMAIP(X,GAMMA)    - incomplete Gamma function 
                                     (alternate definition)
       LET A = TRICOMI(X,GAMMA)    - Tricomi's incomplete gamma
       LET A = GAMMAIC(X,GAMMA)    - complementary incomplete Gamma
       LET A = GAMMAR(X)           - reciprocal Gamma function
       LET A = DIGAMMA(X)          - digamma function
       LET A = POCH(X,A)           - Pochhammer's generalized symbol
       LET A = POCH1(X,A)          - Pochhammer's generalized symbol of
                                     the first order

       LET A = BESSY0(X)           - Bessel function second kind order 0
       LET A = BESSY1(X)           - Bessel function second kind order 1
       LET A = BESSI0(X)           - modified Bessel function of order 0
       LET A = BESSI1(X)           - modified Bessel function of order 1
       LET A = BESSI0E(X)          - exponentially scaled modified Bessel
                                     function of order 0
       LET A = BESSI1E(X)          - exponentially scaled modified Bessel
                                     function of order 1
       LET A = BESSK0(X)           - modified Bessel function of third
                                     kind order 0
       LET A = BESSK1(X)           - modified Bessel function of third
                                     kind order 1
       LET A = BESSK0E(X)          - exponentially scaled modified Bessel
                                     function of third kind order 0
       LET A = BESSK1E(X)          - exponentially scaled modified Bessel
                                     function of third kind order 1
       LET A = BESSJN(X,V)         - Bessel function of first kind of
                                     order V (V can be fractional)
       LET A = BESSYN(X,V)         - Bessel function of second kind of
                                     order V (V can be fractional)
       LET A = BESSIN(X,V)         - modified Bessel function of order V
                                     (V can be fractional)
       LET A = BESSINE(X,V)        - exponentially sclaed modified Bessel
                                     function of order V (V can be
                                     fractional)
       LET A = BESSKN(X,V)         - modified Bessel function of third
                                     kind order V (V can be fractional)
       LET A = BESSKNE(X,V)        - exponentially scaled modified Bessel
                                     function of third kind order V (V
                                     can be fractional)
       LET A = CBESSJR(X,CX,V)     - real part of Bessel function of
                                     first kind of order V (V can be
                                     fractional) and complex argument
       LET A = CBESSJI(X,CX,V)     - imaginary part of Bessel function
                                     of first kind of order V (V can be
                                     fractional) and complex argument
       LET A = CBESSYR(X,CX,V)     - real part of Bessel function of
                                     second kind of order V (V can be
                                     fractional) and complex argument
       LET A = CBESSYI(X,CX,V)     - imaginary part of Bessel function
                                     of second kind of order V (V can be
                                     fractional) and complex argument
       LET A = CBESSIR(X,CX,V)     - real part of modified Bessel function
                                     of order V (V can be fractional) and
                                     complex argument
       LET A = CBESSII(X,CX,V)     - imaginary part of modified Bessel
                                     function of order V (V can be
                                     fractional) and complex argument
       LET A = CBESSKR(X,CX,V)     - real part of modified Bessel function
                                     of third kind and of order V (V can
                                     be fractional) and complex argument
       LET A = CBESSKI(X,CX,V)     - imaginary part of modified Bessel
                                     function of third kind and order V
                                     (V can be fractional) and complex
                                     argument
       LET A = AIRY(X)             - Airy function
       LET A = BAIRY(X)            - Bairy function

       LET A = DAWSON(X)           - Dawson integral
       LET A = SPENCE(X)           - Spence dilogarithm function
       LET A = EXPINT1(X)          - exponential integral of order 1
       LET A = EXPINTE(X)          - exponential integral
       LET A = EXPINTN(X,N)        - exponential integral of order N 
                                     (N = 0, 1, 2, ...)
       LET A = LOGINT(X)           - logarithmic integral
       LET A = SININT(X)           - sine integral
       LET A = COSINT(X)           - cosine integral
       LET A = SINHINT(X)          - hyperbolic sine integral
       LET A = COSHINT(X)          - hyperbolic cosine integral

       LET A = RF(X,Y,Z)           - Carlson's elliptic integral of the
                                     first kind
       LET A = RD(X,Y,Z)           - Carlson's elliptic integral of the
                                     second kind
       LET A = RJ(X,Y,Z,P)         - Carlson's elliptic integral of the
                                     third kind
       LET A = RC(X,Y)             - Carlson's degenerate elliptic
                                     integral
       LET A = ELLIPC1(X)          - Legendre complete elliptic
                                     integral of the first kind. 
       LET A = ELLIPC2(X)          - Legendre complete elliptic
                                     integral of the second kind. 
       LET A = ELLIP1(PHI,ALPHA)   - Legendre elliptic integral of the
                                     first kind. 
       LET A = ELLIP2(PHI,ALPHA)   - Legendre elliptic integral of the
                                     second kind. 
       LET A = ELLIP3(PHI,N,ALPHA) - Legendre elliptic integral of the
                                     third kind. 

       LET A = CHU(X,A,B)          - confluent hypergeometric function

       LET A = CABS(XR,XC)         - complex absolute value
       LET YR = CCOS(XR,XC)        - real component of complex cosine
       LET YC = CCOSI(XR,XC)       - complex component of complex
                                     cosine
       LET YR = CEXP(XR,XC)        - real component of complex
                                     exponential
       LET YC = CEXPI(XR,XC)       - complex component of complex
                                     exponential
       LET YR = CLOG(XR,XC)        - real component of complex
                                     natural logarithm
       LET YC = CLOGI(XR,XC)       - complex component of complex
                                     natural logarithm
       LET YR = CSIN(XR,XC)        - real component of complex sine
       LET YC = CSINI(XR,XC)       - complex component of complex sine
       LET YR = CSQRT(XR,XC)       - real component of complex
                                     square root
       LET YC = CSQRTI(XR,XC)       - complex component of complex
                                     square root

       BESSJ0 and BESSJ1 were added as synonyms for BESS0 and BESS1.

    These new functions are based on code from the SLATEC library.

 2) The following new probability library functions were added:

       LET A = DISCDF(X,N)     - cdf for discrete uniform distribution
       LET A = DISPDF(X,N)     - pdf for discrete uniform distribution
       LET A = DISPPF(P,N)     - ppf for discrete uniform distribution

       LET A = TRICDF(X,C)     - cdf for triangular distribution
       LET A = TRIPDF(X,C)     - pdf for triangular distribution
       LET A = TRIPPF(P,C)     - ppf for triangular distribution

       LET A = BETCDF(X,C)     - cdf for Beta distribution
       LET A = BETPDF(X,C)     - pdf for Beta distribution
       LET A = BETPPF(P,C)     - ppf for Beta distribution

       LET A = HYPCDF(X,K,N,M) - cdf for hypergeometric distribution
       LET A = HYPPDF(X,K,N,M) - pdf for hypergeometric distribution
       LET A = HYPPPF(P,K,N,M) - ppf for hypergeometric distribution

       LET A = GAMPDF(X,GAMMA) - pdf for Gamma distribution

       LET A = NCBCDF(X,ALPHA,BETA,LAMBDA) - cdf for non-central Beta
       LET A = NCBPPF(P,ALPHA,BETA,LAMBDA) - ppf for non-central Beta

       LET A = NCCCDF(X,NU,LAMBDA)         - cdf for non-central
                                             chi-square
       LET A = NCCPPF(P,NU,LAMBDA)         - ppf for non-central 
                                             chi-square
       LET LAMBDA = NCCNCP(P,NU,CDF)       - find non-centrality 
                                             parameter for non-central
                                             chi-square

       LET A = NCFCDF(X,NU1,NU2,LAMBDA)    - cdf for non-central F
       LET A = NCFPPF(P,NU1,NU2,LAMBDA)    - ppf for non-central F
       LET A = DNFCDF(X,NU1,NU2,LAM1,LAM2) - cdf for doubly non-central F
       LET A = NCFPPF(P,NU1,NU2,LAM1,LAM2) - ppf for doubly non-central F

       LET A = NCTCDF(X,NU1,LAMBDA)        - cdf for non-central T
       LET A = NCTPPF(P,NU1,LAMBDA)        - ppf for non-central T
       LET A = DNTCDF(X,NU1,LAM1,LAM2)     - cdf for doubly non-central T
       LET A = NCTPPF(P,NU1,LAM1,LAM2)     - ppf for doubly non-central T

       LET A = VONCDF(X,B)     - cdf for Von Mises distribution
       LET A = VONPDF(X,B)     - pdf for Von Mises distribution
       LET A = VONPPF(P,B)     - ppf for Von Mises distribution

       LET A = BVNCDF(X1,X2,P) - cdf for bivariate normal distribution

 3) The following probability plots were added:

       LET C = <value>
       TRIANGULAR PROBABILITY PLOT Y

       LET N = <value>
       DISCRETE UNIFORM PROBABILITY PLOT Y

       LET ALPHA = <value>
       LET BETA = <value>
       LET LAMBDA = <value>
       NONCENTRAL BETA PROBABILITY PLOT Y

       LET NU = <value>
       LET LAMBDA = <value>
       NONCENTRAL CHI-SQUARE PROBABILITY PLOT Y

       LET NU1 = <value>
       LET NU2 = <value>
       LET LAMBDA = <value>
       NONCENTRAL F PROBABILITY PLOT Y

       LET NU = <value>
       LET LAMBDA = <value>
       NONCENTRAL T PROBABILITY PLOT Y

       LET NU1 = <value>
       LET NU2 = <value>
       LET LAMBDA1 = <value>
       LET LAMBDA2 = <value>
       DOUBLY NONCENTRAL F PROBABILITY PLOT Y

       LET NU = <value>
       LET LAMBDA1 = <value>
       LET LAMBDA2 = <value>
       DOUBLY NONCENTRAL T PROBABILITY PLOT Y

       LET K = <value>
       LET N = <value>
       LET M = <value>
       HYPERGEOMETIC PROBABILITY PLOT Y

       LET B = <value>
       VON MISES PROBABILITY PLOT Y

 4) The DRAWDATA command was added.  This command is similar to
    the DRAW command, but it works in units of the most recent plot
    rather than 0 to 100 units.

 5) The COPY command has changed.  In prior versions, the COPY command
    generated a plot on the Tektronix 4631 harcopy unit.  However, this
    is now an obsolete device.

    The new copy command copies all or portions of a file to another
    file.  That is,

      COPY FILE1.DAT FILE2.DAT

    copies the contents of FILE1.DAT into FILE2.DAT.  Likewise,

      COPY FILE1.DAT FILE2.DAT FOR I = 12 1 24

    copies lines 12 to 24 of FILE1.DAT into FILE2.DAT.

----------------------------------------------------------------
The following enhancements were made to DATAPLOT JUNE, 1994.
----------------------------------------------------------------

 1) The DATAPLOT INTERPOLATION command performs univariate cubic
    spline interpolation.  The following additional interpolation
    commands were added:

      LET Y2 = LINEAR INTERPOLATION Y1 X1 X2
      LET Z2 = BILINEAR INTERPOLATION Z1 Y1 X1 Y2 X2
      LET Z2 = BIVARIATE INTERPOLATION Z1 Y1 X1 Y2 X2
      LET Z2 = 2D INTERPOLATION Z1 Y1 X1 Y2 X2
      
    The LINEAR INTERPOLATION command simply does univariate linear
    interpolation.  In most cases, the cubic spline interpolation is
    preferred.  However, there are occassionally cases where linear
    interpolation may be preferred (typically when the original data
    contains relativelty large gaps in the data).

    For the bivariate case, there are two types of interpolation.  You
    can start with a grid and interpolate points off of the grid.  The
    other type starts with "random" points and interpolates to form a
    grid.  The BILINEAR and BIVARIATE interpolation start with a grid
    while 2D starts with random points and forms a grid.  The BIVARIATE
    case uses the B2INK and B2VAL routines from CMLIB while 2D 
    INTERPOLATION uses the LOTPS routine from the SLATEC library.

    All of these new interpolation commands are documented further
    in the on-line help (e.g., enter HELP BIVARIATE INTERPOLATION).

 2) A univariate function optimization command was added:

        LET A = OPTIMIZE F WRT X FOR X = A TO B

    A command to adjust the convergence tolerance for this
    optimization was also added:

        OPTIMIZATION TOLERANCE <value>
 
     Enter HELP OPTIMIZE for details.

  3) Generally, it is much faster to do solid fills of complex regions
     in hardware rather than software when available.  By default,
     both the Postscript and X11 drivers do this.  However, there may
     ocassionally be cases where the hardware fill does not work
     correctly.  The commands SET X11 HARDWARE FILL <ON/OFF> and 
     SET X11 POSTSCRIPT HARDWARE FILL <ON/OFF> were added to specify
     whether non-convex solid polygon fills are done in software or
     hardware.  Hardware is the default.

 4) Some bug fixes were implemented.

------------------------------------------------------------

      Changes prior to this are no longer in the news file
      because they are documented in the Reference Manual and
      the on-line help.

------------------------------------------------------------

      YOU HAVE JUST ACCESSED THE  FILE DPNEWF.

